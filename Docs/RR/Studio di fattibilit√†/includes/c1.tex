\newpage
\section{Capitolato C1 (Scelto)}

\subsection{Descrizione}

Il capitolato proposto da ItalianaSoftware riguarda la creazione di
una web app per la gestione di microservizi. Le funzionalità che dovranno essere fornite saranno la vendita e l'acquisto di microservizi, tramite un apposito marketplace, e per ogni microservizio, sarà fornita dal fornitore l'interfaccia pubblica delle proprie API. Il progetto non richiede soltanto di gestire la compravendita dei microservizi attraverso il marketplace, ma anche di integrare delle funzioni di controllo tramite un API Gateway.
Quest'ultimo dovrà essere in grado di effettuare un'analisi statistica dei dati di utilizzo più rilevanti e regolare l'accesso alle API registrate
tramite opportune API key, limitando l'utilizzo alle API a coloro che
non sono in possesso di una chiave valida. La creazione della web
app sarà affrontata tramite l'utilizzo delle consuete tecnologie per
lo sviluppo web lato front-end e back-end. La peculiarità del progetto
è la realizzazione dell'API Gateway in linguaggio Jolie, come da richiesta
del committente. Jolie, infatti, rappresenta un nuovo emegente linguaggio
di programmazione open-source orientato ai microservizi.

\subsection{Dominio applicativo}

Lo scopo di questo prodotto è fornire la possibilità a sviluppatori
e utilizzatori di avere a propria disposizione un valido strumento
per l'acquisto e la vendita regolamentata di microservizi. Questo
garantisce degli standard qualitativi al cliente finale, che
può valutare e scegliere ciò che più rispetta le proprie esigenze.
Il bacino di utenza riguarda, dunque, tutte le aziende e gli sviluppatori
che si affacciano al mondo dei microservizi, oltre a coloro che li
metteranno a disposizione.

\subsection{Tecnologie}

Per la realizzazione di questo capitolato sono necessarie conoscenze
di base per lo sviluppo di applicativi web. Nel nostro particolare
caso la scelta può ricadere su:
\begin{itemize}
	\item \textbf{HTML5 e CSS3} per la struttura e l'aspetto grafico.
	\item \textbf{Bootstrap 3} come framework CSS.
	\item \textbf{Javascript e jQuery} per la parte comportamentale front-end.
	\item \textbf{PHP 7} per le funzionalità back-end.
	\item \textbf{Oracle MySQL} come database SQL.
	\item \textbf{Jolie} per la realizzazione dell'API Gateway
\end{itemize}

\subsection{Aspetti critici}

L'aspetto cruciale nella realizzazione del progetto riguarda, secondo la nostra analisi, la corretta implementazione dell'API Gateway, che permetta le funzioni
richieste dal committente. La volontà di introdurre un Service Level
Agreement (SLA) per i servizi, e la conseguente necessità di gestione
avanzata delle statistiche tramite API Gateway, evidenziano come questa
parte del progetto sia in realtà il punto su cui è necessario prestare la massima attenzione.

\subsection{Considerazioni conclusive}

Il capitolato è stato designato come il più allettante da parte del
nostro gruppo per numerosi aspetti. Si manifesta, come prima cosa,
l'interesse condiviso da parte di tutti i componenti per le tecnologie
inerenti alla parte web. Allo stesso tempo, acquisire un nuovo linguaggio
di programmazione per questa sfida risulta un'esperienza
formativa e interessante: di fatto, oggigiorno, queste nuove tecnologie vengono sempre più utilizzate, anche da grosse aziende (vedi Amazon, Netflix, Google, Facebook, etc.). Infine, il gruppo ha manifestato un forte interesse nell'intraprendere un'attività lavorativa con un'azienda emergente, nata proprio da un progetto universitario.