\newpage
\section{Introduzione}
\subsection{Scopo del documento}
Lo scopo del documento è quello di presentare le motivazioni e la breve analisi che ha indirizzato il gruppo verso la scelta del capitolato C1. Tutti i capitolati vengono analizzati con la medesima metodologia, evidenziandone in particolare le tecnologie necessarie, gli aspetti cruciali e una fase conclusiva contenente il giudizio e le opinioni del gruppo.

\subsection{Scopo del prodotto}
Lo scopo del prodotto è la realizzazione di un API Market per l'acquisto e la vendita di microservizi. Il sistema offrirà la possibilità di registrare nuove API per la vendita, permetterà la consultazione e ricerca ai potenziali acquirenti, gestendo i permessi tramite creazione e controllo di relative API key. Il sistema, oltre alla webapp stessa, sarà corredato di un API Gateway per la gestione delle richieste e il controllo delle chiavi, e fornirà funzionalità avanzate di statistiche per il gestore della piattaforma nonchè per gli utilizzatori.

\subsection{Riferimenti normativi}
\begin{itemize}
	\item \textsc{NormeDiProgetto 1\_0\_0.pdf}
\end{itemize}

\subsection{Riferimenti informativi}
\begin{itemize}
	\item \textbf{Capitolato C1:} APIM: An API Market Platform \\ \url{http://www.math.unipd.it/~tullio/IS-1/2016/Progetto/C1.pdf}
	\item \textbf{Capitolato C2:} AtAVi: Accoglienza tramite Assistente Virtuale \\ \url{http://www.math.unipd.it/~tullio/IS-1/2016/Progetto/C2.pdf}
	\item \textbf{Capitolato C3:} DeGeOP: A Designer and Geo-localizer Web App for Organizational Plants \\
	\url{http://www.math.unipd.it/~tullio/IS-1/2016/Progetto/C3.pdf}
	\item \textbf{Capitolato C4:} eBread: applicazione di lettura per dislessici \\
	\url{http://www.math.unipd.it/~tullio/IS-1/2016/Progetto/C4.pdf}
	\item \textbf{Capitolato C5:} Monolith: an interactive bubble provider \\
	\url{http://www.math.unipd.it/~tullio/IS-1/2016/Progetto/C5.pdf}
	\item \textbf{Capitolato C6:} SWEDesigner: editor di diagrammi UML con generazione di codice \\
	\url{http://www.math.unipd.it/~tullio/IS-1/2016/Progetto/C6.pdf}
\end{itemize}

\subsection{Glossario}
Per semplificare la consultazione e disambiguare alcune terminologie tecniche, alcune voci indicate con la lettera \textit{G} a pedice sono descritte approfonditamente nel documento apposito \textsc{Glossario 1\_0\_0.pdf}