\newpage
\section{Introduzione}

\subsection{Scopo del documento}
\textbf{Da fare}

\subsection{Scopo del prodotto}
Lo scopo del prodotto è la realizzazione di un API Market\ped{\textit{G}} per l'acquisto e la vendita di microservizi. Il sistema offrirà la possibilità di registrare nuove API\ped{\textit{G}} per la vendita, permetterà la consultazione e la ricerca di API\ped{\textit{G}} ai potenziali acquirenti, gestendo i permessi di accesso ed utilizzo tramite creazione e controllo di relative API key\ped{\textit{G}}. Il sistema, oltre alla web app stessa, sarà corredato di un API Gateway\ped{\textit{G}} per la gestione delle richieste e il controllo delle chiavi, e fornirà funzionalità avanzate di statistiche per il gestore della piattaforma e per i fornitori dei microservizi.

\subsection{Riferimenti normativi}
\begin{itemize}
\item \textsc{NormeDiProgetto 1\_0\_0.pdf};
\item \textbf{Capitolato d’appalto C1:} APIM: An API Market Platform\\ \url{http://www.math.unipd.it/~tullio/IS-1/2016/Progetto/C1.pdf};
\end{itemize}
\textbf{Da fare}

\subsection{Riferimenti informativi}
\textbf{Da fare}


\subsection{Glossario}
Per semplificare la consultazione e disambiguare alcune terminologie tecniche, le voci indicate con la lettera \textit{G} a pedice sono descritte approfonditamente nel documento \textsc{Glossario1\_0\_0.pdf}.