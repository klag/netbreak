\newpage
\section{Analisi dei rischi}
In questa sezione del documento vengono elencati i potenziali rischi che potrebbero verificarsi durante la realizzazione del prodotto API Market\ped{\textit{G}} e la metodologia adottata per la loro identificazione.

\subsection{Metodologia}

La procedura che il gruppo \gruppo\ intende utilizzare per la gestione dei rischi è composta dalle seguenti fasi:
\begin{itemize}
	\item \textbf{Identificazione:} vengono individuati tutti i potenziali rischi che possono presentarsi durante lo sviluppo del progetto al fine di studiarne la loro natura. Essi, infatti, possono essere di tre tipi:
	\begin{itemize}
		\item \textbf{Progetto:} relativi a pianificazione, strumenti e risorse;
		\item \textbf{Prodotto:} relativi a conformità e aspettative del committente;
		\item \textbf{Mercato:} relativi a costi e concorrenza.
	\end{itemize}
	\item \textbf{Analisi:} per ogni rischio, si studiano le probabilità di avvenimento e le possibili conseguenze, al fine di capirne criticità e grado di incidenza sul progetto;
	\item \textbf{Pianficazione:} vengono istituiti dei metodi  per prevenire i rischi individuati e definiti dei piani alternativi per la loro gestione.
	\item \textbf{Controllo:} ogni rischio viene costantemente monitorato al fine di mitigarne gli effetti. Alcune possibili attività possono essere:
		\begin{itemize}
		\item \textbf{Verifica del livello di rischio};
		\item \textbf{Riconoscimento, trattamento e aggiornamento delle strategie}.
		\end{itemize}
	\end{itemize}
\MakeUppercase{è} inoltre di fondamentale importanza riportare periodicamente ogni rischio serio all'attenzione dell'amministratore di progetto.

\subsection{Fattori di rischio}

Per ogni rischio viene fornito il seguente elenco di informazioni, necessario per comprenderne la natura:
\begin{itemize}
	\item \textbf{Nome:} identificativo per discriminare l'ambito del rischio;
	\item \textbf{Descrizione:} breve descrizione dello scenario con cui si presenta il rischio individuato;
	\item \textbf{Occorrenza:} indica la possibilità che si verifichi il rischio;
	\item \textbf{Pericolosità:} indica il grado di pericolosità del rischio;
	\item \textbf{Riconoscimento:} fornisce un metodo che permette di riconoscere il rischio;
	\item \textbf{Trattamento:} fornisce una soluzione affinchè si riducano ulteriormente le possibilità di occorrenza.
\end{itemize}
\textbf{N.B.:} le voci Occorrenza e Pericolosità possono assumere i valori \{1, 2, 3\}, che corrispondono rispettivamente a livello basso, medio e alto.\\\\
Un rischio può dipendere da diversi fattori:
\begin{itemize}
	\item Tecnologie;
	\item Rapporti personali;
	\item Organizzazione del lavoro;
	\item Requisiti e rapporti con gli stakeholder;
	\item Tempi e costi.
\end{itemize}

\subsubsection{Tecnologie}

Di seguito, sono elencati e descritti i possibili scenari di rischi a livello tecnologico.

\paragraph{Tecnologie adottate}
\begin{itemize}
	\item \textbf{Descrizione:} lo studio e l'utilizzo delle tecnologie web per la realizzazione del prodotto richiesto, può portare delle difficoltà al momento dell'integrazione con la tecnologia a microservizi. Inoltre, è possibile fare affidamento sul committente per problemi e/o incomprensioni riguardanti Jolie e le sue feature;
	\item \textbf{Occorrenza:} 2;
	\item \textbf{Pericolosità:} 2;
	\item \textbf{Riconoscimento:} il Responsabile di Progetto deve verificare il grado di preparazione di ogni componente del gruppo in merito alle tecnologie scelte;
	\item \textbf{Trattamento:} ogni membro del gruppo ha il compito di studiare autonomamente tutte le tecnologie necessarie alla realizzazione del prodotto, facendo uso del materiale e dei documenti forniti dall’Amministratore di Progetto.
\end{itemize}

\paragraph{Guasti hardware}
\begin{itemize}
	\item \textbf{Descrizione:} ogni componente del gruppo è dotato di computer portatili non professionali, quindi bisogna prendere in considerazione il rischio di rottura di uno di questi;
	\item \textbf{Occorrenza:} 1;
	\item \textbf{Pericolosità:} 2;
	\item \textbf{Riconoscimento:} ogni membro del gruppo deve prestare attenzione verso i propri strumenti hardware di lavoro;
	\item \textbf{Trattamento:} ogni membro del gruppo possiede un dispositivo di riserva, in modo da poter proseguire il lavoro in caso di guasti o malfunzionamenti hardware.
\end{itemize}

\paragraph{Malfunzionamenti del server}
\begin{itemize}
	\item \textbf{Descrizione:} il server destinato ad ospitare il progetto presenta dei malfunzionamenti, il che mette a rischio l'intero lavoro per il gruppo;
	\item \textbf{Occorrenza:} 1;
	\item \textbf{Pericolosità:} 3;
	\item \textbf{Riconoscimento:} il progetto risiede anche su un server locale, che funge da alternativa nel caso il server principale presenti dei problemi;
	\item \textbf{Trattamento:} è compito dell'Amministratore risolvere il malfunzionamento nel minor tempo possibile e riportare allo stato funzionante il server principale.
\end{itemize}

\subsubsection{Rapporti personali}

Di seguito, sono elencati e descritti i possibili scenari di rischi a livello del personale.

\paragraph{Problemi interni al team}
\begin{itemize}
	\item \textbf{Descrizione:} dato che, per ogni membro del gruppo, si tratta della prima esperienza di lavoro in un team di queste dimensioni, è importante non sottovalutare gli eventuali problemi di collaborazione. Essi, infatti, potrebbero portare instabilità interna, con conseguenti ritardi nella presentazione del progetto;
	\item \textbf{Occorrenza:} 1;
	\item \textbf{Pericolosità:} 3;
	\item \textbf{Riconoscimento:} è importante che ci sia un rapporto costante con il Responsabile di Progetto, affinchè possa monitorare e gestire ogni tipo di problematica sorta;
	\item \textbf{Trattamento:} in caso di contrasti tra membri del gruppo, è compito del Responsabile di Progetto affidare alle persone coinvolte, delle attività che non siano strettamente legate. Questo fa sì che non venga influenzato il clima di lavoro per gli altri componenti del gruppo.
\end{itemize}

\paragraph{Problemi personali individuali}
\begin{itemize}
	\item \textbf{Descrizione:} possono verificarsi problemi organizzativi dovuti a sovrapposizioni di impegni e necessità proprie di ogni membro del gruppo. Ad esempio, un componente del team è anche un lavoratore presso un'azienda, quindi occorre prestare attenzione alla gestione di casi come questo;
	\item \textbf{Occorrenza:} 2;
	\item \textbf{Pericolosità:} 3;
	\item \textbf{Riconoscimento:} per evitare rischi di disorganizzazione, occorre fornire preventivamente e tempestivamente al \Res, tutti gli impegni di ogni componente del gruppo. Nel caso specifico del membro lavoratore, quest'ultimo dovrà fornire costantemente un calendario aggiornato contenente tutti i suoi impegni lavorativi;
	\item \textbf{Trattamento:} per ogni impegno notificato, il \Res\ avrà il compito di ripianificare parte delle attività da svolgere per sopperire alle mancanze lavorative. Nel caso del membro lavoratore, dovranno essergli forniti gli strumenti necessari, affinchè egli possa essere aggiornato sull'andamento dello sviluppo del progetto e non influenzi negativamente il lavoro del team.
\end{itemize}

\subsubsection{Organizzazione del lavoro}

Di seguito, sono elencati e descritti i possibili scenari di rischi a livello organizzativo.

\paragraph{Pianificazione errata}
\begin{itemize}
	\item \textbf{Descrizione:} durante la fase di Pianificazione, è possibile che i tempi per lo svolgimento di alcune attività vengano calcolati in modo errato;
	\item \textbf{Occorrenza:} 2;
	\item \textbf{Pericolosità:} 3;
	\item \textbf{Riconoscimento:} bisogna monitorare costantemente lo stato delle attività nel programma di project management, in modo da gestire eventuali ritardi nello sviluppo delle attività stesse;
	\item \textbf{Trattamento:} per ogni attività, è previsto un periodo maggiore di quanto normalmente richiesto. Ciò consente ad un eventuale ritardo di non impattare negativamente sulla durata totale del progetto.
\end{itemize}

\subsubsection{Requisiti e rapporti con gli stakeholder}

Di seguito, sono elencati e descritti i possibili scenari di rischi a livello dei requisiti.

\paragraph{Incomprensioni sui requisiti}
\begin{itemize}
	\item \textbf{Descrizione:} alcuni requisiti individuati dagli Analisti possono essere interpretati in modo errato oppure possono, a loro volta, implicare ulteriori requisiti. Inoltre, è possibile che alcuni requisiti vengano aggiunti, modificati o eliminati, a seconda degli accordi presi con il committente;
	\item \textbf{Occorrenza:} 2;
	\item \textbf{Pericolosità:} 3;
	\item \textbf{Riconoscimento:} vengono fissati degli incontri con il proponente, in modo da concordare sulla visione del prodotto per fornire un prodotto conforme alla richieste. Ad ogni revisione prevista, i documenti relativi al progetto verranno consegnati e valutati dal committente;
	\item \textbf{Trattamento:} è indispensabile correggere tutti gli eventuali errori e/o le imprecisioni individuate dal committente in seguito ad ogni revisione.
\end{itemize}

\subsubsection{Tempi e costi}

Di seguito, sono elencati e descritti i possibili scenari di rischi a livello di valutazione dei costi.

\paragraph{Stime e previsioni}
\begin{itemize}
	\item \textbf{Descrizione:} i tempi stabiliti nella pianificazione delle attività per lo svolgimento del progetto vengono sovrastimate o sottostimate. Ciò può comportare una variazione sul costo preventivo presentato;
	\item \textbf{Occorrenza:} 2;
	\item \textbf{Pericolosità:} 2;
	\item \textbf{Riconoscimento:} un’attività si dice sottostimata se occupa molto più tempo di quello preventivato; invece, un'attività si dice sovrastimata se occupa meno tempo rispetto a quello previsto. Il Responsabile di Progetto deve controllare con attenzione il
	programma di project management ed intervenire tempestivamente per modificare la pianificazione e il rendiconto dei costi;
	\item \textbf{Trattamento:} ogni qualvolta viene assegnata un'attività ad un membro del team, quest'ultimo ha l'obbligo di rispettare i tempi e le scadenza stabilite dal Responsabile di Progetto.
\end{itemize}