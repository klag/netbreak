\newpage

% template tabella!! Da cancellare solo quando finito
%\begin{table}[H]
%	\begin{center}
%		\begin{tabular}{|c|c|c|c|c|c|c|c|}
%			\hline
%			\textbf{nome} & \multicolumn{6}{c|}{\textbf{Ore per ruolo}} & \textbf{Ore totali} \\\cline{2-7}
%			& \textbf{Resp} & \textbf{Amm} & \textbf{An} & \textbf{Proj} & \textbf{Prog} & \textbf{Ver} & \\
%			\hline
%			\MC			&		&		&		&		&		&		&		\\
%			\hline
%			\AN			&		&		&		&	 	&		&		& 		\\
%			\hline
%			\DAN		&		&		&		&		&		&		&		\\
%			\hline
%			\AS			&		&	 	&	 	&		&	 	& 		&		\\
%			\hline
%			\NS 		&		&		&		&		&		& 		&		\\
%			\hline
%			\DS			& 		&		&		&		&		&		&		\\
%			\hline
%		\end{tabular}
%	\end{center}
%	\caption{Ore per componente, \AdR}
%\end{table}


\section{Registro Suddivisione Lavoro}
Nei seguenti paragrafi verrà spiegato come il gruppo intende dividere soddisfare alcune regole del progetto:
\begin{itemize}
	\item Tutti i componenti devono ricoprire almeno una volta tutti i ruoli; 
	\item Un componente del gruppo può ricoprire più ruoli contemporaneamente, purchè non entri in conflitto d'interesse, ad esempio non può Verificare il lavoro da lui svolto.
\end{itemize}

\subsection{Periodo di Analisi dei Requisiti}
Durante il periodo di \AdR, il lavoro dei membri sarà suddiviso come segue:

\begin{table}[H]
	\begin{center}
		\begin{tabular}{|c|c|c|c|c|c|c|c|}
			\hline
			\textbf{Nome} & \multicolumn{6}{c|}{\textbf{Ore per ruolo}} & \textbf{Ore totali} \\\cline{2-7}
			& \textbf{Resp} & \textbf{Amm} & \textbf{An} & \textbf{Proj} & \textbf{Prog} & \textbf{Ver} & \\
			\hline
			\MC			&		&		&	16	&		&		&	16	&	32	\\
			\hline
			\AN			&		&	4	&	6	&	 	&		&	22	& 	32	\\
			\hline
			\DAN		&		&	3	&	29	&		&		&		&	32	\\
			\hline
			\AS			&	20	&	 	&	12 	&		&	 	& 		&	32	\\
			\hline
			\NS 		&	18	&	3	&	11	&		&		& 		&	32	\\
			\hline
			\DS			& 		&	2	&	5	&		&		&	25	&	32	\\
			\hline
		\end{tabular}
	\end{center}
	\caption{Ore per componente, \AdR}
\end{table}

\begin{figure}[H]
	\centering
	\includegraphics[scale=0.6]{img/6-1.png}
	\caption{Suddivisione ruoli per componente, Analisi dei Requisiti}
\end{figure}

\subsection{Periodo di Progettazione Architetturale}
Durante il periodo di progettazione architetturale, il lavoro dei membri sarà suddiviso come segue:

\begin{table}[H]
	\begin{center}
		\begin{tabular}{|c|c|c|c|c|c|c|c|}
			\hline
			\textbf{Nome} & \multicolumn{6}{c|}{\textbf{Ore per ruolo}} & \textbf{Ore totali} \\\cline{2-7}
			& \textbf{Resp} & \textbf{Amm} & \textbf{An} & \textbf{Proj} & \textbf{Prog} & \textbf{Ver} & \\
			\hline
			\MC			&		&	3	&		&	31	&		&		&   34	\\
			\hline
			\AN			&	3	&		&		&	31	&		&		& 	34	\\
			\hline
			\DAN		&		&	2	&		&	17	&		&	14	&	33	\\
			\hline
			\AS			&		&	 	&	 	&	14	&	 	& 	19	&	33	\\
			\hline
			\NS 		&		&		&		&	14	&		& 	19	&	33	\\
			\hline
			\DS			& 	2	&		&		&	31	&		&		&	33	\\
			\hline
		\end{tabular}
	\end{center}
	\caption{Ore per componente, Progettazione Architetturale}
\end{table}

\begin{figure}[H]
	\centering
	\includegraphics[scale=0.6]{img/6-2.png}
	\caption{Suddivisione ruoli per componente, Progettazione Architetturale}
\end{figure}

\subsection{Periodo di Progettazione Architetturale Dettagliata}
Durante il periodo di progettazione architetturale, il lavoro dei membri sarà suddiviso come segue:

\begin{table}[H]
	\begin{center}
		\begin{tabular}{|c|c|c|c|c|c|c|c|}
			\hline
			\textbf{Nome} & \multicolumn{6}{c|}{\textbf{Ore per ruolo}} & \textbf{Ore totali} \\\cline{2-7}
			& \textbf{Resp} & \textbf{Amm} & \textbf{An} & \textbf{Proj} & \textbf{Prog} & \textbf{Ver} & \\
			\hline
			\MC			&	2	&		&		&	6	&		&	11	&	19	\\
			\hline
			\AN			&		&		&		&	8	&   	&	11	& 	19	\\
			\hline
			\DAN		&	3	&		&		&	17	&		&		&	20	\\
			\hline
			\AS			&		&	3	&	 	&	17	&	 	& 		&	20	\\
			\hline
			\NS 		&		&	3	&		&	17	&		& 		&	20	\\
			\hline
			\DS			& 		&		&		&	7	&		&	13	&	20	\\
			\hline
		\end{tabular}
	\end{center}
	\caption{Ore per componente, Progettazione Architetturale Dettagliata}
\end{table}

\begin{figure}[H]
	\centering
	\includegraphics[scale=0.6]{img/6-3.png}
	\caption{Suddivisione ruoli per componente, Progettazione Architetturale Dettagliata}
\end{figure}

\subsection{Periodo di Codifica}
Durante il periodo di codifica, il lavoro dei membri sarà suddiviso come segue:

\begin{table}[H]
	\begin{center}
		\begin{tabular}{|c|c|c|c|c|c|c|c|}
			\hline
			\textbf{Nome} & \multicolumn{6}{c|}{\textbf{Ore per ruolo}} & \textbf{Ore totali} \\\cline{2-7}
			& \textbf{Resp} & \textbf{Amm} & \textbf{An} & \textbf{Proj} & \textbf{Prog} & \textbf{Ver} & \\
			\hline
			\MC			&		&		&		&		&	18	&	18	&	36	\\
			\hline
			\AN			&	3	&		&		&	 	&	32	&		& 	35	\\
			\hline
			\DAN		&		&		&		&		&	14	&	21	&	35	\\
			\hline
			\AS			&		&	 	&	 	&		&	14 	& 	22	&	36	\\
			\hline
			\NS 		&		&	3	&		&		&	32	& 		&	35	\\
			\hline
			\DS			& 	3	&		&		&		&	33	&		&	36	\\
			\hline
		\end{tabular}
	\end{center}
	\caption{Ore per componente, Codifica}
\end{table}

\begin{figure}[H]
	\centering
	\includegraphics[scale=0.6]{img/6-4.png}
	\caption{Suddivisione ruoli per componente, Codifica}
\end{figure}

\subsection{Periodo di Verifica}
Durante il periodo di verifica, il lavoro dei membri sarà suddiviso come segue:

\begin{table}[H]
	\begin{center}
		\begin{tabular}{|c|c|c|c|c|c|c|c|}
			\hline
			\textbf{Nome} & \multicolumn{6}{c|}{\textbf{Ore per ruolo}} & \textbf{Ore totali} \\\cline{2-7}
			& \textbf{Resp} & \textbf{Amm} & \textbf{An} & \textbf{Proj} & \textbf{Prog} & \textbf{Ver} & \\
			\hline
			\MC			&		&	3	&		&	7	&		&	6	&	16	\\
			\hline
			\AN			&		&		&		&	 	&		&	17	& 	17	\\
			\hline
			\DAN		&	3	&		&		&		&		&	14	&	17	\\
			\hline
			\AS			&		&	 	&	 	&	4	&	 	& 	12	&	16	\\
			\hline
			\NS 		&		&		&		&	4	&		& 	13	&	17	\\
			\hline
			\DS			& 		&		&		&		&		&	16	&	16	\\
			\hline
		\end{tabular}
	\end{center}
	\caption{Ore per componente, Verifica}
\end{table}

\begin{figure}[H]
	\centering
	\includegraphics[scale=0.6]{img/6-5.png}
	\caption{Suddivisione ruoli per componente, Verifica}
\end{figure}

\subsection{Ore totali per componente}
Il consuntivo delle ore totali raggruppate per ciascun membro del gruppo, e suddiviso per il ruolo assunto durante tutte le fasi del progetto, risulta essere così suddiviso:

\begin{table}[H]
	\begin{center}
		\begin{tabular}{|c|c|c|c|c|c|c|c|}
			\hline
			\textbf{Nome} & \multicolumn{6}{c|}{\textbf{Ore per ruolo}} & \textbf{Ore totali} \\\cline{2-7}
			& \textbf{Resp} & \textbf{Amm} & \textbf{An} & \textbf{Proj} & \textbf{Prog} & \textbf{Ver} & \\
			\hline
			\MC			&	2	&	6	&	16	&	44	&	18	&	51	&	137	\\
			\hline
			\AN			&	6	&	4	&	6	&	39	&	32	&	50	& 	137	\\
			\hline
			\DAN		&	6	&	5	&	29	&	34	&	14	&	49	&	137	\\
			\hline
			\AS			&	20	&	3 	&	12 	&	35	&	14 	& 	53	&	137	\\
			\hline
			\NS 		&	18	&	9	&	11	&	35	&	32	& 	32	&	137	\\
			\hline
			\DS			& 	5	&	2	&	5	&	38	&	33	&	54	&	137	\\
			\hline
		\end{tabular}
	\end{center}
	\caption{Ore totali, per componente}
\end{table}

\subsection{Ore totali remunerabili}
La tabella sottostante, infine, raccoglie il conteggio complessivo delle ore remunerabili suddivise per ruolo e per ciascun componente.

\begin{table}[H]
	\begin{center}
		\begin{tabular}{|c|c|c|c|c|c|c|c|}
			\hline
			\textbf{Nome} & \multicolumn{6}{c|}{\textbf{Ore per ruolo}} & \textbf{Ore totali} \\\cline{2-7}
			& \textbf{Resp} & \textbf{Amm} & \textbf{An} & \textbf{Proj} & \textbf{Prog} & \textbf{Ver} & \\
			\hline
			\MC			&	2	&	6	&	0	&	44	&	18	&	35	&	105	\\
			\hline
			\AN			&	6	&	0	&	0	&	39	&	32	&	28	& 	105	\\
			\hline
			\DAN		&	6	&	2	&	0	&	34	&	14	&	49	&	105	\\
			\hline
			\AS			&	0	&	3 	&	0 	&	35	&	14 	& 	53	&	105	\\
			\hline
			\NS 		&	0	&	6	&	0	&	35	&	32	& 	32	&	105	\\
			\hline
			\DS			& 	5	&	0	&	0	&	38	&	33	&	29	&	105	\\
			\hline
		\end{tabular}
	\end{center}
	\caption{Ore totali remunerabili, per componente}
\end{table}

\section{Quadro economico di progetto}
Si elenca, di seguito, il prospetto economico per il progetto: esso è suddiviso in sezioni riguardanti ciascuna fase dello stesso. Il piano economico così presentato si basa sul calcolo delle ore precedentemente presentato. Sarà inserita, per scopo puramente conoscitivo, anche la fase di Analisi dei Requisiti nonostante non sia rendicontabile e non rientri quindi nel calcolo complessivo. Di seguito si illustrano i valori di riferimento per il calcolo dei costi.

\begin{table}[H]
	\begin{center}
		\begin{tabular}{|c|c|c|}
			\hline
			\textbf{Ruolo}	& \textbf{Costo per ora} \\
			\hline
			\Res	&	30	\\
			\hline
			\Amm	&	20	\\
			\hline
			\Ana	&	25	\\
			\hline
			\Prog	&	22	\\
			\hline
			\Progr	&	15	\\
			\hline
			\Ver	&	15	\\
			\hline
		\end{tabular}
	\end{center}
	\caption{Costo per ora, suddiviso per ruolo}
\end{table}

\subsection{Analisi dei Requisiti}
Il prospetto economico per quanto concerne l'Analisi dei Requisiti è il seguente:


\begin{table}[H]
	\begin{center}
		\begin{tabular}{|c|c|c|}
			\hline
			\textbf{Ruolo}	& \textbf{Numero di ore} & \textbf{Costo} \\
			\hline
			\Res	&	38  &	1140	\\
			\hline
			\Amm	&	12  &	240	\\
			\hline
			\Ana	&	79  &	1975	\\
			\hline
			\Ver	&	63  &	945	\\
			\hline
			\textbf{Totale}  &	192	&	4300	\\
			\hline
		\end{tabular}
	\end{center}
	\caption{Prospetto dei costi, Analisi dei Requisiti }
\end{table}


\subsection{Progettazione Architetturale}
Il prospetto economico per quanto concerne la Progettazione Architetturale è il seguente:


\begin{table}[H]
	\begin{center}
		\begin{tabular}{|c|c|c|}
			\hline
			\textbf{Ruolo}	& \textbf{Numero di ore} & \textbf{Costo} \\
			\hline
			\Res	&	5  &	150	\\
			\hline
			\Amm	&	5  &	240	\\
			\hline
			\Prog	&	138  &	3036	\\
			\hline
			\Ver	&	52  &	780	\\
			\hline
			\textbf{Totale}  &	200 &	4206	\\
			\hline
		\end{tabular}
	\end{center}
	\caption{Prospetto dei costi, Progettazione Architetturale }
\end{table}


\subsection{Progettazione Architetturale Dettagliata}
Il prospetto economico per quanto concerne la Progettazione Architetturale Dettagliata è il seguente:


\begin{table}[H]
	\begin{center}
		\begin{tabular}{|c|c|c|}
			\hline
			\textbf{Ruolo}	& \textbf{Numero di ore} & \textbf{Costo} \\
			\hline
			\Res	&	5  &	150	\\
			\hline
			\Amm	&	6  &	120	\\
			\hline
			\Prog	&	72  &	1584	\\
			\hline
			\Ver	&	35  &	525	\\
			\hline
			\textbf{Totale}  &	118 &	2379	\\
			\hline
		\end{tabular}
	\end{center}
	\caption{Prospetto dei costi, Progettazione Architetturale Dettagliata }
\end{table}

\subsection{Codifica}
Il prospetto economico per quanto concerne la fase di codifica è il seguente:


\begin{table}[H]
	\begin{center}
		\begin{tabular}{|c|c|c|}
			\hline
			\textbf{Ruolo}	& \textbf{Numero di ore} & \textbf{Costo} \\
			\hline
			\Res	&	6  &	180	\\
			\hline
			\Amm	&	3  &	60	\\
			\hline
			\Progr	&	143  &	2145	\\
			\hline
			\Ver	&	61  &	915	\\
			\hline
			\textbf{Totale}  &	213 &	3300	\\
			\hline
		\end{tabular}
	\end{center}
	\caption{Prospetto dei costi, Codifica }
\end{table}


\subsection{Verifica}
Il prospetto economico per quanto concerne la fase di verifica e validazione è il seguente:


\begin{table}[H]
	\begin{center}
		\begin{tabular}{|c|c|c|}
			\hline
			\textbf{Ruolo}	& \textbf{Numero di ore} & \textbf{Costo} \\
			\hline
			\Res	&	6  &	180	\\
			\hline
			\Prog	&	15  &	330	\\
			\hline
			\Ver	&	78  &	1170	\\
			\hline
			\textbf{Totale}  &	99  &	1680	\\
			\hline
		\end{tabular}
	\end{center}
	\caption{Prospetto dei costi, Verifica }
\end{table}


\subsection{Totale e considerazioni conclusive}

Di seguito, la tabella mostra totale delle ore rendicontate con il costo parziale e complessivo, suddiviso per ruolo. 


\begin{table}[H]
	\begin{center}
		\begin{tabular}{|c|c|c|}
			\hline
			\textbf{Ruolo}	& \textbf{Ore rendicontabili} & \textbf{Costo} \\
			\hline
			\Res	&	22  &	660	\\
			\hline
			\Amm	&	14  &	280	\\
			\hline
			\Prog	&	225  &	4950	\\
			\hline
			\Progr	&	143  &	2145	\\
			\hline
			\Ver	&	226  &	3390	\\
			\hline
			\textbf{Totale}  &	630  &	11425	\\
			\hline
		\end{tabular}
	\end{center}
	\caption{Prospetto dei costi complessivo}
\end{table}

Il costo complessivo preventivato per la realizzazione del progetto è quindi di \textbf{€ 11425}