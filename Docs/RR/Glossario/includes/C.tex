\section{C}
\begin{itemize}
	\item \textbf{Client}: indica una componente che accede ai servizi o alle risorse messe a disposizione da un server. Esso fa parte dell'architettura logica di rete client-server. Inoltre, il termine client indica anche il software usato sul computer-client per accedere alle funzionalità offerte da un server.
	\item \textbf{Cloud computing}: letteralmente "nuvola informatica", indica un paradigma di erogazione di risorse informatiche come l'archiviazione, l'elaborazione o la trasmissione di dati, caratterizzato dalla disponibilità on demand attraverso Internet, a partire da un insieme di risorse preesistenti e configurabili.
	\item \textbf{CSS}: acronimo per Cascading Style Sheets (letteralmente fogli di stile a cascata), è un linguaggio usato per definire la formattazione di documenti HTML, XHTML e XML.
	\item \textbf{CSS3}: ultima versione dello standard CSS. \MakeUppercase{è} retrocompatibile con le precedenti versioni di CSS.
	\item \textbf{CSSLint}: strumento che aiuta a rilevare possibili errori nel codice CSS.
	\item \textbf{Customer Communication Dashboard}: E' un’interfaccia grafica che organizza e presenta le informazioni circa l'acquisto o la disdetta delle Api in modo semplice, intuitivo ed immediato, consentendo al management di agire tempestivamente nella correzione della strategia in caso di necessità.
\end{itemize}