\section{M}
\begin{itemize} 
	\item \textbf{machine learning}: apprendimento automatico, l'abilità dei computer nell'apprendere qualcosa senza essere stati esplicitamente programmati.
	\item \textbf{marketplace}: mercato online che raggruppa i prodotti di diversi venditori o siti web, consentendo la compravendita di tali beni o servizi.
	\item \textbf{MeteorJS}: framework web per JavaScript, gratuito ed open source, scritto in Node.js. Permette una prototipazione rapida e la produzione di codice multipiattaforma. Si integra con MongoDB, utilizza il protocollo DDP (Distributed Data Protocol) e un pattern di tipo publish-subscribe per la propagazione automatica delle modifiche dei dati ai clienti, senza richiedere allo sviluppatore di scrivere codice di sincronizzazione.
	\item \textbf{microservizio}: unità software specializzata che viene generalmente eseguita su un processo di sistema. \MakeUppercase{è} prevista comunicazione tra i microservizi e può avvenire attraverso la rete o sulla stessa macchina. Ogni microservizio si propone all’esterno come una black-box, infatti espone solo un API, astraendo rispetto al dettaglio di come le funzionalità siano effettivamente implementate e dallo specifico linguaggio o tecnologia utilizzati. Ciò mira a far sì che il cambiamento di ciascun microservizio non abbia impatto sugli altri microservizi comunicanti.
	\item \textbf{MongoDB}: DBMS non relazionale, orientato ai documenti. Classificato come un database di tipo NoSQL, MongoDB si allontana dalla struttura tradizionale basata su tabelle dei database relazionali in favore di documenti in stile JSON con schema dinamico (BSON), rendendo l'integrazione di dati di alcuni tipi di applicazioni più facile e veloce. 
\end{itemize}
