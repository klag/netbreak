\section{T}
\begin{itemize}
	\item \textbf{team}: gruppo di persone che lavorano insieme.
	\item \textbf{Telegram}: servizio di messaggistica istantanea basato su cloud, completamente gratuito e disponibile per diverse piattaforme. Il codice sorgente della parte lato client è open source, mentre quello lato server no, a differenza di Rocket.chat, per esempio.
	\item \textbf{TeXstudio}: editor gratuito, moderno e multi-piattaforma per Linux, sistemi Mac OS e Microsoft Windows che integra molti strumenti utili per sviluppare documenti in \LaTeX. TeXstudio è uno strumento facile da usare e da configurare.
<<<<<<< HEAD
	\item \textbf{Task}: una task è una attività da svolgere in un determinato periodo da una persona. All'interno di un progetto questa task è assegnata dal \textit{Responsabile di Progetto}.
	\item \textbf{Tomcat}: Tomcat è un software open source. La sua funzionalità di spicco è quella del web application server, ovvero un server capace di gestire e supportare le pagine JSP e le servlet nel rispetto delle specifiche 2.4 (per le servlet) e 2.0 (per le JSP).
=======
	\item \textbf{task}: attività da svolgere in un determinato periodo da un individuo. All'interno di un progetto questa attività è assegnata dal \textit{Responsabile di Progetto}.
>>>>>>> origin/master
\end{itemize}

