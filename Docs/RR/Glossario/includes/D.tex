\hypertarget{D}{}

\newglossaryentry{database NoSQL}
{
	name=database NoSQL,
	description={si differenziano dai database SQL per il fatto che i dati vengono conservati in documenti e non in tabelle. Le informazioni non sono distribuite in differenti strutture logiche, ma vengono aggregate per oggetto in documenti, la cui natura può essere di tipo Key-Value (che rappresenta la forma primitiva di database NoSQL) o Document Store, basati su semantica JSON. Ogni documento aggregato raccoglie tutti i dati associati a un’entità, in modo che qualsiasi applicazione possa trattare l’entità come oggetto e valutare in un sol colpo tutte le informazioni a essa correlate. In questo modo, si evitano anche i fardelli computazionali dovuti ai passaggi di aggregazione delle informazioni tipici del linguaggio SQL, in quanto tutti i dati necessari e corrispondenti a un medesimo oggetto sono già disponibili in un unico documento. L’assenza di tabelle permette ai database non relazionali di essere schemaless, ossia privi di un qualsiasi schema definito a priori, e questa caratteristica conferisce ai database NoSQL un altro vantaggio non trascurabile}
}

\newglossaryentry{database SQL}
{
	name=database SQL,
	description={sono basati sul modello relazionale (RDBMS) progettato per:
		\begin{enumerate}  
			\item Creare e modificare schemi di database (DDL - Data Definition Language);
			\item Inserire, modificare e gestire dati memorizzati (DML - Data Manipulation Language);
			\item Creare e gestire strumenti di controllo ed accesso ai dati (DCL - Data Control Language)
		\end{enumerate}
	}
}

\newglossaryentry{diagramma di Gantt}
{
	name=diagramma di Gantt,
	description={è usato principalmente nelle attività di project management. \MakeUppercase{è} costruito partendo da un asse orizzontale, a rappresentazione dell'arco temporale totale del progetto, suddiviso in fasi incrementali, e da un asse verticale, a rappresentazione delle mansioni o attività che costituiscono il progetto. Un diagramma di Gantt permette la rappresentazione grafica di un calendario di attività, utile al fine di pianificare, coordinare e tracciare specifiche attività in un progetto, dando una chiara illustrazione dello stato d'avanzamento del progetto rappresentato}
}

\newglossaryentry{Django}
{
	name=Django,
	description={web framework open source per lo sviluppo di applicazioni web, scritto in linguaggio Python, seguendo il design pattern MVC (Model-View-Controller)}
}

\newglossaryentry{DynamoDB}
{
	name=DynamoDB,
	description={servizio di database NoSQL completamente gestito e messo a disposizione da Amazon. Esso consente di scaricare gli oneri amministrativi di funzionamento e di scalare un database distribuito, in modo che l'utente non debba preoccuparsi di nulla.
		Con DynamoDB, è possibile creare le tabelle del database in grado di memorizzare e recuperare qualsiasi quantità di dati, servire qualsiasi livello di richiesta di traffico, e utilizzare la console di gestione AWS per monitorare le metriche di utilizzo delle risorse e di prestazioni. Infine, diffonde automaticamente i dati e il traffico per le tabelle su di un numero sufficiente di server per gestire le esigenze dell'utente, mantenendo prestazioni costanti e veloci. Tutti i dati sono memorizzati su dischi SSD e replicati automaticamente su più Availability Zones (zone di disponibilità) in una regione AWS}
}
