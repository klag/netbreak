\section{S}
\begin{itemize}
	\item \textbf{SaaS}: acronimo per Software-as-a-Service, è un modello di distribuzione del software applicativo, dove un produttore di software sviluppa, opera (direttamente o tramite terze parti) e gestisce un'applicazione web che mette a disposizione dei propri clienti via Internet. Si tratta di un servizio di cloud computing.
	\item \textbf{SASS}: acronimo per Syntactically Awesome StyleSheets, è un'estensione del linguaggio CSS che permette di utilizzare variabili, di creare funzioni e di organizzare il foglio di stile in più file. Il linguaggio SASS si basa sul concetto di preprocessore CSS, il quale serve a definire fogli di stile con una forma più semplice, completa e potente rispetto ai CSS e a generare file CSS ottimizzati, aggregando le strutture definite anche in modo complesso. Per poter usare SASS è necessario installarlo tramite Ruby.
	\item \textbf{server}: componente o sottosistema informatico di elaborazione e gestione del traffico di informazioni, che fornisce, a livello logico e fisico, un qualunque tipo di servizio ad altre componenti dette client.
	\item \textbf{SiriSDK}: sistema di comando vocale sviluppato da Apple Inc.
	\item \textbf{Skype}: software gratuito di messaggistica istantanea, chiamate e videochiamate con protocollo VoIP senza costi, semplicemente sfruttando la propria connessione internet. La versione SkypeOut permette di effettuare telefonate anche a numeri fissi.
	\item \textbf{SLA}: acronimo per Service Level Agreement, letteralmente contratto di servizio, è un impegno ufficiale che prevale tra un fornitore di servizi e un cliente, i quali concordano alcuni aspetti particolari del servizio come qualità, disponibilità, responsabilità, etc.
	\item \textbf{Slack}: strumento per la comunicazione interna e il coordinamento di attività di un'azienda o di un team di sviluppo. \MakeUppercase{è} disponibile per la maggior parte dei sistemi operativi, quali Windows, Linux, Mac OS X, come applicazione web o applicazione desktop, mentre su dispositivi mobili come app per iOS, Android e Windows Phone.
	\item \textbf{SQL}: acronimo per Structured Query Language, è un linguaggio standardizzato per database basati sul modello relazionale (RDBMS). Non è semplicemente un linguaggio di interrogazione, ma si occupa anche della creazione, gestione e amministrazione del database.
	\item \textbf{SQLFiddle}: SQL Fiddle e' un simulatore online di ambienti database. L'utente puo' scegliere l'ambiente, generare uno schema test, inserire query ed eseguirle e visualizzare i risultati. La simulazione create puo' essere condivisa con altri user attraverso un link.
	\item \textbf{Swift}: linguaggio di programmazione multiparadigma, compilato, sviluppato da Apple Inc.
<<<<<<< HEAD
	\item \textbf{SmartSheet}: e' uno strumento di gestione e collaborazione per aziende di tutte le dimensioni, che permette la condivisione dei progetti, l'importazione da fogli di calcolo Excel, Project o Google e la generazione di diagrammi Gantt a partire da tabelle simil-Excel.
	\item \textbf{Subtask}: una subtask è una attività minore ma necessaria per il completamento di una task.  
	\item \textbf{Shiny}: Shiny è un framework per applicazioni web per il linguaggio di programmazione R. Shiny trasforma le analisi in applicazioni web interattive senza avere conoscenze di HTML, CSS o Javascript.
=======
	\item \textbf{Smartsheet}: e' uno strumento di gestione e collaborazione per aziende di tutte le dimensioni, che permette la condivisione dei progetti, l'importazione da fogli di calcolo Excel, Project o Google e la generazione di diagrammi Gantt a partire da tabelle simil-Excel.
	\item \textbf{subtask}: sottoattività necessaria per il completamento dell'attività da cui ne deriva.  
>>>>>>> origin/master
\end{itemize}