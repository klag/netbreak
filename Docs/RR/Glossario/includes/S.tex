\section{S}
\begin{itemize}
	\item \textbf{SaaS}: acronimo per Software-as-a-Service, è un modello di distribuzione del software applicativo, dove un produttore di software sviluppa, opera (direttamente o tramite terze parti) e gestisce un'applicazione web che mette a disposizione dei propri clienti via Internet. Si tratta di un servizio di cloud computing.
	\item \textbf{SASS}: acronimo per Syntactically Awesome StyleSheets, è un'estensione del linguaggio CSS che permette di utilizzare variabili, di creare funzioni e di organizzare il foglio di stile in più file. Il linguaggio SASS si basa sul concetto di preprocessore CSS, il quale serve a definire fogli di stile con una forma più semplice, completa e potente rispetto ai CSS e a generare file CSS ottimizzati, aggregando le strutture definite anche in modo complesso. Per poter usare SASS è necessario installarlo tramite Ruby.
	\item \textbf{server}:
	\item \textbf{SiriSDK}:
	\item \textbf{Skype}:
	\item \textbf{SLA}: acronimo per Service Level Agreement.
	\item \textbf{Slack}:
	\item \textbf{SQL}:
	\item \textbf{SQLFiddle}:
	\item \textbf{Swift}:
\end{itemize}