\section{H}
\begin{itemize} 
	\item \textbf{HTML}: acronimo per HyperText Markup Language (HTML) (traduzione letterale: linguaggio a marcatori per ipertesti), è il linguaggio di markup solitamente usato per la formattazione e impaginazione di documenti ipertestuali disponibili nel World Wide Web sotto forma di pagine web, nato assieme al web 1.0. È un linguaggio di pubblico dominio, la cui sintassi è stabilita dal World Wide Web Consortium (W3C), e che è derivato da un altro linguaggio avente scopi più generici, l'SGML. 
	\item \textbf{HTML5}: è un linguaggio di markup per la strutturazione delle pagine web, pubblicato come W3C Recommendation da ottobre 2014. Questo linguaggio introduce notevoli migliorie all'HTML, mantenendo la retrocompatibilità.
	\item \textbf{HTTP}: acronimo per HyperText Transfer Protocol (protocollo di trasferimento di un ipertesto), è usato come principale sistema per la trasmissione di informazioni sul web ovvero in un'architettura tipica client-server. Le specifiche del protocollo sono gestite dal W3C. Un server HTTP generalmente resta in ascolto delle richieste dei client sulla porta 80 usando il protocollo TCP a livello di trasporto.
\end{itemize}
