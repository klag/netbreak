\newpage
\section{Introduzione}

\subsection{Scopo del documento}
Lo scopo del documento è la stesura di norme e regolamenti interni che il team seguirà scrupolosamente per la stesura dei documenti di progetto. Il documento funge da linea guida e da regolamento scrupoloso per ogni membro del gruppo al momento della stesura di ciascun documento o per le comunicazioni esterne. Ciò ha lo scopo di produrre documentazione omogenea, seguendo le medesime linee guida. Nel documento saranno descritti accuratamente: le modalità di stesura dei documenti e di revisione, i ruoli e le mansioni, le metodologie di lavoro, i metodi di comunicazione, i metodi di organizzazione e gli strumenti di lavoro. Ogni sottocategoria sarà comprensiva di eventuali norme e convenzioni utilizzate.

\subsection{Scopo del prodotto}
Lo scopo del prodotto è la realizzazione di un API Market per l'acquisto e la vendita di microservizi. Il sistema offrirà la possibilità di registrare nuove API per la vendita, permetterà la consultazione e ricerca ai potenziali acquirenti, gestendo i permessi tramite creazione e controllo di relative API key. Il sistema, oltre alla webapp stessa, sarà corredato di un API Gateway per la gestione delle richieste e il controllo delle chiavi, e fornirà funzionalità avanzate di statistiche per il gestore della piattaforma nonchè per gli utilizzatori.

\subsection{Riferimenti normativi}
\begin{itemize}
	\item \textbf{ISO 8601} (Rappresentazione di date e orari) \\
	\url{https://it.wikipedia.org/wiki/ISO_8601}
\end{itemize}
\subsection{Riferimenti informativi}

\subsection{Glossario}
Per semplificare la consultazione e disambiguare alcune terminologie tecniche, alcune voci indicate con la lettera \textit{G} a pedice sono descritte approfonditamente nel documento apposito \textsc{Glossario 1\_0\_0.pdf}