\newpage
\section{Introduzione}

\subsection{Scopo del documento}
Lo scopo del presente documento è la definizione di norme e regolamenti interni che il team NetBreak seguirà scrupolosamente al fine di produrre materiale uniforme ed omogeneo. Il documento descrive accuratamente gli strumenti e le convenzioni adottate per la stesura dei vari documenti di progetto, le tecniche di revisione degli stessi, la definizione dei ruoli e la loro suddivisione, gli ambienti utilizzati nella fase di sviluppo del prodotto, le modalità di comunicazione interne ed esterne al gruppo.

\subsection{Scopo del prodotto}
Lo scopo del prodotto è la realizzazione di un API Market\ped{\textit{G}} per l'acquisto e la vendita di microservizi. Il sistema offrirà la possibilità di registrare nuove API\ped{\textit{G}} per la vendita, permetterà la consultazione e la ricerca di API\ped{\textit{G}} ai potenziali acquirenti, gestendo i permessi di accesso ed utilizzo tramite creazione e controllo di relative API key\ped{\textit{G}}. Il sistema, oltre alla web app stessa, sarà corredato di un API Gateway\ped{\textit{G}} per la gestione delle richieste e il controllo delle chiavi, e fornirà funzionalità avanzate di statistiche per il gestore della piattaforma e per i fornitori dei microservizi.

\subsection{Riferimenti normativi}
\begin{itemize}
	\item \textbf{ISO 8601} (Rappresentazione di date e orari)\\
	\url{https://it.wikipedia.org/wiki/ISO\_8601}
\end{itemize}

\subsection{Riferimenti informativi}
\begin{itemize}
	\item \textsc{PianoDiProgetto1\_0\_0.pdf}
	\item \textsc{PianoDiQualifica1\_0\_0.pdf}
	\item \textbf{Amministrazione di Progetto}\\
	\url{http://www.math.unipd.it/~tullio/IS-1/2016/Dispense/L05.pdf}
\end{itemize}

\subsection{Glossario}
Per semplificare la consultazione e disambiguare alcune terminologie tecniche, le voci indicate con la lettera \textit{G} a pedice sono descritte approfonditamente nel documento \textsc{Glossario1\_0\_0.pdf}.