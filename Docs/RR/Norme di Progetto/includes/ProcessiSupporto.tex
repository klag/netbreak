\newpage

\section{Processi di supporto}
%%%%%%%%%%%%%%%%%%%%%%%%%%%%%%%%%%%%%%%%%%%%%%%%%%%%%%
\subsection{Documentazione}
	In questa sezione del documento verrà spiegata la struttura, classificazione e metodo di redazione dei documenti del gruppo \textit{NetBreak}.

	\subsubsection{Nomenclatura e versione del documento}
	Ogni documento deve rispettare una denominazione comune il più possibile chiara affinché venga individuato facilmente il nome e la versione del documento. 
	Il gruppo ha scelto di utilizzare la seguente nomenclatura 
		\begin{center}
			\textit{NomeDelDocumento X\_Y\_Z.pdf}
		\end{center}
	
	All’interno del nome \textit{X\_Y\_Z} indica la versione del documento in questo modo:
	
	\begin{itemize}
	  \item \textit{X}: indica il numero di pubblicazioni del documento. Questo indice è incrementato esclusivamente dal \textit{\RdP} in seguito all’approvazione finale da parte sua. L’incremento di tale indice, azzera automaticamente Y e Z;
	  \item \textit{Y}: indica il numero di verifiche e viene incrementato esclusivamente dai \textit{\Vers}. L’incremento di tale indice, azzera automaticamente Z;
	  \item \textit{Z}: indica il numero di aggiornamenti minori effettuati prima o in seguito a una verifica o approvazione. Viene incrementato progressivamente e ritorna a zero solo in seguito a una modifica di X e Y.
	\end{itemize}
	
	Ogni modifica della versione del documento deve riflettersi nel changelog, nel nome del documento e nel frontespizio nella voce “Ultima modifica e versione”.

	%%%%%%%%%%%%%%%%%%%%%%%%%%%%%%%%%%%%%%%%%%%
	\subsubsection{Classificazione del documento}
	Tutti i documenti redatti dal gruppo devono avere la classificazione, che può essere \textbf{Interna} o \textbf{Esterna}.
	Un documento viene classificato come \textbf{Interno} se la sua consultazione è limitata al gruppo. Al contrario, se il documento è destinato a una consultazione esterna al gruppo, è classificato come \textbf{Esterno}.
	
	%%%%%%%%%%%%%%%%%%%%%%%%%%%%%%%%%%%%%%%%%%%
	\subsubsection{Ciclo di vita del documento}
	Un documento può trovarsi in tre stati. 
	\begin{itemize}
		\item{Documento in Bozza}: il/i redattore/i è in fase di stesura del documento;
		\item{Documento in attesa di verifica}: il/i redattore/i ha concluso la stesura, il documento passa in stato di verifica in attesa del \textit{\Ver};
		\item{Documento approvato}: il \textit{\Ver} ha concluso il suo lavoro e passa il documento dal \textit{\RdP} che ha il compito di approvarlo definitivamente.
	\end{itemize}

	%%%%%%%%%%%%%%%%%%%%%%%%%%%%%%%%%%%%%%%%%
	\subsubsection{Strumenti di sviluppo del documento}
	Per la stesura dei documenti è stato scelto di utilizzare \textit{\LaTeX\ped{G}}, un linguaggio semplice e modulare. \textit{\LaTeX\ped{G}} è in grado di evitare possibili conflitti provenienti dall’utilizzo di software e piattaforme differenti.
	
	%%%%%%%%%%%%%%%%%%%%%%%%%%%%%%%%%%%%%%%%%
	\subsubsection{Struttura del documento}
	Ogni documento redatto dal gruppo avrà una struttura chiara e di facile comprensione per chiunque. Di seguito viene riportata la struttura che sarà utilizzata nei documenti.
		\paragraph{Frontespizio}
			\begin{itemize}
				\item{Nome del gruppo;}
				\item{Nome del progetto;}
				\item{Logo del gruppo;}
				\item{Nome del documento;}
				\item{Nome del file;}
				\item{Data di creazione;}
				\item{Ultima modifica e versione;}
				\item{Stato;}
				\item{Nome e cognome del/i redattore/i;}
				\item{Nome e cognome del verificatori;}
				\item{Classificazione del documento;}
				\item{Distribuzione;}
				\item{Destinatari del documento;}	
			\end{itemize}
		
		\paragraph{Changelog}
		La seconda sezione del documento è una lista delle modifiche effettuate al documento, dalla sua creazione alla sua approvazione finale. Le modifiche devono essere inserite con questo schema:
				\begin{itemize}
				\item{Descrizione}: questo campo indica cosa è stato cambiato nel documento;
				\item{Autore e ruolo}: questo campo indica l’autore e ruolo della persona che ha effettuato la modifica;
				\item{Data e versione}: questo campo indica la data di modifica e la nuova versione del documento in questione.
			\end{itemize}	
		\paragraph{Indice}
			In ogni documento deve essere presente l’indice di tutte le sezioni del documento; ogni immagine o tabella deve essere indicata con il relativo indice.
		\paragraph{Intestazione e piè di pagina}
			Tutte le pagine del documento, eccetto la prima hanno una intestazione e piè di pagina. L’intestazione contiene nell’angolo sinistro il nome del progetto e del documento in questione. Nell’angolo destro è presente il nome del capitolo contente quella sezione, ad esclusione del changelog.
			Il piè di pagina contiene nell’angolo sinistro il nome del gruppo e l’indirizzo email ufficiale del gruppo. Nell’angolo destro è presente una numerazione in numeri romani del changelog e dell’indice, mentre per le pagine rimanenti è utilizzata una numerazione con numeri arabi, che indica la pagina attuale e il numero totale di pagine del documento.
		\paragraph{Norme tipografiche}
			Questa sezione del documento contiene i criteri riguardanti l’ortografica e la tipografia utilizzate nel corso dello sviluppo del progetto.
			\subparagraph{Stili di testo e punteggiatura}
				\begin{itemize}
					\item \textbf{Grassetto}:il grassetto deve essere utilizzato per parole importanti all’interno di frasi o elenchi puntati;
					\item \textit{Corsivo}: il corsivo deve essere utilizzato per:
						\begin{itemize}
							\item Citazioni;
							\item Parola del glossario (unitamente ad una G maiuscola in pedice);
							\item Riferimenti ai ruoli;
							\item Riferimento all’intero gruppo.
						\end{itemize}
					\item MAIUSCOLO: viene utilizzato unicamente per scrivere acronimi e macro \LaTeX presenti nei documenti;
					\item \textsc{maiuscoletto}: viene utilizzato per i riferimenti ad altri documenti;
					\item \LaTeX: viene usanto il comando \textbackslash{LaTeX} per ogni occorenza del termine \LaTeX.
					\item Punteggiatura: Ogni simbolo di punteggiatura è seguito da uno spazio, ad eccezione del punto, del punto interrogativo, del punto esclamativo è seguito da uno spazio e lettera maiuscola. Ogni voce di un elenco puntato deve terminare con punto e virgola, ad eccezione dell’ultima che termina con un punto. 
				\end{itemize}
			\subparagraph{Formato Data}
				All’interno di ogni documento, tutte le date seguiranno lo standard ISO\ped{G} 8601:2004\ped{G}:
					\begin{center}
						YYYY-MM-DD
					\end{center}
				Dove:
					\begin{itemize}
						\item YYYY: indica l’anno;
						\item MM: indica il mese;
						\item DD: indica il giorno;
					\end{itemize}
		
		\subsubsection{Lista documenti da consegnare}
			\paragraph{\SdF}
				\begin{itemize}
					\item Classificazione: Interno;
					\item Destinazione: gruppo e committenti;
					\item Contenuto: questo documento contiene lo studio effettuato dal gruppo su tutti i capitolati e le motivazioni che hanno portato all’accettazione o meno del progetto.
				\end{itemize}
			\paragraph{\NdP}
			\begin{itemize}
				\item Classificazione: Interno;
				\item Destinazione: gruppo e committenti;
				\item Contenuto: l’intento del documento è raccogliere tutte le convenzioni, strumenti e regole che il gruppo adotterà durante la realizzazione del progetto. 
			\end{itemize}	
			\paragraph{\AdR}
			\begin{itemize}
				\item Classificazione: Esterno;
				\item Destinazione: gruppo, committenti e proponente;
				\item Contenuto: questo documento si prefigge lo scopo di dare una visione generale dei requisiti e di tutti i posibili casi d'uso dell'\progetto.
			\end{itemize}
			\paragraph{\PdP}
				\begin{itemize}
					\item Classificazione: Esterno;
					\item Destinazione: gruppo, committenti e proponente;
					\item Contenuto: questo documento descrive come il gruppo ha impiegato tempo e risorse umane.
				\end{itemize}
			\paragraph{\PdQ}
				\begin{itemize}
					\item Classificazione: Esterno;
					\item Destinazione: gruppo, committenti e proponente;
					\item Contenuto: questo documento contiene come il gruppo intende raggiungere gli obbiettivi di qualità prefissati all’inizio del progetto.
				\end{itemize}
			\paragraph{\G}
				\begin{itemize}
					\item Classificazione: Esterno;
					\item Destinazione: gruppo, committenti e proponente;
					\item Contenuto: questo documento intende fornire una definizione di tutti i termini tecnici e acronimi, rendendo la lettura facile a tutti i destinatari.
				\end{itemize}
	%%%%%%%%%%%%%%%%%%%%%
	\subsection{Verifica}
		\subsubsection{Analisi}
			\paragraph{Analisi statica}
				L'analisi statica è applicata ai documenti di testo, consiste nel trovare errori nella sintassi e ortografica. Una prima verifica viene effettuata dai \textit{\Vers} e successivamente dal \textit{\RdP}. Le due tecniche scelte sono la \textbf{\textit{Formal Walkthrough\ped{G}}} e la \textbf{\textit{Fagan Inspection\ped{G}}}.
				\begin{itemize}
					\item \textbf{Formal Walkthrough}: questa tecnica è focalizzata nel trovare errori di sintassi e ortografici, senza concentrarsi sul trovare errori in particolare. Questa tecnica sarà utilizzata dai \textit{\Vers}, i quali trascriveranno gli errori più frequenti in una apposita lista, necessaria per la tecnica di Fagan Inspection;
					\item \textbf{Fagan Inspection}: Questa tecnica si basa su una lettura attenta dei documenti, basandosi sulla lista degli errori stilata durante la \textbf{Formal Walkthrough}. Questo processo acquisirà rilevanza in concomitanza con l'incremento della lista di possibili errori stilata dai \textit{\Vers}.
				\end{itemize}
				
			\paragraph{Analisi dinamica}
				L'analisi dinamica è applicata esclusivamente sul software prodotto, in quanto consiste nell’effettuare test per verificare il corretto funzionamento dell’applicativo.
	
		\subsubsection{Strumenti}
				\begin{itemize}
					\item W3C markup validator service\ped{G}: validatore online di codice HTML\ped{G}, utile per trovare eventuali errori nel codice. L'indirizzo web di riferimento è: \url https://validator.w3.org/;
					\item CSSLint\ped{G}: validatore online di codice CSS\ped{G}, utile per trovare eventuali errori nel codice. L'indirizzo web di riferimento è: \url http://csslint.net/;
					\item SQLFiddle: validatore online di codice MySQL\ped{G}.Utile per verificare la consistenza del codice del database\ped{G}. L'indirizzo web di riferimento è: \url http://sqlfiddle.com/;
					\item PHP Code Checker\ped{G}: strumento online che verifica la presenza di eventuali errori di sintassi del codice PHP\ped{G}. L'indirizzo web di riferimento è: \url http://phpcodechecker.com/.
				\end{itemize}