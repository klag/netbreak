\newpage

\section{Processi organizzativi}

\subsection{Gestione Organizzativa}
	\subsubsection{Comunicazioni}
	Al fine di ottenere comunicazioni chiare e coincise il \textit{\RdP} deve gestirle in modo strutturato, utilizzando la forma che meglio si adatta alla situazione.
	Le comunicazioni possono essere interne o esterne al \textit{team\ped{G}}.
		\paragraph{Comunicazioni interne}
		Le comunicazioni interni sono uno strumento ad uso esclusivo dei membri del gruppo e sono in stile informale in forma scritta o orale. 
		Se la comunicazione avvengono in forma orale tra alcuni membri del gruppo, è loro dovere riportarla  in forma scritta se è una informazione di rilievo riguardo lo sviluppo del software, utilizzando l’applicazione di istant messaging\ped{G} Telegram\ped{G}, sulla quale è presente la chat del gruppo.

		\paragraph{Comunicazioni esterne}
		Per tutte le comunicazioni esterne il gruppo si è dotato di una email ufficiale, necessaria per gestire le comunicazioni con i committenti e proponenti. La gestione di tale casella di posta elettronica è affidata al \textit{\RdP}, il quale ha il compito di tenere informati i membri riguardo le comunicazioni importanti. La casella di posta elettronica è:
		
			\begin{center}
				\url{netbreak.swe@gmail.com} 
			\end{center}
			
	\subsubsection{Riunioni}
		Le riunioni sono un elemento essenziale per il corretto svolgimento del progetto. Hanno il compito di confrontare i membri del gruppo tra di loro e con il proponente e committenti. Una riunione condotta correttamente permette di risparmiare tempo e gestire al meglio le risorse umane.
		\paragraph{Riunioni interne}
		Il \textit{\RdP} ha facoltà di decidere e convocare la riunione del gruppo. Tale convocazione deve essere inviata tramite posta elettronica a tutti i membri del gruppo con almeno due giorni di anticipo e deve contenere l’ordine del giorno. Tutti i membri del gruppo sono obbligati a confermare la loro presenza o meno entro ventiquattro ore dalla ricezione della comunicazione. Se impossibilitati a partecipare, devono motivare tale impedimento al \textit{\RdP}. 
		Il \textit{\RdP} ha il compito di dirigere la riunione, tenendo fede all’ordine del giorno precedentemente inviato, ed evitare perdite di tempo. 
		Tutti i partecipanti devono presentarsi alla riunione preparati, aver rispettato le scadenze decise alla riunione precendente, visibili su sulla App multipiattaforma Asana\ped{G} e caricato il materiale assegnato sulla piattaforma di Web Storage\ped{G} indicata.
		All’inizio di ogni riunione il \textit{\RdP} avrà il compito di nominare un segretario. Il segretario avrà il compito di tenere la minuta dell’incontro, stilare un verbale informale a disposizione esclusivamente del gruppo e creare le task con le scadenze concordate su Asana\ped{G}.  
		
		\paragraph{Riunioni esterne}
		Queste riunioni sono fondamentali per il corretto svolgimento del progetto. Prima di ogni riunione il \textit{\RdP} nominerà un segretario, incaricato di prendere nota dei temi discussi e redigere un verbale, evidenziando gli aspetti più rilevanti e/o modifiche da apportare. Tale documento è ad uso esclusivo del gruppo.
		A questi incontri parteciperanno tutti i membri del gruppo (salvo impedimenti di rilievo) e sarà compito del \textit{\RdP} coordinare e gestire l’interazione tra il gruppo e l’interlocutore. Durante la riunione i membri del gruppo potranno effettuare domande e/o chiedere chiarimenti. 





%%%%%%%%%%%%%%%%%%%%%%%%%%%%%%%%%%%%%%%%%%%%%%%%%%%%%%%%%%%%%%%%%%%%%%%%%%%%%%%%%%%%%%%%%%%%%%%%%%%%%%%%%%%%
\subsection{Processo di pianificazione}

\subsubsection{Descrizione}
Durante l'intero sviluppo del progetto didattico ogni componente del gruppo
dovrà obbligatoriamente cimentarsi in tutti i ruoli elencati di seguito. \\
Inoltre non potrà mai accadere che un membro del gruppo risulti redattore e verificatore di un medesimo documento.
In questo modo si tende ad evitare il conflitto di interessi che potrebbe sorgere se la responsabilità della stesura e della verifica di un documento fosse affidata ad un'unica persona.
Un membro può inoltre ricoprire più ruoli contemporaneamente.

\paragraph{\RdP}
Il \textit{\RdP} è colui che detiene la responsabilità del
lavoro svolto dall'intero \textit{team\ped{G}}. Rappresenta inoltre colui che mantiene i
contatti diretti presso il fornitore e il cliente, ovvero gli enti esterni. Più
in dettaglio, ha responsabilità su:
\begin{itemize}
  \item Pianificazione, coordinamento e controllo generale delle attività;
  \item Gestione delle risorse;
  \item Analisi e gestione dei rischi;
  \item Gestione e approvazione della documentazione;
  \item Contatti con gli enti esterni.
\end{itemize}
Il \textit{\RdP} redige l'\textit{organigramma\ped{G}}, si assicura che
tutte le attività vengano svolte seguendo rigorosamente le \textit{\NdP}, si
assicura che vengano rispettati i ruoli assegnati nel \textit{\PdP} e che non si
presentino conflitti di interesse tra redattori e verificatori. Ha inoltre
l'incarico di creare, assegnare ad ogni membro e gestire i singoli task. Redige
il \textit{\PdP} e collabora alla stesura del \textit{\PdQ}. Il \textit{\RdP} è
l'unica persona in grado di approvare in modo definitivo un documento.

\paragraph{Amministratore}
L'\textit{\Amm} è il responsabile di tutto ciò che riguarda l'ambiente di
lavoro. Più in dettaglio, egli si occupa di:
\begin{itemize}
  \item Controllo dell'ambiente di lavoro;
  \item Gestione del versionamento della documentazione tramite l'uso di
  database;
  \item Controllo delle versioni e delle configurazioni del prodotto;
  \item Risoluzione dei problemi legati alla gestione dei processi.
\end{itemize}
L'\textit{\Amm} redige le \textit{\NdP} e collabora alla stesura del
\textit{\PdP}.

\paragraph{Progettista}
Il \textit{\Prog} è il responsabile di tutto ciò che riguarda la progettazione.
Più in dettaglio, egli si occupa di:
\begin{itemize}
  \item Produrre una soluzione attuabile, robusta e semplice entro i limiti di
  tempo stabiliti;
  \item Effettuare scelte progettuali volte a garantire la manutenibilità e la
  modularità del prodotto software.
\end{itemize}
Il \textit{\Prog} redige la \textit{\ST}, la \textit{\DDP} e la parte
programmatica del \textit{\PdQ}.

\paragraph{Analista}
L'\textit{\Ana} si occupa di tutto ciò che riguarda l'analisi del problema da
affrontare. Le mansioni principali sono quelle di:
\begin{itemize}
  \item Studiare a fondo e capire le problematiche del prodotto da realizzare;
  \item Produrre una specifica di progetto compresibile per il
  \textit{proponente}, per il \textit{committente} e per il
  \textit{Progettista}.
\end{itemize}
L'\textit{\Ana} redige lo \textit{\SdF}, l'\textit{\AdR} e parte del
\textit{\PdQ}.

\paragraph{Verificatore}
Il \textit{\Ver} è responsabile di tutto ciò che riguarda l'attività di verifica.
Effettua la verifica dei documenti utilizzando gli strumenti e i metodi proposti nel
\textit{\PdQ} e seguendo rigorosamente quanto descritto nelle \textit{\NdP}.
Egli ha il compito di garantire la conformità rispetto le \textit{\NdP} dei documenti da lui verificati.
Si occupa di redigere la sezione del \textit{\PdQ} che illustra l'esito delle
verifiche effettuate sui documenti.

\paragraph{Programmatore}
Il \textit{\Progr} si occuperà di implementare le soluzioni del \textit{\Prog}, è quindi
responsabile dell'attività di codifica. In dettaglio, i suoi compiti sono:
\begin{itemize}
  \item implementare le soluzioni descritte dal \textit{\Prog} in maniera
  rigorosa;
  \item Scrivere il codice rispettando le convenzioni prese nel presente
  documento;
  \item Implementare i test per il codice scritto da utilizzare per l'attività
  di verifica.
\end{itemize}
Il \textit{\Progr} redige il \textit{\MU} e produce la documentazione del codice.

\subsection{Strumenti}

\subsubsection{Google Drive}
\textit{Google Drive\ped{G}} è un servizio di storage online utilizzato dal
\textit{team\ped{G}} per condividere file e documenti che non necessitano di tracciamento.
\textit{Google Drive\ped{G}} consente anche il lavoro concorrente su uno stesso
file.
\subsubsection{GitHub}
Gli strumenti software per il controllo versione sono ritenuti molto spesso necessari per la maggior parte dei progetti di sviluppo software.
Per questo motivo, il \textit{team\ped{G}} ha deciso di creare un \textit{repository\ped{G}}, ospitato all'interno di un server, per gestire la continua evoluzione dei documenti digitali come il codice sorgente del software, la documentazione testuale e altre informazioni importanti.
Il software di versionamento scelto è \textit{Git\ped{G}} perchè presenta molti più aspetti positivi rispetto ai \textit{repository\ped{G}} centralizzati. \textit{Git\ped{G}} permette di lavorare anche in assenza di connettività con il server centrale. I membri del \textit{team\ped{G}} possono lavorare sulla propria copia locale del \textit{repository\ped{G}} e rendere pubbliche le modifiche caricandole nel \textit{repository\ped{G}} centrale. Questo approccio risulta molto semplice e veloce per poter collaborare con i membri del \textit{team\ped{G}} simultaneamente.
Inoltre online è possibile reperire moltissima documentazione a riguardo per un rapido apprendimento. Il servizio scelto per il mantenimento dei dai è \textit{GitHub\ped{G}}. \\La struttura base del \textit{repository\ped{G}}:
\begin{itemize}
  \item
	RR;
	\begin{itemize}
		\item
			Interni;
		\item
			Esterni.
	\end{itemize}
  \item
    RQ;
  \item
    RP;
  \item
  	RA;
  \item
  	Template.
\end{itemize}

\subsubsection{Gestione delle attività del progetto}
Per gestire nella maniera più opportuna la divisione del lavoro, si è scelto di
utilizzare il sistema di pianificazione delle attività \textit{Zoho\ped{G}}.

\subsubsection{Task management}
Vengono creati dal \textit{\RdP} e sono assegnati a due singoli membri del gruppo,
il primo in qualità di redattore e il secondo in qualità di \textit{\Ver}.
Per una gestione più chiara di questa divisione delle attività, sono stati creati su \textit{Zoho\ped{G}}
due insiemi di \textit{task\ped{G}} per ogni documento, definiti:
\begin{itemize}
  \item \textit{Nome del documento};
  \item \textit{Nome del documento - da verificare}.
\end{itemize}
Una volta che il proprietario ritiene il suo \textit{task\ped{G}} completato, deve spostarlo
nella relativa sezione \textit{da verificare} in modo tale che il verificatore
interessato possa controllarlo e segnarlo come completato, se risulta idoneo.

\paragraph{Creazione di un elenco di task}
Un elenco di \textit{task\ped{G}} rappresenta l'insieme dei \textit{task\ped{G}} necessari per la realizzazione di un intero documento.
Per la creazione di un nuovo insieme di \textit{task\ped{G}} bisogna seguire le seguenti istruzioni:
\begin{enumerate}
   \item Dalla HomePage di \textit{Zoho\ped{G}} selezionare il progetto interessato (\progetto);
  \item Selezionare la voce \textbf{Compiti e Pietre miliari} dal menù laterale;
   \item Selezionare la voce \textbf{Nuovo elenco di compiti} e compilare l'elenco di \textit{task\ped{G}} nel
  seguente modo:
  \begin{itemize}
    \item \textbf{Elenco dei compiti:} assegnare il nome del documento che si
    vuole rappresentare;
    \item \textbf{Pietra miliare collegata:} selezionare a quale pietra miliare si
    vuole collegare la realizzazione di questo insieme di \textit{task\ped{G}}. Corrispondono alla
    versione finale del documento.
  \end{itemize}
\end{enumerate}
\paragraph{Creazione di un task}
Per la creazione di un nuovo singolo \textit{task\ped{G}} bisogna seguire le seguenti
istruzioni:
\begin{enumerate}
  \item Dalla HomePage di \textit{Zoho\ped{G}} selezionare il progetto interessato (\progetto);
  \item Selezionare la voce \textbf{Compiti e pietre miliari} dal menù laterale;
  \item Selezionare la voce \textbf{Nuovo compito} e compilare il \textit{task\ped{G}} nel
  seguente modo:
    \begin{itemize}
      \item \textbf{Nome task:} assegnare un nome identificativo al \textit{task\ped{G}} seguito dalla \textit{milestone\ped{G}} corrispondente;
      \item \textbf{Aggiungi descrizione:} inserire una descrizione breve ma
      concisa del \textit{task\ped{G}}, deve avere la seguente forma:
        \begin{itemize}
          \item Redattore: nome del redattore;
          \item Verificatore: nome del verificatore.
        \end{itemize}
      \item \textbf{Elenco dei compiti:} selezionare uno degli elenchi di
      \textit{task\ped{G}}, creati precedentemente, che hanno come fine ultimo la realizzazione di un
      documento;
      \item \textbf{Chi è il responsabile?:} inserire l'intestatario del \textit{task\ped{G}}
      come primo utente, nel secondo invece inserire il verificatore assegnato;
      \item \textbf{Data di conclusione:} Inserire la data ultima per il
      completamento del \textit{task\ped{G}};
     \item \textbf{Priorità:} inserire, opzionalmente, una priorità al \textit{task\ped{G}}.
    \end{itemize}
\end{enumerate}


\paragraph{Modifica di un task}
Per modificare un \textit{task\ped{G}} seguire le seguenti istruzioni:
\begin{itemize}
  \item Selezionare il \textit{task\ped{G}} da modificare;
  \item Dalla pagina proposta selezionare il campo che si vuole modificare;
  \item Completata la modifica premere il pulsante \textbf{Invio}.
\end{itemize}

\paragraph{Completamento di un task}
Dopo che il verificatore ha appurato che il \textit{task\ped{G}} soddisfa i requisiti, e quindi è stato redatto secondo le
\textit{\NdP}, può procedere con il suo completamento seguendo le seguenti istruzioni:
\begin{itemize}
  \item Selezionare il \textit{task\ped{G}} da segnare come completato;
  \item Nella parte inferiore della pagina proposta selezionare la voce \textbf{Contrassegna come completato}.
\end{itemize}
Dopo questa operazione, il \textit{task\ped{G}} viene spostato automaticamente nella lista dei \textit{task\ped{G}} completati. Da qui, in
casi eccezionali, può essere rispostato nei \textit{task\ped{G}} \textit{da verificare}. Questi casi eccezionali possono essere:
\begin{itemize}
  \item Il \textit{\RdP} non ha approvato il documento e quindi deve essere
  rivisto;
  \item Il \textit{\Ver}, dopo un secondo controllo sui \textit{task\ped{G}} a lui assegnati,
  può accorgersi di errori e/o incompletezze. Deve quindi essere rivisto.
\end{itemize}

\subsection{Lista di controllo}
Durante l'applicazione della tecnica del \textit{Walkthrough\ped{G}} ai documenti sono stati riportati
più frequentemente i seguenti errori:
\begin{itemize}
  \item\textbf{Norme stilistiche}:
  \begin{itemize}
    \item La prima parola di una voce dell'elenco puntato non inizia con una lettera maiuscola;
    \item La voce dell'elenco puntato termina con un punto anziché con un punto e virgola o viceversa;
    \item I due punti in grassetto;
    \item Errori di battitura.
    
  \end{itemize}

  \item\textbf{Italiano}:
  \begin{itemize}
    \item Maiuscole usate impropriamente.
    
  \end{itemize}

  \item\textbf{\LaTeX}:
  \begin{itemize}
  \item Mancato utilizzo dei comandi \LaTeX{} personalizzati;
  \item Scambiato il comando \verb|\textit{...}| con \verb|\textbf{...}|.
  \end{itemize}
  
  \item\textbf{Casi d'Uso}:
  \begin{itemize}
\item Mancato rispetto del template stabilito per i punti trattati nei casi d'uso.
  \end{itemize}

	\item\textbf{Glossario}:
	\begin{itemize}
		\item Sono stati evidenziati dei termini che non andavano nel \textit{\G{}};
		\item Non sono stati evidenziati dei termini che sono presenti nel \textit{\G{}}.
	\end{itemize}

\end{itemize}
