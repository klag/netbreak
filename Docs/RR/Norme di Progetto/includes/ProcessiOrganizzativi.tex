\newpage

\section{Processi organizzativi}

\subsection{Gestione Organizzativa}
	\subsubsection{Comunicazioni}
	Al fine di ottenere comunicazioni chiare e coincise il \textit{\RdP} deve gestirle in modo strutturato, utilizzando la forma che meglio si adatta alla situazione.
	Le comunicazioni possono essere interne o esterne al \textit{team\ped{G}}.
		\paragraph{Comunicazioni interne}
		Le comunicazioni interni sono uno strumento ad uso esclusivo dei membri del gruppo e sono in stile informale in forma scritta o orale. 
		Se la comunicazione avvengono in forma orale tra alcuni membri del gruppo, è loro dovere riportarla  in forma scritta se è una informazione di rilievo riguardo lo sviluppo del software, utilizzando l’applicazione di istant messaging\ped{G} Telegram\ped{G}, sulla quale è presente la chat del gruppo.

		\paragraph{Comunicazioni esterne}
		Per tutte le comunicazioni esterne il gruppo si è dotato di una email ufficiale, necessaria per gestire le comunicazioni con i committenti e proponenti. La gestione di tale casella di posta elettronica è affidata al \textit{\RdP}, il quale ha il compito di tenere informati i membri riguardo le comunicazioni importanti. La casella di posta elettronica è:
		
			\begin{center}
				\url{netbreak.swe@gmail.com} 
			\end{center}
			
	\subsubsection{Riunioni}
		Le riunioni sono un elemento essenziale per il corretto svolgimento del progetto. Hanno il compito di confrontare i membri del gruppo tra di loro e con il proponente e committenti. Una riunione condotta correttamente permette di risparmiare tempo e gestire al meglio le risorse umane.
		\paragraph{Riunioni interne}
		Il \textit{\RdP} ha facoltà di decidere e convocare la riunione del gruppo. Tale convocazione deve essere inviata tramite posta elettronica a tutti i membri del gruppo con almeno due giorni di anticipo e deve contenere l’ordine del giorno. Tutti i membri del gruppo sono obbligati a confermare la loro presenza o meno entro ventiquattro ore dalla ricezione della comunicazione. Se impossibilitati a partecipare, devono motivare tale impedimento al \textit{\RdP}. 
		Il \textit{\RdP} ha il compito di dirigere la riunione, tenendo fede all’ordine del giorno precedentemente inviato, ed evitare perdite di tempo. 
		Tutti i partecipanti devono presentarsi alla riunione preparati, aver rispettato le scadenze decise alla riunione precendente, visibili su sulla App multipiattaforma Asana\ped{G} e caricato il materiale assegnato sulla piattaforma di Web Storage\ped{G} indicata.
		All’inizio di ogni riunione il \textit{\RdP} avrà il compito di nominare un segretario. Il segretario avrà il compito di tenere la minuta dell’incontro, stilare un verbale informale a disposizione esclusivamente del gruppo e creare le task con le scadenze concordate su Asana\ped{G}.  
		
		\paragraph{Riunioni esterne}
		Queste riunioni sono fondamentali per il corretto svolgimento del progetto. Prima di ogni riunione il \textit{\RdP} nominerà un segretario, incaricato di prendere nota dei temi discussi e redigere un verbale, evidenziando gli aspetti più rilevanti e/o modifiche da apportare. Tale documento è ad uso esclusivo del gruppo.
		A questi incontri parteciperanno tutti i membri del gruppo (salvo impedimenti di rilievo) e sarà compito del \textit{\RdP} coordinare e gestire l’interazione tra il gruppo e l’interlocutore. Durante la riunione i membri del gruppo potranno effettuare domande e/o chiedere chiarimenti. 





\subsection{Ruoli e mansioni}

Nel corso del progetto, ogni membro del team svolgerà diverse mansioni. I ruoli sono standard, stabiliti con criterio e con mansioni mutuamente esclusive. Lo scopo di questa sezione è descrivere tutti i possibili ruoli che un membro del team può assumere durante lo svolgimento del progetto. Essi verranno organizzati in modo tale da:
\begin{itemize}
	\item Evitare conflitti di interesse quali concomitanza dei ruoli di stesura e verifica
	\item Ruotare i ruoli per permettere a ciascun membro di assumere qualsiasi posizione, durante il corso del progetto
	\item Permettere ad una persona di assumere più mansioni contemporaneamente
\end{itemize}

\paragraph{\RdP}
Il Responsabile di Progetto organizza il lavoro interno del team, occupandosi della pianificazione delle attività. Egli mantiene i contatti con tutti gli enti esterni, gestisce le risorse, i rischi e la documentazione. Di quest'ultima, è inoltre l'unico incaricato di apporre l'approvazione definitiva. Le ulteriori mansioni organizzative del Responsabile di Progetto sono quelle di: garantire il rispetto dei ruoli, ed assicurarsi che le attività assegnate vengano svolte scrupolosamente nei tempi stabiliti. I documenti di diretta responsabilità del Responsabile di Progetto sono il Piano di Progetto ed il Piano di Qualifica.

\paragraph{Amministratore}
L'Amministratore svolge un attività di controllo sull'ambiente di lavoro come mansione principale. Altre dirette responsabilità dell'Amministratore sono relative al versionamento di prodotto e documentazione. I documenti di sua diretta responsabilità, inoltre, sono le Norme di Progetto e il Piano di Progetto.

\paragraph{Progettista}
Il ruolo di Progettista è una mansione che ha come compiti primari quello di realizzare una produzione astratta del progetto , effettuando le scelte necessarie a mantenere il prodotto modulare, di facile manutenzione e ampliamento.

\paragraph{Programmatore}
Il programmatore svolge le attività di produzione vera e propria della soluzione finale. Seguendo in maniera attenta la progettazione pregressa, effettua la codifica in rispetto delle convenzioni descritte nelle Norme di Progetto. Si occupa inoltre di predisporre la fase di test per la successiva fase di verifica, e di documentare quanto viene prodotto.

\paragraph{Verificatore}
Il Verificatore svolge per l'appunto l'attività di verifica di tutta la documentazione prodotta. Si attiene scrupolosamente alle Norme di Progetto per assicurarsi che i testi prodotti siano conformi agli standard prefissati.

\paragraph{Analista}
La figura di Analista ha il ruolo di esaminare caratteristiche e problematiche del prodotto. Si occupa dunque della fase di analisi per ogni aspetto del progetto. Si occupa direttamente della Specifica di progetto e della parte di Analisi dei Requisiti.

\subsection{Strumenti}

\subsubsection{GitHub}
Per il controllo di versione e la gestione semplificata dei file di progetto e della documentazione, si è scelto l'utilizzo di GitHub. La scelta è ricaduta su questa soluzione poichè altre piattaforme non consentono la collaborazione di 6 membri in modo gratuito. Il team lavorerà dunque su un repository pubblico. I membri del team potranno, tramite questa soluzione, collaborare simultaneamente sincronizzando le versioni in tempo reale, o in un secondo momento qualora mancasse la connessione. GitHub offre inoltre una sezione Wiki per la gestione di eventuale documentazione, e un sistema di issue tracking sufficientemente performante ai fini del progetto.

\subsubsection{Google Drive}
Per lo storage di ulteriori files e documenti, o per tutto ciò che non è strettamente correlato alla produzione e consegna del progetto, sarà salvato su un sistema di Cloud storage. Ciò permette l'accesso dei file a tutti i membri, la possibilità di modifica, pur non avendo una gestione capillare della versione (che è affidata a GitHub per i file designati).

\subsubsection{Asana}
Per permettere una più semplice divisione del lavoro e assegnazione delle mansioni, si è scelto di utilizzare il software Asana. Asana permette la creazione, modifica e assegnazione di tasks e subtasks. Ad essi va indicato un assegnatario e una data di scadenza. La creazione di tasks è limitata alla figura di Responsabile di Progetto che coordina il lavoro tra i membri del gruppo, evitando squilibri e conflitti di interessi. I tasks devono essere quanto più specifici possibile, e descrivere nel dettaglio i ruoli che l'assegnatario andrà a svolgere. Ognuno è tenuto a verificare le mansioni assegnate con frequenza giornaliera, ad aggiornare lo stato dei tasks ed un eventuale completamento, e a rispondere a dubbi e interrogativi sollevati nella conversazione dei task assegnati.

