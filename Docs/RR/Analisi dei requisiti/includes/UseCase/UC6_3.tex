\subsubsection{Caso d'uso UC6.3: Visualizzazione risultati}
\label{UC6_3}

\begin{minipage}{\linewidth}
	\begin{tabular}{ l | p{11cm}}
		\hline
		\rowcolor{Gray}
		\multicolumn{2}{c}{UC6.3 - Visualizzazione risultati} \\
		\hline
		\textbf{Attori} & Utente generico, Utente non autenticato, Utente autenticato, Amministratore API Market \\
		\textbf{Descrizione} & L'attore visualizza le API che corrispondono alle keywords della ricerca effettuata \\
		\textbf{Pre-Condizioni} & L'attore ha confermato la ricerca API \\
		\textbf{Post-Condizioni} & L'attore ha visualizzato le API che corrispondono alle keywords della ricerca effettuata \\
		\textbf{Scenario Principale} & 
		\begin{enumerate*}[label=(\arabic*.),itemjoin={\newline}]
			\item L'attore può visualizzare le API che corrispondono alle keywords della ricerca effettuata, che possono essere anche vuoti nel caso di stringhe non consone
		\end{enumerate*}\\
		\textbf{Scenari Alternativi} & 
		\begin{enumerate*}[label=(\arabic*.),itemjoin={\newline}]
		\item L'attore può accedere alla schermata delle singole API tramite la lista delle API (UC7)
	\end{enumerate*}\\
	\end{tabular}
\end{minipage}