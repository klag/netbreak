\paragraph{Caso d'uso UC7.1.5:  Modifica Email}
\label{UC7_1_5}
\begin{figure}[ht]
	\centering
	\includegraphics[scale=0.45]{UML/UC7_1_5.png}
	\caption{UC7.1.5:  Modifica Email}
\end{figure}
\FloatBarrier
\begin{tabular}{ l | p{11cm}}
	\hline
	\rowcolor{Gray}
	 \multicolumn{2}{c}{UC7.1.5 - Modifica Email} \\
	 \hline
		\textbf{Attori} & Utente Autenticato \\
	\textbf{Descrizione} & Gli utenti possono modificare la propria Email\\
	\textbf{Pre-Condizioni} & L'utente e' nella schermata di gestione Profilo\\
	\textbf{Post-Condizioni} & L'utente ha modificato con successo la Propria Email\\
	\textbf{Scenario Principale} & 
	\begin{enumerate*}[label=(\arabic*.),itemjoin={\newline}]
		\item L'utente puo' inserire la nuova Email (UC7.1.5.1)
		\item L'utente conferma la modifica al'Email (UC7.1.5.2)
	\end{enumerate*}\\
	\textbf{Scenari Alternativi} & 
	\begin{enumerate*}[label=(\arabic*.),itemjoin={\newline}]
		\item L'utente visualizza un errore nella modifica dell'Email (UC7.1.5.3)
	\end{enumerate*}\\
\end{tabular}

