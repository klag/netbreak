\subsubsection{Caso d'uso UC5.2: Conferma inserimento email}
\label{UC5_2}

\begin{minipage}{\linewidth}
\begin{longtable}{ l | p{11cm}}
	\hline
	\rowcolor{Gray}
	 \multicolumn{2}{c}{UC5.2 - Conferma inserimento email} \\
	 \hline
	\textbf{Attori} & Utente non autenticato \\
	\textbf{Descrizione} & L'attore conferma l'inserimento del proprio indirizzo email tramite un apposito pulsante \\
	\textbf{Pre-Condizioni} & L'attore ha inserito l'email dell'account che intende recuperare\\
	\textbf{Post-Condizioni} & L'attore ha compilato il campo email, relativo all'account che desidera recuperare\\
	\textbf{Scenario Principale} & \begin{enumerate*}[label=(\arabic*.),itemjoin={\newline}]
		\item L'attore può confermare i dati immessi, visualizzando un messaggio di successo e ricevendo tramite email la procedura di recupero password (UC5.4) e viene reindirizzato alla schermata principale (UC1)
	\end{enumerate*}\\
	\textbf{Scenari Alternativi} & 
	\begin{enumerate*}[label=(\arabic*.),itemjoin={\newline}]
		\item L'attore ha inserito un email non valida o inesistente, e visualizza un messaggio d'errore (UC5.3)
	\end{enumerate*}\\
\end{longtable}
\end{minipage}