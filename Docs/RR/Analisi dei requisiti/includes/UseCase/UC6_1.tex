\subsubsection{Caso d'uso UC6.1:  Inserimento Nome API}
\label{UC6_1}

\begin{tabular}{ l | p{11cm}}
	\hline
	\rowcolor{Gray}
	 \multicolumn{2}{c}{UC6.1 - Inserimento Nome API} \\
	 \hline
	\textbf{Attori} & Utente Non Autenticato, Utente Autenticato \\
	\textbf{Descrizione} & Gli utenti possono effettuare una ricerca delle API usandone il nome\\
	\textbf{Pre-Condizioni} & L'utente ha scelto fare una ricerca di API\\
	\textbf{Post-Condizioni} & L'utente ha inserito il nome dell'API nella barra di ricerca \\
	\textbf{Scenario Principale} & 
	\begin{enumerate*}[label=(\arabic*.),itemjoin={\newline}]
		\item L'utente puo' inserire il Nome dell API nella barra di ricerca (UC6.1)
	\end{enumerate*}\\
\end{tabular}