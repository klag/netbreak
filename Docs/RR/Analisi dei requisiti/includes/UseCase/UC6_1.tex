\subsubsection{Caso d'uso UC6.1: Inserimento keywords}
\label{UC6_1}

\begin{minipage}{\linewidth}
\begin{tabular}{ l | p{11cm}}
	\hline
	\rowcolor{Gray}
	 \multicolumn{2}{c}{UC6.1 - Inserimento keywords} \\
	 \hline
	\textbf{Attori} & Utente generico, Utente non autenticato, Utente autenticato, Amministratore API Market \\
	\textbf{Descrizione} & L'attore effettua una ricerca delle API inserendo nella barra di ricerca una stringa contenente le keywords desiderate \\
	\textbf{Pre-Condizioni} & L'attore ha scelto di effettuare una ricerca di API \\
	\textbf{Post-Condizioni} & L'attore ha inserito nella barra di ricerca una stringa contenente le keywords desiderate \\
	\textbf{Scenario Principale} & 
	\begin{enumerate*}[label=(\arabic*.),itemjoin={\newline}]
		\item L'attore può inserire nella barra di ricerca una stringa con tenente le keywords desiderate: esse verranno ricercate sul nome dell'API, sul nome dell'autore e su eventuali tag
	\end{enumerate*}\\
\end{tabular}
\end{minipage}