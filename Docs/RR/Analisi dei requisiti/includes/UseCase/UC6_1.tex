\subsubsection{Caso d'uso UC6.1: Inserimento keyword}
\label{UC6_1}

\begin{minipage}{\linewidth}
\begin{tabular}{ l | p{11cm}}
	\hline
	\rowcolor{Gray}
	 \multicolumn{2}{c}{UC6.1 - Inserimento keyword} \\
	 \hline
	\textbf{Attori} & Utente non autenticato, Utente autenticato \\
	\textbf{Descrizione} & L'attore può effettuare una ricerca delle API una stringa contenente le keyword desiderate\\
	\textbf{Pre-Condizioni} & L'attore ha scelto di effettuare una ricerca di API\\
	\textbf{Post-Condizioni} & L'attore ha inserito una stringa nella barra di ricerca \\
	\textbf{Scenario Principale} & 
	\begin{enumerate*}[label=(\arabic*.),itemjoin={\newline}]
		\item L'attore può inserire una stringa nella barra di ricerca: esse possono contenere il nome dell'API, dell'autore, di eventuali tag o parole chiave presenti nella descrizione del prodotto.
	\end{enumerate*}\\
\end{tabular}
\end{minipage}