\subsubsection{Caso d'uso UC6.6:  Visualizzazione Errore Ricerca}
\label{UC6_6}

\begin{tabular}{ l | p{11cm}}
	\hline
	\rowcolor{Gray}
	 \multicolumn{2}{c}{UC6.6 - Visualizzazione Errore Ricerca} \\
	 \hline
	\textbf{Attori} & Utente Non Autenticato, Utente Autenticato \\
	\textbf{Descrizione} & Gli utenti visualizzano un errore in seguito a una ricerca illegale o in seguito a un errore a livello software\\
	\textbf{Pre-Condizioni} & L'utente ha scelto fare una ricerca di API\\
	\textbf{Post-Condizioni} & L'utente deve rifare la ricerca\\
	\textbf{Scenario Principale} & 
	\begin{enumerate*}[label=(\arabic*.),itemjoin={\newline}]
		\item L'utente visualizza un Errore Ricerca (UC6.6)
	\end{enumerate*}\\
\end{tabular}