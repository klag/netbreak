\newpage
\subsection{Caso d'uso UC2: Main post-autenticazione }
\label{UC2}
\begin{figure}[ht]
	\centering
	\includegraphics[scale=0.45]{UML/UC2.png}
	\caption{UC2: Main post-autenticazione}
\end{figure}

\begin{longtable}{ l | p{11cm}}
	\hline
	\rowcolor{Gray}
	 \multicolumn{2}{c}{UC2 - Main post-autenticazione} \\
	 \hline
	\textbf{Attori} & Utente autenticato, Amministratore API Market \\
	\textbf{Descrizione} & L'attore tramite la schermata principale
	dell'applicazione, può accedere e sfruttare le funzionalità a lui disponibili: l'interazione
	con il proprio profilo utente, con le API non acquistate e non, con le API registrate, la
	registrazione di una nuova API, il logout. 
	L'Amministratore API Market, oltre alle funzionalità offerte all'utente autenticato, può
	visualizzare i dati di utilizzo delle API ed amministrare l'applicazione web.  \\
	\textbf{Pre-Condizioni} & L'attore ha avviato l'applicazione web e si è autenticato \\
	\textbf{Post-Condizioni} & L'applicazione ha eseguito le richieste dell'attore \\
	\textbf{Scenario Principale} & 
	\begin{enumerate*}[label=(\arabic*.),itemjoin={\newline}]
		\item L'attore può effettuare una ricerca sulle API presenti nell'applicazione
(UC6)
		\item L'attore può visualizzare le API da lui acquistate (UC7)
		\item L'attore può visualizzare le API da lui registrate (UC8)
		\item L'attore può registrare una nuova API (UC9)
		\item L'attore può effettuare una ricerca sugli utenti registrati all'applicazione (UC10)
		\item L'attore può visualizzare il proprio profilo utente (UC11)
		\item L'attore può effettuare il logout (UC12)
		\item L'amministratore API Market può accedere ai servizi di amministrazione dell'applicazione web (UC13)
	\end{enumerate*}\\
\end{longtable}