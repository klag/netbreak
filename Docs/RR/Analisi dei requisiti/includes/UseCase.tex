\newpage
\section{Casi d'uso}
Vengono elencati i casi d'uso ricavati dall'analisi del capitolato C1.

\subsection{Attori}
Di seguito è riportato il diagramma UML che descrive la gerarchia degli attori. Sono sono state individuate 3 tipologie differenti di attori con funzionalità crescenti. Gli attori sono: l'utente non autenticato, l'utente autenticato e l'amministratore API Market (un attore autenticato con alcune funzionalità superuser)
\label{Attori}

\begin{figure}[ht]
	\centering
	\includegraphics[scale=0.45]{UML/attori.png}
	\caption{Attori}
\end{figure}


\newpage
\subsection{Caso d'uso UC1:  Main }
\label{UC1}
\begin{figure}[ht]
	\centering
	\includegraphics[scale=0.45]{UML/UC1.png}
	\caption{UC1: Main}
\end{figure}

\renewcommand*{\arraystretch}{1.6}
\begin{longtable}{ l | p{11cm}}
	\hline
	\rowcolor{Gray}
	 \multicolumn{2}{c}{UC1 - Main} \\
	 \hline
	\textbf{Attori} & Utente Non Autenticato  \\
	\textbf{Descrizione} & L'attore, tramite la schermata principale dell'applicazione, può accedere e sfruttare le funzionalità a lui disponibili: la registrazione, il login, il recupero password, la ricerca API  \\
	\textbf{Pre-Condizioni} & L'attore ha avviato l'applicazione web e non si è ancora autenticato \\
	\textbf{Post-Condizioni}&L'applicazione ha eseguito le richieste dell'attore\\
	\textbf{Scenario Principale} & \begin{enumerate*}[label=(\arabic*.),itemjoin={\newline}]
		\item L'attore può registrarsi all'applicazione (UC2)
		\item L'attore può effettuare il login all'applicazione (UC3)
		\item L'attore può recuperare la propria password (UC4)
		\item L'attore può effettuare una ricerca sulle API presenti nell'applicazione (UC6)
	\end{enumerate*}\\
\end{longtable}
\newpage
\subsection{Caso d'uso UC2:  Main post-autenticazione }
\label{UC2}
\begin{figure}[ht]
	\centering
	\includegraphics[scale=0.45]{UML/UC2.png}
	\caption{UC2: Main post-autenticazione}
\end{figure}

\begin{longtable}{ l | p{11cm}}
	\hline
	\rowcolor{Gray}
	 \multicolumn{2}{c}{UC2 - Main post-autenticazione} \\
	 \hline
	\textbf{Attori} & Utente autenticato, Amministratore API Market \\
	\textbf{Descrizione} & L'attore tramite la schermata principale
	dell'applicazione, può accedere e sfruttare le funzionalità a lui disponibili: l'interazione
	con il proprio profilo utente, con le API non acquistate e non, con le API registrate, la
	registrazione di una nuova API, il logout. 
	L'Amministratore API Market, oltre alle funzionalità offerte all'utente autenticato, può
	visualizzare i dati di utilizzo delle API ed amministrare l'applicazione web.  \\
	\textbf{Pre-Condizioni} & L'attore ha avviato l'applicazione web e si è autenticato \\
	\textbf{Post-Condizioni} & L'applicazione ha eseguito le richieste dell'attore \\
	\textbf{Scenario Principale} & 
	\begin{enumerate*}[label=(\arabic*.),itemjoin={\newline}]
		\item L'attore può interagire con il proprio profilo utente
		\item L'attore può interagire con le API da lui non acquistate
		\item L'attore può interagire con le API da lui acquistate
		\item L'attore può interagire con le API da lui registrate
		\item L'attore può registrare una nuova API
		\item L'attore può effettuare il logout
		\item L'attore Amministratore API Market può visualizzare i dati di utilizzo delle singole API 
		\item L'attore Amministratore API Market può amministrare l'applicazione web
	\end{enumerate*}\\
\end{longtable}
\newpage
\subsection{Caso d'uso UC3: Registrazione utente }
\label{UC3}
\begin{figure}[ht]
	\centering
	\includegraphics[scale=0.45]{UML/UC3.png}
	\caption{UC3: Registrazione utente}
\end{figure}

\begin{longtable}{ l | p{11cm}}
	\hline
	\rowcolor{Gray}
	 \multicolumn{2}{c}{UC3 - Registrazione utente} \\
	 \hline
	\textbf{Attori} & Utente non autenticato \\
	\textbf{Descrizione} & L'attore inserisce le sue informazioni personali per potersi registrare all'applicazione web, così da poter successivamente effettuare il login ed evolversi in un utente autenticato. \\
	\textbf{Pre-Condizioni} & L'attore ha scelto di registrarsi e l'applicazione web mostra la schermata di registrazione \\
	\textbf{Post-Condizioni} & L'attore si è registrato all'applicazione web \\
	\textbf{Scenario Principale} & \begin{enumerate*}[label=(\arabic*.),itemjoin={\newline}]
		\item L'attore può inserire il proprio nome (UC3.1)
		\item L'attore può inserire il proprio cognome (UC3.2)
		\item L'attore può inserire lo username desiderato (UC3.3)
		\item L'attore può inserire la propria email (UC3.4) 
		\item L'attore può inserire la password desiderata (UC3.5)
		\item L'attore può reinserire la password desiderata per conferma (UC3.6)
		\item L'attore può confermare la registrazione (UC3.7)
	\end{enumerate*}\\
	\textbf{Scenari Alternativi} & 
	\begin{enumerate*}[label=(\arabic*.),itemjoin={\newline}]
		\item L'attore può visualizzare un messaggio d'errore che segnala i campi dati non validi (UC3.8)
	\end{enumerate*}\\
\end{longtable}
\subsubsection{Caso d'uso UC3.1:  Inserimento Nome}
\label{UC3_1}

\begin{longtable}{ l | p{11cm}}
	\hline
	\rowcolor{Gray}
	 \multicolumn{2}{c}{UC3.1 - Inserimento Nome} \\
	 \hline
	\textbf{Attori} & Utente Non Autenticato \\
	\textbf{Descrizione} & L'utente non autenticato inserisce il suo nome  \\
	\textbf{Pre-Condizioni} & L'utente ha scelto di registrarsi e l'applicazione web mostra la schermata di registrazione \\
	\textbf{Post-Condizioni} & L'utente visualizza area per registrazione a applicazione web \\
	\textbf{Scenario Principale} & \begin{enumerate*}[label=(\arabic*.),itemjoin={\newline}]
		\item L'utente non autenticato può inserire il proprio Nome (UC3.1)
	\end{enumerate*}\\
	\textbf{Scenari Alternativi} & 
	\begin{enumerate*}[label=(\arabic*.),itemjoin={\newline}]
		\item Il Nome non e' valido perche' contiene caratteri particolari
	\end{enumerate*}\\
\end{longtable}
\subsubsection{Caso d'uso UC3.2: Inserimento cognome}
\label{UC3_2}

\begin{longtable}{ l | p{11cm}}
	\hline
	\rowcolor{Gray}
	 \multicolumn{2}{c}{UC3.2 - Inserimento cognome} \\
	 \hline
	\textbf{Attori} & Utente non autenticato \\
	\textbf{Descrizione} & L'attore inserisce il suo cognome \\
	\textbf{Pre-Condizioni} & L'applicazione mostra il campo dati per l'inserimento del cognome \\
	\textbf{Post-Condizioni} & L'attore ha inserito il proprio cognome \\
	\textbf{Scenario Principale} & \begin{enumerate*}[label=(\arabic*.),itemjoin={\newline}]
		\item L'attore può inserire il proprio cognome
	\end{enumerate*}\\
\end{longtable}
\subsubsection{Caso d'uso UC3.3:  Inserimento username}
\label{UC3_3}

\begin{longtable}{ l | p{11cm}}
	\hline
	\rowcolor{Gray}
	 \multicolumn{2}{c}{UC3.3:  Inserimento username} \\
	 \hline
	\textbf{Attori} & Utente non autenticato \\
	\textbf{Descrizione} & L'attore inserisce il suo Username  \\
	\textbf{Pre-Condizioni} & L'applicazione visualizza i form per l'inserimento del campo per lo username \\
	\textbf{Post-Condizioni} & L'attore ha inserito lo username desiderato \\
	\textbf{Scenario Principale} & \begin{enumerate*}[label=(\arabic*.),itemjoin={\newline}]
		\item L'utente non autenticato può inserire il proprio username
	\end{enumerate*}\\
\end{longtable}
\subsubsection{Caso d'uso UC3.4:  Registrazione email}
\label{UC3_4}

\begin{longtable}{ l | p{11cm}}
	\hline
	\rowcolor{Gray}
	 \multicolumn{2}{c}{UC3.4 - Inserimento email} \\
	 \hline
	\textbf{Attori} & Utente non autenticato \\
	\textbf{Descrizione} & L'attore inserisce la propria email  \\
	\textbf{Pre-Condizioni} & L'applicazione visualizza i form per l'inserimento del campo per l'email \\
	\textbf{Post-Condizioni} & L'attore ha inserito la propria email \\
	\textbf{Scenario Principale} & \begin{enumerate*}[label=(\arabic*.),itemjoin={\newline}]
		\item L'utente non autenticato può inserire la propria email
	\end{enumerate*}\\
\end{longtable}

\subsubsection{Caso d'uso UC3.5: Inserimento password}
\label{UC3_5}

\begin{longtable}{ l | p{11cm}}
	\hline
	\rowcolor{Gray}
	 \multicolumn{2}{c}{UC3.5 - Inserimento password} \\
	 \hline
	\textbf{Attori} & Utente non autenticato \\
	\textbf{Descrizione} & L'attore inserisce la password desiderata  \\
	\textbf{Pre-Condizioni} & L'applicazione mostra il campo dati per l'inserimento della password \\
	\textbf{Post-Condizioni} & L'attore ha inserito la password desiderata \\
	\textbf{Scenario Principale} & \begin{enumerate*}[label=(\arabic*.),itemjoin={\newline}]
		\item L'attore può inserire la password desiderata
	\end{enumerate*}\\
\end{longtable}
\subsubsection{Caso d'uso UC3.6: Reinserimento password}
\label{UC3_6}

\begin{longtable}{ l | p{11cm}}
	\hline
	\rowcolor{Gray}
	 \multicolumn{2}{c}{UC3.6 - Reinserimento password} \\
	 \hline
	\textbf{Attori} & Utente non autenticato \\
	\textbf{Descrizione} & L'attore reinserisce la password desiderata  \\
	\textbf{Pre-Condizioni} & L'applicazione mostra il campo dati per il reinserimento della password \\
	\textbf{Post-Condizioni} & L'attore ha reinserito la password desiderata \\
	\textbf{Scenario Principale} & \begin{enumerate*}[label=(\arabic*.),itemjoin={\newline}]
		\item L'attore può reinserire la password desiderata
	\end{enumerate*}\\
\end{longtable}
\subsubsection{Caso d'uso UC3.7: Conferma registrazione}
\label{UC3_7}

\begin{longtable}{ l | p{11cm}}
	\hline
	\rowcolor{Gray}
	 \multicolumn{2}{c}{UC3.7 - Conferma registrazione} \\
	 \hline
	\textbf{Attori} & Utente non autenticato \\
	\textbf{Descrizione} & L'attore conferma i dati inseriti per la registrazione \\
	\textbf{Pre-Condizioni} & L'applicazione mostra il pulsante per la conferma della registrazione (con le informazioni inserite nei campi dati) \\
	\textbf{Post-Condizioni} & L'attore ha confermato la registrazione, visualizzando il messaggio di successo, ed è stato reindirizzato alla pagina principale come utente non autenticato (UC1) \\
	\textbf{Scenario Principale} & \begin{enumerate*}[label=(\arabic*.),itemjoin={\newline}]
		\item L'attore può confermare la registrazione, visualizzando il messaggio di successo, e venendo reindirizzato alla pagina principale come utente non autenticato (UC1)
	\end{enumerate*}\\
\end{longtable}
\subsubsection{Caso d'uso UC3.8:  Visualizza Errore Registrazione}
\label{UC3_8}

\begin{longtable}{ l | p{11cm}}
	\hline
	\rowcolor{Gray}
	 \multicolumn{2}{c}{UC3.8 - Visualizza Errore Registrazione} \\
	 \hline
	\textbf{Attori} & Utente Non Autenticato \\
	\textbf{Descrizione} & L'utente non autenticato visualizza un messaggio d'errore circa la sua registrazione  \\
	\textbf{Pre-Condizioni} & L'utente ha scelto di registrarsi e inserito dati errati di registrazione \\
	\textbf{Post-Condizioni} & L'utente visualizza un messaggio d'errore \\
	\textbf{Scenario Principale} & \begin{enumerate*}[label=(\arabic*.),itemjoin={\newline}]
		\item L'utente non autenticato Visualizza un Errore di Registrazione (UC3.1)
	\end{enumerate*}\\
\end{longtable}
\newpage
\subsection{Caso d'uso UC4: Login}
\label{UC4}
\begin{figure}[ht]
	\centering
	\includegraphics[scale=0.45]{UML/UC4.png}
	\caption{UC4: Login}
\end{figure}

\begin{longtable}{ l | p{11cm}}
	\hline
	\rowcolor{Gray}
	 \multicolumn{2}{c}{UC4 - Login}\\
	 \hline
	\textbf{Attori} & Utente non autenticato \\
	\textbf{Descrizione} & L'attore inserisce le sue informazioni personali per poter accedere all'applicazione web ed evolversi in un utente autenticato. Può effettuare login tramite Facebook, Twitter, LinkedIn, Google+ e tramite l'API Market stesso. \\
	\textbf{Pre-Condizioni} & L'attore ha scelto di effettuare l'accesso e l'applicazione web mostra la schermata di login \\
	\textbf{Post-Condizioni} & L'attore ha effettuato l'accesso all'applicazione web \\
	\textbf{Scenario Principale} & \begin{enumerate*}[label=(\arabic*.),itemjoin={\newline}]
		\item L'attore può effettuare il login tramite API Market (UC4.1)
		\item L'attore può effettuare il login tramite Facebook (UC4.2)
		\item L'attore può effettuare il login tramite Twitter (UC4.3)
		\item L'attore può effettuare il login tramite LinkedIn (UC4.4)
		\item L'attore può effettuare il login tramite Google+ (UC4.5)
	\end{enumerate*}\\
\end{longtable}
\newpage
\subsubsection{Caso d'uso UC4.1: Login tramite API Market}
\label{UC4_1}
\begin{figure}[!htbp]
	\centering
	\includegraphics[scale=0.45]{UML/UC4_1.png}
	\caption{UC4.1: Login tramite API Market}
\end{figure}

\begin{tabular}{ l | p{11cm}}
	\hline
	\rowcolor{Gray}
	 \multicolumn{2}{c}{UC4.1 - Login tramite API Market} \\
	 \hline
	\textbf{Attori} & Utente non autenticato \\
	\textbf{Descrizione} & L'attore effettua il login all'applicazione web, così da evolversi in un utente autenticato\\
	\textbf{Pre-Condizioni} & L'attore ha scelto di eseguire il login all'applicazione web e non è autenticato \\
	\textbf{Post-Condizioni} & L'attore ha effettuato il login all'applicazione web, evolvendosi in un utente autenticato \\
	\textbf{Scenario Principale} & 
	\begin{enumerate*}[label=(\arabic*.),itemjoin={\newline}]
		\item L'attore può inserire l'email o username (UC4.1.1)
		\item L'attore non autenticato può inserire la password (UC4.1.2)
		\item L'attore può confermare i dati inseriti per procedere al login (UC4.1.3)
	\end{enumerate*}\\
\end{tabular}

\paragraph{Caso d'uso UC4.1.1:  Inserimento username o email}
\label{UC4_1_1}

\begin{longtable}{ l | p{11cm}}
	\hline
	\rowcolor{Gray}
	\multicolumn{2}{c}{Caso d'uso UC4.1.1:  Inserimento username o email} \\
	\hline
	\textbf{Attori} & Utente non autenticato \\
	\textbf{Descrizione} & L'attore inserisce la propria username o email  \\
	\textbf{Pre-Condizioni} & L'applicazione visualizza i form per l'inserimento del campo per lo username \\
	\textbf{Post-Condizioni} & L'attore ha inserito il proprio username o email \\
	\textbf{Scenario Principale} & \begin{enumerate*}[label=(\arabic*.),itemjoin={\newline}]
		\item L'utente non autenticato può inserire il proprio username o la propria email
	\end{enumerate*}\\
\end{longtable}

\paragraph{Caso d'uso UC4.1.2:  Inserimento password}
\label{UC4_1_2}

\begin{longtable}{ l | p{11cm}}
	\hline
	\rowcolor{Gray}
	\multicolumn{2}{c}{Caso d'uso UC4.1.2:  Inserimento password} \\
	\hline
	\textbf{Attori} & Utente non autenticato \\
	\textbf{Descrizione} & L'attore inserisce la propria password  \\
	\textbf{Pre-Condizioni} & L'applicazione visualizza i form per l'inserimento del campo per la password \\
	\textbf{Post-Condizioni} & L'attore ha inserito la propria password \\
	\textbf{Scenario Principale} & \begin{enumerate*}[label=(\arabic*.),itemjoin={\newline}]
		\item L'utente non autenticato può inserire la propria password
	\end{enumerate*}\\
\end{longtable}

\paragraph{Caso d'uso UC4.1.3:  Inserimento username o email}
\label{UC4_1_3}

\begin{longtable}{ l | p{11cm}}
	\hline
	\rowcolor{Gray}
	\multicolumn{2}{c}{Caso d'uso UC4.1.3:  Inserimento username o email} \\
	\hline
	\textbf{Attori} & Utente non autenticato \\
	\textbf{Descrizione} & L'attore conferma i dati inseriti per il login  \\
	\textbf{Pre-Condizioni} & L'applicazione visualizza il pulsante per confermare i dati di accesso \\
	\textbf{Post-Condizioni} & L'attore ha confermato i dati inseriti \\
	\textbf{Scenario Principale} & \begin{enumerate*}[label=(\arabic*.),itemjoin={\newline}]
	\item L'attore convalida i dati inseriti per procedere con il login, e verrà reindirizzato alla schermata principale post-autenticazione (UC2) oppure, in caso di login fallito, viene consentito all'utente di ritentare dall'apposita schermata (UC4.1)
\end{enumerate*}\\
\end{longtable}
\subsubsection{Caso d'uso UC4.2:  Login Tramite Facebook }
\label{UC4_2}
\begin{figure}[ht]
	\centering
	\includegraphics[scale=0.45]{UML/UC4_2.png}
	\caption{UC4.2: Login Tramite Facebook}
\end{figure}

\begin{tabular}{ l | p{11cm}}
	\hline
	\rowcolor{Gray}
	 \multicolumn{2}{c}{UC4.2 - Login Tramite Facebook} \\
	 \hline
	\textbf{Attori} & Utente Non Autenticato, Facebook \\
	\textbf{Descrizione} & L'utente non autenticato effettua il login all'applicazione web tramite Facebook, così da evolversi in un utente autenticato\\
	\textbf{Pre-Condizioni} & L'utente ha scelto di eseguire il login all'applicazione web e non è autenticato \\
	\textbf{Post-Condizioni} & L'utente ha effettuato il login all'applicazione web tramite Facebook, evolvendosi in un utente autenticato \\
	\textbf{Scenario Principale} & \begin{enumerate*}[label=(\arabic*.),itemjoin={\newline}]
		\item L'utente non autenticato può effettuare il login all'applicazione web tramite Facebook (UC4.2.1)
	\end{enumerate*}\\
\end{tabular}
\newpage
\subsubsection{Caso d'uso UC4.3: Login tramite Twitter }
\label{UC4_2}
\begin{figure}[!htbp]
	\centering
	\includegraphics[scale=0.45]{UML/UC4_3.png}
	\caption{UC4.3: Login tramite Twitter}
\end{figure}

\begin{tabular}{ l | p{11cm}}
	\hline
	\rowcolor{Gray}
	\multicolumn{2}{c}{UC4.3 - Login tramite Twitter} \\
	\hline
	\textbf{Attori} & Utente non autenticato, Twitter \\
	\textbf{Descrizione} & L'attore Utente non autenticato effettua il login all'applicazione web tramite Twitter, così da evolversi in un utente autenticato\\
	\textbf{Pre-Condizioni} & L'attore Utente non autenticato ha scelto di eseguire il login all'applicazione web e non è autenticato \\
	\textbf{Post-Condizioni} & L'attore Utente non autenticato ha effettuato il login all'applicazione web tramite Twitter, evolvendosi in un utente autenticato \\
	\textbf{Scenario Principale} & \begin{enumerate*}[label=(\arabic*.),itemjoin={\newline}]
		\item L'attore effettua il login con successo tramite Twitter, evolvendosi in un Utente autenticato (UC2)
	\end{enumerate*}\\
	\textbf{Scenari Alternativi} & \begin{enumerate*}[label=(\arabic*.),itemjoin={\newline}]
		\item L'attore ha fallito il login tramite Twitter (E.g: Mancanza di privilegi e autorizzazioni, utente non loggato correttamente a Twitter)
	\end{enumerate*}\\
\end{tabular}
\subsubsection{Caso d'uso UC4.4: Login Tramite LinkedIn }
\label{UC4_4}
\begin{figure}[ht]
	\centering
	\includegraphics[scale=0.45]{UML/UC4_4.png}
	\caption{UC4.4: Login Tramite LinkedIn}
\end{figure}

\begin{longtable}{ l | p{11cm}}
	\hline
	\rowcolor{Gray}
	 \multicolumn{2}{c}{UC4.4 - Login Tramite LinkedIn} \\
	 \hline
	\textbf{Attori} & Utente Non Autenticato, LinkedIn \\
	\textbf{Descrizione} & L'utente non autenticato effettua il login all'applicazione web tramite LinkedIn, così da evolversi in un utente autenticato\\
	\textbf{Pre-Condizioni} & L'utente ha scelto di eseguire il login all'applicazione web e non è autenticato \\
	\textbf{Post-Condizioni} & L'utente ha effettuato il login all'applicazione web tramite LinkedIn, evolvendosi in un utente autenticato \\
	\textbf{Scenario Principale} & \begin{enumerate*}[label=(\arabic*.),itemjoin={\newline}]
		\item L'utente non autenticato può effettuare il login all'applicazione web tramite LinkedIn (UC4.4.1)
	\end{enumerate*}\\
\end{longtable}

\subsubsection{Caso d'uso UC4.5: Login Tramite Google+ }
\label{UC4_5}
\begin{figure}[ht]
	\centering
	\includegraphics[scale=0.45]{UML/UC4_5.png}
	\caption{UC4.5: Login Tramite Google+ }
\end{figure}

\begin{longtable}{ l | p{11cm}}
	\hline
	\rowcolor{Gray}
	 \multicolumn{2}{c}{UC4.5 - Login Tramite Google+} \\
	 \hline
	\textbf{Attori} & Utente Non Autenticato, Google+ \\
	\textbf{Descrizione} & L'utente non autenticato effettua il login all'applicazione web tramite Google+, così da evolversi in un utente autenticato\\
	\textbf{Pre-Condizioni} & L'utente ha scelto di eseguire il login all'applicazione web e non è autenticato \\
	\textbf{Post-Condizioni} & L'utente ha effettuato il login all'applicazione web tramite Google+, evolvendosi in un utente autenticato\\
	\textbf{Scenario Principale} & \begin{enumerate*}[label=(\arabic*.),itemjoin={\newline}]
		\item L'utente non autenticato può effettuare il login all'applicazione web tramite Google+ (UC4.5.1)
	\end{enumerate*}\\
\end{longtable}
\newpage
\subsection{Caso d'uso UC5: Recupero password }
\label{UC5}
\begin{figure}[ht]
	\centering
	\includegraphics[scale=0.45]{UML/UC5.png}
	\caption{UC5: Recupero password }
\end{figure}

\begin{longtable}{ l | p{11cm}}
	\hline
	\rowcolor{Gray}
	 \multicolumn{2}{c}{UC5 - Recupero password} \\
	 \hline
	\textbf{Attori} & Utente non autenticato \\
	\textbf{Descrizione} & L'attore tenta il recupero della propria password tramite l'invio di una email \\
	\textbf{Pre-Condizioni} & L'attore ha scelto di recuperare la sua password e non è autenticato \\
	\textbf{Post-Condizioni} & L'attore ha ricevuto nella propria casella email un link per reimpostare la propria password, oppure la procedura è fallita \\
	\textbf{Scenario Principale} & 
	\begin{enumerate*}[label=(\arabic*.),itemjoin={\newline}]
		\item L'attore può inserire la propria email di registrazione (UC5.1)
		\item L'attore può confermare l'indirizzo email inserito, al quale l'applicazione web ha inviato un link per reimpostare la password (UC5.2)
	\end{enumerate*}\\
	\textbf{Scenari Alternativi} & 
	\begin{enumerate*}[label=(\arabic*.),itemjoin={\newline}]
		\item L'attore visualizza un errore e l'invio della email di recupero password non avviene (UC5.3) 
	\end{enumerate*}\\
\end{longtable}
\subsubsection{Caso d'uso UC5.1:  Inserimento Email}
\label{UC5_1}

\begin{longtable}{ l | p{11cm}}
	\hline
	\rowcolor{Gray}
	 \multicolumn{2}{c}{UC5.1 - Inserimento Email} \\
	 \hline
	\textbf{Attori} & Utente Non Autenticato \\
	\textbf{Descrizione} & L'utente non autenticato inserisce la sua Email  \\
	\textbf{Pre-Condizioni} & L'utente ha dimenticato la password e l'applicazione web mostra la schermata di Recupero Password\\
	\textbf{Post-Condizioni} & L'utente visualizza area per Recupero Password per l'applicazione web \\
	\textbf{Scenario Principale} & \begin{enumerate*}[label=(\arabic*.),itemjoin={\newline}]
		\item L'utente non autenticato può inserire la propria Email(UC5.1)
	\end{enumerate*}\\
	\textbf{Scenari Alternativi} & 
	\begin{enumerate*}[label=(\arabic*.),itemjoin={\newline}]
		\item L'email inserita non e' valida
		\item L'email inserita non esiste
	\end{enumerate*}\\
\end{longtable}
\subsubsection{Caso d'uso UC5.2:  Conferma Inserimento Email}
\label{UC5_2}

\begin{longtable}{ l | p{11cm}}
	\hline
	\rowcolor{Gray}
	 \multicolumn{2}{c}{UC5.2 - Conferma Inserimento Email} \\
	 \hline
	\textbf{Attori} & Utente Non Autenticato \\
	\textbf{Descrizione} & L'utente non autenticato re-inserisce la sua Email per confermarla \\
	\textbf{Pre-Condizioni} & L'utente ha dimenticato la password e l'applicazione web mostra la schermata di Recupero Password\\
	\textbf{Post-Condizioni} & L'utente visualizza un Messaggio di conferma che invita a controllare la propria email\\
	\textbf{Scenario Principale} & \begin{enumerate*}[label=(\arabic*.),itemjoin={\newline}]
		\item L'utente conferma inserimento Email (UC5.2)
	\end{enumerate*}\\
	\textbf{Scenari Alternativi} & 
	\begin{enumerate*}[label=(\arabic*.),itemjoin={\newline}]
		\item L'operazione non e' andata a buon fine
	\end{enumerate*}\\
\end{longtable}
\subsubsection{Caso d'uso UC5.3:  Errore inserimento email}
\label{UC5_3}

\begin{minipage}{\linewidth}
\begin{longtable}{ l | p{11cm}}
	\hline
	\rowcolor{Gray}
	 \multicolumn{2}{c}{UC5.3 - Errore inserimento email} \\
	 \hline
	\textbf{Attori} & Utente non autenticato \\
	\textbf{Descrizione} & L'attore riceve un messaggio di errore dovuto all'inserimento di un'email non valida \\
	\textbf{Pre-Condizioni} & L'attore ha dimenticato la password e ha inserito l'email per poterla recuperare\\
	\textbf{Post-Condizioni} & L'attore riceve un messaggio di errore e può eventualmente ripetere la procedura di recupero password\\
	\textbf{Scenario Principale} & \begin{enumerate*}[label=(\arabic*.),itemjoin={\newline}]
		\item L'attore visualizza un messaggio d'errore per aver lasciato il campo vuoto o per aver inserito un indirizzo inesistente. Può eventualmente ripetere la procedura (UC5)
	\end{enumerate*}\\
\end{longtable}
\end{minipage}

\subsubsection{Caso d'uso UC5.4: Reset password}
\label{UC5_4}

\begin{minipage}{\linewidth}
	\begin{longtable}{ l | p{11cm}}
		\hline
		\rowcolor{Gray}
		\multicolumn{2}{c}{UC5.4 - Reset password} \\
		\hline
		\textbf{Attori} & Utente non autenticato \\
		\textbf{Descrizione} & L'attore ha ricevuto un link per la schermata di reset della propria password \\
		\textbf{Pre-Condizioni} & L'attore ha richiesto il recupero della password ed ha aperto il link per poterla resettare\\
		\textbf{Post-Condizioni} & L'attore ha resettato con successo la password\\
		\textbf{Scenario Principale} & \begin{enumerate*}[label=(\arabic*.),itemjoin={\newline}]
			\item L'attore può inserire la nuova password desiderata (UC5.4.1)
			\item L'attore può reinserire la nuova password desiderata (UC5.4.2)
			\item L'attore può confermare i dati inseriti e completare la procedura con successo (UC5.4.3)
		\end{enumerate*}\\
		\textbf{Scenari Alternativi} & \begin{enumerate*}[label=(\arabic*.),itemjoin={\newline}]
			\item L'attore visualizza un errore qualora le password inserite non coincidano o i campi risultino vuoti (UC5.4.4)
		\end{enumerate*}\\
	\end{longtable}
\end{minipage}

\paragraph{Caso d'uso UC5.4.1: Inserimento nuova password}
\label{UC5_4_1}

\begin{minipage}{\linewidth}
	\begin{longtable}{ l | p{11cm}}
		\hline
		\rowcolor{Gray}
		\multicolumn{2}{c}{UC5.4.1 - Inserimento nuova password} \\
		\hline
		\textbf{Attori} & Utente non autenticato \\
		\textbf{Descrizione} & L'attore inserisce la nuova password per il proprio account \\
		\textbf{Pre-Condizioni} & L'attore ha visualizzato la schermata di reset password\\
		\textbf{Post-Condizioni} & L'attore ha inserito la nuova password\\
		\textbf{Scenario Principale} & \begin{enumerate*}[label=(\arabic*.),itemjoin={\newline}]
			\item L'attore può inserire la nuova password per il proprio account
		\end{enumerate*}\\
	\end{longtable}
\end{minipage}

\paragraph{Caso d'uso UC5.4.2: Reinserimento nuova password}
\label{UC5_4_2}

\begin{minipage}{\linewidth}
	\begin{longtable}{ l | p{11cm}}
		\hline
		\rowcolor{Gray}
		\multicolumn{2}{c}{UC5.4.2 - Reinserimento nuova password} \\
		\hline
		\textbf{Attori} & Utente non autenticato \\
		\textbf{Descrizione} & L'attore reinserisce la nuova password per il proprio account \\
		\textbf{Pre-Condizioni} & L'attore ha visualizzato la schermata di reset password\\
		\textbf{Post-Condizioni} & L'attore ha reinserito la nuova password\\
		\textbf{Scenario Principale} & \begin{enumerate*}[label=(\arabic*.),itemjoin={\newline}]
			\item L'attore può reinserire la nuova password per il proprio account
		\end{enumerate*}\\
	\end{longtable}
\end{minipage}

\paragraph{Caso d'uso UC5.4.3: Conferma nuova password}
\label{UC5_4_3}

\begin{minipage}{\linewidth}
	\begin{longtable}{ l | p{11cm}}
		\hline
		\rowcolor{Gray}
		\multicolumn{2}{c}{UC5.4.3 - Conferma nuova password} \\
		\hline
		\textbf{Attori} & Utente non autenticato \\
		\textbf{Descrizione} & L'attore conferma i dati inseriti per poter resettare la propria password \\
		\textbf{Pre-Condizioni} & L'attore ha inserito due volte la propria nuova password\\
		\textbf{Post-Condizioni} & L'attore ha confermato i nuovi dati per il proprio account\\
		\textbf{Scenario Principale} & \begin{enumerate*}[label=(\arabic*.),itemjoin={\newline}]
			\item L'attore può confermare i nuovi dati per il proprio account, visualizzando un messaggio di successo in caso di esito positivo
		\end{enumerate*}\\
	\end{longtable}
\end{minipage}

\paragraph{Caso d'uso UC5.4.4: Errore reset password}
\label{UC5_4_4}

\begin{minipage}{\linewidth}
	\begin{longtable}{ l | p{11cm}}
		\hline
		\rowcolor{Gray}
		\multicolumn{2}{c}{UC5.4.4 - Errore reset password} \\
		\hline
		\textbf{Attori} & Utente non autenticato \\
		\textbf{Descrizione} & L'attore conferma i dati inseriti per poter resettare la propria password \\
		\textbf{Pre-Condizioni} & L'attore ha inserito due volte la propria nuova password\\
		\textbf{Post-Condizioni} & L'attore non ha confermato con successo i propri nuovi dati di accesso\\
		\textbf{Scenario Principale} & \begin{enumerate*}[label=(\arabic*.),itemjoin={\newline}]
			\item L'attore può confermare i nuovi dati per il proprio account, ma visualizza un messaggio di errore poichè le password non coincidono o i campi sono vuoti
		\end{enumerate*}\\
	\end{longtable}
\end{minipage}
\newpage
\subsection{Caso d'uso UC6: Ricerca API}
\label{UC6}
\begin{figure}[ht]
	\centering
	\includegraphics[scale=0.45]{UML/UC6.png}
	\caption{UC6: Ricerca API}
\end{figure}

\begin{longtable}{ l | p{11cm}}
	\hline
	\rowcolor{Gray}
	 \multicolumn{2}{c}{UC6 - Ricerca API} \\
	 \hline
	\textbf{Attori} & Utente non autenticato, Utente autenticato \\
	\textbf{Descrizione} & L'attore inserisce le keyword necessarie alla ricerca di API. \\
	\textbf{Pre-Condizioni} & L'attore ha scelto di effettuare una ricerca tra le API \\
	\textbf{Post-Condizioni} & L'attore ha effettuato la ricerca di API e ha visualizzato il risultato\\
	\textbf{Scenario Principale} & 
	\begin{enumerate*}[label=(\arabic*.),itemjoin={\newline}]
		\item L'attore può inserire la stringa di ricerca desiderata (UC6.1)
		\item L'attore può confermare i dati inseriti (UC6.2) e visualizzare i risultati forniti dall'applicazione web (UC6.3)
	\end{enumerate*}\\
\end{longtable}
\subsubsection{Caso d'uso UC6.1:  Inserimento Nome API}
\label{UC6_1}

\begin{tabular}{ l | p{11cm}}
	\hline
	\rowcolor{Gray}
	 \multicolumn{2}{c}{UC6.1 - Inserimento Nome API} \\
	 \hline
	\textbf{Attori} & Utente Non Autenticato, Utente Autenticato \\
	\textbf{Descrizione} & Gli utenti possono effettuare una ricerca delle API usandone il nome\\
	\textbf{Pre-Condizioni} & L'utente ha scelto fare una ricerca di API\\
	\textbf{Post-Condizioni} & L'utente ha inserito il nome dell'API nella barra di ricerca \\
	\textbf{Scenario Principale} & 
	\begin{enumerate*}[label=(\arabic*.),itemjoin={\newline}]
		\item L'utente puo' inserire il Nome dell API nella barra di ricerca (UC6.1)
	\end{enumerate*}\\
\end{tabular}
\subsubsection{Caso d'uso UC6.2:  Inserimento Nome Utente}
\label{UC6_2}

\begin{tabular}{ l | p{11cm}}
	\hline
	\rowcolor{Gray}
	 \multicolumn{2}{c}{UC6.2 - Inserimento Nome Utente} \\
	 \hline
	\textbf{Attori} & Utente Non Autenticato, Utente Autenticato \\
	\textbf{Descrizione} & Gli utenti possono effettuare una ricerca delle API usandone il nome dell'Utente che le ha create. Tale utente puo' rappresentare un'azienda\\
	\textbf{Pre-Condizioni} & L'utente ha scelto fare una ricerca di API\\
	\textbf{Post-Condizioni} & L'utente ha inserito il nome dell'Utente creatore dell'API nella barra di ricerca \\
	\textbf{Scenario Principale} & 
	\begin{enumerate*}[label=(\arabic*.),itemjoin={\newline}]
		\item L'utente puo' inserire il Nome dell'Utente Creatore dell'API nella barra di ricerca (UC6.2)
	\end{enumerate*}\\
\end{tabular}
\subsubsection{Caso d'uso UC6.3:  Scelta Categoria}
\label{UC6_3}

\begin{tabular}{ l | p{11cm}}
	\hline
	\rowcolor{Gray}
	 \multicolumn{2}{c}{UC6.3 - Scelta Categoria} \\
	 \hline
	\textbf{Attori} & Utente Non Autenticato, Utente Autenticato \\
	\textbf{Descrizione} & Gli utenti possono effettuare una ricerca delle API usandone la categoria\\
	\textbf{Pre-Condizioni} & L'utente ha scelto fare una ricerca di API\\
	\textbf{Post-Condizioni} & L'utente ha inserito il nome della categoria dell'API nella barra di ricerca \\
	\textbf{Scenario Principale} & 
	\begin{enumerate*}[label=(\arabic*.),itemjoin={\newline}]
		\item L'utente puo' inserire il Nome della categoria dell'API nella barra di ricerca (UC6.3)
	\end{enumerate*}\\
\end{tabular}
\newpage
\subsection{Caso d'uso UC7 - Visualizzazione API}
\label{UC7}
\begin{figure}[ht]
	\centering
	\includegraphics[scale=0.45]{UML/UC7.png}
	\caption{UC7: Visualizzazione API}
\end{figure}

\begin{longtable}{ l | p{11cm}}
	\hline
	\rowcolor{Gray}
	\multicolumn{2}{c}{UC7 - Visualizzazione API}\\
	\hline
	
	 \textbf{Attori} & Utente non autenticato, Utente autenticato  \\
	\textbf{Descrizione} & L'attore può visualizzare i dati relativi a un API che ha selezionato tramite la homepage o i risultati di una ricerca  \\
	\textbf{Pre-Condizioni} & L'attore ha selezionato un prodotto per la consultazione \\
	\textbf{Post-Condizioni} & L'attore visualizza la pagina relativa all'API selezionata\\
	\textbf{Scenario Principale} & 
	\begin{enumerate*}[label=(\arabic*.),itemjoin={\newline}]
		\item L'attore visualizza il nome dell'API (UC7.1)
		\item L'attore visualizza la descrizione dell'API (UC7.2)
		\item L'attore visualizza l'autore dell'API (UC7.3)
		\item L'attore visualizza i tag registrati relativi all'API (UC7.4)
		\item L'attore può visualizzare l'interfaccia dell'API (UC7.5)
		\item L'attore può consultare la documentazione fornita dall'utente (UC7.6)
		\item L'attore può visualizzare i dati di utilizzo dell'API  (UC7.7)
		\item L'attore può acquistare l'API visualizzata (UC9)
	\end{enumerate*}\\
	\textbf{Scenari Alternativi} & 
	\begin{enumerate*}[label=(\arabic*.),itemjoin={\newline}]
		\item L'attore possiede una licenza attiva per il prodotto, e può visualizzarne la chiave personale di utilizzo (UC7.8)
	\end{enumerate*}\\
\end{longtable}


\subsubsection{Caso d'uso UC7.1: Visualizzazione nome API}
\label{UC7_1}

\begin{minipage}{\linewidth}
	\begin{tabular}{ l | p{11cm}}
		\hline
		\rowcolor{Gray}
		\multicolumn{2}{c}{UC7.1 - Visualizza nome API} \\
		\hline
		\textbf{Attori} & Utente non autenticato, Utente autenticato \\
		\textbf{Descrizione} & L'attore visualizza il nome dell'API nella schermata relativa\\
		\textbf{Pre-Condizioni} & L'attore ha selezionato un API per poterla visualizzare\\
		\textbf{Post-Condizioni} & L'attore visualizza il nome dell'API selezionata \\
		\textbf{Scenario Principale} & 
		\begin{enumerate*}[label=(\arabic*.),itemjoin={\newline}]
			\item L'attore può visualizzare il nome dell'API selezionata
		\end{enumerate*}\\
	\end{tabular}
\end{minipage}

\subsubsection{Caso d'uso UC7.2: Visualizzazione descrizione API}
\label{UC7_2}

\begin{minipage}{\linewidth}
	\begin{tabular}{ l | p{11cm}}
		\hline
		\rowcolor{Gray}
		\multicolumn{2}{c}{UC7.2 - Visualizzazione descrizione API} \\
		\hline
		\textbf{Attori} & Utente non autenticato, Utente autenticato \\
		\textbf{Descrizione} & L'attore visualizza la descrizione dell'API nella schermata relativa\\
		\textbf{Pre-Condizioni} & L'attore ha selezionato un API per poterla visualizzare\\
		\textbf{Post-Condizioni} & L'attore visualizza la descrizione dell'API selezionata \\
		\textbf{Scenario Principale} & 
		\begin{enumerate*}[label=(\arabic*.),itemjoin={\newline}]
			\item L'attore può visualizzare la descrizione dell'API selezionata
		\end{enumerate*}\\
	\end{tabular}
\end{minipage}

\subsubsection{Caso d'uso UC7.3: Visualizzazione autore API}
\label{UC7_3}

\begin{minipage}{\linewidth}
	\begin{tabular}{ l | p{11cm}}
		\hline
		\rowcolor{Gray}
		\multicolumn{2}{c}{UC7.3 - Visualizzazione autore API} \\
		\hline
		\textbf{Attori} & Utente non autenticato, Utente autenticato \\
		\textbf{Descrizione} & L'attore visualizza il nome dell'autore dell'API nella schermata relativa\\
		\textbf{Pre-Condizioni} & L'attore ha selezionato un API per poterla visualizzare\\
		\textbf{Post-Condizioni} & L'attore visualizza il nome dell'autore per l'API selezionata \\
		\textbf{Scenario Principale} & 
		\begin{enumerate*}[label=(\arabic*.),itemjoin={\newline}]
			\item L'attore può visualizzare il nome dell'autore per l'API selezionata
		\end{enumerate*}\\
	\end{tabular}
\end{minipage}

\subsubsection{Caso d'uso UC7.4: Visualizzazione tag API}
\label{UC7_4}

\begin{minipage}{\linewidth}
	\begin{tabular}{ l | p{11cm}}
		\hline
		\rowcolor{Gray}
		\multicolumn{2}{c}{UC7.4 - Visualizzazione tag API} \\
		\hline
		\textbf{Attori} & Utente non autenticato, Utente autenticato \\
		\textbf{Descrizione} & L'attore visualizza i tag dell'API nella schermata relativa\\
		\textbf{Pre-Condizioni} & L'attore ha selezionato un API per poterla visualizzare\\
		\textbf{Post-Condizioni} & L'attore visualizza i tag per l'API selezionata \\
		\textbf{Scenario Principale} & 
		\begin{enumerate*}[label=(\arabic*.),itemjoin={\newline}]
			\item L'attore può visualizzare i tag assegnati all'API selezionata
		\end{enumerate*}\\
	\end{tabular}
\end{minipage}

\subsubsection{Caso d'uso UC7.5: Visualizzazione interfaccia API}
\label{UC7_5}

\begin{minipage}{\linewidth}
	\begin{tabular}{ l | p{11cm}}
		\hline
		\rowcolor{Gray}
		\multicolumn{2}{c}{UC7.5 - Visualizzazione interfaccia API} \\
		\hline
		\textbf{Attori} & Utente non autenticato, Utente autenticato \\
		\textbf{Descrizione} & L'attore visualizza l'interfaccia dell'API nella schermata relativa\\
		\textbf{Pre-Condizioni} & L'attore ha selezionato un API per poterla visualizzare\\
		\textbf{Post-Condizioni} & L'attore visualizza l'interfaccia dell'API selezionata \\
		\textbf{Scenario Principale} & 
		\begin{enumerate*}[label=(\arabic*.),itemjoin={\newline}]
			\item L'attore può visualizzare l'interfaccia dell'API selezionata
		\end{enumerate*}\\
	\end{tabular}
\end{minipage}

\subsubsection{Caso d'uso UC7.6: Visualizzazione documentazione API}
\label{UC7_6}

\begin{minipage}{\linewidth}
	\begin{tabular}{ l | p{11cm}}
		\hline
		\rowcolor{Gray}
		\multicolumn{2}{c}{UC7.6 - Visualizzazione documentazione API} \\
		\hline
		\textbf{Attori} & Utente non autenticato, Utente autenticato \\
		\textbf{Descrizione} & L'attore può visualizzare la documentazione dell'API\\
		\textbf{Pre-Condizioni} & L'attore ha selezionato un API per poterla visualizzare\\
		\textbf{Post-Condizioni} & L'attore visualizza la documentazione dell'API selezionata \\
		\textbf{Scenario Principale} & 
		\begin{enumerate*}[label=(\arabic*.),itemjoin={\newline}]
			\item L'attore può visualizzare la documentazione dell'API selezionata tramite un link esterno fornita dall'autore (UC7.6.1)
			\item L'attore può visualizzare la documentazione dell'API selezionata scaricando il file PDF fornito dall'autore (UC7.6.2)
		\end{enumerate*}\\
	\end{tabular}
\end{minipage}

\subsubsection{Caso d'uso UC7.6.1: Visualizzazione esterna documentazione}
\label{UC7_6_1}

\begin{minipage}{\linewidth}
	\begin{tabular}{ l | p{11cm}}
		\hline
		\rowcolor{Gray}
		\multicolumn{2}{c}{UC7.6.1 - Visualizzazione esterna documentazione} \\
		\hline
		\textbf{Attori} & Utente non autenticato, Utente autenticato, Pagina web esterna \\
		\textbf{Descrizione} & L'attore può visualizzare la documentazione dell'API tramite un link esterno fornito dall'autore\\
		\textbf{Pre-Condizioni} & L'attore visualizza un link relativo alla documentazione esterna\\
		\textbf{Post-Condizioni} & L'attore apre il link, e viene reindirizzato ad una pagina esterna fornita dall'autore dell'API \\
		\textbf{Scenario Principale} & 
		\begin{enumerate*}[label=(\arabic*.),itemjoin={\newline}]
			\item L'attore visita la pagina esterna per consultare la documentazione fornita dall'autore
		\end{enumerate*}\\
	\end{tabular}
\end{minipage}

\subsubsection{Caso d'uso UC7.6.2: Scarica documentazione PDF}
\label{UC7_6_2}

\begin{minipage}{\linewidth}
	\begin{tabular}{ l | p{11cm}}
		\hline
		\rowcolor{Gray}
		\multicolumn{2}{c}{UC7.6.2 - Scarica documentazione PDF} \\
		\hline
		\textbf{Attori} & Utente non autenticato, Utente autenticato \\
		\textbf{Descrizione} & L'attore può scaricare la documentazione dell'API in formato PDF\\
		\textbf{Pre-Condizioni} & L'attore visualizza un link relativo alla documentazione in PDF\\
		\textbf{Post-Condizioni} & L'attore seleziona il link di download e scarica la documentazione in formato PDF \\
		\textbf{Scenario Principale} & 
		\begin{enumerate*}[label=(\arabic*.),itemjoin={\newline}]
			\item L'attore può scaricare la documentazione fornita dall'autore dell'API in formato PDF
		\end{enumerate*}\\
	\end{tabular}
\end{minipage}

\subsubsection{Caso d'uso UC7.7: Visualizzazione dati utilizzo API}
\label{UC7_7}

\begin{minipage}{\linewidth}
	\begin{tabular}{ l | p{11cm}}
		\hline
		\rowcolor{Gray}
		\multicolumn{2}{c}{UC7.7 - Visualizzazione dati utilizzo API} \\
		\hline
		\textbf{Attori} & Utente non autenticato, Utente autenticato \\
		\textbf{Descrizione} & L'attore visualizza i dati di utilizzo dell'API nella schermata relativa\\
		\textbf{Pre-Condizioni} & L'attore ha selezionato un API per poterla visualizzare\\
		\textbf{Post-Condizioni} & L'attore visualizza i dati di utilizzo per l'API selezionata \\
		\textbf{Scenario Principale} & 
		\begin{enumerate*}[label=(\arabic*.),itemjoin={\newline}]
			\item L'attore può visualizzare il numero di licenze attive (UC7.7.1)
			\item L'attore può visualizzare il numero di chiamate giornaliere effettuate (UC7.7.2)
			\item L'attore può visualizzare il tempo medio di utilizzo (UC7.7.3)
			\item L'attore può visualizzare il quantitativo di dati scambiati (UC7.7.4)
		\end{enumerate*}\\
	\end{tabular}
\end{minipage}

\subsubsection{Caso d'uso UC7.7.1: Visualizzazione licenze attive}
\label{UC7_7.1}

\begin{minipage}{\linewidth}
	\begin{tabular}{ l | p{11cm}}
		\hline
		\rowcolor{Gray}
		\multicolumn{2}{c}{UC7.7.1 - Visualizzazione licenze attive} \\
		\hline
		\textbf{Attori} & Utente non autenticato, Utente autenticato \\
		\textbf{Descrizione} & L'attore visualizza il numero di licenze attive \\
		\textbf{Pre-Condizioni} & L'attore si trova nella schermata relativa ad una singola API\\
		\textbf{Post-Condizioni} & L'attore visualizza il numero di licenze attive per l'API selezionata \\
		\textbf{Scenario Principale} & 
		\begin{enumerate*}[label=(\arabic*.),itemjoin={\newline}]
			\item L'attore può visualizzare il numero di licenze attive per l'API selezionata
		\end{enumerate*}\\
	\end{tabular}
\end{minipage}

\subsubsection{Caso d'uso UC7.7.2: Visualizzazione chiamate giornaliere}
\label{UC7_7.2}

\begin{minipage}{\linewidth}
	\begin{tabular}{ l | p{11cm}}
		\hline
		\rowcolor{Gray}
		\multicolumn{2}{c}{UC7.7.2 - Visualizzazione chiamate giornaliere} \\
		\hline
		\textbf{Attori} & Utente non autenticato, Utente autenticato \\
		\textbf{Descrizione} & L'attore visualizza il numero di chiamate giornaliere \\
		\textbf{Pre-Condizioni} & L'attore si trova nella schermata relativa ad una singola API\\
		\textbf{Post-Condizioni} & L'attore visualizza il numero di chiamate giornaliere per l'API selezionata \\
		\textbf{Scenario Principale} & 
		\begin{enumerate*}[label=(\arabic*.),itemjoin={\newline}]
			\item L'attore può visualizzare il numero di chiamate giornaliere per l'API selezionata
		\end{enumerate*}\\
	\end{tabular}
\end{minipage}

\subsubsection{Caso d'uso UC7.7.3: Visualizzazione tempo medio di utilizzo}
\label{UC7_7.3}

\begin{minipage}{\linewidth}
	\begin{tabular}{ l | p{11cm}}
		\hline
		\rowcolor{Gray}
		\multicolumn{2}{c}{UC7.7.3 - Visualizzazione tempo medio di utilizzo} \\
		\hline
		\textbf{Attori} & Utente non autenticato, Utente autenticato \\
		\textbf{Descrizione} & L'attore visualizza il tempo medio di utilizzo \\
		\textbf{Pre-Condizioni} & L'attore si trova nella schermata relativa ad una singola API\\
		\textbf{Post-Condizioni} & L'attore visualizza il tempo medio di utilizzo per l'API selezionata \\
		\textbf{Scenario Principale} & 
		\begin{enumerate*}[label=(\arabic*.),itemjoin={\newline}]
			\item L'attore può visualizzare il tempo medio di utilizzo per l'API selezionata
		\end{enumerate*}\\
	\end{tabular}
\end{minipage}

\subsubsection{Caso d'uso UC7.7.4: Visualizzazione dati scambiati}
\label{UC7_7.4}

\begin{minipage}{\linewidth}
	\begin{tabular}{ l | p{11cm}}
		\hline
		\rowcolor{Gray}
		\multicolumn{2}{c}{UC7.7.4 - Visualizzazione dati scambiati} \\
		\hline
		\textbf{Attori} & Utente non autenticato, Utente autenticato \\
		\textbf{Descrizione} & L'attore visualizza il quantitativo di dati scambiati tra l'utilizzatore e il microservizio \\
		\textbf{Pre-Condizioni} & L'attore si trova nella schermata relativa ad una singola API\\
		\textbf{Post-Condizioni} & L'attore visualizza il quantitativo di dati scambiati per l'API selezionata \\
		\textbf{Scenario Principale} & 
		\begin{enumerate*}[label=(\arabic*.),itemjoin={\newline}]
			\item L'attore può visualizzare il quantitativo di dati scambiati per l'API selezionata
		\end{enumerate*}\\
	\end{tabular}
\end{minipage}

\subsubsection{Caso d'uso UC7.8: Visualizzazione chiave API}
\label{UC7_8}

\begin{minipage}{\linewidth}
	\begin{tabular}{ l | p{11cm}}
		\hline
		\rowcolor{Gray}
		\multicolumn{2}{c}{UC7.8 - Visualizzazione chiave API} \\
		\hline
		\textbf{Attori} & Utente autenticato \\
		\textbf{Descrizione} & L'attore visualizza la propria chiave di utilizzo per l'API\\
		\textbf{Pre-Condizioni} & L'attore ha selezionato un API per poterla visualizzare e possiede una licenza attualmente attiva\\
		\textbf{Post-Condizioni} & L'attore visualizza la propria chiave API \\
		\textbf{Scenario Principale} & 
		\begin{enumerate*}[label=(\arabic*.),itemjoin={\newline}]
			\item L'attore può visualizzare la propria chiave API personale
		\end{enumerate*}\\
	\end{tabular}
\end{minipage}
\newpage
\subsection{Caso d'uso UC8 - Visualizzazione API acquistate}
\label{UC8}
\begin{figure}[ht]
	\centering
	\includegraphics[scale=0.45]{UML/UC8.png}
	\caption{UC8: Visualizzazione API}
\end{figure}

\begin{longtable}{ l | p{11cm}}
	\hline
	\rowcolor{Gray}
	\multicolumn{2}{c}{UC8 - Visualizzazione API acquistate}\\
	\hline
	
	 \textbf{Attori} & Utente autenticato  \\
	\textbf{Descrizione} & L'attore può visualizzare la schermata relativa alle API da lui acquistate \\
	\textbf{Pre-Condizioni} & L'attore seleziona il link per visualizzare i propri acquisti  \\
	\textbf{Post-Condizioni} & L'attore visualizza la pagina relativa alle API da lui acquistate\\
	\textbf{Scenario Principale} & 
	\begin{enumerate*}[label=(\arabic*.),itemjoin={\newline}]
		\item L'attore visualizza il numero di API acquistate e attive (UC8.1)
		\item L'attore visualizza la lista di API acquistate e attive (UC8.2)
		\item L'attore visualizza la chiave per ciascuna API attiva (UC8.3)
	\end{enumerate*}\\
	\textbf{Scenari Alternativi} & 
	\begin{enumerate*}[label=(\arabic*.),itemjoin={\newline}]
		\item L'attore può visualizzare i dati relativi a una singola API (UC7)
	\end{enumerate*}\\
\end{longtable}


\subsubsection{Caso d'uso UC8.1: Visualizzazione numero API acquistate}
\label{UC8_1}

\begin{minipage}{\linewidth}
	\begin{tabular}{ l | p{11cm}}
		\hline
		\rowcolor{Gray}
		\multicolumn{2}{c}{UC8.1 - Visualizzazione numero API acquistate} \\
		\hline
		\textbf{Attori} & Utente autenticato \\
		\textbf{Descrizione} & L'attore visualizza il numero di API acquistate nella schermata relativa\\
		\textbf{Pre-Condizioni} & L'attore si trova nel menù relativo alle API acquistate e attive\\
		\textbf{Post-Condizioni} & L'attore visualizza il numero di API acquistate \\
		\textbf{Scenario Principale} & 
		\begin{enumerate*}[label=(\arabic*.),itemjoin={\newline}]
			\item L'attore può visualizzare il numero di API acquistate
		\end{enumerate*}\\
	\end{tabular}
\end{minipage}

\subsubsection{Caso d'uso UC8.2: Visualizzazione numero API acquistate}
\label{UC8_2}

\begin{minipage}{\linewidth}
	\begin{tabular}{ l | p{11cm}}
		\hline
		\rowcolor{Gray}
		\multicolumn{2}{c}{UC8.2 - Visualizzazione lista API acquistate} \\
		\hline
		\textbf{Attori} & Utente autenticato \\
		\textbf{Descrizione} & L'attore visualizza la lista di API acquistate nella schermata relativa\\
		\textbf{Pre-Condizioni} & L'attore si trova nel menù relativo alle API acquistate e attive\\
		\textbf{Post-Condizioni} & L'attore visualizza la lista di API acquistate \\
		\textbf{Scenario Principale} & 
		\begin{enumerate*}[label=(\arabic*.),itemjoin={\newline}]
			\item L'attore visualizza il nome dell'API (8.2.1)
			\item L'attore visualizza il link alla pagina dell'API (8.2.2)
			\item L'attore visualizza la scadenza della propria licenza (8.2.3)
		\end{enumerate*}\\
	\end{tabular}
\end{minipage}

\paragraph{Caso d'uso UC8.2.1: Visualizzazione nome API}
\label{UC8_2_1}

\begin{minipage}{\linewidth}
	\begin{tabular}{ l | p{11cm}}
		\hline
		\rowcolor{Gray}
		\multicolumn{2}{c}{UC8.2.1 - Visualizzazione nome API} \\
		\hline
		\textbf{Attori} & Utente autenticato \\
		\textbf{Descrizione} & L'attore visualizza il nome dell'API\\
		\textbf{Pre-Condizioni} & L'attore visualizza la lista API nel menù relativo alle API acquistate e attive\\
		\textbf{Post-Condizioni} & L'attore visualizza il nome dell'API nella lista \\
		\textbf{Scenario Principale} & 
		\begin{enumerate*}[label=(\arabic*.),itemjoin={\newline}]
			\item L'attore può visualizzare il nome dell'API
		\end{enumerate*}\\
	\end{tabular}
\end{minipage}

\paragraph{Caso d'uso UC8.2.2: Visualizzazione link API}
\label{UC8_2_2}

\begin{minipage}{\linewidth}
	\begin{tabular}{ l | p{11cm}}
		\hline
		\rowcolor{Gray}
		\multicolumn{2}{c}{UC8.2.2 - Visualizzazione link API} \\
		\hline
		\textbf{Attori} & Utente autenticato \\
		\textbf{Descrizione} & L'attore visualizza il link dell'API\\
		\textbf{Pre-Condizioni} & L'attore visualizza la lista API nel menù relativo alle API acquistate e attive\\
		\textbf{Post-Condizioni} & L'attore visualizza il link dell'API nella lista \\
		\textbf{Scenario Principale} & 
		\begin{enumerate*}[label=(\arabic*.),itemjoin={\newline}]
			\item L'attore può visualizzare il link dell'API
		\end{enumerate*}\\
	\end{tabular}
\end{minipage}

\paragraph{Caso d'uso UC8.2.3: Visualizzazione scadenza licenza}
\label{UC8_2_3}

\begin{minipage}{\linewidth}
	\begin{tabular}{ l | p{11cm}}
		\hline
		\rowcolor{Gray}
		\multicolumn{2}{c}{UC8.2.3 - Visualizzazione scadenza licenza} \\
		\hline
		\textbf{Attori} & Utente autenticato \\
		\textbf{Descrizione} & L'attore visualizza la scadenza dell'API\\
		\textbf{Pre-Condizioni} & L'attore visualizza la lista API nel menù relativo alle API acquistate e attive\\
		\textbf{Post-Condizioni} & L'attore visualizza la scadenza dell'API nella lista \\
		\textbf{Scenario Principale} & 
		\begin{enumerate*}[label=(\arabic*.),itemjoin={\newline}]
			\item L'attore può visualizzare la scadenza della propria licenza per l'API
		\end{enumerate*}\\
	\end{tabular}
\end{minipage}

\subsubsection{Caso d'uso UC8.3: Visualizzazione chiave API}
\label{UC8_3}

\begin{minipage}{\linewidth}
	\begin{tabular}{ l | p{11cm}}
		\hline
		\rowcolor{Gray}
		\multicolumn{2}{c}{UC8.3 - Visualizzazione chiave API} \\
		\hline
		\textbf{Attori} & Utente autenticato \\
		\textbf{Descrizione} & L'attore visualizza la propria chiave di utilizzo per l'API\\
		\textbf{Pre-Condizioni} & L'attore si trova nel menù relativo alle API acquistate e attive\\
		\textbf{Post-Condizioni} & L'attore visualizza la propria chiave API \\
		\textbf{Scenario Principale} & 
		\begin{enumerate*}[label=(\arabic*.),itemjoin={\newline}]
			\item L'attore può visualizzare la propria chiave API personale
		\end{enumerate*}\\
	\end{tabular}
\end{minipage}
\newpage
\subsection{Caso d'uso UC9 - Acquisto API}
\label{UC9}
\begin{figure}[ht]
	\centering
	\includegraphics[scale=0.45]{UML/UC9.png}
	\caption{UC9: Visualizzazione API registrate}
\end{figure}

\begin{longtable}{ l | p{11cm}}
	\hline
	\rowcolor{Gray}
	\multicolumn{2}{c}{UC9 - Acquisto API}\\
	\hline
	
	 \textbf{Attori} & Utente autenticato  \\
	\textbf{Descrizione} & L'attore può effettuare l'acquisto dell'API selezionata tramite i crediti da lui posseduti \\
	\textbf{Pre-Condizioni} & L'attore seleziona il pulsante relativo all'acquisto dalla pagina della singola API  \\
	\textbf{Post-Condizioni} & L'attore visualizza la pagina per poter effettuare l'acquisto\\
	\textbf{Scenario Principale} & 
	\begin{enumerate*}[label=(\arabic*.),itemjoin={\newline}]
		\item L'attore visualizza il nome dell'API (UC7.1)
		\item L'attore visualizza l'autore dell'API (UC7.3)
		\item L'attore visualizza un menù per la scelta della licenza da acquistare (UC9.1)
		\item L'attore visualizza il saldo disponibile nel suo portafoglio virtuale (UC12.2.1)
		\item L'attore visualizza il prezzo dell'API selezionata (UC7.8)
		\item L'attore visualizza il saldo preventivato in seguito all'acquisto della licenza selezionata (UC9.2)
		\item L'attore può confermare l'acquisto qualora abbia crediti sufficienti, e venire reindirizzato ad una schermata di riepilogo (UC9.3)
	\end{enumerate*}\\
	\textbf{Scenari Alternativi} & 
	\begin{enumerate*}[label=(\arabic*.),itemjoin={\newline}]
		\item L'attore visualizza un apposito pulsante per poter ricaricare i propri crediti (tramite opportuna pagina) qualora essi non siano sufficienti a completare la transazione (UC9.4)
		\item L'attore visualizza un messaggio ad indicare che la transazione non è andata a buon fine, per motivi differenti dal proprio saldo (UC9.5)
	\end{enumerate*}\\
\end{longtable}


\subsubsection{Caso d'uso UC9.1: Visualizzazione menù licenza}
\label{UC9_1}

\begin{minipage}{\linewidth}
	\begin{tabular}{ l | p{11cm}}
		\hline
		\rowcolor{Gray}
		\multicolumn{2}{c}{UC9.1 - Visualizzazione menù licenza} \\
		\hline
		\textbf{Attori} & Utente autenticato \\
		\textbf{Descrizione} & L'attore visualizza un menù relativo alla scelta della licenza da acquistare e può effettuare la scelta più consona secondo il proprio interesse\\
		\textbf{Pre-Condizioni} & L'attore visualizza il menù per la scelta della licenza\\
		\textbf{Post-Condizioni} & L'attore ha selezionato la licenza a lui più consona, oppure accetta la scelta di default \\
		\textbf{Scenario Principale} & 
		\begin{enumerate*}[label=(\arabic*.),itemjoin={\newline}]
			\item L'attore può selezionare la licenza più consona alle sue necessità, tra quelle rese disponibili dall'autore, tramite il menù preposto
		\end{enumerate*}\\
		\textbf{Scenari Alternativi} & 
		\begin{enumerate*}[label=(\arabic*.),itemjoin={\newline}]
			\item L'attore non modifica la scelta di default, accettandola
		\end{enumerate*}\\
	\end{tabular}
\end{minipage}

\subsubsection{Caso d'uso UC9.2: Previsione saldo}
\label{UC9_2}

\begin{minipage}{\linewidth}
	\begin{tabular}{ l | p{11cm}}
		\hline
		\rowcolor{Gray}
		\multicolumn{2}{c}{UC9.2 - Previsione saldo} \\
		\hline
		\textbf{Attori} & Utente autenticato \\
		\textbf{Descrizione} & L'attore visualizza una previsione del proprio saldo crediti in seguito all'acquisto\\
		\textbf{Pre-Condizioni} & L'attore si trova nella pagina preposta all'acquisto di un'API\\
		\textbf{Post-Condizioni} & L'attore visualizza una previsione del proprio saldo in seguito all'acquisto \\
		\textbf{Scenario Principale} & 
		\begin{enumerate*}[label=(\arabic*.),itemjoin={\newline}]
			\item L'attore può visualizzare una previsione del proprio saldo qualora portasse a termine l'acquisto della licenza selezionata nel caso d'uso UC9.1
		\end{enumerate*}\\
	\end{tabular}
\end{minipage}

\paragraph{Caso d'uso UC9.3: Riepilogo acquisto}
\label{UC9_3}

\begin{minipage}{\linewidth}
	\begin{tabular}{ l | p{11cm}}
		\hline
		\rowcolor{Gray}
		\multicolumn{2}{c}{UC9.3 - Riepilogo acquisto} \\
		\hline
		\textbf{Attori} & Utente autenticato \\
		\textbf{Descrizione} & L'attore visualizza un riepilogo dell'acquisto appena effettuato\\
		\textbf{Pre-Condizioni} & L'attore ha confermato l'acquisto per un API\\
		\textbf{Post-Condizioni} & L'attore visualizza un messaggio di conferma e la chiave per l'API, che verrà inviata anche tramite email dal sistema\\
		\textbf{Scenario Principale} & 
		\begin{enumerate*}[label=(\arabic*.),itemjoin={\newline}]
			\item L'attore può visualizzare un messaggio di ringraziamento e riepilogo per l'acquisto effettuato.
			\item L'attore può visualizzare l'API key relativa all'acquisto andato a buon fine (UC7.8)
		\end{enumerate*}\\
	\end{tabular}
\end{minipage}

\paragraph{Caso d'uso UC9.4: Pulsante saldo insufficiente}
\label{UC9_4}

\begin{minipage}{\linewidth}
	\begin{tabular}{ l | p{11cm}}
		\hline
		\rowcolor{Gray}
		\multicolumn{2}{c}{UC9.4 - Pulsante saldo insufficiente} \\
		\hline
		\textbf{Attori} & Utente autenticato \\
		\textbf{Descrizione} & L'attore visualizza il pulsante per poter ricaricare il proprio saldo\\
		\textbf{Pre-Condizioni} & L'attore ha confermato l'acquisto per un API, ma non possiede un saldo sufficiente\\
		\textbf{Post-Condizioni} & L'attore visualizza un pulsante per essere reindirizzato alla pagina preposta alla ricarica dei crediti\\
		\textbf{Scenario Principale} & 
		\begin{enumerate*}[label=(\arabic*.),itemjoin={\newline}]
			\item L'attore può selezionare il pulsante ed essere reindirizzato alla pagina preposta alla ricarica dei crediti personali (UC12.2.2)
		\end{enumerate*}\\
	\end{tabular}
\end{minipage}

\paragraph{Caso d'uso UC9.5: Errore acquisto}
\label{UC9_5}

\begin{minipage}{\linewidth}
	\begin{tabular}{ l | p{11cm}}
		\hline
		\rowcolor{Gray}
		\multicolumn{2}{c}{UC9.5 - Errore acquisto} \\
		\hline
		\textbf{Attori} & Utente autenticato \\
		\textbf{Descrizione} & L'attore visualizza un errore relativo all'acquisto, non correlato al proprio saldo personale\\
		\textbf{Pre-Condizioni} & L'attore ha confermato l'acquisto per un API, si è verificato un errore\\
		\textbf{Post-Condizioni} & L'attore visualizza un errore relativo all'acquisto, con opportuna descrizione\\
		\textbf{Scenario Principale} & 
		\begin{enumerate*}[label=(\arabic*.),itemjoin={\newline}]
			\item L'attore può visualizzare un errore relativo all'acquisto, non inerente al proprio saldo, con opportuna descrizione
		\end{enumerate*}\\
	\end{tabular}
\end{minipage}

\newpage
\subsection{Caso d'uso UC10 - Visualizzazione API registrate}
\label{UC10}
\begin{figure}[ht]
	\centering
	\includegraphics[scale=0.45]{UML/UC10.png}
	\caption{UC10: Visualizzazione API registrate}
\end{figure}

\begin{longtable}{ l | p{11cm}}
	\hline
	\rowcolor{Gray}
	\multicolumn{2}{c}{UC10 - Visualizzazione API registrate}\\
	\hline
	
	 \textbf{Attori} & Utente autenticato  \\
	\textbf{Descrizione} & L'attore può visualizzare la schermata relativa alle API da lui registrate \\
	\textbf{Pre-Condizioni} & L'attore seleziona il link per visualizzare le API registrate  \\
	\textbf{Post-Condizioni} & L'attore visualizza la pagina relativa alle API da lui registrate\\
	\textbf{Scenario Principale} & 
	\begin{enumerate*}[label=(\arabic*.),itemjoin={\newline}]
		\item L'attore visualizza il numero di API registrate (UC10.1)
		\item L'attore visualizza la lista di API registrate (UC10.2)
	\end{enumerate*}\\
	\textbf{Scenari Alternativi} & 
	\begin{enumerate*}[label=(\arabic*.),itemjoin={\newline}]
		\item L'attore può visualizzare i dati relativi a una singola API (UC7)
	\end{enumerate*}\\
\end{longtable}


\subsubsection{Caso d'uso UC10.1: Visualizzazione numero API registrate}
\label{UC10_1}

\begin{minipage}{\linewidth}
	\begin{tabular}{ l | p{11cm}}
		\hline
		\rowcolor{Gray}
		\multicolumn{2}{c}{UC10.1 - Visualizzazione numero API registrate} \\
		\hline
		\textbf{Attori} & Utente autenticato \\
		\textbf{Descrizione} & L'attore visualizza il numero di API da lui registrate nella schermata relativa\\
		\textbf{Pre-Condizioni} & L'attore si trova nel menù relativo alle API da lui registrate\\
		\textbf{Post-Condizioni} & L'attore visualizza il numero di API registrate \\
		\textbf{Scenario Principale} & 
		\begin{enumerate*}[label=(\arabic*.),itemjoin={\newline}]
			\item L'attore può visualizzare il numero di API da lui registrate nella piattaforma
		\end{enumerate*}\\
	\end{tabular}
\end{minipage}

\subsubsection{Caso d'uso UC10.2: Visualizzazione lista API registrate}
\label{UC10_2}

\begin{minipage}{\linewidth}
	\begin{tabular}{ l | p{11cm}}
		\hline
		\rowcolor{Gray}
		\multicolumn{2}{c}{UC10.2 - Visualizzazione lista API registrate} \\
		\hline
		\textbf{Attori} & Utente autenticato \\
		\textbf{Descrizione} & L'attore visualizza la lista di API da lui registrate nella schermata relativa\\
		\textbf{Pre-Condizioni} & L'attore si trova nel menù relativo alle API da lui registrate\\
		\textbf{Post-Condizioni} & L'attore visualizza la lista di API registrate \\
		\textbf{Scenario Principale} & 
		\begin{enumerate*}[label=(\arabic*.),itemjoin={\newline}]
			\item L'attore visualizza il nome dell'API (10.2.1)
			\item L'attore visualizza il link alla pagina dell'API (10.2.2)
			\item L'attore visualizza il numero di licenze attive (10.2.3)
		\end{enumerate*}\\
	\end{tabular}
\end{minipage}

\paragraph{Caso d'uso UC10.2.1: Visualizzazione nome API}
\label{UC10_2_1}

\begin{minipage}{\linewidth}
	\begin{tabular}{ l | p{11cm}}
		\hline
		\rowcolor{Gray}
		\multicolumn{2}{c}{UC10.2.1 - Visualizzazione nome API} \\
		\hline
		\textbf{Attori} & Utente autenticato \\
		\textbf{Descrizione} & L'attore visualizza il nome dell'API\\
		\textbf{Pre-Condizioni} & L'attore visualizza la lista API nel menù relativo alle API da lui registrate\\
		\textbf{Post-Condizioni} & L'attore visualizza il nome dell'API nella lista \\
		\textbf{Scenario Principale} & 
		\begin{enumerate*}[label=(\arabic*.),itemjoin={\newline}]
			\item L'attore può visualizzare il nome dell'API
		\end{enumerate*}\\
	\end{tabular}
\end{minipage}

\paragraph{Caso d'uso UC10.2.2: Visualizzazione link API}
\label{UC10_2_2}

\begin{minipage}{\linewidth}
	\begin{tabular}{ l | p{11cm}}
		\hline
		\rowcolor{Gray}
		\multicolumn{2}{c}{UC10.2.2 - Visualizzazione link API} \\
		\hline
		\textbf{Attori} & Utente autenticato \\
		\textbf{Descrizione} & L'attore visualizza il link dell'API\\
		\textbf{Pre-Condizioni} & L'attore visualizza la lista API nel menù relativo alle API da lui registrate\\
		\textbf{Post-Condizioni} & L'attore visualizza il link dell'API nella lista \\
		\textbf{Scenario Principale} & 
		\begin{enumerate*}[label=(\arabic*.),itemjoin={\newline}]
			\item L'attore può visualizzare il link dell'API
		\end{enumerate*}\\
	\end{tabular}
\end{minipage}

\paragraph{Caso d'uso UC10.2.3: Visualizzazione numero licenze attive}
\label{UC10_2_3}

\begin{minipage}{\linewidth}
	\begin{tabular}{ l | p{11cm}}
		\hline
		\rowcolor{Gray}
		\multicolumn{2}{c}{UC10.2.3 - Visualizzazione scadenza licenza} \\
		\hline
		\textbf{Attori} & Utente autenticato \\
		\textbf{Descrizione} & L'attore visualizza il numero di licenze attive per ogni voce della lista\\
		\textbf{Pre-Condizioni} & L'attore visualizza la lista API nel menù relativo alle API da lui registrate\\
		\textbf{Post-Condizioni} & L'attore visualizza il numero delle API vendute e attualmente attive \\
		\textbf{Scenario Principale} & 
		\begin{enumerate*}[label=(\arabic*.),itemjoin={\newline}]
			\item L'attore può visualizzare il numero delle API vendute e attualmente attive
		\end{enumerate*}\\
	\end{tabular}
\end{minipage}

\input{includes/UseCase/UC50(UC11).tex}
%\newpage
\subsection{Caso d'uso UC7 - Visualizzazione API}
\label{UC7}
\begin{figure}[ht]
	\centering
	\includegraphics[scale=0.45]{UML/UC7.png}
	\caption{UC7: Visualizzazione API}
\end{figure}

\begin{longtable}{ l | p{11cm}}
	\hline
	\rowcolor{Gray}
	\multicolumn{2}{c}{UC7 - Visualizzazione API}\\
	\hline
	
	 \textbf{Attori} & Utente non autenticato, Utente autenticato, Amministratore API Market \\
	\textbf{Descrizione} & L'attore può visualizzare i dati relativi a un API che ha selezionato tramite la homepage o i risultati di una ricerca  \\
	\textbf{Pre-Condizioni} & L'attore ha selezionato un prodotto per la consultazione \\
	\textbf{Post-Condizioni} & L'attore visualizza la pagina relativa all'API selezionata\\
	\textbf{Scenario Principale} & 
	\begin{enumerate*}[label=(\arabic*.),itemjoin={\newline}]
		\item L'attore visualizza il nome dell'API (UC7.1)
		\item L'attore visualizza la descrizione dell'API (UC7.2)
		\item L'attore visualizza l'autore dell'API (UC7.3)
		\item L'attore può visualizzare l'interfaccia dell'API (UC7.4)
		\item L'attore può consultare la documentazione fornita dall'utente (UC7.5)
		\item L'attore può visualizzare i dati di utilizzo dell'API  (UC7.6)
	\end{enumerate*}\\
\end{longtable}

%\newpage
\subsubsection{UC7.1 - Gestione Proprio Profilo Utente}
\label{UC7.1}

\begin{figure}[ht]
	\centering
	\includegraphics[scale=0.45]{UML/UC7_1.png}
	\caption{UC7.1 - Gestione Proprio Profilo Utente}
\end{figure}
\FloatBarrier
\begin{longtable}{ l | p{11cm}}
	\hline
	\rowcolor{Gray}
	 \multicolumn{2}{c}{UC7.1 - Gestione Proprio Profilo Utente} \\
	 \hline
	 \textbf{Attori} & Utente autenticato  \\
	\textbf{Descrizione} & L’utente pu\'{o} modificare le proprie informazioni personali  \\
	\textbf{Pre-Condizioni} & L’Utente \`{e} nel proprio profilo \\
	\textbf{Post-Condizioni} & L’Utente ha aggiornato il proprio profilo \\
	\textbf{Scenario Principale} & 
	\begin{enumerate*}[label=(\arabic*.),itemjoin={\newline}]
		\item L'utente pu\`{o} modificare il Nome (UC7.1.1)
		\item L'utente pu\`{o} modificare il Cognome (UC7.1.2)
		\item L'utente pu\`{o} modificare l'Username (UC7.1.3)
		\item L'utente pu\`{o} modificare la Foto Utente (UC7.1.4)
		\item L'utente pu\`{o} modificare l'Email (UC7.1.5)
		\item L'utente pu\`{o} modificare la Password (UC7.1.6)
		\item L'utente pu\`{o} confermare le Proprie Modifiche (UC7.1.7)
		\item L'utente pu\`{o} eliminare il Proprio Account (UC7.1.9)
	\end{enumerate*}\\
	\textbf{Scenari Alternativi} & 
		\begin{enumerate*}[label=(\arabic*.),itemjoin={\newline}]
		\item L'utente visualizza un Errore di Modifica Account se le modifiche risultano illegali (UC7.1.8)
		\item L'utente visualizza un Errore di Modifica Account se risulta un errore durante eliminazione (UC7.1.10)
	\end{enumerate*}\\
\end{longtable}




%\newpage
\paragraph{Caso d'uso UC7.1.1:  Modifica Nome}
\label{UC7_1_1}
\begin{figure}[ht]
	\centering
	\includegraphics[scale=0.45]{UML/UC7_1_1.png}
	\caption{UC7.1.1:  Modifica Nome}
\end{figure}
\FloatBarrier
\begin{tabular}{ l | p{11cm}}
	\hline
	\rowcolor{Gray}
	 \multicolumn{2}{c}{UC7.1.1 - Modifica Nome} \\
	 \hline
	\textbf{Attori} & Utente Autenticato \\
	\textbf{Descrizione} & Gli utenti possono modificare il proprio Nome\\
	\textbf{Pre-Condizioni} & L'utente e' nella schermata di gestione Profilo\\
	\textbf{Post-Condizioni} & L'utente ha modificato con successo il proprio Nome \\
	\textbf{Scenario Principale} & 
	\begin{enumerate*}[label=(\arabic*.),itemjoin={\newline}]
		\item L'utente puo' inserire il Nuovo Nome (UC7.1.1.1)
		\item L'utente conferma la modifica al Nome (UC7.1.1.2)
	\end{enumerate*}\\
	\textbf{Scenari Alternativi} & 
	\begin{enumerate*}[label=(\arabic*.),itemjoin={\newline}]
		\item L'utente visualizza un errore nella modifica del Nome (UC7.1.1.3)
	\end{enumerate*}\\
\end{tabular}

%\subparagraph{Caso d'uso UC7.1.1.1:  Inserimento Nuovo Nome}
\label{UC7_1_1_1}

\begin{tabular}{ l | p{11cm}}
	\hline
	\rowcolor{Gray}
	 \multicolumn{2}{c}{UC7.1.1.1 - Inserimento Nuovo Nome} \\
	 \hline
	\textbf{Attori} & Utente Autenticato \\
	\textbf{Descrizione} & L'utente inserisce un nuovo Nome\\
	\textbf{Pre-Condizioni} & L'utente e' nella processo di modifica del Nome\\
	\textbf{Post-Condizioni} & L'utente ha scritto un nuovo Nome nell'editor\\
	\textbf{Scenario Principale} & 
	\begin{enumerate*}[label=(\arabic*.),itemjoin={\newline}]
		\item L'utente scrive un nuovo nome 
	\end{enumerate*}\\
\end{tabular}
%\subparagraph{Caso d'uso UC7.1.1.2:  Conferma Modifica Nome}
\label{UC7_1_1_2}

\begin{tabular}{ l | p{11cm}}
	\hline
	\rowcolor{Gray}
	 \multicolumn{2}{c}{UC7.1.1.2 - Conferma Modifica Nome} \\
	 \hline
	\textbf{Attori} & Utente Autenticato \\
	\textbf{Descrizione} & L'utente conferma il Nuovo Nome inserito\\
	\textbf{Pre-Condizioni} & L'utente ha scritto il proprio nome in Input\\
	\textbf{Post-Condizioni} & L'utente ha modificato il proprio Nome\\
	\textbf{Scenario Principale} & 
	\begin{enumerate*}[label=(\arabic*.),itemjoin={\newline}]
		\item L'utente conferma il nuovo Nome inserito
	\end{enumerate*}\\
	\textbf{Scenari Alternativi} & 
	\begin{enumerate*}[label=(\arabic*.),itemjoin={\newline}]
		\item L'utente scritto contiene simboli particolari
	\end{enumerate*}\\
\end{tabular}
%\paragraph{Caso d'uso UC7.1.2:  Modifica Cognome}
\label{UC7_1_2}
\begin{figure}[ht]
	\centering
	\includegraphics[scale=0.45]{UML/UC7_1_2.png}
	\caption{UC7.1.2:  Modifica Cognome}
\end{figure}

\FloatBarrier
\begin{tabular}{ l | p{11cm}}
	\hline
	\rowcolor{Gray}
	 \multicolumn{2}{c}{UC7.1.2 - Modifica Cognome} \\
	 \hline
	\textbf{Attori} & Utente Autenticato \\
	\textbf{Descrizione} & Gli utenti possono modificare il proprio Cognome\\
	\textbf{Pre-Condizioni} & L'utente e' nella schermata di gestione Profilo\\
	\textbf{Post-Condizioni} & L'utente ha modificato con successo il proprio Cognome \\
	\textbf{Scenario Principale} & 
	\begin{enumerate*}[label=(\arabic*.),itemjoin={\newline}]
		\item L'utente puo' inserire il Nuovo Cognome (UC7.1.2.1)
		\item L'utente conferma la modifica al Cognome (UC7.1.2.2)
	\end{enumerate*}\\
	\textbf{Scenari Alternativi} & 
	\begin{enumerate*}[label=(\arabic*.),itemjoin={\newline}]
		\item L'utente visualizza un errore nella modifica del Cognome (UC7.1.2.3)
	\end{enumerate*}\\
\end{tabular}

%\subparagraph{Caso d'uso UC7.1.2.1:  Inserimento Nuovo Cognome}
\label{UC7_1_2_1}

\begin{tabular}{ l | p{11cm}}
	\hline
	\rowcolor{Gray}
	 \multicolumn{2}{c}{UC7.1.2.1 - Inserimento Nuovo Cognome} \\
	 \hline
	\textbf{Attori} & Utente Autenticato \\
	\textbf{Descrizione} & L'utente inserisce un nuovo Cognome in Input\\
	\textbf{Pre-Condizioni} & L'utente e' nella processo di modifica del Cognome\\
	\textbf{Post-Condizioni} & L'utente ha scritto un nuovo Nome nell'editor\\
	\textbf{Scenario Principale} & 
	\begin{enumerate*}[label=(\arabic*.),itemjoin={\newline}]
		\item L'utente scrive un nuovo Cognome (UC7.1.1.1)
	\end{enumerate*}\\
\end{tabular}
%\subparagraph{Caso d'uso UC7.1.2.2:  Conferma Modifica Cognome}
\label{UC7_1_2_2}

\begin{tabular}{ l | p{11cm}}
	\hline
	\rowcolor{Gray}
	 \multicolumn{2}{c}{UC7.1.2.2:  Conferma Modifica Cognome} \\
	 \hline
	\textbf{Attori} & Utente Autenticato \\
	\textbf{Descrizione} & L'utente inserisce un nuovo nome\\
	\textbf{Pre-Condizioni} & L'utente e' nella schermata di gestione Profilo\\
	\textbf{Post-Condizioni} & L'utente ha scritto un nuovo Cognome nell'editor\\
	\textbf{Scenario Principale} & 
	\begin{enumerate*}[label=(\arabic*.),itemjoin={\newline}]
		\item L'utente conferma il nuovo Cognome (UC7.1.1.1)
	\end{enumerate*}\\
\end{tabular}
%\paragraph{Caso d'uso UC7.1.3:  Modifica Username}
\label{UC7_1_3}
\begin{figure}[ht]
	\centering
	\includegraphics[scale=0.45]{UML/UC7_1_3.png}
	\caption{UC7.1.3:  Modifica Username}
\end{figure}
\FloatBarrier
\begin{tabular}{ l | p{11cm}}
	\hline
	\rowcolor{Gray}
	 \multicolumn{2}{c}{UC7.1.3 - Modifica Username} \\
	 \hline
		\textbf{Attori} & Utente Autenticato \\
	\textbf{Descrizione} & Gli utenti possono modificare il proprio Username\\
	\textbf{Pre-Condizioni} & L'utente e' nella schermata di gestione Profilo\\
	\textbf{Post-Condizioni} & L'utente ha modificato con successo il proprio Username \\
	\textbf{Scenario Principale} & 
	\begin{enumerate*}[label=(\arabic*.),itemjoin={\newline}]
		\item L'utente puo' inserire il Nuovo Username (UC7.1.3.1)
		\item L'utente conferma la modifica al Username (UC7.1.3.2)
	\end{enumerate*}\\
	\textbf{Scenari Alternativi} & 
	\begin{enumerate*}[label=(\arabic*.),itemjoin={\newline}]
		\item L'utente visualizza un errore nella modifica del Username (UC7.1.3.3)
	\end{enumerate*}\\
\end{tabular}

%\subparagraph{Caso d'uso UC7.1.3.1:  Inserimento Nuovo Username}
\label{UC7_1_3_1}

\begin{tabular}{ l | p{11cm}}
	\hline
	\rowcolor{Gray}
	 \multicolumn{2}{c}{UC7.1.3.1:  Inserimento Nuovo Username} \\
	 \hline
	\textbf{Attori} & Utente Autenticato \\
	\textbf{Descrizione} & L'utente inserisce un nuovo Username\\
	\textbf{Pre-Condizioni} & L'utente e' nella processo di modifica del Username\\
	\textbf{Post-Condizioni} & L'utente ha scritto un nuovo Username nell'editor\\
	\textbf{Scenario Principale} & 
	\begin{enumerate*}[label=(\arabic*.),itemjoin={\newline}]
		\item L'utente scrive un nuovo Username
	\end{enumerate*}\\
\end{tabular}
%\subparagraph{Caso d'uso UC7.1.3.2:  Conferma Modifica Username}
\label{UC7_1_3_2}

\begin{tabular}{ l | p{11cm}}
	\hline
	\rowcolor{Gray}
	 \multicolumn{2}{c}{UC7.1.3.2 - Conferma Modifica Username} \\
	 \hline
	\textbf{Attori} & Utente Autenticato \\
	\textbf{Descrizione} & L'utente conferma l'username\\
	\textbf{Pre-Condizioni} & L'utente ha scritto un nuovo username nell'editor\\
	\textbf{Post-Condizioni} & L'utente ha modificato l'username\\
	\textbf{Scenario Principale} & 
	\begin{enumerate*}[label=(\arabic*.),itemjoin={\newline}]
		\item L'utente conferma il nuovo username appena scritto
	\end{enumerate*}\\
\end{tabular}
%\paragraph{Caso d'uso UC7.1.4:  Modifica Foto Utente}
\label{UC7_1_4}
\begin{figure}[ht]
	\centering
	\includegraphics[scale=0.45]{UML/UC7_1_4.png}
	\caption{UC4: Login}
\end{figure}
\FloatBarrier
\begin{tabular}{ l | p{11cm}}
	\hline
	\rowcolor{Gray}
	 \multicolumn{2}{c}{UC7.1.4 - Modifica Foto Utente} \\
	 \hline
		\textbf{Attori} & Utente Autenticato \\
	\textbf{Descrizione} & Gli utenti possono modificare la propria Foto Profilo Utente selezionandola dal loro dispositivo\\
	\textbf{Pre-Condizioni} & L'utente e' nella schermata di gestione Profilo\\
	\textbf{Post-Condizioni} & L'utente ha modificato con successo la propria Foto Profilo \\
	\textbf{Scenario Principale} & 
	\begin{enumerate*}[label=(\arabic*.),itemjoin={\newline}]
		\item L'utente puo' inserire selezionare una Nuova Foto (UC7.1.4.1)
		\item L'utente conferma la modifica alla Foto Profilo (UC7.1.2.2)
	\end{enumerate*}\\
	\textbf{Scenari Alternativi} & 
	\begin{enumerate*}[label=(\arabic*.),itemjoin={\newline}]
		\item L'utente visualizza un errore nella modifica della Foto (UC7.1.2.3)
	\end{enumerate*}\\
\end{tabular}

%\subparagraph{Caso d'uso UC7.1.4.1:  Seleziona Nuova Foto Utente}
\label{UC7_1_4_1}

\begin{tabular}{ l | p{11cm}}
	\hline
	\rowcolor{Gray}
	 \multicolumn{2}{c}{UC7.1.4.1:  Seleziona Nuova Foto Utente} \\
	 \hline
	\textbf{Attori} & Utente Autenticato \\
	\textbf{Descrizione} & L'attore sceglie una Nuova Foto Utente localmente\\
	\textbf{Pre-Condizioni} & L'utente e' nella processo di modifica della Foto Utente\\
	\textbf{Post-Condizioni} & L'utente ha scelto una Nuova Foto Utente\\
	\textbf{Scenario Principale} & 
	\begin{enumerate*}[label=(\arabic*.),itemjoin={\newline}]
		\item L'utente sceglie una foto 
	\end{enumerate*}\\
\end{tabular}
%\subparagraph{Caso d'uso UC7.1.4.2:  Conferma Modifica Foto Utente}
\label{UC7_1_4_2}

\begin{tabular}{ l | p{11cm}}
	\hline
	\rowcolor{Gray}
	 \multicolumn{2}{c}{UC7.1.4.2:  Conferma Modifica Foto Utente} \\
	 \hline
	\textbf{Attori} & Utente Autenticato \\
	\textbf{Descrizione} & L'utente conferma la modifica alla foto\\
	\textbf{Pre-Condizioni} & L'utente ha scelto una foto utente\\
	\textbf{Post-Condizioni} & L'utente ha modificato con successo la propria foto\\
	\textbf{Scenario Principale} & 
	\begin{enumerate*}[label=(\arabic*.),itemjoin={\newline}]
		\item L'utente conferma la modifica della foto
	\end{enumerate*}\\
\end{tabular}
%\paragraph{Caso d'uso UC7.1.5:  Modifica Email}
\label{UC7_1_5}
\begin{figure}[ht]
	\centering
	\includegraphics[scale=0.45]{UML/UC7_1_5.png}
	\caption{UC7.1.5:  Modifica Email}
\end{figure}
\FloatBarrier
\begin{tabular}{ l | p{11cm}}
	\hline
	\rowcolor{Gray}
	 \multicolumn{2}{c}{UC7.1.5 - Modifica Email} \\
	 \hline
		\textbf{Attori} & Utente Autenticato \\
	\textbf{Descrizione} & Gli utenti possono modificare la propria Email\\
	\textbf{Pre-Condizioni} & L'utente e' nella schermata di gestione Profilo\\
	\textbf{Post-Condizioni} & L'utente ha modificato con successo la Propria Email\\
	\textbf{Scenario Principale} & 
	\begin{enumerate*}[label=(\arabic*.),itemjoin={\newline}]
		\item L'utente puo' inserire la nuova Email (UC7.1.5.1)
		\item L'utente conferma la modifica al'Email (UC7.1.5.2)
	\end{enumerate*}\\
	\textbf{Scenari Alternativi} & 
	\begin{enumerate*}[label=(\arabic*.),itemjoin={\newline}]
		\item L'utente visualizza un errore nella modifica dell'Email (UC7.1.5.3)
	\end{enumerate*}\\
\end{tabular}


%\subparagraph{Caso d'uso UC7.1.5.1:  Inserimento Nuova Email}
\label{UC7_1_5_1}

\begin{tabular}{ l | p{11cm}}
	\hline
	\rowcolor{Gray}
	 \multicolumn{2}{c}{UC7.1.5.1 - Inserimento Nuova Email} \\
	 \hline
	\textbf{Attori} & Utente Autenticato \\
	\textbf{Descrizione} & L'utente inserisce un nuova Email nell'Editor\\
	\textbf{Pre-Condizioni} & L'utente e' nella Editor di modifica dell'Email\\
	\textbf{Post-Condizioni} & L'utente ha scritto una nuova Email nell'Editor\\
	\textbf{Scenario Principale} & 
	\begin{enumerate*}[label=(\arabic*.),itemjoin={\newline}]
		\item L'utente scrive una nuova Email
	\end{enumerate*}\\
\end{tabular}
%\subparagraph{Caso d'uso UC7.1.5.2:  Conferma Modifica Email}
\label{UC7_1_5_2}

\begin{tabular}{ l | p{11cm}}
	\hline
	\rowcolor{Gray}
	 \multicolumn{2}{c}{UC7.1.5.2:  Conferma Modifica Email} \\
	 \hline
	\textbf{Attori} & Utente Autenticato \\
	\textbf{Descrizione} & L'utente inserisce un nuovo nome\\
	\textbf{Pre-Condizioni} & L'utente e' nella schermata di gestione Profilo\\
	\textbf{Post-Condizioni} & L'utente ha scritto un nuovo Nome nell'editor\\
	\textbf{Scenario Principale} & 
	\begin{enumerate*}[label=(\arabic*.),itemjoin={\newline}]
		\item L'utente scrive un nuovo nome (UC7.1.1.1)
	\end{enumerate*}\\
\end{tabular}
%\paragraph{Caso d'uso UC7.1.6:  Modifica Password}
\label{UC7_1_6}
\begin{figure}[ht]
	\centering
	\includegraphics[scale=0.45]{UML/UC7_1_6.png}
	\caption{UC7.1.6:  Modifica Password}
\end{figure}
\FloatBarrier
\begin{tabular}{ l | p{11cm}}
	\hline
	\rowcolor{Gray}
	 \multicolumn{2}{c}{UC7.1.6 - Modifica Password} \\
	 \hline
		\textbf{Attori} & Utente Autenticato \\
	\textbf{Descrizione} & Gli utenti possono modificare il proprio Cognome\\
	\textbf{Pre-Condizioni} & L'utente e' nella schermata di gestione Profilo\\
	\textbf{Post-Condizioni} & L'utente ha modificato con successo il proprio Cognome \\
	\textbf{Scenario Principale} & 
	\begin{enumerate*}[label=(\arabic*.),itemjoin={\newline}]
		\item L'utente deve inserire la Vecchia Password (UC7.1.6.1)
		\item L'utente deve inserire la Nuova Password (UC7.1.6.2)
		\item L'utente deve re-inserire la Nuova Password (UC7.1.6.3)
		\item L'utente conferma la modifica alla Password(UC7.1.6.4)
	\end{enumerate*}\\
	\textbf{Scenari Alternativi} & 
	\begin{enumerate*}[label=(\arabic*.),itemjoin={\newline}]
		\item L'utente visualizza un errore nella modifica della Password (UC7.1.6.5)
	\end{enumerate*}\\
\end{tabular}

%\subparagraph{Caso d'uso UC7.1.6.1:  Inserimento Vecchia Password}
\label{UC7_1_6_1}

\begin{tabular}{ l | p{11cm}}
	\hline
	\rowcolor{Gray}
	 \multicolumn{2}{c}{UC7.1.6.1:  Inserimento Vecchia Password} \\
	 \hline
	\textbf{Attori} & Utente Autenticato \\
	\textbf{Descrizione} & L'utente inserisce la vecchia Password in Input nell'Editor\\
	\textbf{Pre-Condizioni} & L'utente e' nella schermata di modifica Password\\
	\textbf{Post-Condizioni} & L'utente ha scritto la Vecchia Password nell'editor\\
	\textbf{Scenario Principale} & 
	\begin{enumerate*}[label=(\arabic*.),itemjoin={\newline}]
		\item L'utente scrive la vecchia Password
	\end{enumerate*}\\
\end{tabular}
%\subparagraph{Caso d'uso UC7.1.6.2:  Inserimento Nuova Password}
\label{UC7_1_6_2}

\begin{tabular}{ l | p{11cm}}
	\hline
	\rowcolor{Gray}
	 \multicolumn{2}{c}{UC7.1.6.2 - Inserimento Nuova Password} \\
	 \hline
	\textbf{Attori} & Utente Autenticato \\
	\textbf{Descrizione} & L'utente inserisce una Nuova Password\\
	\textbf{Pre-Condizioni} & L'utente e' nella schermata di modifica Password\\
	\textbf{Post-Condizioni} & L'utente ha scritto una nuova Password nell'editor\\
	\textbf{Scenario Principale} & 
	\begin{enumerate*}[label=(\arabic*.),itemjoin={\newline}]
		\item L'utente scrive una Nuova Password
	\end{enumerate*}\\
\end{tabular}
%\subparagraph{Caso d'uso UC7.1.6.3:  Re-Inserimento Nuovo Password}
\label{UC7_1_6_3}

\begin{tabular}{ l | p{11cm}}
	\hline
	\rowcolor{Gray}
	 \multicolumn{2}{c}{UC7.1.6.3 - Re-Inserimento Nuova Password} \\
	 \hline
	\textbf{Attori} & Utente Autenticato \\
	\textbf{Descrizione} & L'utente inserisce una Nuova Password\\
	\textbf{Pre-Condizioni} & L'utente e' nella schermata di gestione Profilo\\
	\textbf{Post-Condizioni} & L'utente ha scritto un nuovo Nome nell'editor\\
	\textbf{Scenario Principale} & 
	\begin{enumerate*}[label=(\arabic*.),itemjoin={\newline}]
		\item L'utente conferma l'inserimento della Nuova PassWord
	\end{enumerate*}\\
\end{tabular}
%\subparagraph{Caso d'uso UC7.1.6.4:  Conferma Modifica Password}
\label{UC7_1_6_4}

\begin{tabular}{ l | p{11cm}}
	\hline
	\rowcolor{Gray}
	 \multicolumn{2}{c}{UC7.1.6.4 - Conferma Modifica Password} \\
	 \hline
	\textbf{Attori} & Utente Autenticato \\
	\textbf{Descrizione} & L'utente inserisce conferma la password nuova\\
	\textbf{Pre-Condizioni} & L'utente e' nella schermata di modifica Password\\
	\textbf{Post-Condizioni} & L'utente ha modificato la Password\\
	\textbf{Scenario Principale} & 
	\begin{enumerate*}[label=(\arabic*.),itemjoin={\newline}]
		\item L'utente conferma la modifica della Password
	\end{enumerate*}\\
\end{tabular}
%\paragraph{Caso d'uso UC7.1.7:  Conferma Modifiche}
\label{UC7_1_7}

\begin{tabular}{ l | p{11cm}}
	\hline
	\rowcolor{Gray}
	 \multicolumn{2}{c}{UC7.1.7 - Conferma Modifiche} \\
	 \hline
	\textbf{Attori} & Utente Autenticato \\
	\textbf{Descrizione} & Gli utenti confermano e rendono permanenti le modifiche al proprio Profilo Utente\\
	\textbf{Pre-Condizioni} & L'utente si trova nella schermata di gestione Profilo\\
	\textbf{Post-Condizioni} & L'utente ha reso permanenti le modifiche al suo Profilo\\
	\textbf{Scenario Principale} & 
	\begin{enumerate*}[label=(\arabic*.),itemjoin={\newline}]
		\item L'utente conferma le modifiche al proprio Profilo (UC7.1.7)
	\end{enumerate*}\\
\end{tabular}
%\paragraph{Caso d'uso UC7.1.8:  Visualizza Errore Modifica Account}
\label{UC7_1_8}

\begin{tabular}{ l | p{11cm}}
	\hline
	\rowcolor{Gray}
	 \multicolumn{2}{c}{UC7.1.8 - Visualizza Errore Eliminazione Account} \\
	 \hline
	\textbf{Attori} & Utente Autenticato \\
	\textbf{Descrizione} & L'utente ha confermato le modifiche al proprio Profilo ma c'e' stato un errore e lo visualizza a schermo con dettagli\\
	\textbf{Pre-Condizioni} & L'utente ha confermato le modifiche al proprio account\\
	\textbf{Post-Condizioni} & L'utente si trova nella schermata di Gestione Account e puo' scegliere se rifare le modifiche \\
	\textbf{Scenario Principale} & 
	\begin{enumerate*}[label=(\arabic*.),itemjoin={\newline}]
		\item L'utente visualizza un errore delle modifche all'Account (UC7.1.8)
	\end{enumerate*}\\
\end{tabular}
%\paragraph{Caso d'uso UC7.1.9:  Eliminazione Proprio Account}
\label{UC7_1_9}

\begin{tabular}{ l | p{11cm}}
	\hline
	\rowcolor{Gray}
	 \multicolumn{2}{c}{UC7.1.9 - Eliminazione Proprio Account} \\
	 \hline
	\textbf{Attori} & Utente Autenticato \\
	\textbf{Descrizione} & Gli utenti possono eliminare il proprio Account\\
	\textbf{Pre-Condizioni} & L'utente si trova nella schermata di gestione del proprio Profilo\\
	\textbf{Post-Condizioni} & L'utente ha eliminato il proprio Account \\
	\textbf{Scenario Principale} & 
	\begin{enumerate*}[label=(\arabic*.),itemjoin={\newline}]
		\item L'utente puo' eliminare il proprio Account (UC7.1.9)
	\end{enumerate*}\\
\end{tabular}
%\paragraph{Caso d'uso UC7.1.10:  Visualizza Errore Eliminazione Account}
\label{UC7_1_10}

\begin{tabular}{ l | p{11cm}}
	\hline
	\rowcolor{Gray}
	 \multicolumn{2}{c}{UC7.1.10 - Visualizza Errore Eliminazione Account} \\
	 \hline
	\textbf{Attori} & Utente Autenticato \\
	\textbf{Descrizione} & L'utente ha provato a elimare il proprio Account ma c'e' stata un'illegalita'. Ad esempio l'utente e' in saldo negativo.\\
	\textbf{Pre-Condizioni} & L'utente ha provato a elimare il proprio Account\\
	\textbf{Post-Condizioni} & L'utente visualizza una schermata con i motivi della negazione d'elimazione account\\
	\textbf{Scenario Principale} & 
	\begin{enumerate*}[label=(\arabic*.),itemjoin={\newline}]
		\item L'utente visualizza un errore di Eliminazione Account (UC7.1.9)
	\end{enumerate*}\\
\end{tabular}
%\newpage
\subsubsection{UC7.2 - Visualizzazione Proprio Profilo Utente}
\label{UC7.2}

\begin{figure}[ht]
	\centering
	\includegraphics[scale=0.45]{UML/UC7_2.png}
	\caption{UC7.2 - Visualizzazione Proprio Profilo Utente}
\end{figure}
\FloatBarrier
\begin{longtable}{ l | p{11cm}}
	\hline
	\rowcolor{Gray}
	 \multicolumn{2}{c}{UC7.2 - Visualizzazione Proprio Profilo Utente} \\
	 \hline
	 \textbf{Attori} & Utente autenticato \\
	\textbf{Descrizione} & L’utente pu\'{o} visualizzare le proprie informazioni personali, interagire con le API da lui registrate e con le API acquistate  \\
	\textbf{Pre-Condizioni} & L’Utente \`{e} nel proprio profilo \\
	\textbf{Post-Condizioni} & L’Utente ha scelto l'interazione con il proprio profilo \\
	\textbf{Scenario Principale} & 
	\begin{enumerate*}[label=(\arabic*.),itemjoin={\newline}]
		\item L'utente pu\`{o} interagire con le API acquistate (UC9)
		\item L'utente pu\`{o} interagire con le API registrate (UC10)
		\item L'utente pu\`{o} visualizzare le Informazioni personali (UC7.2.3)
	\end{enumerate*}\\
\end{longtable}



%\paragraph{Caso d'uso UC7.2.3:  Visualizza Informazioni Personali}
\label{UC7_2_3}
\begin{figure}[ht]
	\centering
	\includegraphics[scale=0.45]{UML/UC7_2_3.png}
	\caption{UC7.2.3: Visualizza Informazioni Personali}
\end{figure}

\FloatBarrier
\begin{tabular}{ l | p{11cm}}
	\hline
	\rowcolor{Gray}
	 \multicolumn{2}{c}{UC7.2.3:  VIsualizza Informazioni Personali} \\
	 \hline
	\textbf{Attori} & Utente Autenticato \\
	\textbf{Descrizione} & Gli utenti visualizzano le informazioni del proprio Profilo Utente\\
	\textbf{Pre-Condizioni} & L'utente e' nella schermata di gestione Profilo\\
	\textbf{Post-Condizioni} & L'utente ha modificato con successo il proprio Cognome \\
	\textbf{Scenario Principale} & 
	\begin{enumerate*}[label=(\arabic*.),itemjoin={\newline}]
		\item L'utente puo' Visualizzare il Nome (UC7.1.3.1)
		\item L'utente puo' Visualizzare il Cognome (UC7.1.3.2)
		\item L'utente puo' Visualizzare l'Username (UC7.1.3.3)
		\item L'utente puo' Visualizzare la Foto Utente (UC7.1.3.4)
		\item L'utente puo' Visualizzare il suo Credito Disponibile (UC7.1.3.4)
	\end{enumerate*}\\
\end{tabular}

%\newpage
\subsection{Caso d'uso UC8 - Interazione Con API non acquistate}
\label{UC8}
\begin{figure}[ht]
	\centering
	\includegraphics[scale=0.45]{UML/UC8.png}
	\caption{UC8 - Interazione Con API non acquistate}
\end{figure}

\begin{longtable}{ l | p{11cm}}
	\hline
	\rowcolor{Gray}
	\multicolumn{2}{c}{UC8 - Interazione Con API non acquistate}\\
	\hline
	
	 \textbf{Attori} & Utente autenticato  \\
	\textbf{Descrizione} & L'utente puo' interagire con le API non aquistate in vari modi. Visualizza una lista di API e puo' filtrarle, puo' consultarne la documentazione di ognuna, puo' acquistarle \\
	\textbf{Pre-Condizioni} & L'utente e' nella schermata di interazione con le API\\
	\textbf{Post-Condizioni} & L'utente ha scelto l'interazione con le API\\
	\textbf{Scenario Principale} & 
	\begin{enumerate*}[label=(\arabic*.),itemjoin={\newline}]
		\item L'utente puo' cercare una API (UC6)
		\item L'utente puo' cercare una API (UC8.1)
		\item L'utente puo' cercare una API (UC8.2)
	\end{enumerate*}\\
\end{longtable}


%\subsubsection{Caso d'uso UC8.1 - Consultazione Documentazione API}
\label{UC8.1}
\begin{figure}[ht]
	\centering
	\includegraphics[scale=0.45]{UML/UC8_1.png}
	\caption{UC8.1 - Consultazione Documentazione API}
\end{figure}
\FloatBarrier
\begin{longtable}{ l | p{11cm}}
	\hline
	\rowcolor{Gray}
	\multicolumn{2}{c}{UC8.1 - Consultazione Documentazione API}\\
	\hline
	
	 \textbf{Attori} & Utente autenticato  \\
	\textbf{Descrizione} & L'utente puo' consultare la Documentazione delle API \\
	\textbf{Pre-Condizioni} & L'utente e' autenticato in APIMarket e ha scelto una API dalla lista API\\
	\textbf{Post-Condizioni} & L'utente ha visualizzato la documetazione e ora puo' scegliere un'altra interazione\\
	\textbf{Scenario Principale} & 
	\begin{enumerate*}[label=(\arabic*.),itemjoin={\newline}]
		\item L'utente puo' Consultare la versione PDF della Documentazione (UC8.1.1)
		\item L'utente puo' Consultare la versione Web della Documentazione (UC8.1.2)
	\end{enumerate*}\\
\end{longtable}


%\input{includes/UseCase/UC8_1_1.tex}
%\newpage
\paragraph{Caso d'uso UC8.1.2: Consultazione Versione Web}
\label{UC8.1.2}

\renewcommand*{\arraystretch}{1.6}
\begin{longtable}{ l | p{11cm}}
	\hline
	\rowcolor{Gray}
	\multicolumn{2}{c}{UC8.1.2: Consultazione Versione Web} \\
	\hline
	\textbf{Attori} &Utente Autenticato, Amministratore APIMarket \\
	\textbf{Descrizione} & l'attore visualizza la documentazione dell'API \\
	\textbf{Pre-Condizioni} &  l'attore ha scelto di consultare la documentazione di un'API\\
	\textbf{Post-Condizioni}& l'attore vede la versione Web della documentazione\\
	\textbf{Scenario Principale} & \begin{enumerate*}[label=(\arabic*.),itemjoin={\newline}]
		\item L'attore vede la documentazione in versione Web
	\end{enumerate*}\\
\end{longtable}

%
\subsubsection{Caso d'uso UC8.2 - Procedura di Acquisto API}
\label{UC8.2}
\begin{figure}[ht]
	\centering
	\includegraphics[scale=0.45]{UML/UC8_2.png}
	\caption{UC8.2 - Procedura di Acquisto API}
\end{figure}
\FloatBarrier
\begin{longtable}{ l | p{11cm}}
	\hline
	\rowcolor{Gray}
	\multicolumn{2}{c}{UC8.2 - Procedura di Acquisto API}\\
	\hline
	 \textbf{Attori} & Utente autenticat, Amministratore APIMarket, Sistema \\
	\textbf{Descrizione} & L'utente puo' comprare un'API e puo' farlo in diversi modi, anche secondo una procedura esterna (un esempio di procedura esterna e' Paypal)\\
	\textbf{Pre-Condizioni} & L'utente ha selezionato un'API\\
	\textbf{Post-Condizioni} & L'utente ha acquistato l'API e ricevuto la chiave API associata\\
	\textbf{Scenario Principale} & 
	\begin{enumerate*}[label=(\arabic*.),itemjoin={\newline}]
		\item L'utente puo' scegliere il Piano di Acquisto(UC8.2.1)
		\item L'utente puo' scegliere la modalita' di Acquisto (UC8.2.2)
		\item L'utente puo' confermare l'Acquisto (UC8.2.3)
		\item L'utente puo' Visualizzare l'Errore di Acquisto (UC8.2.4)
		\item L'utente puo' confermare l'Acquisto Avvenuto (UC8.2.5)
		\item Il sistema puo' eseguire l'Acquisto usando una procedura d'Acquisto esterna (UC8.2.6)
		\item L'utente puo' ricevere una Chiave API dopo l'Acquisto (UC8.2.7)
	\end{enumerate*}\\
	\textbf{Scenari Alternativi} & 
	\begin{enumerate*}[label=(\arabic*.),itemjoin={\newline}]
		\item L'utente puo' Visualizzare l'Errore di Acquisto (UC8.2.4)
	\end{enumerate*}\\
\end{longtable}

%\paragraph{Caso d'uso UC8.2.1: Scelta Piano di Acquisto}
\label{UC8.2.1}

\renewcommand*{\arraystretch}{1.6}
\begin{longtable}{ l | p{11cm}}
	\hline
	\rowcolor{Gray}
	\multicolumn{2}{c}{UC8.2.1: Scelta Piano di Acquisto} \\
	\hline
	\textbf{Attori} &Utente Autenticato, Amministratore APIMarket\\
	\textbf{Descrizione} & ll'Attore deve scelgliere un Piano d'Acquisto da un insieme di Piani d'Acquisto\\
	\textbf{Pre-Condizioni} &  L'attore ha avviato la procedura di Acquisto\\
	\textbf{Post-Condizioni}& L'attore ha scelto un piano di Acquisto\\
	\textbf{Scenario Principale} & \begin{enumerate*}[label=(\arabic*.),itemjoin={\newline}]
		\item L'attore ha una lista di Piani D'Acquisto e ne sceglie uno
	\end{enumerate*}\\
\end{longtable}

%\paragraph{Caso d'uso UC8.2.2: Scelta' Modalita' di Acquisto}
\label{UC8.2.2}

\renewcommand*{\arraystretch}{1.6}
\begin{longtable}{ l | p{11cm}}
	\hline
	\rowcolor{Gray}
	\multicolumn{2}{c}{UC8.2.2: Scelta' Modalita' di Acquisto} \\
	\hline
	\textbf{Attori} &Utente Autenticato, Amministratore APIMarket, Interfacce API Presente In APIMarket \\
	\textbf{Descrizione} & l'attore visualizza un insieme di modalita' di acquisto (es. Paypal, carta) e sceglie quella che preferisce\\
	\textbf{Pre-Condizioni} & L'attore e' nel processo di Acquisto di una API\\
	\textbf{Post-Condizioni}& L'attore ha scelto la sua modalita' di Acquisto\\
	\textbf{Scenario Principale} & \begin{enumerate*}[label=(\arabic*.),itemjoin={\newline}]
		\item L'attore sceglie la modalita' di acquisto
	\end{enumerate*}\\
\end{longtable}

%\paragraph{Caso d'uso UC8.2.3: Conferma Acquisto}
\label{UC8.2.3}

\renewcommand*{\arraystretch}{1.6}
\begin{longtable}{ l | p{11cm}}
	\hline
	\rowcolor{Gray}
	\multicolumn{2}{c}{UC8.2.3: Conferma Acquisto} \\
	\hline
	\textbf{Attori} &Utente Autenticato, Amministratore APIMarket \\
	\textbf{Descrizione} & l'attore sceglie conferma l'acquisto e il processo di pagamento puo' avvenire\\
	\textbf{Pre-Condizioni} &  l'attore ha scelto di acquistare una API e scelto le modalita' di acquisto\\
	\textbf{Post-Condizioni}& l'attore ha pagato e acquistato la API\\
	\textbf{Scenario Principale} & \begin{enumerate*}[label=(\arabic*.),itemjoin={\newline}]
		\item L'attore conferma l'acquisto
	\end{enumerate*}\\
\end{longtable}

%\paragraph{Caso d'uso UC8.2.4: Visualizzazione Errore Acquisto}
\label{UC8.2.4}

\renewcommand*{\arraystretch}{1.6}
\begin{longtable}{ l | p{11cm}}
	\hline
	\rowcolor{Gray}
	\multicolumn{2}{c}{UC8.2.4: Visualizzazione Errore Acquisto} \\
	\hline
	\textbf{Attori} &Utente Autenticato, Amministratore APIMarket\\
	\textbf{Descrizione} & L'attore ha provato a effettuare l'acquisto ma i dati inseriti sono errati\\
	\textbf{Pre-Condizioni} &  l'attore ha confermato l'acquisto\\
	\textbf{Post-Condizioni}& l'attore deve reinserire le informazioni d'Acquisto\\
	\textbf{Scenario Principale} & \begin{enumerate*}[label=(\arabic*.),itemjoin={\newline}]
		\item L'attore visualizza l'Errore d'Acquisto
	\end{enumerate*}\\
	\textbf{Scenari Alternativi} & \begin{enumerate*}[label=(\arabic*.),itemjoin={\newline}]
		\item L'attore sceglie di annullare l'Acquisto
	\end{enumerate*}\\
\end{longtable}

%\paragraph{Caso d'uso UC8.2.5: Conferma Acquisto Avvenuto}
\label{UC8.2.5}

\renewcommand*{\arraystretch}{1.6}
\begin{longtable}{ l | p{11cm}}
	\hline
	\rowcolor{Gray}
	\multicolumn{2}{c}{UC8.2.5: Conferma Acquisto Avvenuto} \\
	\hline
	\textbf{Attori} &Utente Autenticato, Amministratore APIMarket \\
	\textbf{Descrizione} & l'attore riceve una conferma dal sistema che segnala il successo dell'Acquisto  \\
	\textbf{Pre-Condizioni} &  l'attore ha confermato l'acquisto\\
	\textbf{Post-Condizioni}& l'attore riceve la chiave dell'API\\
	\textbf{Scenario Principale} & \begin{enumerate*}[label=(\arabic*.),itemjoin={\newline}]
		\item L'attore visualizza una conferma che segnala un Acquisto di successo
	\end{enumerate*}\\
\end{longtable}

%\paragraph{Caso d'uso UC8.2.6: Procedura Acquisto Esterna}
\label{UC8.2.6}

\renewcommand*{\arraystretch}{1.6}
\begin{longtable}{ l | p{11cm}}
	\hline
	\rowcolor{Gray}
	\multicolumn{2}{c}{UC8.2.6: Procedura Acquisto Esterna} \\
	\hline
	\textbf{Attori} &Utente Autenticato, Amministratore APIMarket\\
	\textbf{Descrizione} & l'attore usa una procedura d'acquisto esterna per acquistare l'API \\
	\textbf{Pre-Condizioni} &  l'attore ha confermato l'acquisto\\
	\textbf{Post-Condizioni}& l'attore ha pagato l'API e riceve la chiave API\\
	\textbf{Scenario Principale} & \begin{enumerate*}[label=(\arabic*.),itemjoin={\newline}]
		\item L'attore può pagare usando una procedura d'acquisto esterna
	\end{enumerate*}\\
\end{longtable}

%\paragraph{Caso d'uso UC8.2.7: Rilascio Chiave API}
\label{UC8.2.7}

\renewcommand*{\arraystretch}{1.6}
\begin{longtable}{ l | p{11cm}}
	\hline
	\rowcolor{Gray}
	\multicolumn{2}{c}{UC8.2.7: Rilascio Chiave API} \\
	\hline
	\textbf{Attori} &Utente Autenticato, Amministratore APIMarket \\
	\textbf{Descrizione} & l'attore riceve una chiave API e puo' iniziare a usare l'API \\
	\textbf{Pre-Condizioni} &  l'attore ha comprato l'API\\
	\textbf{Post-Condizioni}& l'attore ha la chiave API e puo' usarla fino alla scadenza della chiave\\
	\textbf{Scenario Principale} & \begin{enumerate*}[label=(\arabic*.),itemjoin={\newline}]
		\item L'attore riceve una chiave API valida fino a una certa data o fino al rinnovo
	\end{enumerate*}\\
\end{longtable}

%\newpage
\subsection{Caso d'uso UC9: Interazione con API Acquistate}
\label{UC9}
\begin{figure}[ht]
	\centering
	\includegraphics[scale=0.45]{UML/UC9.png}
	\caption{UC9: Interazione con API Acquistate}
\end{figure}
\FloatBarrier
\renewcommand*{\arraystretch}{1.6}
\begin{longtable}{ l | p{11cm}}
	\hline
	\rowcolor{Gray}
	\multicolumn{2}{c}{UC9: Interazione con API Acquistate} \\
	\hline
	\textbf{Attori} &Utente Autenticato, Amministratore APIMarket, Interfacce API Presente In APIMarket \\
	\textbf{Descrizione} & l'attore sceglie attraverso quale modalità interagire con le API acquistate \\
	\textbf{Pre-Condizioni} & l'attore si trova nella schermata di gestione di una API acquistata\\
	\textbf{Post-Condizioni}&l'attore ha scelto l'interazione\\
	\textbf{Scenario Principale} & \begin{enumerate*}[label=(\arabic*.),itemjoin={\newline}]
		\item L'attore può ricercare un'API(UC6)
		\item L'attore può consultare la documentazione di un'API(UC8.1)
		\item L'attore può Annullare il rinnovo di un'API(UC9.1)
		\item L'attore può utilizzare un'API(UC9.2)
		\item L'attore può Rimuovere un'API dalle Acquistate
		\item L'attore può vedere la Scadenza di un'API(UC9.4);
	\end{enumerate*}\\
\end{longtable}



%\newpage
\subsubsection{Caso d'uso UC9.1: Annullamento Rinnovo API}
\label{UC9.1}
\begin{figure}[ht]
	\centering
	\includegraphics[scale=0.45]{UML/UC9_1.png}
	\caption{UC9.1: Annullamento Rinnovo API}
\end{figure}
\FloatBarrier
\renewcommand*{\arraystretch}{1.6}
\begin{longtable}{ l | p{11cm}}
	\hline
	\rowcolor{Gray}
	\multicolumn{2}{c}{UC9.1: Annullamento Rinnovo API} \\
	\hline
	\textbf{Attori} &Utente Autenticato, Amministratore APIMarket, Interfacce API Presente In APIMarket \\
	\textbf{Descrizione} & l'attore sceglie attraverso quale modalità interagire con le API acquistate \\
	\textbf{Pre-Condizioni} & l'attore ha scelto di gestire il rinnovo automatico di una API\\
	\textbf{Post-Condizioni}&l'attore ha effettuato le operazioni desiderate\\
	\textbf{Scenario Principale} & \begin{enumerate*}[label=(\arabic*.),itemjoin={\newline}]
		\item L'attore può visualizzare i dati riguardanti il rinnovo automatico di una API (UC9.1.1);
		\item L'attore può confermare l'annullamento del rinnovo automatico di una API (UC9.1.2)
	\end{enumerate*}\\
\end{longtable}


%\paragraph{Caso d'uso UC9.1.1: Visualizzazione Dati Rinnovo API}
\label{UC9.1.1}

\renewcommand*{\arraystretch}{1.6}
\begin{longtable}{ l | p{11cm}}
	\hline
	\rowcolor{Gray}
	\multicolumn{2}{c}{UC9.1.1: Visualizzazione Dati Rinnovo API} \\
	\hline
	\textbf{Attori} &Utente Autenticato, Amministratore APIMarket, Interfacce API Presente In APIMarket \\
	\textbf{Descrizione} & l'attore visualizza i dati del rinnovo automatico di una API acquistata \\
	\textbf{Pre-Condizioni} &  l'attore ha scelto di visualizzare i dati del rinnovo automatico di una API acquistata\\
	\textbf{Post-Condizioni}& l'attore ha visualizzato i dati del rinnovo automatico di una API acquistata\\
	\textbf{Scenario Principale} & \begin{enumerate*}[label=(\arabic*.),itemjoin={\newline}]
		\item L'attore può visualizzare i dati riguardanti il rinnovo automatico di una API
	\end{enumerate*}\\
\end{longtable}

%\newpage
\paragraph{Caso d'uso UC9.1.2: Conferma Annullamento Rinnovo API}
\label{UC9.1.2}

\renewcommand*{\arraystretch}{1.6}
\begin{longtable}{ l | p{11cm}}
	\hline
	\rowcolor{Gray}
	\multicolumn{2}{c}{UC9.1.2: Conferma Annullamento Rinnovo API} \\
	\hline
	\textbf{Attori} &Utente Autenticato, Amministratore APIMarket, Interfacce API Presente In APIMarket \\
	\textbf{Descrizione} & l'attore conferma l'annullamento del rinnovo automatico di una API acquistata \\
	\textbf{Pre-Condizioni} & l'attore ha scelto di confermare l'annullamento del rinnovo automatico di una API acquistata\\
	\textbf{Post-Condizioni}& l'attore ha confermato l'annullamento del rinnovo automatico di una API acquistata\\
	\textbf{Scenario Principale} & \begin{enumerate*}[label=(\arabic*.),itemjoin={\newline}]
		\item l'attore può confermare l'annullamento del rinnovo automatico di una API acquistata
	\end{enumerate*}\\
\end{longtable}

%\subsubsection{Caso d'uso UC9.2: Interazione Interfaccia API}
\label{UC9.2}

\renewcommand*{\arraystretch}{1.6}
\begin{longtable}{ l | p{11cm}}
	\hline
	\rowcolor{Gray}
	\multicolumn{2}{c}{UC9.2: Interazione Interfaccia API} \\
	\hline
	\textbf{Attori} &Utente Autenticato, Amministratore APIMarket, Interfacce API Presente In APIMarket \\
	\textbf{Descrizione} & l'attore interagisce con l'interfaccia API \\
	\textbf{Pre-Condizioni} & l'attore ha scelto di interagire con l'interfaccia API acquistata\\
	\textbf{Post-Condizioni}& l'attore ha finito di interagire con l'interfaccia API acquistata\\
	\textbf{Scenario Principale} & \begin{enumerate*}[label=(\arabic*.),itemjoin={\newline}]
			\item l'attore può interagire con l'interfaccia di una API acquistata
	\end{enumerate*}\\
\end{longtable}
%\subsubsection{Caso d'uso UC9.3: Rimozione API Dalle API Acquistate}
\label{UC9.3}

\renewcommand*{\arraystretch}{1.6}
\begin{longtable}{ l | p{11cm}}
	\hline
	\rowcolor{Gray}
	\multicolumn{2}{c}{UC9.3: Rimozione API Dalle API Acquistate} \\
	\hline
	\textbf{Attori} &Utente Autenticato, Amministratore APIMarket, Interfacce API Presente In APIMarket \\
	\textbf{Descrizione} & l'attore sceglie se rimuovere una API dalle API acquistate \\
	\textbf{Pre-Condizioni} & l'attore ha scelto di rimuovere una API dalle API acquistate\\
	\textbf{Post-Condizioni}& l'attore ha rimosso una API dalle API acquistate oppure ha annullato l'operazione\\
	\textbf{Scenario Principale} & \begin{enumerate*}[label=(\arabic*.),itemjoin={\newline}]
			\item l'attore può confermare la rimozione di una API dalle API acquistate
	\end{enumerate*}\\
\end{longtable}



%\subsubsection{Caso d'uso UC9.4: Visualizzazione Scadenza API}
\label{UC9.4}

\renewcommand*{\arraystretch}{1.6}
\begin{longtable}{ l | p{11cm}}
	\hline
	\rowcolor{Gray}
	\multicolumn{2}{c}{UC9.4: Visualizzazione Scadenza API} \\
	\hline
	\textbf{Attori} &Utente Autenticato, Amministratore APIMarket, Interfacce API Presente In APIMarket \\
	\textbf{Descrizione} & l'attore visualizza la data di scadenza della chiave di una API \\
	\textbf{Pre-Condizioni} &l'attore ha scelto di visualizzare la scadenza di una API\\
	\textbf{Post-Condizioni}& l'attore ha visualizzato la scadenza di una API\\
	\textbf{Scenario Principale} & \begin{enumerate*}[label=(\arabic*.),itemjoin={\newline}]
		\item L'attore può visualizzare la data di scadenza della chiave di una API
	\end{enumerate*}\\
\end{longtable}
%\newpage
\subsection{Caso d'uso UC10: Interazione Con API Registrate}
\label{UC10}
\begin{figure}[ht]
	\centering
	\includegraphics[scale=0.45]{UML/UC10.png}
	\caption{UC10: Interazione Con API Registrate}
\end{figure}

\renewcommand*{\arraystretch}{1.6}
\begin{longtable}{ l | p{11cm}}
	\hline
	\rowcolor{Gray}
	\multicolumn{2}{c}{UC10: Interazione Con API Registrate} \\
	\hline
	\textbf{Attori} &Utente Autenticato, Amministratore APIMarket, Interfacce API Presente In APIMarket \\
	\textbf{Descrizione} & l'attore sceglie attraverso quale modalità interagire con le API acquistate \\
	\textbf{Pre-Condizioni} & l'attore si trova nella schermata di gestione di una API acquistata\\
	\textbf{Post-Condizioni}&l'attore ha scelto l'interazione\\
	\textbf{Scenario Principale} & \begin{enumerate*}[label=(\arabic*.),itemjoin={\newline}]
			\item L'attore può rimuovere una propria API dall'APIMarket (UC10.1), inviando una notifica agli utenti di quella API (UC10.4);
		\item L'attore può modificare una propria API (UC10.2);
		\item L'attore può visualizzare i dati di utilizzo di una propria API (UC10.3);
	\end{enumerate*}\\
\end{longtable}




%\subsubsection{Caso d'uso UC10.1: Rimozione API Da APIMarket}
\label{UC10.1}

\renewcommand*{\arraystretch}{1.6}
\begin{longtable}{ l | p{11cm}}
	\hline
	\rowcolor{Gray}
	\multicolumn{2}{c}{UC10.1: Rimozione API Da APIMarket} \\
	\hline
	\textbf{Attori} &Utente Autenticato, Amministratore APIMarket \\
	\textbf{Descrizione} & l'attore rimuove dall'APIMarket una propria API\\
	\textbf{Pre-Condizioni} & l'attore ha scelto di rimuovere dall'APIMarket una propria API\\
	\textbf{Post-Condizioni}&l'attore ha rimosso dall'APIMarket una propria API\\
	\textbf{Scenario Principale} & \begin{enumerate*}[label=(\arabic*.),itemjoin={\newline}]
			\item L'attore può rimuovere dall'APIMarket una propria API
	\end{enumerate*}\\
\end{longtable}




%\subsubsection{Caso d'uso UC10.2: Modifica API}
\label{UC10.2}

\begin{figure}[ht]
	\centering
	\includegraphics[scale=0.45]{UML/UC10_2.png}
	\caption{UC10.2: Modifica API}
\end{figure}

\renewcommand*{\arraystretch}{1.6}
\begin{longtable}{ l | p{11cm}}
	\hline
	\rowcolor{Gray}
	\multicolumn{2}{c}{UC10.2: Modifica API} \\
	\hline
	\textbf{Attori} &Utente Autenticato, Amministratore APIMarket \\
	\textbf{Descrizione} & l'attore modifica una propria API\\
	\textbf{Pre-Condizioni} & l'attore ha scelto di modificare una propria API\\
	\textbf{Post-Condizioni}&l'attore ha modificato una propria API oppure l'operazione è fallita\\
	\textbf{Scenario Principale} & \begin{enumerate*}[label=(\arabic*.),itemjoin={\newline}]
		\item l'attore può modificare la documentazione della propria API (UC10.2.1)
		\item l'attore può modificare l'interfaccia della propria API (UC10.2.2)
		\item l'attore può confermare le modifiche alla propria API (UC10.2.3)
	\end{enumerate*}\\
	\textbf{Scenari Alternativi} & \begin{enumerate*}[label=(\arabic*.),itemjoin={\newline}]
		\item l'attore può visualizzare l'errore di modifica API (UC10.2.4)
	\end{enumerate*}\\
\end{longtable}

%\paragraph{Caso d'uso UC10.2.1: Modifica Documentazione API}
\label{UC10.2.1}

\renewcommand*{\arraystretch}{1.6}
\begin{longtable}{ l | p{11cm}}
	\hline
	\rowcolor{Gray}
	\multicolumn{2}{c}{UC10.2.1: Modifica Documentazione API} \\
	\hline
	\textbf{Attori} &Utente Autenticato, Amministratore APIMarket \\
	\textbf{Descrizione} &  l'attore modifica la documentazione della propria API\\
	\textbf{Pre-Condizioni} & l'attore ha scelto di modificare la documentazione della propria API\\
	\textbf{Post-Condizioni}& l'attore ha modificato la documentazione della propria API\\
	\textbf{Scenario Principale} & \begin{enumerate*}[label=(\arabic*.),itemjoin={\newline}]
		\item L'attore può rimuovere dall'APIMarket una propria API
	\end{enumerate*}\\
\end{longtable}




%\paragraph{Caso d'uso UC10.2.2: Modifica Interfaccia API}
\label{UC10.2.2}

\renewcommand*{\arraystretch}{1.6}
\begin{longtable}{ l | p{11cm}}
	\hline
	\rowcolor{Gray}
	\multicolumn{2}{c}{UC10.2.2: Modifica Interfaccia API} \\
	\hline
	\textbf{Attori} &Utente Autenticato, Amministratore APIMarket \\
	\textbf{Descrizione} &  l'attore modifica l'interfaccia della propria API\\
	\textbf{Pre-Condizioni} & l'attore ha scelto di modificare l'interfaccia della propria API\\
	\textbf{Post-Condizioni}& l'attore ha modificato l'interfaccia della propria API\\
	\textbf{Scenario Principale} & \begin{enumerate*}[label=(\arabic*.),itemjoin={\newline}]
		\item l'attore può modificare l'interfaccia della propria API
	\end{enumerate*}\\
\end{longtable}



%\paragraph{Caso d'uso UC10.2.3: Conferma Modifica API}
\label{UC10.2.3}

\renewcommand*{\arraystretch}{1.6}
\begin{longtable}{ l | p{11cm}}
	\hline
	\rowcolor{Gray}
	\multicolumn{2}{c}{UC10.2.3: Conferma Modifica API} \\
	\hline
	\textbf{Attori} &Utente Autenticato, Amministratore APIMarket \\
	\textbf{Descrizione} &  l'attore conferma le modifiche alla propria API\\
	\textbf{Pre-Condizioni} & l'attore ha scelto di confermare le modifiche alla propria API\\
	\textbf{Post-Condizioni}& l'attore ha confermato le modifiche alla propria API\\
	\textbf{Scenario Principale} & \begin{enumerate*}[label=(\arabic*.),itemjoin={\newline}]
		\item l'attore può confermare le modifiche alla propria API
	\end{enumerate*}\\
\end{longtable}



%\paragraph{Caso d'uso UC10.2.4: Visualizzazione Errore Modifica API}
\label{UC10.2.4}

\renewcommand*{\arraystretch}{1.6}
\begin{longtable}{ l | p{11cm}}
	\hline
	\rowcolor{Gray}
	\multicolumn{2}{c}{UC10.2.4: Visualizzazione Errore Modifica API} \\
	\hline
	\textbf{Attori} &Utente Autenticato, Amministratore APIMarket \\
	\textbf{Descrizione} &  l'attore visualizza l'errore nella modifica della propria API\\
	\textbf{Pre-Condizioni} & l'attore ha scelto di visualizzare l'errore nella modifica della propria API\\
	\textbf{Post-Condizioni}& l'attore ha visualizzato l'errore nella modifica della propria API ed è stato reindirizzato ad UC10.2\\
	\textbf{Scenario Principale} & \begin{enumerate*}[label=(\arabic*.),itemjoin={\newline}]
		\item l'attore può visualizzare l'errore nella modifica della propria API
	\end{enumerate*}\\
\end{longtable}



%\subsubsection{Caso d'uso UC10.3: Visualizzazione Dati Utilizzo API}
\label{UC10.3}

\renewcommand*{\arraystretch}{1.6}
\begin{longtable}{ l | p{11cm}}
	\hline
	\rowcolor{Gray}
	\multicolumn{2}{c}{UC10.3: Visualizzazione Dati Utilizzo API} \\
	\hline
	\textbf{Attori} &Utente Autenticato, Amministratore APIMarket \\
	\textbf{Descrizione} & l'attore visualizza i dati di utilizzo di una propria API\\
	\textbf{Pre-Condizioni} & l'attore ha scelto di visualizzare i dati di utilizzo di una propria API\\
	\textbf{Post-Condizioni}& l'attore ha visualizzato i dati di utilizzo di una propria API\\
	\textbf{Scenario Principale} & \begin{enumerate*}[label=(\arabic*.),itemjoin={\newline}]
		\item l'attore può visualizzare i dati di utilizzo di una propria API
	\end{enumerate*}\\
\end{longtable}

%\subsubsection{Caso d'uso UC10.4: Notifica Della Rimozione API}
\label{UC10.4}

\renewcommand*{\arraystretch}{1.6}
\begin{longtable}{ l | p{11cm}}
	\hline
	\rowcolor{Gray}
	\multicolumn{2}{c}{UC10.4: Notifica Della Rimozione API} \\
	\hline
	\textbf{Attori} &Utente Autenticato, Amministratore APIMarket \\
	\textbf{Descrizione} & l'applicazione web notifica gli utenti dell'API rimossa riguardo all'avvenuta rimozione\\
	\textbf{Pre-Condizioni} & l'attore ha rimosso dall'APIMarket una propria API\\
	\textbf{Post-Condizioni}& l'applicazione web ha notificato gli utenti dell'API rimossa riguardo all'avvenuta rimozione\\
	\textbf{Scenario Principale} & \begin{enumerate*}[label=(\arabic*.),itemjoin={\newline}]
		\item L'applicazione web può notificare gli utenti dell'API rimossa riguardo all'avvenuta rimozione
	\end{enumerate*}\\
\end{longtable}




%\subsubsection{Caso d'uso UC10.5: Visualizza Errore Cancellazione API}
\label{UC10.5}

\renewcommand*{\arraystretch}{1.6}
\begin{longtable}{ l | p{11cm}}
	\hline
	\rowcolor{Gray}
	\multicolumn{2}{c}{UC10.5: Visualizza Errore Cancellazione API} \\
	\hline
	\textbf{Attori} &Utente Autenticato, Amministratore APIMarket \\
	\textbf{Descrizione} & l'attore visualizza l'errore nella rimozione di una propria API\\
	\textbf{Pre-Condizioni} & l'attore ha tentato di rimuovere una propria API dall'APIMarket e l'operazione è fallita\\
	\textbf{Post-Condizioni}& l'attore ha visualizzato l'errore nella rimozione dell'API ed è stato reindirizzato ad UC10\\
	\textbf{Scenario Principale} & \begin{enumerate*}[label=(\arabic*.),itemjoin={\newline}]
		\item L'attore può visualizzare l'errore nella rimozione dell'API
	\end{enumerate*}\\
\end{longtable}



%\newpage
\subsection{Caso d'uso UC11 - Registrazione nuova API}
\label{UC11}
\begin{figure}[ht]
	\centering
	\includegraphics[scale=0.45]{UML/UC11.png}
	\caption{UC11: Registrazione nuova API}
\end{figure}

\begin{longtable}{ l | p{11cm}}
	\hline
	\rowcolor{Gray}
	\multicolumn{2}{c}{UC11 - Registrazione nuova API}\\
	\hline
	\textbf{Attori} & Utente autenticato, Amministratore API Market \\
	\textbf{Descrizione} & L'attore registra una nuova API su API Market \\
	\textbf{Pre-Condizioni} & L'attore si trova nella schermata relativa alla registrazione di una nuova API \\
	\textbf{Post-Condizioni} & L'attore ha registrato una nuova API su API Market \\
	\textbf{Scenario Principale} & 
	\begin{enumerate*}[label=(\arabic*.),itemjoin={\newline}]
		\item L'attore può inserire il nome della nuova API (UC11.1)
		\item L'attore può inserire la descrizione della nuova API (UC11.2)
		\item L'attore può inserire i tag della nuova API (UC11.3)
		\item L'attore può inserire l'interfaccia della nuova API (UC11.4)
		\item L'attore può inserire il file per la documentazione PDF della nuova API (UC11.5)
		\item L'attore può inserire il link per la documentazione esterna della nuova API (UC11.7)
		\item L'attore può inserire il prezzo base della nuova API (UC11.8)
		\item L'attore può confermare la registrazione della nuova API (UC11.9)
	\end{enumerate*}\\
	\textbf{Scenario Principale} & 
	\begin{enumerate*}[label=(\arabic*.),itemjoin={\newline}]
			\item L'attore può visualizzare un messaggio di errore riguardo al caricamento del file di documentazione PDF dell'API, ed il caricamento del file non avviene (UC11.6)
			\item L'attore può visualizzare un messaggio d'errore informativo riguardo la conferma della registrazione dell'API, e la registrazione non avviene (UC11.10)
	\end{enumerate*}\\
\end{longtable}

\subsubsection{Caso d'uso UC11.1: Inserimento nome API}
\label{UC11_1}

\begin{minipage}{\linewidth}
	\begin{tabular}{ l | p{11cm}}
		\hline
		\rowcolor{Gray}
		\multicolumn{2}{c}{UC11.1 - Inserimento nome API} \\
		\hline
		\textbf{Attori} & Utente autenticato, Amministratore API Market \\
		\textbf{Descrizione} & L'attore inserisce il nome della nuova API \\
		\textbf{Pre-Condizioni} & L'attore si trova nella schermata relativa alla registrazione di una nuova API \\
		\textbf{Post-Condizioni} & L'attore ha inserito il nome della nuova API \\
		\textbf{Scenario Principale} & 
		\begin{enumerate*}[label=(\arabic*.),itemjoin={\newline}]
			\item L'attore può inserire il nome della nuova API
		\end{enumerate*}\\
	\end{tabular}
\end{minipage}

\subsubsection{Caso d'uso UC11.2: Inserimento descrizione API}
\label{UC11_2}

\begin{minipage}{\linewidth}
	\begin{tabular}{ l | p{11cm}}
		\hline
		\rowcolor{Gray}
		\multicolumn{2}{c}{UC11.2 - Inserimento descrizione API} \\
		\hline
		\textbf{Attori} & Utente autenticato, Amministratore API Market \\
		\textbf{Descrizione} & L'attore inserisce la descrizione della nuova API \\
		\textbf{Pre-Condizioni} & L'attore si trova nella schermata relativa alla registrazione di una nuova API \\
		\textbf{Post-Condizioni} & L'attore ha inserito la descrizione della nuova API \\
		\textbf{Scenario Principale} & 
		\begin{enumerate*}[label=(\arabic*.),itemjoin={\newline}]
			\item L'attore può inserire la descrizione della nuova API
		\end{enumerate*}\\
	\end{tabular}
\end{minipage}

\subsubsection{Caso d'uso UC11.3: Inserimento tag API}
\label{UC11_3}

\begin{minipage}{\linewidth}
	\begin{tabular}{ l | p{11cm}}
		\hline
		\rowcolor{Gray}
		\multicolumn{2}{c}{UC11.3 - Inserimento tag API} \\
		\hline
		\textbf{Attori} & Utente autenticato, Amministratore API Market \\
		\textbf{Descrizione} & L'attore inserisce i tag della nuova API \\
		\textbf{Pre-Condizioni} & L'attore si trova nella schermata relativa alla registrazione di una nuova API \\
		\textbf{Post-Condizioni} & L'attore ha inserito i tag della nuova API \\
		\textbf{Scenario Principale} & 
		\begin{enumerate*}[label=(\arabic*.),itemjoin={\newline}]
			\item L'attore può inserire i tag della nuova API
		\end{enumerate*}\\
	\end{tabular}
\end{minipage}

\subsubsection{Caso d'uso UC11.4: Inserimento interfaccia API}
\label{UC11_4}

\begin{minipage}{\linewidth}
	\begin{tabular}{ l | p{11cm}}
		\hline
		\rowcolor{Gray}
		\multicolumn{2}{c}{UC11.4 - Inserimento interfaccia API} \\
		\hline
		\textbf{Attori} & Utente autenticato, Amministratore API Market \\
		\textbf{Descrizione} & L'attore inserisce l'interfaccia della nuova API \\
		\textbf{Pre-Condizioni} & L'attore si trova nella schermata relativa alla registrazione di una nuova API \\
		\textbf{Post-Condizioni} & L'attore ha inserito l'interfaccia della nuova API \\
		\textbf{Scenario Principale} & 
		\begin{enumerate*}[label=(\arabic*.),itemjoin={\newline}]
			\item L'attore può inserire l'interfaccia della nuova API
		\end{enumerate*}\\
	\end{tabular}
\end{minipage}

\subsubsection{Caso d'uso UC11.5: Inserimento documentazione PDF API}
\label{UC11_5}

\begin{minipage}{\linewidth}
	\begin{tabular}{ l | p{11cm}}
		\hline
		\rowcolor{Gray}
		\multicolumn{2}{c}{UC11.5 - Inserimento documentazione PDF API} \\
		\hline
		\textbf{Attori} & Utente autenticato, Amministratore API Market \\
		\textbf{Descrizione} & L'attore carica su API Market un file PDF contenente la documentazione PDF della nuova API \\
		\textbf{Pre-Condizioni} & L'attore si trova nella schermata relativa alla registrazione di una nuova API \\
		\textbf{Post-Condizioni} & L'attore ha caricato su API Market un file PDF contenente la documentazione PDF della nuova API \\
		\textbf{Scenario Principale} & 
		\begin{enumerate*}[label=(\arabic*.),itemjoin={\newline}]
			\item L'attore può caricare su API Market un file PDF contenente la documentazione PDF della nuova API
		\end{enumerate*}\\
		\textbf{Scenari Alternativi} & 
		\begin{enumerate*}[label=(\arabic*.),itemjoin={\newline}]
		\item L'attore può visualizzare un messaggio di errore ed il caricamento del file non avviene (UC11.10)
		\end{enumerate*}\\
	\end{tabular}
\end{minipage}

\subsubsection{Caso d'uso UC11.6: Errore inserimento PDF API}
\label{UC11_6}

\begin{minipage}{\linewidth}
	\begin{tabular}{ l | p{11cm}}
		\hline
		\rowcolor{Gray}
		\multicolumn{2}{c}{UC11.6 - Errore inserimento PDF API} \\
		\hline
		\textbf{Attori} & Utente autenticato, Amministratore API Market \\
		\textbf{Descrizione} & L'attore visualizza un messaggio di errore e l'inserimento della documentazione PDF della nuova API non avviene \\
		\textbf{Pre-Condizioni} & L'attore ha cercato di caricare su API Market un file contenente la documentazione della nuova API ma si è verificato un errore \\
		\textbf{Post-Condizioni} & L'attore ha visualizzato un messaggio di errore \\
		\textbf{Scenario Principale} & 
		\begin{enumerate*}[label=(\arabic*.),itemjoin={\newline}]
			\item L'attore può visualizzare un messaggio di errore
		\end{enumerate*}\\
	\end{tabular}
\end{minipage}

\subsubsection{Caso d'uso UC11.7: Inserimento documentazione esterna API}
\label{UC11_7}

\begin{minipage}{\linewidth}
	\begin{tabular}{ l | p{11cm}}
		\hline
		\rowcolor{Gray}
		\multicolumn{2}{c}{UC11.7 - Inserimento documentazione esterna API} \\
		\hline
		\textbf{Attori} & Utente autenticato, Amministratore API Market \\
		\textbf{Descrizione} & L'attore inserisce il link alla documentazione esterna della nuova API \\
		\textbf{Pre-Condizioni} & L'attore si trova nella schermata relativa alla registrazione di una nuova API \\
		\textbf{Post-Condizioni} & L'attore ha inserito il link alla documentazione esterna della nuova API \\
		\textbf{Scenario Principale} & 
		\begin{enumerate*}[label=(\arabic*.),itemjoin={\newline}]
			\item L'attore può inserire il link alla documentazione esterna della nuova API
		\end{enumerate*}\\
	\end{tabular}
\end{minipage}

\subsubsection{Caso d'uso UC11.8: Inserimento prezzo base API}
\label{UC11_8}

\begin{minipage}{\linewidth}
	\begin{tabular}{ l | p{11cm}}
		\hline
		\rowcolor{Gray}
		\multicolumn{2}{c}{UC11.8 - Inserimento prezzo base API} \\
		\hline
		\textbf{Attori} & Utente autenticato, Amministratore API Market \\
		\textbf{Descrizione} & L'attore inserisce il prezzo base della nuova API \\
		\textbf{Pre-Condizioni} & L'attore si trova nella schermata relativa alla registrazione di una nuova API \\
		\textbf{Post-Condizioni} & L'attore ha inserito il prezzo base della nuova API \\
		\textbf{Scenario Principale} & 
		\begin{enumerate*}[label=(\arabic*.),itemjoin={\newline}]
			\item L'attore può inserire il prezzo base della nuova API
		\end{enumerate*}\\
	\end{tabular}
\end{minipage}

\subsubsection{Caso d'uso UC11.9: Conferma registrazione nuova API}
\label{UC11_9}

\begin{minipage}{\linewidth}
	\begin{tabular}{ l | p{11cm}}
		\hline
		\rowcolor{Gray}
		\multicolumn{2}{c}{UC11.9 - Conferma registrazione nuova API} \\
		\hline
		\textbf{Attori} & Utente autenticato, Amministratore API Market \\
		\textbf{Descrizione} & L'attore conferma la registrazione della nuova API \\
		\textbf{Pre-Condizioni} & L'attore si trova nella schermata relativa alla registrazione di una nuova API \\
		\textbf{Post-Condizioni} & L'attore ha confermato la registrazione della nuova API \\
		\textbf{Scenario Principale} & 
		\begin{enumerate*}[label=(\arabic*.),itemjoin={\newline}]
			\item L'attore può confermare la registrazione della nuova API, visualizzando un messaggio di successo e venendo reindirizzato alla schermata di visualizzazione API registrate (UC10)
		\end{enumerate*}\\
	\end{tabular}
\end{minipage}

\subsubsection{Caso d'uso UC11.10: Errore registrazione nuova API}
\label{UC11_10}

\begin{minipage}{\linewidth}
	\begin{tabular}{ l | p{11cm}}
		\hline
		\rowcolor{Gray}
		\multicolumn{2}{c}{UC11.10 - Errore registrazione nuova API} \\
		\hline
		\textbf{Attori} & Utente autenticato, Amministratore API Market \\
		\textbf{Descrizione} & L'attore visualizza un messaggio di errore informativo e la registrazione della nuova API non avviene \\
		\textbf{Pre-Condizioni} & L'attore ha confermato la registrazione della una nuova API ma si è verificato un errore \\
		\textbf{Post-Condizioni} & L'attore ha visualizzato un messaggio di errore informativo \\
		\textbf{Scenario Principale} & 
		\begin{enumerate*}[label=(\arabic*.),itemjoin={\newline}]
			\item L'attore può visualizzare un messaggio di errore informativo e la registrazione della nuova API non avviene
		\end{enumerate*}\\
	\end{tabular}
\end{minipage}
%\subsubsection{Caso d'uso UC11.1: Inserimento Documentazione API}
\label{UC11.1}
\begin{figure}[ht]
	\centering
	\includegraphics[scale=0.45]{UML/UC11_1.png}
	\caption{Caso d'uso UC11: Registrazione Nuova API}
\end{figure}

\renewcommand*{\arraystretch}{1.6}
\begin{longtable}{ l | p{11cm}}
	\hline
	\rowcolor{Gray}
	\multicolumn{2}{c}{UC11.1: Inserimento Documentazione API} \\
	\hline
	\textbf{Attori} &Utente Autenticato, Amministratore APIMarket \\
	\textbf{Descrizione} & l'attore inserisce la documentazione della propria nuova API \\
	\textbf{Pre-Condizioni} &  l'attore ha scelto di inserire la documentazione della propria nuova API\\
	\textbf{Post-Condizioni}&l'attore ha inserito la documentazione della propria nuova API\\
	\textbf{Scenario Principale} & \begin{enumerate*}[label=(\arabic*.),itemjoin={\newline}]
			\item l'attore può inserire la documentazione della propria nuova API
	\end{enumerate*}\\
\end{longtable}



%\paragraph{Caso d'uso UC11.1.1: Inserimento Documentazione PDF}
\label{UC11.1.1}

\renewcommand*{\arraystretch}{1.6}
\begin{longtable}{ l | p{11cm}}
	\hline
	\rowcolor{Gray}
	\multicolumn{2}{c}{Caso d'uso UC11.1.1: Inserimento Documentazione PDF} \\
	\hline
	\textbf{Attori} &Utente Autenticato, Amministratore APIMarket \\
	\textbf{Descrizione} & l'attore inserisce la documentazione PDF della propria nuova API\\
	\textbf{Pre-Condizioni} &  l'attore ha scelto di inserire la documentazione PDF della propria nuova API\\
	\textbf{Post-Condizioni}&l'attore ha inserito la documentazione PDF della propria nuova API\\
	\textbf{Scenario Principale} & \begin{enumerate*}[label=(\arabic*.),itemjoin={\newline}]
		\item l'attore può inserire la documentazione PDF della propria nuova API
	\end{enumerate*}\\
\end{longtable}



%\subsubsection{Caso d'uso UC11.1.2: Inserimento Documentazione Web}
\label{UC11.1.2}

\renewcommand*{\arraystretch}{1.6}
\begin{longtable}{ l | p{11cm}}
	\hline
	\rowcolor{Gray}
	\multicolumn{2}{c}{UC11.1.2: Inserimento Documentazione Web} \\
	\hline
	\textbf{Attori} &Utente Autenticato, Amministratore APIMarket \\
	\textbf{Descrizione} & l'attore inserisce la documentazione Web della propria nuova API\\
	\textbf{Pre-Condizioni} & 'attore ha scelto di inserire la documentazione Web della propria nuova API\\
	\textbf{Post-Condizioni}&l'attore ha inserito la documentazione Web della propria nuova API\\
	\textbf{Scenario Principale} & \begin{enumerate*}[label=(\arabic*.),itemjoin={\newline}]
		\item l'attore può inserire la documentazione Web della propria nuova API
	\end{enumerate*}\\
\end{longtable}



%\subsubsection{Caso d'uso UC11.2: Registrazione Interfaccia API}
\label{UC11.2}

\renewcommand*{\arraystretch}{1.6}
\begin{longtable}{ l | p{11cm}}
	\hline
	\rowcolor{Gray}
	\multicolumn{2}{c}{UC11.2: Registrazione Interfaccia API} \\
	\hline
	\textbf{Attori} &Utente Autenticato, Amministratore APIMarket \\
	\textbf{Descrizione} & l'attore registra l'interfaccia della propria nuova API \\
	\textbf{Pre-Condizioni} & l'attore ha scelto di registrare l'interfaccia della propria nuova API\\
	\textbf{Post-Condizioni}&l'attore ha registrato l'interfaccia della propria nuova API\\
	\textbf{Scenario Principale} & \begin{enumerate*}[label=(\arabic*.),itemjoin={\newline}]
		\item l'attore può registrare l'interfaccia della propria nuova API
	\end{enumerate*}\\
\end{longtable}




%\subsubsection{Caso d'uso UC11.3: Conferma Registrazione Nuova API}
\label{UC11.3}

\renewcommand*{\arraystretch}{1.6}
\begin{longtable}{ l | p{11cm}}
	\hline
	\rowcolor{Gray}
	\multicolumn{2}{c}{UC11.3: Conferma Registrazione Nuova API} \\
	\hline
	\textbf{Attori} &Utente Autenticato, Amministratore APIMarket \\
	\textbf{Descrizione} & l'attore conferma la registrazione della propria nuova API \\
	\textbf{Pre-Condizioni} & l'attore ha scelto di confermare la registrazione della propria nuova API\\
	\textbf{Post-Condizioni}&l'attore ha confermato la registrazione della propria nuova API\\
	\textbf{Scenario Principale} & \begin{enumerate*}[label=(\arabic*.),itemjoin={\newline}]
		\item l'attore può confermare la registrazione della propria nuova API
	\end{enumerate*}\\
\end{longtable}



%\subsubsection{Caso d'uso UC11.4: Visualizzazione Errore Registrazione Nuova API}
\label{UC11.4}

\renewcommand*{\arraystretch}{1.6}
\begin{longtable}{ l | p{11cm}}
	\hline
	\rowcolor{Gray}
	\multicolumn{2}{c}{UC11.4: Visualizzazione Errore Registrazione Nuova API} \\
	\hline
	\textbf{Attori} &Utente Autenticato, Amministratore APIMarket \\
	\textbf{Descrizione} &  l'attore visualizza l'errore nella registrazione della propria nuova API \\
	\textbf{Pre-Condizioni} & l'attore ha scelto di visualizzare l'errore nella registrazione della propria nuova API;\\
	\textbf{Post-Condizioni}&l'attore ha visualizzato l'errore nella registrazione della propria nuova API\\
	\textbf{Scenario Principale} & \begin{enumerate*}[label=(\arabic*.),itemjoin={\newline}]
		\item l'attore può visualizzare l'errore nella registrazione della propria nuova API
	\end{enumerate*}\\
\end{longtable}

\newpage
\subsection{Caso d'uso UC12: Logout}
\label{UC12}
\begin{figure}[ht]
	\centering
	\includegraphics[scale=0.45]{UML/UC12.png}
	\caption{Caso d'uso UC11: Registrazione Nuova API}
\end{figure}

\renewcommand*{\arraystretch}{1.6}
\begin{longtable}{ l | p{11cm}}
	\hline
	\rowcolor{Gray}
	\multicolumn{2}{c}{UC12: Logout} \\
	\hline
	\textbf{Attori} &Utente Autenticato, Amministratore APIMarket \\
	\textbf{Descrizione} &l'attore effettua il logout \\
	\textbf{Pre-Condizioni} &  l'attore ha scelto di effettuare il logout\\
	\textbf{Post-Condizioni}&l'attore ha effettuato il logout\\
	\textbf{Scenario Principale} & \begin{enumerate*}[label=(\arabic*.),itemjoin={\newline}]
			\item L'attore può confermare il logout (UC12.1)
	\end{enumerate*}\\
\end{longtable}


\newpage
\subsubsection{Caso d'uso UC12.1: Conferma Logout}
\label{UC12.1}

\renewcommand*{\arraystretch}{1.6}
\begin{longtable}{ l | p{11cm}}
	\hline
	\rowcolor{Gray}
	\multicolumn{2}{c}{UC12.1: Conferma Logout} \\
	\hline
	\textbf{Attori} &Utente Autenticato, Amministratore APIMarket \\
	\textbf{Descrizione} & l'attore conferma il logout, trasformandosi in un utente non autenticato e venendo indirizzato ad UC1\\
	\textbf{Pre-Condizioni} & l'attore ha scelto di effettuare il logout\\
	\textbf{Post-Condizioni}&l'attore effettua il logout\\
	\textbf{Scenario Principale} & \begin{enumerate*}[label=(\arabic*.),itemjoin={\newline}]
		\item L'attore può confermare il logout, trasformandosi in un utente non autenticato e venendo indirizzato ad UC1
	\end{enumerate*}\\
\end{longtable}

\newpage
\subsection{Caso d'uso UC13: Visualizzazione Dati Utilizzo API}
\label{UC13}
\begin{figure}[ht]
	\centering
	\includegraphics[scale=0.45]{UML/UC13.png}
	\caption{UC13: Visualizzazione Dati Utilizzo API}
\end{figure}

\renewcommand*{\arraystretch}{1.6}
\begin{longtable}{ l | p{11cm}}
	\hline
	\rowcolor{Gray}
	\multicolumn{2}{c}{UC13: Visualizzazione Dati Utilizzo API} \\
	\hline
	\textbf{Attori} &Utente Autenticato, Amministratore APIMarket \\
	\textbf{Descrizione} &l''attore visualizza i dati dell'utilizzo delle API in APIMarket \\
	\textbf{Pre-Condizioni} &   l'attore ha scelto di visualizzare i dati dell'utilizzo delle API in APIMarket\\
	\textbf{Post-Condizioni}& l'attore ha visualizzato i dati dell'utilizzo delle API in APIMarket\\
	\textbf{Scenario Principale} & \begin{enumerate*}[label=(\arabic*.),itemjoin={\newline}]
		\item L'attore può visualizzare il numero di utenti che hanno acquistato una API (UC13.1)
		\item L'attore può registrare l'interfaccia della propria nuova API (UC13.2)
	\end{enumerate*}\\
\end{longtable}





\subsubsection{Caso d'uso UC13.1: Visualizzazione Numero Utenti Acquirenti API}
\label{UC13.1}

\renewcommand*{\arraystretch}{1.6}
\begin{longtable}{ l | p{11cm}}
	\hline
	\rowcolor{Gray}
	\multicolumn{2}{c}{UC13.1: Visualizzazione Numero Utenti} \\
	\hline
	\textbf{Attori} &Utente Autenticato, Amministratore APIMarket \\
	\textbf{Descrizione} & l'attore visualizza i dati dell'utilizzo delle API in APIMarket\\
	\textbf{Pre-Condizioni} & l'attore ha scelto di visualizzare i dati dell'utilizzo delle API in APIMarket\\
	\textbf{Post-Condizioni}&l'attore ha visualizzato i dati dell'utilizzo delle API in APIMarket\\
	\textbf{Scenario Principale} & \begin{enumerate*}[label=(\arabic*.),itemjoin={\newline}]
		\item L'attore p
	\end{enumerate*}\\
\end{longtable}



\newpage
\subsubsection{Caso d'uso UC13.2: Visualizzazione Tempo Utilizzo API}
\label{UC13.2}

\renewcommand*{\arraystretch}{1.6}
\begin{longtable}{ l | p{11cm}}
	\hline
	\rowcolor{Gray}
	\multicolumn{2}{c}{UC13.2: Visualizzazione Tempo Utilizzo API} \\
	\hline
	\textbf{Attori} &Utente Autenticato, Amministratore APIMarket \\
	\textbf{Descrizione} & l'attore visualizza i dati dell'utilizzo delle API in APIMarket\\
	\textbf{Pre-Condizioni} & l'attore ha scelto di visualizzare i dati dell'utilizzo delle API in APIMarket\\
	\textbf{Post-Condizioni}&l'attore ha visualizzato i dati dell'utilizzo delle API in APIMarket\\
	\textbf{Scenario Principale} & \begin{enumerate*}[label=(\arabic*.),itemjoin={\newline}]
		\item L'attore p
	\end{enumerate*}\\
\end{longtable}

\newpage
\subsection{Caso d'uso UC14: Amministrazione applicazione web}
\label{UC14}
\begin{figure}[ht]
	\centering
	\includegraphics[scale=0.45]{UML/UC14.png}
	\caption{UC14: Amministrazione applicazione web}
\end{figure}

\renewcommand*{\arraystretch}{1.6}
\begin{longtable}{ l | p{11cm}}
	\hline
	\rowcolor{Gray}
	\multicolumn{2}{c}{UC14: Amministrazione applicazione web} \\
	\hline
	\textbf{Attori} & Amministratore API Market \\
	\textbf{Descrizione} & L'attore può gestire la parte riservata della piattaforma, ed effettuare operazioni super-user su utenza, prodotti registrati e sulla piattaforma stessa \\
	\textbf{Pre-Condizioni} & L'attore visita la pagina relativa all'amministrazione della piattaforma API Market\\
	\textbf{Post-Condizioni}& L'attore ha effettuato le modifiche desiderate, o ha consultato i dati desiderati, all'interno della piattaforma\\
	\textbf{Scenario Principale} & \begin{enumerate*}[label=(\arabic*.),itemjoin={\newline}]
		\item L'attore può consultare i dati di utilizzo avanzati per un API (UC14.1)
		\item L'attore può moderare l'utenza predisponendo sospensioni (UC14.2)
	\end{enumerate*}\\
\end{longtable}


\subsubsection{Caso d'uso UC14.1: Visualizzazione dati di utilizzo avanzati}
\label{UC14_1}

\begin{minipage}{\linewidth}
	\begin{tabular}{ l | p{11cm}}
		\hline
		\rowcolor{Gray}
		\multicolumn{2}{c}{UC14.1 - Visualizzazione dati di utilizzo avanzati} \\
		\hline
		\textbf{Attori} &  Amministratore API Market \\
		\textbf{Descrizione} & L'attore visualizza nella schermata relativa ai dati di utilizzo dell'API \\
		\textbf{Pre-Condizioni} & L'attore ha selezionato la visualizzazione dati per un API \\
		\textbf{Post-Condizioni} & L'attore ha visualizzato i dati di utilizzo avanzati dell'API selezionata \\
		\textbf{Scenario Principale} & 
		\begin{enumerate*}[label=(\arabic*.),itemjoin={\newline}]
			\item L'attore può visualizzare il numero di licenze attive per l'API selezionata (UC7.7.1)
			\item L'attore può visualizzare il numero di chiamate giornaliere effettuate all'API selezionata (UC7.7.2)
			\item L'attore può visualizzare il tempo medio di utilizzo dell'API selezionata (UC7.7.3)
			\item L'attore può visualizzare il traffico medio giornaliero dell'API selezionata (UC7.7.4)
			\item L'attore può visualizzare la lista di utenti che hanno una licenza attiva (UC14.1.1)
			\item L'attore può visualizzare il tempo medio di risposta (UC14.1.2)
		\end{enumerate*}\\
	\end{tabular}
\end{minipage}

\paragraph{Caso d'uso UC14.1.1: Visualizzazione utenti attivi per API}
\label{UC14_1_1}

\begin{minipage}{\linewidth}
	\begin{tabular}{ l | p{11cm}}
		\hline
		\rowcolor{Gray}
		\multicolumn{2}{c}{UC14.1.1 - Visualizzazione utenti attivi per API} \\
		\hline
		\textbf{Attori} & Amministratore API Market \\
		\textbf{Descrizione} & L'attore visualizza una lista di utenti attivi per l'API selezionata \\
		\textbf{Pre-Condizioni} & L'attore ha selezionato un API per il quale visualizzare i dati di utilizzo avanzati\\
		\textbf{Post-Condizioni} & L'attore ha visualizzato la lista di licenze attive per l'API selezionata \\
		\textbf{Scenario Principale} & 
		\begin{enumerate*}[label=(\arabic*.),itemjoin={\newline}]
			\item L'attore può visualizzare il nome dell'utente (UC14.1.1.1)
			\item L'attore può visualizzare la durata della licenza (UC14.1.1.2)
		\end{enumerate*}\\
	\end{tabular}
\end{minipage}

\subparagraph{Caso d'uso UC14.1.1.1: Visualizzazione nome}
\label{UC14_1_1_1}

\begin{minipage}{\linewidth}
	\begin{tabular}{ l | p{11cm}}
		\hline
		\rowcolor{Gray}
		\multicolumn{2}{c}{UC14.1.1.1 - Visualizzazione nome} \\
		\hline
		\textbf{Attori} & Amministratore API Market \\
		\textbf{Descrizione} & L'attore può visualizzare il nome dell'utente interessato\\
		\textbf{Pre-Condizioni} & L'attore è nella schermata di visualizzazione degli utenti con licenza attiva per l'API selezionata\\
		\textbf{Post-Condizioni} & L'attore ha visualizzato il nome interessato \\
		\textbf{Scenario Principale} & 
		\begin{enumerate*}[label=(\arabic*.),itemjoin={\newline}]
			\item L'attore può visualizzare il nome dell'utente corrispondente
		\end{enumerate*}
	\end{tabular}
\end{minipage}

\subparagraph{Caso d'uso UC14.1.1.2: Visualizzazione durata residua licenza}
\label{UC14_1_1_2}

\begin{minipage}{\linewidth}
	\begin{tabular}{ l | p{11cm}}
		\hline
		\rowcolor{Gray}
		\multicolumn{2}{c}{UC14.1.1.2 -  Visualizzazione durata residua licenza} \\
		\hline
		\textbf{Attori} & Amministratore API Market \\
		\textbf{Descrizione} & L'attore può visualizzare la durata residua della licenza dell'utente interessato\\
		\textbf{Pre-Condizioni} & L'attore è nella schermata di visualizzazione degli utenti con licenza attiva per l'API selezionata\\
		\textbf{Post-Condizioni} & L'attore ha visualizzato la data di scadenza \\
		\textbf{Scenario Principale} & 
		\begin{enumerate*}[label=(\arabic*.),itemjoin={\newline}]
			\item L'attore può visualizzare la scadenza per l'utente visualizzato riguardante l'API selezionata
		\end{enumerate*}
	\end{tabular}
\end{minipage}

\paragraph{Caso d'uso UC14.1.2: Visualizzazione tempo medio di risposta}
\label{UC14_1_2}

\begin{minipage}{\linewidth}
	\begin{tabular}{ l | p{11cm}}
		\hline
		\rowcolor{Gray}
		\multicolumn{2}{c}{UC14.1.2 - Visualizzazione tempo medio di risposta} \\
		\hline
		\textbf{Attori} & Amministratore API Market \\
		\textbf{Descrizione} & L'attore visualizza il tempo medio di risposta per l'API selezionata \\
		\textbf{Pre-Condizioni} & L'attore ha selezionato un API per il quale visualizzare i dati di utilizzo avanzati\\
		\textbf{Post-Condizioni} & L'attore ha visualizzato il tempo medio di risposta per l'API selezionata \\
		\textbf{Scenario Principale} & 
		\begin{enumerate*}[label=(\arabic*.),itemjoin={\newline}]
			\item L'attore può visualizzare il tempo medio di risposta per l'API selezionata
		\end{enumerate*}\\
	\end{tabular}
\end{minipage}

\subsubsection{Caso d'uso UC14.2: Azioni utente}
\label{UC14_2}

\begin{minipage}{\linewidth}
	\begin{tabular}{ l | p{11cm}}
		\hline
		\rowcolor{Gray}
		\multicolumn{2}{c}{UC14.2 - Azioni utente} \\
		\hline
		\textbf{Attori} &  Amministratore API Market \\
		\textbf{Descrizione} & L'attore visualizza nella schermata relativa ai dati di utilizzo dell'API \\
		\textbf{Pre-Condizioni} & L'attore ha selezionato la visualizzazione dati per un API \\
		\textbf{Post-Condizioni} & L'attore ha visualizzato i dati di utilizzo avanzati dell'API selezionata \\
		\textbf{Scenario Principale} & 
		\begin{enumerate*}[label=(\arabic*.),itemjoin={\newline}]
			\item L'attore può inserire il nome di un utente su cui effettuare un azione (UC14.2.1)
			\item L'attore può sospendere l'utente selezionato (UC14.2.2)
			\item L'attore può sospendere i prelievi di denaro dal proprio conto per l'utente selezionato (UC14.2.3)
			\item L'attore può rimuovere una sospensione utente (UC14.2.4)
			\item L'attore può rimuovere la sospensione dei prelievi (UC14.2.5)
		\end{enumerate*}\\
		\textbf{Scenari Alternativi} & 
		\begin{enumerate*}[label=(\arabic*.),itemjoin={\newline}]
			\item L'attore riceve un messaggio d'errore qualora l'utente inserito non esista. Può dunque ritentare la procedura
		\end{enumerate*}\\
	\end{tabular}
\end{minipage}

\paragraph{Caso d'uso UC14.2.1: Inserimento username}
\label{UC14_2_1}

\begin{minipage}{\linewidth}
	\begin{tabular}{ l | p{11cm}}
		\hline
		\rowcolor{Gray}
		\multicolumn{2}{c}{UC14.2.1 - Inserimento username} \\
		\hline
		\textbf{Attori} & Amministratore API Market \\
		\textbf{Descrizione} & L'attore inserisce l'username di un utente sul quale effettuare operazioni \\
		\textbf{Pre-Condizioni} & L'attore ha scelto di effettuare azioni su un utente\\
		\textbf{Post-Condizioni} & L'attore ha inserito il nome di un utente \\
		\textbf{Scenario Principale} & 
		\begin{enumerate*}[label=(\arabic*.),itemjoin={\newline}]
			\item L'attore può inserire il nome di un utente
		\end{enumerate*}\\
	\end{tabular}
\end{minipage}

\paragraph{Caso d'uso UC14.2.2: Sospensione utente}
\label{UC14_2_2}

\begin{minipage}{\linewidth}
	\begin{tabular}{ l | p{11cm}}
		\hline
		\rowcolor{Gray}
		\multicolumn{2}{c}{UC14.2.2 - Sospensione utente} \\
		\hline
		\textbf{Attori} & Amministratore API Market \\
		\textbf{Descrizione} & L'attore può sospendere l'utente indicato \\
		\textbf{Pre-Condizioni} & L'attore ha inserito il nome utente da sospendere\\
		\textbf{Post-Condizioni} & L'attore ha sospeso un utente \\
		\textbf{Scenario Principale} & 
		\begin{enumerate*}[label=(\arabic*.),itemjoin={\newline}]
			\item L'attore può inserire la durata in giorni della sospensione (UC14.2.2.1)
			\item L'attore può confermare la scelta (UC14.2.2.2)
		\end{enumerate*}\\
	\end{tabular}
\end{minipage}

\subparagraph{Caso d'uso UC14.2.2.1: Durata sospensione utente}
\label{UC14_2_2_1}

\begin{minipage}{\linewidth}
	\begin{tabular}{ l | p{11cm}}
		\hline
		\rowcolor{Gray}
		\multicolumn{2}{c}{UC14.2.2.1 - Durata sospensione utente} \\
		\hline
		\textbf{Attori} & Amministratore API Market \\
		\textbf{Descrizione} & L'attore può indicare quanto deve durare la sospensione per l'utente indicato \\
		\textbf{Pre-Condizioni} & L'attore si trova nella schermata per sospendere un utente\\
		\textbf{Post-Condizioni} & L'attore ha indicato la durata della sospensione \\
		\textbf{Scenario Principale} & 
		\begin{enumerate*}[label=(\arabic*.),itemjoin={\newline}]
			\item L'attore può inserire il numero di giorni di durata della sospensione
		\end{enumerate*}\\
	\end{tabular}
\end{minipage}

\subparagraph{Caso d'uso UC14.2.2.2: Conferma sospensione utente}
\label{UC14_2_2_2}

\begin{minipage}{\linewidth}
	\begin{tabular}{ l | p{11cm}}
		\hline
		\rowcolor{Gray}
		\multicolumn{2}{c}{UC14.2.2.2 - Conferma sospensione utente} \\
		\hline
		\textbf{Attori} & Amministratore API Market \\
		\textbf{Descrizione} & L'attore può confermare la sospensione indicata \\
		\textbf{Pre-Condizioni} & L'attore si trova nella schermata per sospendere un utente e ha inserito la durata in giorni\\
		\textbf{Post-Condizioni} & L'attore ha sospeso con successo un utente \\
		\textbf{Scenario Principale} & 
		\begin{enumerate*}[label=(\arabic*.),itemjoin={\newline}]
			\item L'attore conferma la scelta e viene notificato dell'avvenuta sospensione
		\end{enumerate*}\\
		\textbf{Scenari Alternativi} & 
		\begin{enumerate*}[label=(\arabic*.),itemjoin={\newline}]
			\item L'attore non ha inserito la durata correttamente e viene notificato dell'errore.
		\end{enumerate*}\\
	\end{tabular}
\end{minipage}

\paragraph{Caso d'uso UC14.2.3: Sospensione pagamenti utente}
\label{UC14_2_3}

\begin{minipage}{\linewidth}
	\begin{tabular}{ l | p{11cm}}
		\hline
		\rowcolor{Gray}
		\multicolumn{2}{c}{UC14.2.3 - Sospensione pagamenti utente} \\
		\hline
		\textbf{Attori} & Amministratore API Market \\
		\textbf{Descrizione} & L'attore può sospendere i prelievi per l'utente indicato \\
		\textbf{Pre-Condizioni} & L'attore ha inserito il nome utente da sospendere\\
		\textbf{Post-Condizioni} & L'attore ha sospeso i pagamenti per l'utente selezionato per la risoluzione di eventuali contestazioni \\
		\textbf{Scenario Principale} & 
		\begin{enumerate*}[label=(\arabic*.),itemjoin={\newline}]
			\item L'attore può confermare la sospensione dei prelievi di denaro di un utente, per risolvere eventuali contestazioni
		\end{enumerate*}\\
		\textbf{Scenari Alternativi} & 
		\begin{enumerate*}[label=(\arabic*.),itemjoin={\newline}]
			\item L'attore viene notificato che l'utente possiede una sospensione già in atto.
		\end{enumerate*}\\
	\end{tabular}
\end{minipage}

\paragraph{Caso d'uso UC14.2.4: Revoca sospensione utente}
\label{UC14_2_4}

\begin{minipage}{\linewidth}
	\begin{tabular}{ l | p{11cm}}
		\hline
		\rowcolor{Gray}
		\multicolumn{2}{c}{UC14.2.4 - Revoca sospensione utente} \\
		\hline
		\textbf{Attori} & Amministratore API Market \\
		\textbf{Descrizione} & L'attore può revocare una sospensione per l'utente indicato \\
		\textbf{Pre-Condizioni} & L'attore ha inserito il nome utente su cui revocare la sospensione\\
		\textbf{Post-Condizioni} & L'attore revocato la sospensione per l'utente selezionato \\
		\textbf{Scenario Principale} & 
		\begin{enumerate*}[label=(\arabic*.),itemjoin={\newline}]
			\item L'attore può revocare una sospensione per un utente
		\end{enumerate*}\\
		\textbf{Scenari Alternativi} & 
		\begin{enumerate*}[label=(\arabic*.),itemjoin={\newline}]
			\item L'attore viene notificato che l'utente selezionato non ha una sospensione in atto
		\end{enumerate*}\\
	\end{tabular}
\end{minipage}

\paragraph{Caso d'uso UC14.2.5: Revoca sospensione pagamenti utente}
\label{UC14_2_5}

\begin{minipage}{\linewidth}
	\begin{tabular}{ l | p{11cm}}
		\hline
		\rowcolor{Gray}
		\multicolumn{2}{c}{UC14.2.5 - Revoca sospensione pagamenti utente} \\
		\hline
		\textbf{Attori} & Amministratore API Market \\
		\textbf{Descrizione} & L'attore può revocare una sospensione di pagamenti per l'utente indicato \\
		\textbf{Pre-Condizioni} & L'attore ha inserito il nome utente su cui revocare la sospensione dei pagamenti\\
		\textbf{Post-Condizioni} & L'attore revocato la sospensione per l'utente selezionato in seguito alla risoluzione delle contestazioni in atto \\
		\textbf{Scenario Principale} & 
		\begin{enumerate*}[label=(\arabic*.),itemjoin={\newline}]
			\item L'attore può revocare una sospensione per un utente
		\end{enumerate*}\\
		\textbf{Scenari Alternativi} & 
		\begin{enumerate*}[label=(\arabic*.),itemjoin={\newline}]
			\item L'attore viene notificato che l'utente selezionato non ha una sospensione dei pagamenti in atto
		\end{enumerate*}\\
	\end{tabular}
\end{minipage}

