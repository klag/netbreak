\newpage
\section{Casi d'uso}
Vengono elencati i casi d'uso ricavati dall'analisi del capitolato C1.

\subsection{Attori}
Di seguito è riportato il diagramma UML che descrive la gerarchia degli attori. Sono sono state individuate 3 tipologie differenti di attori con funzionalità crescenti. Gli attori sono: l'utente non autenticato, l'utente autenticato e l'amministratore API Market (un attore autenticato con alcune funzionalità superuser). Un quarto attore, chiamato utente generico, viene utilizzato nei diagrammi UML per indicare funzionalità fruibili indipendentemente dallo status di autenticazione.

\label{Attori}

\begin{figure}[ht]
	\centering
	\includegraphics[scale=0.45]{UML/attori.png}
	\caption{Attori}
\end{figure}


\newpage
\subsection{Caso d'uso UC1:  Main }
\label{UC1}
\begin{figure}[ht]
	\centering
	\includegraphics[scale=0.45]{UML/UC1.png}
	\caption{UC1: Main}
\end{figure}

\renewcommand*{\arraystretch}{1.6}
\begin{longtable}{ l | p{11cm}}
	\hline
	\rowcolor{Gray}
	 \multicolumn{2}{c}{UC1 - Main} \\
	 \hline
	\textbf{Attori} & Utente Non Autenticato  \\
	\textbf{Descrizione} & L'attore, tramite la schermata principale dell'applicazione, può accedere e sfruttare le funzionalità a lui disponibili: la registrazione, il login, il recupero password, la ricerca API  \\
	\textbf{Pre-Condizioni} & L'attore ha avviato l'applicazione web e non si è ancora autenticato \\
	\textbf{Post-Condizioni}&L'applicazione ha eseguito le richieste dell'attore\\
	\textbf{Scenario Principale} & \begin{enumerate*}[label=(\arabic*.),itemjoin={\newline}]
		\item L'attore può registrarsi all'applicazione (UC2)
		\item L'attore può effettuare il login all'applicazione (UC3)
		\item L'attore può recuperare la propria password (UC4)
		\item L'attore può effettuare una ricerca sulle API presenti nell'applicazione (UC6)
	\end{enumerate*}\\
\end{longtable}
\newpage
\subsection{Caso d'uso UC2:  Main post-autenticazione }
\label{UC2}
\begin{figure}[ht]
	\centering
	\includegraphics[scale=0.45]{UML/UC2.png}
	\caption{UC2: Main post-autenticazione}
\end{figure}

\begin{longtable}{ l | p{11cm}}
	\hline
	\rowcolor{Gray}
	 \multicolumn{2}{c}{UC2 - Main post-autenticazione} \\
	 \hline
	\textbf{Attori} & Utente autenticato, Amministratore API Market \\
	\textbf{Descrizione} & L'attore tramite la schermata principale
	dell'applicazione, può accedere e sfruttare le funzionalità a lui disponibili: l'interazione
	con il proprio profilo utente, con le API non acquistate e non, con le API registrate, la
	registrazione di una nuova API, il logout. 
	L'Amministratore API Market, oltre alle funzionalità offerte all'utente autenticato, può
	visualizzare i dati di utilizzo delle API ed amministrare l'applicazione web.  \\
	\textbf{Pre-Condizioni} & L'attore ha avviato l'applicazione web e si è autenticato \\
	\textbf{Post-Condizioni} & L'applicazione ha eseguito le richieste dell'attore \\
	\textbf{Scenario Principale} & 
	\begin{enumerate*}[label=(\arabic*.),itemjoin={\newline}]
		\item L'attore può interagire con il proprio profilo utente
		\item L'attore può interagire con le API da lui non acquistate
		\item L'attore può interagire con le API da lui acquistate
		\item L'attore può interagire con le API da lui registrate
		\item L'attore può registrare una nuova API
		\item L'attore può effettuare il logout
		\item L'attore Amministratore API Market può visualizzare i dati di utilizzo delle singole API 
		\item L'attore Amministratore API Market può amministrare l'applicazione web
	\end{enumerate*}\\
\end{longtable}
\newpage
\subsection{Caso d'uso UC3: Registrazione utente }
\label{UC3}
\begin{figure}[ht]
	\centering
	\includegraphics[scale=0.45]{UML/UC3.png}
	\caption{UC3: Registrazione utente}
\end{figure}

\begin{longtable}{ l | p{11cm}}
	\hline
	\rowcolor{Gray}
	 \multicolumn{2}{c}{UC3 - Registrazione utente} \\
	 \hline
	\textbf{Attori} & Utente non autenticato \\
	\textbf{Descrizione} & L'attore inserisce le sue informazioni personali per potersi registrare all'applicazione web, così da poter successivamente effettuare il login ed evolversi in un utente autenticato. \\
	\textbf{Pre-Condizioni} & L'attore ha scelto di registrarsi e l'applicazione web mostra la schermata di registrazione \\
	\textbf{Post-Condizioni} & L'attore si è registrato all'applicazione web \\
	\textbf{Scenario Principale} & \begin{enumerate*}[label=(\arabic*.),itemjoin={\newline}]
		\item L'attore può inserire il proprio nome (UC3.1)
		\item L'attore può inserire il proprio cognome (UC3.2)
		\item L'attore può inserire lo username desiderato (UC3.3)
		\item L'attore può inserire la propria email (UC3.4) 
		\item L'attore può inserire la password desiderata (UC3.5)
		\item L'attore può reinserire la password desiderata per conferma (UC3.6)
		\item L'attore può confermare la registrazione (UC3.7)
	\end{enumerate*}\\
	\textbf{Scenari Alternativi} & 
	\begin{enumerate*}[label=(\arabic*.),itemjoin={\newline}]
		\item L'attore può visualizzare un messaggio d'errore che segnala i campi dati non validi (UC3.8)
	\end{enumerate*}\\
\end{longtable}
\subsubsection{Caso d'uso UC3.1:  Inserimento Nome}
\label{UC3_1}

\begin{longtable}{ l | p{11cm}}
	\hline
	\rowcolor{Gray}
	 \multicolumn{2}{c}{UC3.1 - Inserimento Nome} \\
	 \hline
	\textbf{Attori} & Utente Non Autenticato \\
	\textbf{Descrizione} & L'utente non autenticato inserisce il suo nome  \\
	\textbf{Pre-Condizioni} & L'utente ha scelto di registrarsi e l'applicazione web mostra la schermata di registrazione \\
	\textbf{Post-Condizioni} & L'utente visualizza area per registrazione a applicazione web \\
	\textbf{Scenario Principale} & \begin{enumerate*}[label=(\arabic*.),itemjoin={\newline}]
		\item L'utente non autenticato può inserire il proprio Nome (UC3.1)
	\end{enumerate*}\\
	\textbf{Scenari Alternativi} & 
	\begin{enumerate*}[label=(\arabic*.),itemjoin={\newline}]
		\item Il Nome non e' valido perche' contiene caratteri particolari
	\end{enumerate*}\\
\end{longtable}
\subsubsection{Caso d'uso UC3.2: Inserimento cognome}
\label{UC3_2}

\begin{longtable}{ l | p{11cm}}
	\hline
	\rowcolor{Gray}
	 \multicolumn{2}{c}{UC3.2 - Inserimento cognome} \\
	 \hline
	\textbf{Attori} & Utente non autenticato \\
	\textbf{Descrizione} & L'attore inserisce il suo cognome \\
	\textbf{Pre-Condizioni} & L'applicazione mostra il campo dati per l'inserimento del cognome \\
	\textbf{Post-Condizioni} & L'attore ha inserito il proprio cognome \\
	\textbf{Scenario Principale} & \begin{enumerate*}[label=(\arabic*.),itemjoin={\newline}]
		\item L'attore può inserire il proprio cognome
	\end{enumerate*}\\
\end{longtable}
\subsubsection{Caso d'uso UC3.3:  Inserimento username}
\label{UC3_3}

\begin{longtable}{ l | p{11cm}}
	\hline
	\rowcolor{Gray}
	 \multicolumn{2}{c}{UC3.3:  Inserimento username} \\
	 \hline
	\textbf{Attori} & Utente non autenticato \\
	\textbf{Descrizione} & L'attore inserisce il suo Username  \\
	\textbf{Pre-Condizioni} & L'applicazione visualizza i form per l'inserimento del campo per lo username \\
	\textbf{Post-Condizioni} & L'attore ha inserito lo username desiderato \\
	\textbf{Scenario Principale} & \begin{enumerate*}[label=(\arabic*.),itemjoin={\newline}]
		\item L'utente non autenticato può inserire il proprio username
	\end{enumerate*}\\
\end{longtable}
\subsubsection{Caso d'uso UC3.4:  Registrazione email}
\label{UC3_4}

\begin{longtable}{ l | p{11cm}}
	\hline
	\rowcolor{Gray}
	 \multicolumn{2}{c}{UC3.4 - Inserimento email} \\
	 \hline
	\textbf{Attori} & Utente non autenticato \\
	\textbf{Descrizione} & L'attore inserisce la propria email  \\
	\textbf{Pre-Condizioni} & L'applicazione visualizza i form per l'inserimento del campo per l'email \\
	\textbf{Post-Condizioni} & L'attore ha inserito la propria email \\
	\textbf{Scenario Principale} & \begin{enumerate*}[label=(\arabic*.),itemjoin={\newline}]
		\item L'utente non autenticato può inserire la propria email
	\end{enumerate*}\\
\end{longtable}

\subsubsection{Caso d'uso UC3.5: Inserimento password}
\label{UC3_5}

\begin{longtable}{ l | p{11cm}}
	\hline
	\rowcolor{Gray}
	 \multicolumn{2}{c}{UC3.5 - Inserimento password} \\
	 \hline
	\textbf{Attori} & Utente non autenticato \\
	\textbf{Descrizione} & L'attore inserisce la password desiderata  \\
	\textbf{Pre-Condizioni} & L'applicazione mostra il campo dati per l'inserimento della password \\
	\textbf{Post-Condizioni} & L'attore ha inserito la password desiderata \\
	\textbf{Scenario Principale} & \begin{enumerate*}[label=(\arabic*.),itemjoin={\newline}]
		\item L'attore può inserire la password desiderata
	\end{enumerate*}\\
\end{longtable}
\subsubsection{Caso d'uso UC3.6: Reinserimento password}
\label{UC3_6}

\begin{longtable}{ l | p{11cm}}
	\hline
	\rowcolor{Gray}
	 \multicolumn{2}{c}{UC3.6 - Reinserimento password} \\
	 \hline
	\textbf{Attori} & Utente non autenticato \\
	\textbf{Descrizione} & L'attore reinserisce la password desiderata  \\
	\textbf{Pre-Condizioni} & L'applicazione mostra il campo dati per il reinserimento della password \\
	\textbf{Post-Condizioni} & L'attore ha reinserito la password desiderata \\
	\textbf{Scenario Principale} & \begin{enumerate*}[label=(\arabic*.),itemjoin={\newline}]
		\item L'attore può reinserire la password desiderata
	\end{enumerate*}\\
\end{longtable}
\subsubsection{Caso d'uso UC3.7: Conferma registrazione}
\label{UC3_7}

\begin{longtable}{ l | p{11cm}}
	\hline
	\rowcolor{Gray}
	 \multicolumn{2}{c}{UC3.7 - Conferma registrazione} \\
	 \hline
	\textbf{Attori} & Utente non autenticato \\
	\textbf{Descrizione} & L'attore conferma i dati inseriti per la registrazione \\
	\textbf{Pre-Condizioni} & L'applicazione mostra il pulsante per la conferma della registrazione (con le informazioni inserite nei campi dati) \\
	\textbf{Post-Condizioni} & L'attore ha confermato la registrazione, visualizzando il messaggio di successo, ed è stato reindirizzato alla pagina principale come utente non autenticato (UC1) \\
	\textbf{Scenario Principale} & \begin{enumerate*}[label=(\arabic*.),itemjoin={\newline}]
		\item L'attore può confermare la registrazione, visualizzando il messaggio di successo, e venendo reindirizzato alla pagina principale come utente non autenticato (UC1)
	\end{enumerate*}\\
\end{longtable}
\subsubsection{Caso d'uso UC3.8:  Visualizza Errore Registrazione}
\label{UC3_8}

\begin{longtable}{ l | p{11cm}}
	\hline
	\rowcolor{Gray}
	 \multicolumn{2}{c}{UC3.8 - Visualizza Errore Registrazione} \\
	 \hline
	\textbf{Attori} & Utente Non Autenticato \\
	\textbf{Descrizione} & L'utente non autenticato visualizza un messaggio d'errore circa la sua registrazione  \\
	\textbf{Pre-Condizioni} & L'utente ha scelto di registrarsi e inserito dati errati di registrazione \\
	\textbf{Post-Condizioni} & L'utente visualizza un messaggio d'errore \\
	\textbf{Scenario Principale} & \begin{enumerate*}[label=(\arabic*.),itemjoin={\newline}]
		\item L'utente non autenticato Visualizza un Errore di Registrazione (UC3.1)
	\end{enumerate*}\\
\end{longtable}
\newpage
\subsection{Caso d'uso UC4: Login}
\label{UC4}
\begin{figure}[ht]
	\centering
	\includegraphics[scale=0.45]{UML/UC4.png}
	\caption{UC4: Login}
\end{figure}

\begin{longtable}{ l | p{11cm}}
	\hline
	\rowcolor{Gray}
	 \multicolumn{2}{c}{UC4 - Login}\\
	 \hline
	\textbf{Attori} & Utente non autenticato \\
	\textbf{Descrizione} & L'attore inserisce le sue informazioni personali per poter accedere all'applicazione web ed evolversi in un utente autenticato. Può effettuare login tramite Facebook, Twitter, LinkedIn, Google+ e tramite l'API Market stesso. \\
	\textbf{Pre-Condizioni} & L'attore ha scelto di effettuare l'accesso e l'applicazione web mostra la schermata di login \\
	\textbf{Post-Condizioni} & L'attore ha effettuato l'accesso all'applicazione web \\
	\textbf{Scenario Principale} & \begin{enumerate*}[label=(\arabic*.),itemjoin={\newline}]
		\item L'attore può effettuare il login tramite API Market (UC4.1)
		\item L'attore può effettuare il login tramite Facebook (UC4.2)
		\item L'attore può effettuare il login tramite Twitter (UC4.3)
		\item L'attore può effettuare il login tramite LinkedIn (UC4.4)
		\item L'attore può effettuare il login tramite Google+ (UC4.5)
	\end{enumerate*}\\
\end{longtable}
\newpage
\subsubsection{Caso d'uso UC4.1: Login tramite API Market}
\label{UC4_1}
\begin{figure}[!htbp]
	\centering
	\includegraphics[scale=0.45]{UML/UC4_1.png}
	\caption{UC4.1: Login tramite API Market}
\end{figure}

\begin{tabular}{ l | p{11cm}}
	\hline
	\rowcolor{Gray}
	 \multicolumn{2}{c}{UC4.1 - Login tramite API Market} \\
	 \hline
	\textbf{Attori} & Utente non autenticato \\
	\textbf{Descrizione} & L'attore effettua il login all'applicazione web, così da evolversi in un utente autenticato\\
	\textbf{Pre-Condizioni} & L'attore ha scelto di eseguire il login all'applicazione web e non è autenticato \\
	\textbf{Post-Condizioni} & L'attore ha effettuato il login all'applicazione web, evolvendosi in un utente autenticato \\
	\textbf{Scenario Principale} & 
	\begin{enumerate*}[label=(\arabic*.),itemjoin={\newline}]
		\item L'attore può inserire l'email o username (UC4.1.1)
		\item L'attore non autenticato può inserire la password (UC4.1.2)
		\item L'attore può confermare i dati inseriti per procedere al login (UC4.1.3)
	\end{enumerate*}\\
\end{tabular}

\paragraph{Caso d'uso UC4.1.1:  Inserimento username o email}
\label{UC4_1_1}

\begin{longtable}{ l | p{11cm}}
	\hline
	\rowcolor{Gray}
	\multicolumn{2}{c}{Caso d'uso UC4.1.1:  Inserimento username o email} \\
	\hline
	\textbf{Attori} & Utente non autenticato \\
	\textbf{Descrizione} & L'attore inserisce la propria username o email  \\
	\textbf{Pre-Condizioni} & L'applicazione visualizza i form per l'inserimento del campo per lo username \\
	\textbf{Post-Condizioni} & L'attore ha inserito il proprio username o email \\
	\textbf{Scenario Principale} & \begin{enumerate*}[label=(\arabic*.),itemjoin={\newline}]
		\item L'utente non autenticato può inserire il proprio username o la propria email
	\end{enumerate*}\\
\end{longtable}

\paragraph{Caso d'uso UC4.1.2:  Inserimento password}
\label{UC4_1_2}

\begin{longtable}{ l | p{11cm}}
	\hline
	\rowcolor{Gray}
	\multicolumn{2}{c}{Caso d'uso UC4.1.2:  Inserimento password} \\
	\hline
	\textbf{Attori} & Utente non autenticato \\
	\textbf{Descrizione} & L'attore inserisce la propria password  \\
	\textbf{Pre-Condizioni} & L'applicazione visualizza i form per l'inserimento del campo per la password \\
	\textbf{Post-Condizioni} & L'attore ha inserito la propria password \\
	\textbf{Scenario Principale} & \begin{enumerate*}[label=(\arabic*.),itemjoin={\newline}]
		\item L'utente non autenticato può inserire la propria password
	\end{enumerate*}\\
\end{longtable}

\paragraph{Caso d'uso UC4.1.3:  Inserimento username o email}
\label{UC4_1_3}

\begin{longtable}{ l | p{11cm}}
	\hline
	\rowcolor{Gray}
	\multicolumn{2}{c}{Caso d'uso UC4.1.3:  Inserimento username o email} \\
	\hline
	\textbf{Attori} & Utente non autenticato \\
	\textbf{Descrizione} & L'attore conferma i dati inseriti per il login  \\
	\textbf{Pre-Condizioni} & L'applicazione visualizza il pulsante per confermare i dati di accesso \\
	\textbf{Post-Condizioni} & L'attore ha confermato i dati inseriti \\
	\textbf{Scenario Principale} & \begin{enumerate*}[label=(\arabic*.),itemjoin={\newline}]
	\item L'attore convalida i dati inseriti per procedere con il login, e verrà reindirizzato alla schermata principale post-autenticazione (UC2) oppure, in caso di login fallito, viene consentito all'utente di ritentare dall'apposita schermata (UC4.1)
\end{enumerate*}\\
\end{longtable}
\subsubsection{Caso d'uso UC4.2:  Login Tramite Facebook }
\label{UC4_2}
\begin{figure}[ht]
	\centering
	\includegraphics[scale=0.45]{UML/UC4_2.png}
	\caption{UC4.2: Login Tramite Facebook}
\end{figure}

\begin{tabular}{ l | p{11cm}}
	\hline
	\rowcolor{Gray}
	 \multicolumn{2}{c}{UC4.2 - Login Tramite Facebook} \\
	 \hline
	\textbf{Attori} & Utente Non Autenticato, Facebook \\
	\textbf{Descrizione} & L'utente non autenticato effettua il login all'applicazione web tramite Facebook, così da evolversi in un utente autenticato\\
	\textbf{Pre-Condizioni} & L'utente ha scelto di eseguire il login all'applicazione web e non è autenticato \\
	\textbf{Post-Condizioni} & L'utente ha effettuato il login all'applicazione web tramite Facebook, evolvendosi in un utente autenticato \\
	\textbf{Scenario Principale} & \begin{enumerate*}[label=(\arabic*.),itemjoin={\newline}]
		\item L'utente non autenticato può effettuare il login all'applicazione web tramite Facebook (UC4.2.1)
	\end{enumerate*}\\
\end{tabular}
\newpage
\subsubsection{Caso d'uso UC4.3: Login tramite Twitter }
\label{UC4_2}
\begin{figure}[!htbp]
	\centering
	\includegraphics[scale=0.45]{UML/UC4_3.png}
	\caption{UC4.3: Login tramite Twitter}
\end{figure}

\begin{tabular}{ l | p{11cm}}
	\hline
	\rowcolor{Gray}
	\multicolumn{2}{c}{UC4.3 - Login tramite Twitter} \\
	\hline
	\textbf{Attori} & Utente non autenticato, Twitter \\
	\textbf{Descrizione} & L'attore Utente non autenticato effettua il login all'applicazione web tramite Twitter, così da evolversi in un utente autenticato\\
	\textbf{Pre-Condizioni} & L'attore Utente non autenticato ha scelto di eseguire il login all'applicazione web e non è autenticato \\
	\textbf{Post-Condizioni} & L'attore Utente non autenticato ha effettuato il login all'applicazione web tramite Twitter, evolvendosi in un utente autenticato \\
	\textbf{Scenario Principale} & \begin{enumerate*}[label=(\arabic*.),itemjoin={\newline}]
		\item L'attore effettua il login con successo tramite Twitter, evolvendosi in un Utente autenticato (UC2)
	\end{enumerate*}\\
	\textbf{Scenari Alternativi} & \begin{enumerate*}[label=(\arabic*.),itemjoin={\newline}]
		\item L'attore ha fallito il login tramite Twitter (E.g: Mancanza di privilegi e autorizzazioni, utente non loggato correttamente a Twitter)
	\end{enumerate*}\\
\end{tabular}
\subsubsection{Caso d'uso UC4.4: Login Tramite LinkedIn }
\label{UC4_4}
\begin{figure}[ht]
	\centering
	\includegraphics[scale=0.45]{UML/UC4_4.png}
	\caption{UC4.4: Login Tramite LinkedIn}
\end{figure}

\begin{longtable}{ l | p{11cm}}
	\hline
	\rowcolor{Gray}
	 \multicolumn{2}{c}{UC4.4 - Login Tramite LinkedIn} \\
	 \hline
	\textbf{Attori} & Utente Non Autenticato, LinkedIn \\
	\textbf{Descrizione} & L'utente non autenticato effettua il login all'applicazione web tramite LinkedIn, così da evolversi in un utente autenticato\\
	\textbf{Pre-Condizioni} & L'utente ha scelto di eseguire il login all'applicazione web e non è autenticato \\
	\textbf{Post-Condizioni} & L'utente ha effettuato il login all'applicazione web tramite LinkedIn, evolvendosi in un utente autenticato \\
	\textbf{Scenario Principale} & \begin{enumerate*}[label=(\arabic*.),itemjoin={\newline}]
		\item L'utente non autenticato può effettuare il login all'applicazione web tramite LinkedIn (UC4.4.1)
	\end{enumerate*}\\
\end{longtable}

\subsubsection{Caso d'uso UC4.5: Login Tramite Google+ }
\label{UC4_5}
\begin{figure}[ht]
	\centering
	\includegraphics[scale=0.45]{UML/UC4_5.png}
	\caption{UC4.5: Login Tramite Google+ }
\end{figure}

\begin{longtable}{ l | p{11cm}}
	\hline
	\rowcolor{Gray}
	 \multicolumn{2}{c}{UC4.5 - Login Tramite Google+} \\
	 \hline
	\textbf{Attori} & Utente Non Autenticato, Google+ \\
	\textbf{Descrizione} & L'utente non autenticato effettua il login all'applicazione web tramite Google+, così da evolversi in un utente autenticato\\
	\textbf{Pre-Condizioni} & L'utente ha scelto di eseguire il login all'applicazione web e non è autenticato \\
	\textbf{Post-Condizioni} & L'utente ha effettuato il login all'applicazione web tramite Google+, evolvendosi in un utente autenticato\\
	\textbf{Scenario Principale} & \begin{enumerate*}[label=(\arabic*.),itemjoin={\newline}]
		\item L'utente non autenticato può effettuare il login all'applicazione web tramite Google+ (UC4.5.1)
	\end{enumerate*}\\
\end{longtable}
\newpage
\subsection{Caso d'uso UC5: Recupero password }
\label{UC5}
\begin{figure}[ht]
	\centering
	\includegraphics[scale=0.45]{UML/UC5.png}
	\caption{UC5: Recupero password }
\end{figure}

\begin{longtable}{ l | p{11cm}}
	\hline
	\rowcolor{Gray}
	 \multicolumn{2}{c}{UC5 - Recupero password} \\
	 \hline
	\textbf{Attori} & Utente non autenticato \\
	\textbf{Descrizione} & L'attore tenta il recupero della propria password tramite l'invio di una email \\
	\textbf{Pre-Condizioni} & L'attore ha scelto di recuperare la sua password e non è autenticato \\
	\textbf{Post-Condizioni} & L'attore ha ricevuto nella propria casella email un link per reimpostare la propria password, oppure la procedura è fallita \\
	\textbf{Scenario Principale} & 
	\begin{enumerate*}[label=(\arabic*.),itemjoin={\newline}]
		\item L'attore può inserire la propria email di registrazione (UC5.1)
		\item L'attore può confermare l'indirizzo email inserito, al quale l'applicazione web ha inviato un link per reimpostare la password (UC5.2)
	\end{enumerate*}\\
	\textbf{Scenari Alternativi} & 
	\begin{enumerate*}[label=(\arabic*.),itemjoin={\newline}]
		\item L'attore visualizza un errore e l'invio della email di recupero password non avviene (UC5.3) 
	\end{enumerate*}\\
\end{longtable}
\subsubsection{Caso d'uso UC5.1:  Inserimento Email}
\label{UC5_1}

\begin{longtable}{ l | p{11cm}}
	\hline
	\rowcolor{Gray}
	 \multicolumn{2}{c}{UC5.1 - Inserimento Email} \\
	 \hline
	\textbf{Attori} & Utente Non Autenticato \\
	\textbf{Descrizione} & L'utente non autenticato inserisce la sua Email  \\
	\textbf{Pre-Condizioni} & L'utente ha dimenticato la password e l'applicazione web mostra la schermata di Recupero Password\\
	\textbf{Post-Condizioni} & L'utente visualizza area per Recupero Password per l'applicazione web \\
	\textbf{Scenario Principale} & \begin{enumerate*}[label=(\arabic*.),itemjoin={\newline}]
		\item L'utente non autenticato può inserire la propria Email(UC5.1)
	\end{enumerate*}\\
	\textbf{Scenari Alternativi} & 
	\begin{enumerate*}[label=(\arabic*.),itemjoin={\newline}]
		\item L'email inserita non e' valida
		\item L'email inserita non esiste
	\end{enumerate*}\\
\end{longtable}
\subsubsection{Caso d'uso UC5.2:  Conferma Inserimento Email}
\label{UC5_2}

\begin{longtable}{ l | p{11cm}}
	\hline
	\rowcolor{Gray}
	 \multicolumn{2}{c}{UC5.2 - Conferma Inserimento Email} \\
	 \hline
	\textbf{Attori} & Utente Non Autenticato \\
	\textbf{Descrizione} & L'utente non autenticato re-inserisce la sua Email per confermarla \\
	\textbf{Pre-Condizioni} & L'utente ha dimenticato la password e l'applicazione web mostra la schermata di Recupero Password\\
	\textbf{Post-Condizioni} & L'utente visualizza un Messaggio di conferma che invita a controllare la propria email\\
	\textbf{Scenario Principale} & \begin{enumerate*}[label=(\arabic*.),itemjoin={\newline}]
		\item L'utente conferma inserimento Email (UC5.2)
	\end{enumerate*}\\
	\textbf{Scenari Alternativi} & 
	\begin{enumerate*}[label=(\arabic*.),itemjoin={\newline}]
		\item L'operazione non e' andata a buon fine
	\end{enumerate*}\\
\end{longtable}
\subsubsection{Caso d'uso UC5.3:  Errore inserimento email}
\label{UC5_3}

\begin{minipage}{\linewidth}
\begin{longtable}{ l | p{11cm}}
	\hline
	\rowcolor{Gray}
	 \multicolumn{2}{c}{UC5.3 - Errore inserimento email} \\
	 \hline
	\textbf{Attori} & Utente non autenticato \\
	\textbf{Descrizione} & L'attore riceve un messaggio di errore dovuto all'inserimento di un'email non valida \\
	\textbf{Pre-Condizioni} & L'attore ha dimenticato la password e ha inserito l'email per poterla recuperare\\
	\textbf{Post-Condizioni} & L'attore riceve un messaggio di errore e può eventualmente ripetere la procedura di recupero password\\
	\textbf{Scenario Principale} & \begin{enumerate*}[label=(\arabic*.),itemjoin={\newline}]
		\item L'attore visualizza un messaggio d'errore per aver lasciato il campo vuoto o per aver inserito un indirizzo inesistente. Può eventualmente ripetere la procedura (UC5)
	\end{enumerate*}\\
\end{longtable}
\end{minipage}

\subsubsection{Caso d'uso UC5.4: Reset password}
\label{UC5_4}

\begin{minipage}{\linewidth}
	\begin{longtable}{ l | p{11cm}}
		\hline
		\rowcolor{Gray}
		\multicolumn{2}{c}{UC5.4 - Reset password} \\
		\hline
		\textbf{Attori} & Utente non autenticato \\
		\textbf{Descrizione} & L'attore ha ricevuto un link per la schermata di reset della propria password \\
		\textbf{Pre-Condizioni} & L'attore ha richiesto il recupero della password ed ha aperto il link per poterla resettare\\
		\textbf{Post-Condizioni} & L'attore ha resettato con successo la password\\
		\textbf{Scenario Principale} & \begin{enumerate*}[label=(\arabic*.),itemjoin={\newline}]
			\item L'attore può inserire la nuova password desiderata (UC5.4.1)
			\item L'attore può reinserire la nuova password desiderata (UC5.4.2)
			\item L'attore può confermare i dati inseriti e completare la procedura con successo (UC5.4.3)
		\end{enumerate*}\\
		\textbf{Scenari Alternativi} & \begin{enumerate*}[label=(\arabic*.),itemjoin={\newline}]
			\item L'attore visualizza un errore qualora le password inserite non coincidano o i campi risultino vuoti (UC5.4.4)
		\end{enumerate*}\\
	\end{longtable}
\end{minipage}

\paragraph{Caso d'uso UC5.4.1: Inserimento nuova password}
\label{UC5_4_1}

\begin{minipage}{\linewidth}
	\begin{longtable}{ l | p{11cm}}
		\hline
		\rowcolor{Gray}
		\multicolumn{2}{c}{UC5.4.1 - Inserimento nuova password} \\
		\hline
		\textbf{Attori} & Utente non autenticato \\
		\textbf{Descrizione} & L'attore inserisce la nuova password per il proprio account \\
		\textbf{Pre-Condizioni} & L'attore ha visualizzato la schermata di reset password\\
		\textbf{Post-Condizioni} & L'attore ha inserito la nuova password\\
		\textbf{Scenario Principale} & \begin{enumerate*}[label=(\arabic*.),itemjoin={\newline}]
			\item L'attore può inserire la nuova password per il proprio account
		\end{enumerate*}\\
	\end{longtable}
\end{minipage}

\paragraph{Caso d'uso UC5.4.2: Reinserimento nuova password}
\label{UC5_4_2}

\begin{minipage}{\linewidth}
	\begin{longtable}{ l | p{11cm}}
		\hline
		\rowcolor{Gray}
		\multicolumn{2}{c}{UC5.4.2 - Reinserimento nuova password} \\
		\hline
		\textbf{Attori} & Utente non autenticato \\
		\textbf{Descrizione} & L'attore reinserisce la nuova password per il proprio account \\
		\textbf{Pre-Condizioni} & L'attore ha visualizzato la schermata di reset password\\
		\textbf{Post-Condizioni} & L'attore ha reinserito la nuova password\\
		\textbf{Scenario Principale} & \begin{enumerate*}[label=(\arabic*.),itemjoin={\newline}]
			\item L'attore può reinserire la nuova password per il proprio account
		\end{enumerate*}\\
	\end{longtable}
\end{minipage}

\paragraph{Caso d'uso UC5.4.3: Conferma nuova password}
\label{UC5_4_3}

\begin{minipage}{\linewidth}
	\begin{longtable}{ l | p{11cm}}
		\hline
		\rowcolor{Gray}
		\multicolumn{2}{c}{UC5.4.3 - Conferma nuova password} \\
		\hline
		\textbf{Attori} & Utente non autenticato \\
		\textbf{Descrizione} & L'attore conferma i dati inseriti per poter resettare la propria password \\
		\textbf{Pre-Condizioni} & L'attore ha inserito due volte la propria nuova password\\
		\textbf{Post-Condizioni} & L'attore ha confermato i nuovi dati per il proprio account\\
		\textbf{Scenario Principale} & \begin{enumerate*}[label=(\arabic*.),itemjoin={\newline}]
			\item L'attore può confermare i nuovi dati per il proprio account, visualizzando un messaggio di successo in caso di esito positivo
		\end{enumerate*}\\
	\end{longtable}
\end{minipage}

\paragraph{Caso d'uso UC5.4.4: Errore reset password}
\label{UC5_4_4}

\begin{minipage}{\linewidth}
	\begin{longtable}{ l | p{11cm}}
		\hline
		\rowcolor{Gray}
		\multicolumn{2}{c}{UC5.4.4 - Errore reset password} \\
		\hline
		\textbf{Attori} & Utente non autenticato \\
		\textbf{Descrizione} & L'attore conferma i dati inseriti per poter resettare la propria password \\
		\textbf{Pre-Condizioni} & L'attore ha inserito due volte la propria nuova password\\
		\textbf{Post-Condizioni} & L'attore non ha confermato con successo i propri nuovi dati di accesso\\
		\textbf{Scenario Principale} & \begin{enumerate*}[label=(\arabic*.),itemjoin={\newline}]
			\item L'attore può confermare i nuovi dati per il proprio account, ma visualizza un messaggio di errore poichè le password non coincidono o i campi sono vuoti
		\end{enumerate*}\\
	\end{longtable}
\end{minipage}
\newpage
\subsection{Caso d'uso UC6: Ricerca API}
\label{UC6}
\begin{figure}[ht]
	\centering
	\includegraphics[scale=0.45]{UML/UC6.png}
	\caption{UC6: Ricerca API}
\end{figure}

\begin{longtable}{ l | p{11cm}}
	\hline
	\rowcolor{Gray}
	 \multicolumn{2}{c}{UC6 - Ricerca API} \\
	 \hline
	\textbf{Attori} & Utente non autenticato, Utente autenticato \\
	\textbf{Descrizione} & L'attore inserisce le keyword necessarie alla ricerca di API. \\
	\textbf{Pre-Condizioni} & L'attore ha scelto di effettuare una ricerca tra le API \\
	\textbf{Post-Condizioni} & L'attore ha effettuato la ricerca di API e ha visualizzato il risultato\\
	\textbf{Scenario Principale} & 
	\begin{enumerate*}[label=(\arabic*.),itemjoin={\newline}]
		\item L'attore può inserire la stringa di ricerca desiderata (UC6.1)
		\item L'attore può confermare i dati inseriti (UC6.2) e visualizzare i risultati forniti dall'applicazione web (UC6.3)
	\end{enumerate*}\\
\end{longtable}
\subsubsection{Caso d'uso UC6.1:  Inserimento Nome API}
\label{UC6_1}

\begin{tabular}{ l | p{11cm}}
	\hline
	\rowcolor{Gray}
	 \multicolumn{2}{c}{UC6.1 - Inserimento Nome API} \\
	 \hline
	\textbf{Attori} & Utente Non Autenticato, Utente Autenticato \\
	\textbf{Descrizione} & Gli utenti possono effettuare una ricerca delle API usandone il nome\\
	\textbf{Pre-Condizioni} & L'utente ha scelto fare una ricerca di API\\
	\textbf{Post-Condizioni} & L'utente ha inserito il nome dell'API nella barra di ricerca \\
	\textbf{Scenario Principale} & 
	\begin{enumerate*}[label=(\arabic*.),itemjoin={\newline}]
		\item L'utente puo' inserire il Nome dell API nella barra di ricerca (UC6.1)
	\end{enumerate*}\\
\end{tabular}
\subsubsection{Caso d'uso UC6.2:  Inserimento Nome Utente}
\label{UC6_2}

\begin{tabular}{ l | p{11cm}}
	\hline
	\rowcolor{Gray}
	 \multicolumn{2}{c}{UC6.2 - Inserimento Nome Utente} \\
	 \hline
	\textbf{Attori} & Utente Non Autenticato, Utente Autenticato \\
	\textbf{Descrizione} & Gli utenti possono effettuare una ricerca delle API usandone il nome dell'Utente che le ha create. Tale utente puo' rappresentare un'azienda\\
	\textbf{Pre-Condizioni} & L'utente ha scelto fare una ricerca di API\\
	\textbf{Post-Condizioni} & L'utente ha inserito il nome dell'Utente creatore dell'API nella barra di ricerca \\
	\textbf{Scenario Principale} & 
	\begin{enumerate*}[label=(\arabic*.),itemjoin={\newline}]
		\item L'utente puo' inserire il Nome dell'Utente Creatore dell'API nella barra di ricerca (UC6.2)
	\end{enumerate*}\\
\end{tabular}
\subsubsection{Caso d'uso UC6.3:  Scelta Categoria}
\label{UC6_3}

\begin{tabular}{ l | p{11cm}}
	\hline
	\rowcolor{Gray}
	 \multicolumn{2}{c}{UC6.3 - Scelta Categoria} \\
	 \hline
	\textbf{Attori} & Utente Non Autenticato, Utente Autenticato \\
	\textbf{Descrizione} & Gli utenti possono effettuare una ricerca delle API usandone la categoria\\
	\textbf{Pre-Condizioni} & L'utente ha scelto fare una ricerca di API\\
	\textbf{Post-Condizioni} & L'utente ha inserito il nome della categoria dell'API nella barra di ricerca \\
	\textbf{Scenario Principale} & 
	\begin{enumerate*}[label=(\arabic*.),itemjoin={\newline}]
		\item L'utente puo' inserire il Nome della categoria dell'API nella barra di ricerca (UC6.3)
	\end{enumerate*}\\
\end{tabular}
\newpage
\subsection{Caso d'uso UC7 - Visualizzazione API}
\label{UC7}
\begin{figure}[ht]
	\centering
	\includegraphics[scale=0.45]{UML/UC7.png}
	\caption{UC7: Visualizzazione API}
\end{figure}

\begin{longtable}{ l | p{11cm}}
	\hline
	\rowcolor{Gray}
	\multicolumn{2}{c}{UC7 - Visualizzazione API}\\
	\hline
	
	 \textbf{Attori} & Utente non autenticato, Utente autenticato, Amministratore API Market \\
	\textbf{Descrizione} & L'attore può visualizzare i dati relativi a un API che ha selezionato tramite la homepage o i risultati di una ricerca  \\
	\textbf{Pre-Condizioni} & L'attore ha selezionato un prodotto per la consultazione \\
	\textbf{Post-Condizioni} & L'attore visualizza la pagina relativa all'API selezionata\\
	\textbf{Scenario Principale} & 
	\begin{enumerate*}[label=(\arabic*.),itemjoin={\newline}]
		\item L'attore visualizza il nome dell'API (UC7.1)
		\item L'attore visualizza la descrizione dell'API (UC7.2)
		\item L'attore visualizza l'autore dell'API (UC7.3)
		\item L'attore può visualizzare l'interfaccia dell'API (UC7.4)
		\item L'attore può consultare la documentazione fornita dall'utente (UC7.5)
		\item L'attore può visualizzare i dati di utilizzo dell'API  (UC7.6)
	\end{enumerate*}\\
\end{longtable}

\newpage
\subsection{Caso d'uso UC8 - Interazione Con API non acquistate}
\label{UC8}
\begin{figure}[ht]
	\centering
	\includegraphics[scale=0.45]{UML/UC8.png}
	\caption{UC8 - Interazione Con API non acquistate}
\end{figure}

\begin{longtable}{ l | p{11cm}}
	\hline
	\rowcolor{Gray}
	\multicolumn{2}{c}{UC8 - Interazione Con API non acquistate}\\
	\hline
	
	 \textbf{Attori} & Utente autenticato  \\
	\textbf{Descrizione} & L'utente puo' interagire con le API non aquistate in vari modi. Visualizza una lista di API e puo' filtrarle, puo' consultarne la documentazione di ognuna, puo' acquistarle \\
	\textbf{Pre-Condizioni} & L'utente e' nella schermata di interazione con le API\\
	\textbf{Post-Condizioni} & L'utente ha scelto l'interazione con le API\\
	\textbf{Scenario Principale} & 
	\begin{enumerate*}[label=(\arabic*.),itemjoin={\newline}]
		\item L'utente puo' cercare una API (UC6)
		\item L'utente puo' cercare una API (UC8.1)
		\item L'utente puo' cercare una API (UC8.2)
	\end{enumerate*}\\
\end{longtable}


\newpage
\subsection{Caso d'uso UC9: Interazione con API Acquistate}
\label{UC9}
\begin{figure}[ht]
	\centering
	\includegraphics[scale=0.45]{UML/UC9.png}
	\caption{UC9: Interazione con API Acquistate}
\end{figure}
\FloatBarrier
\renewcommand*{\arraystretch}{1.6}
\begin{longtable}{ l | p{11cm}}
	\hline
	\rowcolor{Gray}
	\multicolumn{2}{c}{UC9: Interazione con API Acquistate} \\
	\hline
	\textbf{Attori} &Utente Autenticato, Amministratore APIMarket, Interfacce API Presente In APIMarket \\
	\textbf{Descrizione} & l'attore sceglie attraverso quale modalità interagire con le API acquistate \\
	\textbf{Pre-Condizioni} & l'attore si trova nella schermata di gestione di una API acquistata\\
	\textbf{Post-Condizioni}&l'attore ha scelto l'interazione\\
	\textbf{Scenario Principale} & \begin{enumerate*}[label=(\arabic*.),itemjoin={\newline}]
		\item L'attore può ricercare un'API(UC6)
		\item L'attore può consultare la documentazione di un'API(UC8.1)
		\item L'attore può Annullare il rinnovo di un'API(UC9.1)
		\item L'attore può utilizzare un'API(UC9.2)
		\item L'attore può Rimuovere un'API dalle Acquistate
		\item L'attore può vedere la Scadenza di un'API(UC9.4);
	\end{enumerate*}\\
\end{longtable}



\newpage
\subsection{Caso d'uso UC10: Interazione Con API Registrate}
\label{UC10}
\begin{figure}[ht]
	\centering
	\includegraphics[scale=0.45]{UML/UC10.png}
	\caption{UC10: Interazione Con API Registrate}
\end{figure}

\renewcommand*{\arraystretch}{1.6}
\begin{longtable}{ l | p{11cm}}
	\hline
	\rowcolor{Gray}
	\multicolumn{2}{c}{UC10: Interazione Con API Registrate} \\
	\hline
	\textbf{Attori} &Utente Autenticato, Amministratore APIMarket, Interfacce API Presente In APIMarket \\
	\textbf{Descrizione} & l'attore sceglie attraverso quale modalità interagire con le API acquistate \\
	\textbf{Pre-Condizioni} & l'attore si trova nella schermata di gestione di una API acquistata\\
	\textbf{Post-Condizioni}&l'attore ha scelto l'interazione\\
	\textbf{Scenario Principale} & \begin{enumerate*}[label=(\arabic*.),itemjoin={\newline}]
			\item L'attore può rimuovere una propria API dall'APIMarket (UC10.1), inviando una notifica agli utenti di quella API (UC10.4);
		\item L'attore può modificare una propria API (UC10.2);
		\item L'attore può visualizzare i dati di utilizzo di una propria API (UC10.3);
	\end{enumerate*}\\
\end{longtable}




\newpage
\subsection{Caso d'uso UC11 - Registrazione nuova API}
\label{UC11}
\begin{figure}[ht]
	\centering
	\includegraphics[scale=0.45]{UML/UC11.png}
	\caption{UC11: Registrazione nuova API}
\end{figure}

\begin{longtable}{ l | p{11cm}}
	\hline
	\rowcolor{Gray}
	\multicolumn{2}{c}{UC11 - Registrazione nuova API}\\
	\hline
	\textbf{Attori} & Utente autenticato, Amministratore API Market \\
	\textbf{Descrizione} & L'attore registra una nuova API su API Market \\
	\textbf{Pre-Condizioni} & L'attore si trova nella schermata relativa alla registrazione di una nuova API \\
	\textbf{Post-Condizioni} & L'attore ha registrato una nuova API su API Market \\
	\textbf{Scenario Principale} & 
	\begin{enumerate*}[label=(\arabic*.),itemjoin={\newline}]
		\item L'attore può inserire il nome della nuova API (UC11.1)
		\item L'attore può inserire la descrizione della nuova API (UC11.2)
		\item L'attore può inserire i tag della nuova API (UC11.3)
		\item L'attore può inserire l'interfaccia della nuova API (UC11.4)
		\item L'attore può inserire il file per la documentazione PDF della nuova API (UC11.5)
		\item L'attore può inserire il link per la documentazione esterna della nuova API (UC11.7)
		\item L'attore può inserire il prezzo base della nuova API (UC11.8)
		\item L'attore può confermare la registrazione della nuova API (UC11.9)
	\end{enumerate*}\\
	\textbf{Scenario Principale} & 
	\begin{enumerate*}[label=(\arabic*.),itemjoin={\newline}]
			\item L'attore può visualizzare un messaggio di errore riguardo al caricamento del file di documentazione PDF dell'API, ed il caricamento del file non avviene (UC11.6)
			\item L'attore può visualizzare un messaggio d'errore informativo riguardo la conferma della registrazione dell'API, e la registrazione non avviene (UC11.10)
	\end{enumerate*}\\
\end{longtable}

\subsubsection{Caso d'uso UC11.1: Inserimento nome API}
\label{UC11_1}

\begin{minipage}{\linewidth}
	\begin{tabular}{ l | p{11cm}}
		\hline
		\rowcolor{Gray}
		\multicolumn{2}{c}{UC11.1 - Inserimento nome API} \\
		\hline
		\textbf{Attori} & Utente autenticato, Amministratore API Market \\
		\textbf{Descrizione} & L'attore inserisce il nome della nuova API \\
		\textbf{Pre-Condizioni} & L'attore si trova nella schermata relativa alla registrazione di una nuova API \\
		\textbf{Post-Condizioni} & L'attore ha inserito il nome della nuova API \\
		\textbf{Scenario Principale} & 
		\begin{enumerate*}[label=(\arabic*.),itemjoin={\newline}]
			\item L'attore può inserire il nome della nuova API
		\end{enumerate*}\\
	\end{tabular}
\end{minipage}

\subsubsection{Caso d'uso UC11.2: Inserimento descrizione API}
\label{UC11_2}

\begin{minipage}{\linewidth}
	\begin{tabular}{ l | p{11cm}}
		\hline
		\rowcolor{Gray}
		\multicolumn{2}{c}{UC11.2 - Inserimento descrizione API} \\
		\hline
		\textbf{Attori} & Utente autenticato, Amministratore API Market \\
		\textbf{Descrizione} & L'attore inserisce la descrizione della nuova API \\
		\textbf{Pre-Condizioni} & L'attore si trova nella schermata relativa alla registrazione di una nuova API \\
		\textbf{Post-Condizioni} & L'attore ha inserito la descrizione della nuova API \\
		\textbf{Scenario Principale} & 
		\begin{enumerate*}[label=(\arabic*.),itemjoin={\newline}]
			\item L'attore può inserire la descrizione della nuova API
		\end{enumerate*}\\
	\end{tabular}
\end{minipage}

\subsubsection{Caso d'uso UC11.3: Inserimento tag API}
\label{UC11_3}

\begin{minipage}{\linewidth}
	\begin{tabular}{ l | p{11cm}}
		\hline
		\rowcolor{Gray}
		\multicolumn{2}{c}{UC11.3 - Inserimento tag API} \\
		\hline
		\textbf{Attori} & Utente autenticato, Amministratore API Market \\
		\textbf{Descrizione} & L'attore inserisce i tag della nuova API \\
		\textbf{Pre-Condizioni} & L'attore si trova nella schermata relativa alla registrazione di una nuova API \\
		\textbf{Post-Condizioni} & L'attore ha inserito i tag della nuova API \\
		\textbf{Scenario Principale} & 
		\begin{enumerate*}[label=(\arabic*.),itemjoin={\newline}]
			\item L'attore può inserire i tag della nuova API
		\end{enumerate*}\\
	\end{tabular}
\end{minipage}

\subsubsection{Caso d'uso UC11.4: Inserimento interfaccia API}
\label{UC11_4}

\begin{minipage}{\linewidth}
	\begin{tabular}{ l | p{11cm}}
		\hline
		\rowcolor{Gray}
		\multicolumn{2}{c}{UC11.4 - Inserimento interfaccia API} \\
		\hline
		\textbf{Attori} & Utente autenticato, Amministratore API Market \\
		\textbf{Descrizione} & L'attore inserisce l'interfaccia della nuova API \\
		\textbf{Pre-Condizioni} & L'attore si trova nella schermata relativa alla registrazione di una nuova API \\
		\textbf{Post-Condizioni} & L'attore ha inserito l'interfaccia della nuova API \\
		\textbf{Scenario Principale} & 
		\begin{enumerate*}[label=(\arabic*.),itemjoin={\newline}]
			\item L'attore può inserire l'interfaccia della nuova API
		\end{enumerate*}\\
	\end{tabular}
\end{minipage}

\subsubsection{Caso d'uso UC11.5: Inserimento documentazione PDF API}
\label{UC11_5}

\begin{minipage}{\linewidth}
	\begin{tabular}{ l | p{11cm}}
		\hline
		\rowcolor{Gray}
		\multicolumn{2}{c}{UC11.5 - Inserimento documentazione PDF API} \\
		\hline
		\textbf{Attori} & Utente autenticato, Amministratore API Market \\
		\textbf{Descrizione} & L'attore carica su API Market un file PDF contenente la documentazione PDF della nuova API \\
		\textbf{Pre-Condizioni} & L'attore si trova nella schermata relativa alla registrazione di una nuova API \\
		\textbf{Post-Condizioni} & L'attore ha caricato su API Market un file PDF contenente la documentazione PDF della nuova API \\
		\textbf{Scenario Principale} & 
		\begin{enumerate*}[label=(\arabic*.),itemjoin={\newline}]
			\item L'attore può caricare su API Market un file PDF contenente la documentazione PDF della nuova API
		\end{enumerate*}\\
		\textbf{Scenari Alternativi} & 
		\begin{enumerate*}[label=(\arabic*.),itemjoin={\newline}]
		\item L'attore può visualizzare un messaggio di errore ed il caricamento del file non avviene (UC11.10)
		\end{enumerate*}\\
	\end{tabular}
\end{minipage}

\subsubsection{Caso d'uso UC11.6: Errore inserimento PDF API}
\label{UC11_6}

\begin{minipage}{\linewidth}
	\begin{tabular}{ l | p{11cm}}
		\hline
		\rowcolor{Gray}
		\multicolumn{2}{c}{UC11.6 - Errore inserimento PDF API} \\
		\hline
		\textbf{Attori} & Utente autenticato, Amministratore API Market \\
		\textbf{Descrizione} & L'attore visualizza un messaggio di errore e l'inserimento della documentazione PDF della nuova API non avviene \\
		\textbf{Pre-Condizioni} & L'attore ha cercato di caricare su API Market un file contenente la documentazione della nuova API ma si è verificato un errore \\
		\textbf{Post-Condizioni} & L'attore ha visualizzato un messaggio di errore \\
		\textbf{Scenario Principale} & 
		\begin{enumerate*}[label=(\arabic*.),itemjoin={\newline}]
			\item L'attore può visualizzare un messaggio di errore
		\end{enumerate*}\\
	\end{tabular}
\end{minipage}

\subsubsection{Caso d'uso UC11.7: Inserimento documentazione esterna API}
\label{UC11_7}

\begin{minipage}{\linewidth}
	\begin{tabular}{ l | p{11cm}}
		\hline
		\rowcolor{Gray}
		\multicolumn{2}{c}{UC11.7 - Inserimento documentazione esterna API} \\
		\hline
		\textbf{Attori} & Utente autenticato, Amministratore API Market \\
		\textbf{Descrizione} & L'attore inserisce il link alla documentazione esterna della nuova API \\
		\textbf{Pre-Condizioni} & L'attore si trova nella schermata relativa alla registrazione di una nuova API \\
		\textbf{Post-Condizioni} & L'attore ha inserito il link alla documentazione esterna della nuova API \\
		\textbf{Scenario Principale} & 
		\begin{enumerate*}[label=(\arabic*.),itemjoin={\newline}]
			\item L'attore può inserire il link alla documentazione esterna della nuova API
		\end{enumerate*}\\
	\end{tabular}
\end{minipage}

\subsubsection{Caso d'uso UC11.8: Inserimento prezzo base API}
\label{UC11_8}

\begin{minipage}{\linewidth}
	\begin{tabular}{ l | p{11cm}}
		\hline
		\rowcolor{Gray}
		\multicolumn{2}{c}{UC11.8 - Inserimento prezzo base API} \\
		\hline
		\textbf{Attori} & Utente autenticato, Amministratore API Market \\
		\textbf{Descrizione} & L'attore inserisce il prezzo base della nuova API \\
		\textbf{Pre-Condizioni} & L'attore si trova nella schermata relativa alla registrazione di una nuova API \\
		\textbf{Post-Condizioni} & L'attore ha inserito il prezzo base della nuova API \\
		\textbf{Scenario Principale} & 
		\begin{enumerate*}[label=(\arabic*.),itemjoin={\newline}]
			\item L'attore può inserire il prezzo base della nuova API
		\end{enumerate*}\\
	\end{tabular}
\end{minipage}

\subsubsection{Caso d'uso UC11.9: Conferma registrazione nuova API}
\label{UC11_9}

\begin{minipage}{\linewidth}
	\begin{tabular}{ l | p{11cm}}
		\hline
		\rowcolor{Gray}
		\multicolumn{2}{c}{UC11.9 - Conferma registrazione nuova API} \\
		\hline
		\textbf{Attori} & Utente autenticato, Amministratore API Market \\
		\textbf{Descrizione} & L'attore conferma la registrazione della nuova API \\
		\textbf{Pre-Condizioni} & L'attore si trova nella schermata relativa alla registrazione di una nuova API \\
		\textbf{Post-Condizioni} & L'attore ha confermato la registrazione della nuova API \\
		\textbf{Scenario Principale} & 
		\begin{enumerate*}[label=(\arabic*.),itemjoin={\newline}]
			\item L'attore può confermare la registrazione della nuova API, visualizzando un messaggio di successo e venendo reindirizzato alla schermata di visualizzazione API registrate (UC10)
		\end{enumerate*}\\
	\end{tabular}
\end{minipage}

\subsubsection{Caso d'uso UC11.10: Errore registrazione nuova API}
\label{UC11_10}

\begin{minipage}{\linewidth}
	\begin{tabular}{ l | p{11cm}}
		\hline
		\rowcolor{Gray}
		\multicolumn{2}{c}{UC11.10 - Errore registrazione nuova API} \\
		\hline
		\textbf{Attori} & Utente autenticato, Amministratore API Market \\
		\textbf{Descrizione} & L'attore visualizza un messaggio di errore informativo e la registrazione della nuova API non avviene \\
		\textbf{Pre-Condizioni} & L'attore ha confermato la registrazione della una nuova API ma si è verificato un errore \\
		\textbf{Post-Condizioni} & L'attore ha visualizzato un messaggio di errore informativo \\
		\textbf{Scenario Principale} & 
		\begin{enumerate*}[label=(\arabic*.),itemjoin={\newline}]
			\item L'attore può visualizzare un messaggio di errore informativo e la registrazione della nuova API non avviene
		\end{enumerate*}\\
	\end{tabular}
\end{minipage}
\newpage
\subsection{Caso d'uso UC12: Logout}
\label{UC12}
\begin{figure}[ht]
	\centering
	\includegraphics[scale=0.45]{UML/UC12.png}
	\caption{Caso d'uso UC11: Registrazione Nuova API}
\end{figure}

\renewcommand*{\arraystretch}{1.6}
\begin{longtable}{ l | p{11cm}}
	\hline
	\rowcolor{Gray}
	\multicolumn{2}{c}{UC12: Logout} \\
	\hline
	\textbf{Attori} &Utente Autenticato, Amministratore APIMarket \\
	\textbf{Descrizione} &l'attore effettua il logout \\
	\textbf{Pre-Condizioni} &  l'attore ha scelto di effettuare il logout\\
	\textbf{Post-Condizioni}&l'attore ha effettuato il logout\\
	\textbf{Scenario Principale} & \begin{enumerate*}[label=(\arabic*.),itemjoin={\newline}]
			\item L'attore può confermare il logout (UC12.1)
	\end{enumerate*}\\
\end{longtable}


\newpage
\subsection{Caso d'uso UC13: Visualizzazione Dati Utilizzo API}
\label{UC13}
\begin{figure}[ht]
	\centering
	\includegraphics[scale=0.45]{UML/UC13.png}
	\caption{UC13: Visualizzazione Dati Utilizzo API}
\end{figure}

\renewcommand*{\arraystretch}{1.6}
\begin{longtable}{ l | p{11cm}}
	\hline
	\rowcolor{Gray}
	\multicolumn{2}{c}{UC13: Visualizzazione Dati Utilizzo API} \\
	\hline
	\textbf{Attori} &Utente Autenticato, Amministratore APIMarket \\
	\textbf{Descrizione} &l''attore visualizza i dati dell'utilizzo delle API in APIMarket \\
	\textbf{Pre-Condizioni} &   l'attore ha scelto di visualizzare i dati dell'utilizzo delle API in APIMarket\\
	\textbf{Post-Condizioni}& l'attore ha visualizzato i dati dell'utilizzo delle API in APIMarket\\
	\textbf{Scenario Principale} & \begin{enumerate*}[label=(\arabic*.),itemjoin={\newline}]
		\item L'attore può visualizzare il numero di utenti che hanno acquistato una API (UC13.1)
		\item L'attore può registrare l'interfaccia della propria nuova API (UC13.2)
	\end{enumerate*}\\
\end{longtable}





\newpage
\subsection{Caso d'uso UC14: Amministrazione applicazione web}
\label{UC14}
\begin{figure}[ht]
	\centering
	\includegraphics[scale=0.45]{UML/UC14.png}
	\caption{UC14: Amministrazione applicazione web}
\end{figure}

\renewcommand*{\arraystretch}{1.6}
\begin{longtable}{ l | p{11cm}}
	\hline
	\rowcolor{Gray}
	\multicolumn{2}{c}{UC14: Amministrazione applicazione web} \\
	\hline
	\textbf{Attori} & Amministratore API Market \\
	\textbf{Descrizione} & L'attore può gestire la parte riservata della piattaforma, ed effettuare operazioni super-user su utenza, prodotti registrati e sulla piattaforma stessa \\
	\textbf{Pre-Condizioni} & L'attore visita la pagina relativa all'amministrazione della piattaforma API Market\\
	\textbf{Post-Condizioni}& L'attore ha effettuato le modifiche desiderate, o ha consultato i dati desiderati, all'interno della piattaforma\\
	\textbf{Scenario Principale} & \begin{enumerate*}[label=(\arabic*.),itemjoin={\newline}]
		\item L'attore può consultare i dati di utilizzo avanzati per un API (UC14.1)
		\item L'attore può moderare l'utenza predisponendo sospensioni (UC14.2)
	\end{enumerate*}\\
\end{longtable}


\subsubsection{Caso d'uso UC14.1: Visualizzazione dati di utilizzo avanzati}
\label{UC14_1}

\begin{minipage}{\linewidth}
	\begin{tabular}{ l | p{11cm}}
		\hline
		\rowcolor{Gray}
		\multicolumn{2}{c}{UC14.1 - Visualizzazione dati di utilizzo avanzati} \\
		\hline
		\textbf{Attori} &  Amministratore API Market \\
		\textbf{Descrizione} & L'attore visualizza nella schermata relativa ai dati di utilizzo dell'API \\
		\textbf{Pre-Condizioni} & L'attore ha selezionato la visualizzazione dati per un API \\
		\textbf{Post-Condizioni} & L'attore ha visualizzato i dati di utilizzo avanzati dell'API selezionata \\
		\textbf{Scenario Principale} & 
		\begin{enumerate*}[label=(\arabic*.),itemjoin={\newline}]
			\item L'attore può visualizzare il numero di licenze attive per l'API selezionata (UC7.7.1)
			\item L'attore può visualizzare il numero di chiamate giornaliere effettuate all'API selezionata (UC7.7.2)
			\item L'attore può visualizzare il tempo medio di utilizzo dell'API selezionata (UC7.7.3)
			\item L'attore può visualizzare il traffico medio giornaliero dell'API selezionata (UC7.7.4)
			\item L'attore può visualizzare la lista di utenti che hanno una licenza attiva (UC14.1.1)
			\item L'attore può visualizzare il tempo medio di risposta (UC14.1.2)
		\end{enumerate*}\\
	\end{tabular}
\end{minipage}

\paragraph{Caso d'uso UC14.1.1: Visualizzazione utenti attivi per API}
\label{UC14_1_1}

\begin{minipage}{\linewidth}
	\begin{tabular}{ l | p{11cm}}
		\hline
		\rowcolor{Gray}
		\multicolumn{2}{c}{UC14.1.1 - Visualizzazione utenti attivi per API} \\
		\hline
		\textbf{Attori} & Amministratore API Market \\
		\textbf{Descrizione} & L'attore visualizza una lista di utenti attivi per l'API selezionata \\
		\textbf{Pre-Condizioni} & L'attore ha selezionato un API per il quale visualizzare i dati di utilizzo avanzati\\
		\textbf{Post-Condizioni} & L'attore ha visualizzato la lista di licenze attive per l'API selezionata \\
		\textbf{Scenario Principale} & 
		\begin{enumerate*}[label=(\arabic*.),itemjoin={\newline}]
			\item L'attore può visualizzare il nome dell'utente (UC14.1.1.1)
			\item L'attore può visualizzare la durata della licenza (UC14.1.1.2)
		\end{enumerate*}\\
	\end{tabular}
\end{minipage}

\subparagraph{Caso d'uso UC14.1.1.1: Visualizzazione nome}
\label{UC14_1_1_1}

\begin{minipage}{\linewidth}
	\begin{tabular}{ l | p{11cm}}
		\hline
		\rowcolor{Gray}
		\multicolumn{2}{c}{UC14.1.1.1 - Visualizzazione nome} \\
		\hline
		\textbf{Attori} & Amministratore API Market \\
		\textbf{Descrizione} & L'attore può visualizzare il nome dell'utente interessato\\
		\textbf{Pre-Condizioni} & L'attore è nella schermata di visualizzazione degli utenti con licenza attiva per l'API selezionata\\
		\textbf{Post-Condizioni} & L'attore ha visualizzato il nome interessato \\
		\textbf{Scenario Principale} & 
		\begin{enumerate*}[label=(\arabic*.),itemjoin={\newline}]
			\item L'attore può visualizzare il nome dell'utente corrispondente
		\end{enumerate*}
	\end{tabular}
\end{minipage}

\subparagraph{Caso d'uso UC14.1.1.2: Visualizzazione durata residua licenza}
\label{UC14_1_1_2}

\begin{minipage}{\linewidth}
	\begin{tabular}{ l | p{11cm}}
		\hline
		\rowcolor{Gray}
		\multicolumn{2}{c}{UC14.1.1.2 -  Visualizzazione durata residua licenza} \\
		\hline
		\textbf{Attori} & Amministratore API Market \\
		\textbf{Descrizione} & L'attore può visualizzare la durata residua della licenza dell'utente interessato\\
		\textbf{Pre-Condizioni} & L'attore è nella schermata di visualizzazione degli utenti con licenza attiva per l'API selezionata\\
		\textbf{Post-Condizioni} & L'attore ha visualizzato la data di scadenza \\
		\textbf{Scenario Principale} & 
		\begin{enumerate*}[label=(\arabic*.),itemjoin={\newline}]
			\item L'attore può visualizzare la scadenza per l'utente visualizzato riguardante l'API selezionata
		\end{enumerate*}
	\end{tabular}
\end{minipage}

\paragraph{Caso d'uso UC14.1.2: Visualizzazione tempo medio di risposta}
\label{UC14_1_2}

\begin{minipage}{\linewidth}
	\begin{tabular}{ l | p{11cm}}
		\hline
		\rowcolor{Gray}
		\multicolumn{2}{c}{UC14.1.2 - Visualizzazione tempo medio di risposta} \\
		\hline
		\textbf{Attori} & Amministratore API Market \\
		\textbf{Descrizione} & L'attore visualizza il tempo medio di risposta per l'API selezionata \\
		\textbf{Pre-Condizioni} & L'attore ha selezionato un API per il quale visualizzare i dati di utilizzo avanzati\\
		\textbf{Post-Condizioni} & L'attore ha visualizzato il tempo medio di risposta per l'API selezionata \\
		\textbf{Scenario Principale} & 
		\begin{enumerate*}[label=(\arabic*.),itemjoin={\newline}]
			\item L'attore può visualizzare il tempo medio di risposta per l'API selezionata
		\end{enumerate*}\\
	\end{tabular}
\end{minipage}

\subsubsection{Caso d'uso UC14.2: Azioni utente}
\label{UC14_2}

\begin{minipage}{\linewidth}
	\begin{tabular}{ l | p{11cm}}
		\hline
		\rowcolor{Gray}
		\multicolumn{2}{c}{UC14.2 - Azioni utente} \\
		\hline
		\textbf{Attori} &  Amministratore API Market \\
		\textbf{Descrizione} & L'attore visualizza nella schermata relativa ai dati di utilizzo dell'API \\
		\textbf{Pre-Condizioni} & L'attore ha selezionato la visualizzazione dati per un API \\
		\textbf{Post-Condizioni} & L'attore ha visualizzato i dati di utilizzo avanzati dell'API selezionata \\
		\textbf{Scenario Principale} & 
		\begin{enumerate*}[label=(\arabic*.),itemjoin={\newline}]
			\item L'attore può inserire il nome di un utente su cui effettuare un azione (UC14.2.1)
			\item L'attore può sospendere l'utente selezionato (UC14.2.2)
			\item L'attore può sospendere i prelievi di denaro dal proprio conto per l'utente selezionato (UC14.2.3)
			\item L'attore può rimuovere una sospensione utente (UC14.2.4)
			\item L'attore può rimuovere la sospensione dei prelievi (UC14.2.5)
		\end{enumerate*}\\
		\textbf{Scenari Alternativi} & 
		\begin{enumerate*}[label=(\arabic*.),itemjoin={\newline}]
			\item L'attore riceve un messaggio d'errore qualora l'utente inserito non esista. Può dunque ritentare la procedura
		\end{enumerate*}\\
	\end{tabular}
\end{minipage}

\paragraph{Caso d'uso UC14.2.1: Inserimento username}
\label{UC14_2_1}

\begin{minipage}{\linewidth}
	\begin{tabular}{ l | p{11cm}}
		\hline
		\rowcolor{Gray}
		\multicolumn{2}{c}{UC14.2.1 - Inserimento username} \\
		\hline
		\textbf{Attori} & Amministratore API Market \\
		\textbf{Descrizione} & L'attore inserisce l'username di un utente sul quale effettuare operazioni \\
		\textbf{Pre-Condizioni} & L'attore ha scelto di effettuare azioni su un utente\\
		\textbf{Post-Condizioni} & L'attore ha inserito il nome di un utente \\
		\textbf{Scenario Principale} & 
		\begin{enumerate*}[label=(\arabic*.),itemjoin={\newline}]
			\item L'attore può inserire il nome di un utente
		\end{enumerate*}\\
	\end{tabular}
\end{minipage}

\paragraph{Caso d'uso UC14.2.2: Sospensione utente}
\label{UC14_2_2}

\begin{minipage}{\linewidth}
	\begin{tabular}{ l | p{11cm}}
		\hline
		\rowcolor{Gray}
		\multicolumn{2}{c}{UC14.2.2 - Sospensione utente} \\
		\hline
		\textbf{Attori} & Amministratore API Market \\
		\textbf{Descrizione} & L'attore può sospendere l'utente indicato \\
		\textbf{Pre-Condizioni} & L'attore ha inserito il nome utente da sospendere\\
		\textbf{Post-Condizioni} & L'attore ha sospeso un utente \\
		\textbf{Scenario Principale} & 
		\begin{enumerate*}[label=(\arabic*.),itemjoin={\newline}]
			\item L'attore può inserire la durata in giorni della sospensione (UC14.2.2.1)
			\item L'attore può confermare la scelta (UC14.2.2.2)
		\end{enumerate*}\\
	\end{tabular}
\end{minipage}

\subparagraph{Caso d'uso UC14.2.2.1: Durata sospensione utente}
\label{UC14_2_2_1}

\begin{minipage}{\linewidth}
	\begin{tabular}{ l | p{11cm}}
		\hline
		\rowcolor{Gray}
		\multicolumn{2}{c}{UC14.2.2.1 - Durata sospensione utente} \\
		\hline
		\textbf{Attori} & Amministratore API Market \\
		\textbf{Descrizione} & L'attore può indicare quanto deve durare la sospensione per l'utente indicato \\
		\textbf{Pre-Condizioni} & L'attore si trova nella schermata per sospendere un utente\\
		\textbf{Post-Condizioni} & L'attore ha indicato la durata della sospensione \\
		\textbf{Scenario Principale} & 
		\begin{enumerate*}[label=(\arabic*.),itemjoin={\newline}]
			\item L'attore può inserire il numero di giorni di durata della sospensione
		\end{enumerate*}\\
	\end{tabular}
\end{minipage}

\subparagraph{Caso d'uso UC14.2.2.2: Conferma sospensione utente}
\label{UC14_2_2_2}

\begin{minipage}{\linewidth}
	\begin{tabular}{ l | p{11cm}}
		\hline
		\rowcolor{Gray}
		\multicolumn{2}{c}{UC14.2.2.2 - Conferma sospensione utente} \\
		\hline
		\textbf{Attori} & Amministratore API Market \\
		\textbf{Descrizione} & L'attore può confermare la sospensione indicata \\
		\textbf{Pre-Condizioni} & L'attore si trova nella schermata per sospendere un utente e ha inserito la durata in giorni\\
		\textbf{Post-Condizioni} & L'attore ha sospeso con successo un utente \\
		\textbf{Scenario Principale} & 
		\begin{enumerate*}[label=(\arabic*.),itemjoin={\newline}]
			\item L'attore conferma la scelta e viene notificato dell'avvenuta sospensione
		\end{enumerate*}\\
		\textbf{Scenari Alternativi} & 
		\begin{enumerate*}[label=(\arabic*.),itemjoin={\newline}]
			\item L'attore non ha inserito la durata correttamente e viene notificato dell'errore.
		\end{enumerate*}\\
	\end{tabular}
\end{minipage}

\paragraph{Caso d'uso UC14.2.3: Sospensione pagamenti utente}
\label{UC14_2_3}

\begin{minipage}{\linewidth}
	\begin{tabular}{ l | p{11cm}}
		\hline
		\rowcolor{Gray}
		\multicolumn{2}{c}{UC14.2.3 - Sospensione pagamenti utente} \\
		\hline
		\textbf{Attori} & Amministratore API Market \\
		\textbf{Descrizione} & L'attore può sospendere i prelievi per l'utente indicato \\
		\textbf{Pre-Condizioni} & L'attore ha inserito il nome utente da sospendere\\
		\textbf{Post-Condizioni} & L'attore ha sospeso i pagamenti per l'utente selezionato per la risoluzione di eventuali contestazioni \\
		\textbf{Scenario Principale} & 
		\begin{enumerate*}[label=(\arabic*.),itemjoin={\newline}]
			\item L'attore può confermare la sospensione dei prelievi di denaro di un utente, per risolvere eventuali contestazioni
		\end{enumerate*}\\
		\textbf{Scenari Alternativi} & 
		\begin{enumerate*}[label=(\arabic*.),itemjoin={\newline}]
			\item L'attore viene notificato che l'utente possiede una sospensione già in atto.
		\end{enumerate*}\\
	\end{tabular}
\end{minipage}

\paragraph{Caso d'uso UC14.2.4: Revoca sospensione utente}
\label{UC14_2_4}

\begin{minipage}{\linewidth}
	\begin{tabular}{ l | p{11cm}}
		\hline
		\rowcolor{Gray}
		\multicolumn{2}{c}{UC14.2.4 - Revoca sospensione utente} \\
		\hline
		\textbf{Attori} & Amministratore API Market \\
		\textbf{Descrizione} & L'attore può revocare una sospensione per l'utente indicato \\
		\textbf{Pre-Condizioni} & L'attore ha inserito il nome utente su cui revocare la sospensione\\
		\textbf{Post-Condizioni} & L'attore revocato la sospensione per l'utente selezionato \\
		\textbf{Scenario Principale} & 
		\begin{enumerate*}[label=(\arabic*.),itemjoin={\newline}]
			\item L'attore può revocare una sospensione per un utente
		\end{enumerate*}\\
		\textbf{Scenari Alternativi} & 
		\begin{enumerate*}[label=(\arabic*.),itemjoin={\newline}]
			\item L'attore viene notificato che l'utente selezionato non ha una sospensione in atto
		\end{enumerate*}\\
	\end{tabular}
\end{minipage}

\paragraph{Caso d'uso UC14.2.5: Revoca sospensione pagamenti utente}
\label{UC14_2_5}

\begin{minipage}{\linewidth}
	\begin{tabular}{ l | p{11cm}}
		\hline
		\rowcolor{Gray}
		\multicolumn{2}{c}{UC14.2.5 - Revoca sospensione pagamenti utente} \\
		\hline
		\textbf{Attori} & Amministratore API Market \\
		\textbf{Descrizione} & L'attore può revocare una sospensione di pagamenti per l'utente indicato \\
		\textbf{Pre-Condizioni} & L'attore ha inserito il nome utente su cui revocare la sospensione dei pagamenti\\
		\textbf{Post-Condizioni} & L'attore revocato la sospensione per l'utente selezionato in seguito alla risoluzione delle contestazioni in atto \\
		\textbf{Scenario Principale} & 
		\begin{enumerate*}[label=(\arabic*.),itemjoin={\newline}]
			\item L'attore può revocare una sospensione per un utente
		\end{enumerate*}\\
		\textbf{Scenari Alternativi} & 
		\begin{enumerate*}[label=(\arabic*.),itemjoin={\newline}]
			\item L'attore viene notificato che l'utente selezionato non ha una sospensione dei pagamenti in atto
		\end{enumerate*}\\
	\end{tabular}
\end{minipage}

