\subsection{Requisiti di vincolo}
\begin{longtable}{|c|m{8cm}|c|}
\caption{Tabella dei requisiti di vincolo} \\

\hline
\thead*{\textbf{Codice Requisito}} &\thead{\textbf{Descrizione}}  &\thead{\textbf{Fonti}} \\
\hline
\endhead

\hline
\endfoot
\hline
\endlastfoot

\hypertarget{RVO1}{RVO1} & Il sistema deve avere un'architettura a microservizi & \makecell*{Capitolato} \\
\hline

\hypertarget{RVO2}{RVO2} &  Nel caso di più gruppi per questo progetto, le specifiche di design iniziali dovranno essere condivise fra tutti i partecipanti al fine di rendere il progetto finale integrabile ed omogeneo & \makecell*{Capitolato} \\
\hline

\hypertarget{RVD3}{RVD3} & Si suggerisce di utilizzare Jolie per le interfacce dei servizi e per l'API gateway &\makecell*{Capitolato} \\
\hline

\hypertarget{RVD4}{RVD4} & Le componenti web possono essere realizzate utilizzando Javascript, HTML, css3 &\makecell*{Capitolato} \\
\hline

\hypertarget{RVD5}{RVD5} & Come database possono essere utilizzati sia NoSQL che SQL &\makecell*{Capitolato} \\
\hline

\hypertarget{RVO6}{RVO6} & Il codice finale deve essere depositato su una repository git &\makecell*{Capitolato} \\
\hline

\hypertarget{RVO7}{RVO7} & Deve essere stilato breve report tecnico che evidenzi gli aspetti positivi e gli aspetti negativi di un'architettura a microservizi  &\makecell*{Capitolato} \\
\hline

\end{longtable}
