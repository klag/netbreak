\newpage
\section{Descrizione generale}
\subsection{Obbiettivo del prodotto}
Con la realizzazione di questo progetto, si vuole principalmente fornire un ambiente online di archiviazione e compravendita di API di microservizi. Un secondo obiettivo implicito riguarda il voler favorire la diffusione e lo sviluppo del linguaggio Jolie, la tecnologia principale del committente ItalianaSoftware. Jolie è un progetto internazionale ed open-source, ed introduce un paradigma di programmazione orientato ai microservizi. Questa idea non è nuova, ma in fase di sviluppo da anni, tuttavia solo recentemente ha assunto una forma definita e dei risultati concreti. Approfittando dell'interesse per le architetture a microservizi, Jolie vorrebbe affermarsi come uno dei principali linguaggi di programmazione ed il marketplace proposto da ItalianaSoftware viaggia precisamente verso questo traguardo. E' richiesto lo sviluppo di una applicazione web che favorisca la compravendita di API di microservizi Jolie. Per garantire ciò, sarà necessario poter consultare i microservizi presenti e la rispettiva documentazione, permettere la gestione di tutte le operazioni legate alle API (inserimento, modifica...) e monitorarne l'uso (per estrapolarne dati utili a verificarne il corretto funzionamento e per eventuali indagini di mercato).

\subsection{Funzioni del prodotto}
Le caratteristiche peculiari del progetto riguardano la possibilità non solo di gestire i microservizi alla stregua di un marketplace (aprendo la strada di fatto a questa nuova emergente tecnologia), ma di integrare funzioni di controllo tramite un API Gateway che si occupi di: effettuare un'analisi statistica dei dati di utilizzo più rilevanti, e regolare l'accesso alle API registrate tramite opportune API key limitando l'accesso alle API a coloro che non sono in possesso di una chiave valida. La creazione della web app sarà affrontata tramite l'utilizzo delle consuete tecnologie per lo sviluppo web in lato front-end e back-end. Il progetto adempie alle necessità di un comune marketplace (la cui merce siano API di microservizi):
\begin{itemize}
\item Gestione profili utenti \\
\item Inserimento e modifica di API (di microservizi) nel database \\
\item Ricerca e consultazione di API e relativa documentazione \\
\item Compravendita delle API tramite API key \\
\item Monitoraggio dell'uso delle API in base alle API key \\
\item Visualizzazione dati d'uso delle API \\
\end{itemize}
Un'applicazione web permette facile acceso a tali funzionalità; il suo front-end comunica con gli utenti mentre il suo back-end svolge il suo lavoro su un apposito server di ItalianaSoftware. Ad occuparsi di coordinare e presentare i risultati delle varie funzionalità è l'API Gateway, realizzata con un'architettura a microservizi.

\subsection{Caratteristiche degli utenti}
Il bacino degli utenti dell'API marketplace sarà molto specializzato, composto quasi esclusivamente da aziende e privati nell'ambito informatico. Sia Jolie, che è un linguaggio in fase di sviluppo, sia l'architettura a microservizi, per quanto interessante ed innovativa, non sono ancora elementi molto diffusi. Solo chi lavora nel loro ambito specifico troverà interessante il nostro progetto e potrà usufruire appieno delle sue potenzialità. Per esempio, un utente che conosca le architetture a microservizi non potrà inserire comunque alcuna propria API se non sarà scritta in linguaggio Jolie. Tutte le funzionalità dell'API marketplace saranno disponibili a qualsiasi utente, ma ciascuna ne identifica una categoria generale. Gli utenti meno familiari con le tecnologie impiegate potranno consultare facilmente la documentazione delle API per valutarne l'utilità senza impegno. Gli utenti più esperti ed ambiziosi potranno sfruttare al meglio la programmazione a microservizi costruendo nuove API a partire da quelle già presenti nel database. Infine gli utenti intenzionati a diventare veri e propri seller, avranno a disposizione alcuni dati riguardo alle proprio API su cui basare i prodotti futuri. Data l'estrema versatilità e componibilità della programmazione a microservizi, non esistono distinzioni di particolare rilievo tra utenti come privati o come aziende (che dovranno aver adottato, o voler estendersi con, un'architettura a microservizi).

\subsection{Piattaforma di esecuzione}
Il capitolato dovrà eseguire su una macchina fornita da ItalianaSoftware.

\subsection{Vincoli generali}
\begin{itemize}
\item Per le interfacce delle API è molto incoraggiato l'uso di Jolie \\
\item Per l'API Gateway è molto incoraggiato l'uso di Jolie, ed obbligatorio l'uso di un'architettura a microservizi \\
\item Le componenti web possono essere realizzate utilizzando Javascript, HTML, css3 (con possibile uso di framework) \\
\item Per il database possono essere utilizzati sia DB NoSQL che database SQL \\
\item Nel caso ItalianaSoftware riesca a fornirli, il progetto dovrà superare i loro test \\
\item Il progetto finalizzato dovrà essere depositato in un repository git \\
\item E' richiesto un breve report tecnico che evidenzi gli aspetti positivi e gli aspetti negativi di un'architettura a microservizi \\
\end{itemize}
