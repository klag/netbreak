\newpage
\section{Introduzione}

\subsection{Scopo del documento}
Lo scopo del documento è descrivere i casi d'uso ed i requisiti del capitolato, in accordo con le decisioni del gruppo e le indicazioni di ItalianaSoftware. In questo documento inoltre sono catalogate le corrispondenze tra casi d'uso e requisiti, così da permetterne una facile ricerca.
\subsection{Scopo del prodotto}
Con la realizzazione di questo progetto, si vuole principalmente fornire un ambiente online di archiviazione e compravendita di API di microservizi. Un secondo obiettivo implicito riguarda il voler favorire la diffusione e lo sviluppo del linguaggio Jolie, la tecnologia principale del committente ItalianaSoftware. Jolie è un progetto internazionale ed open-source, ed introduce un paradigma di programmazione orientato ai microservizi. Questa idea non è nuova, ma in fase di sviluppo da anni, tuttavia solo recentemente ha assunto una forma definita e dei risultati concreti. Approfittando dell'interesse per le architetture a microservizi, Jolie vorrebbe affermarsi come uno dei principali linguaggi di programmazione ed il marketplace proposto da ItalianaSoftware viaggia precisamente verso questo traguardo. E' richiesto lo sviluppo di una applicazione web che favorisca la compravendita di API di microservizi Jolie. Per garantire ciò, sarà necessario poter consultare i microservizi presenti e la rispettiva documentazione, permettere la gestione di tutte le operazioni legate alle API (inserimento, modifica...) e monitorarne l'uso (per estrapolarne dati utili a verificarne il corretto funzionamento e per eventuali indagini di mercato).

\subsection{Glossario}
Al fine di evitare ogni ambiguità i termini tecnici del dominio del progetto, gli acronimi e le parole che necessitano di ulteriori spiegazioni saranno nei vari documenti marcate con il pedice \ped{G} e quindi presenti nel documento \textit{\G}.

\subsection{Riferimenti}

\subsubsection{Normativi}

\begin{itemize}
	
	\item NormeDiProgetto.
	
\end{itemize}

\subsubsection{Informativi}
\begin{itemize}
	\item
	\textbf{Capitolato d'appalto C1}: \progetto. Reperibile all'indirizzo: \\
	\url{http://www.math.unipd.it/~tullio/IS-1/2016/Progetto/C1.pdf}.;
	\item 	
	\textbf{Studio di Fattibilita'}: StudioDiFattibilita
	\item
	\textbf{Ingegneria del Software - Ian Sommerville - Ottava edizione}:
	\begin{itemize}
		\item Capitolo 6: Requisiti del software;
		\item Capitolo 7: Processi di ingegneria dei requisiti.
	\end{itemize} 
	\item
	\textbf{Slide dell’insegnamento - Diagrammi dei casi d’uso:}  Reperibili all'indirizzo: \\ \url{http://www.math.unipd.it/~tullio/IS-1/2016/Dispense/E01b.pdf}.
\end{itemize}