\section{Introduzione}

	\begin{itemize}
		\item Tipologia: riunione informativa e decisionale;
		\item Redatto da: \NS;
		\item Data: 31 marzo 2017;
		\item Luogo: LabP140 - Paolotti;
		\item Ora inizio: 10.00;
		\item Ora fine: 15.00;
		\item Presenti: \AS, \DS, \MC, \AN, \NS, \DAN;	
		\item Assenti: nessuno.
	\end{itemize}

\section{Riassunto}
Il giorno venerdì 31 marzo, il gruppo si riunisce dopo aver concluso il periodo di correzione della documentazione post Revisione di Progettazione. Il \RdP\ suggerisce al team di formare al suo interno dei mini-team, al fine di poter procedere in modo parallelo ed ordinato alla codifica della piattaforma \progetto.\\
Si formano, così, 2 sotto-gruppi: uno responsabile del back-end e l'altro del front-end. Inoltre, viene decisa una data di scadenza nella quale i gruppi dovranno cominciare a collaborare per integrare le componenti sviluppate.
Infine, vengono tracciate le linee guida del nuovo documento Definizione di Prodotto.

\section{Tracciamento decisioni}
Di seguito, vengono evidenziate le principali decisioni prese durante la riunione interna del 31 marzo 2017.

\begin{table}[H]
	\begin{center}
		\begin{tabular}{|p{3cm}| p{11cm}|}
			\hline
			\textbf{Identificativo}	& \textbf{Descrizione} \\
			\hline
			VI\_03	& Creazione di due sotto-gruppi (uno per il lato back-end e uno per il lato front-end) per un avanzamento parallelo ed ordinato	\\
			\hline
			VI\_04 & Assegnamento dei componenti del team a ciascun sotto-gruppo formato \\
			\hline
			VI\_05 & Viene avviata la stesura del nuovo documento Definizione di Prodotto \\
			\hline
		\end{tabular}
		\caption{Tracciamento decisioni riunione 31 marzo 2017}
	\end{center}
\end{table}