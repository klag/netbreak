\newglossaryentry{PDCA}
{
	name=PDCA,
	description={Noto anche come ciclo di Deming, è un metodo di gestione in quattro fasi iterativo, utilizzato in attività per il controllo e il miglioramento continuo dei processi e dei prodotti}
}

\newglossaryentry{PHP}
{
	name=PHP,
	description={Acronimo ricorsivo per Hypertext Preprocessor (preprocessore di ipertesti). \MakeUppercase{è} un linguaggio di scripting interpretato, concepito per la programmazione di pagine web dinamiche. L'interprete PHP è un software gratuito, distribuito sotto licenza PHP. Attualmente, è principalmente utilizzato per sviluppare applicazioni web lato server, ma può essere usato anche per scrivere script a riga di comando o applicazioni stand-alone con interfaccia grafica}
}

\newglossaryentry{PHP 7}
{
	name=PHP 7,
	description={Al momento è l'ultima versione stabile di PHP, rilasciata nel dicembre 2015}
}

\newglossaryentry{PHP Code Checker}
{
	name=PHP Code Checker,
	description={Servizio gratuito online, che aiuta a controllare la validità dei documenti PHP}
}

\newglossaryentry{PostgreSQL}
{
	name=PostgreSQL,
	description={Completo sistema di gestione di basi di dati ad oggetti}
}

\newglossaryentry{Python 3}
{
	name=Python 3,
	description={Linguaggio di programmazione ad alto livello, orientato agli oggetti, adatto per sviluppare applicazioni distribuite, scripting, computazione numerica e system testing}
}

