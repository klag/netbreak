\newpage
\section{F}

\begin{itemize}
	\item \textbf{Fagan Inspection}: tecnica di analisi statica che consiste nella lettura dettagliata del documento o del codice, al fine di trovare gli errori indicati su una lista costruita sulla base di un'analisi di tipo Walkthrough.
	\item \textbf{FA-TTS}: acronimo per Flexible and Adaptive Text To Speech, è un servizio di tipo SaaS che permette la creazione semplice e veloce di sintesi vocale basata su un input di testo. I vari parametri quali la lingua, lo stile, il genere, l'età e la voce, possono essere modificati al fine di raggiungere la voce sintetica più adatta.
	\item \textbf{Formal Walkthrough}: tecnica di analisi statica che consiste nella lettura a largo spettro del documento o del codice, al fine di individuare erorri, senza avere un'idea precisa di cosa cercare.
	\item \textbf{Framework}: ambiente software universale e riusabile, che fornisce particolari funzionalità al fine di facilitare lo sviluppo di applicazioni software complesse. Un framework può contenere programmi di supporto, compilatori, librerie, set di strumenti e API.
	\item \textbf{Front-end}: parte di un sistema software che gestisce l'interazione con l'utente o con sistemi esterni in grado di produrre dati di ingresso.
\end{itemize}
