\newglossaryentry{SaaS}
{
	name=SaaS,
	description={Acronimo per Software-as-a-Service, è un modello di distribuzione del software applicativo, dove un produttore di software sviluppa, opera (direttamente o tramite terze parti) e gestisce un'applicazione web che mette a disposizione dei propri clienti via Internet. Si tratta di un servizio di cloud computing}
}

\newglossaryentry{SASS}
{
	name=SASS,
	description={Acronimo per Syntactically Awesome StyleSheets, è un'estensione del linguaggio CSS che permette di utilizzare variabili, di creare funzioni e di organizzare il foglio di stile in più file. Il linguaggio SASS si basa sul concetto di preprocessore CSS, il quale serve a definire fogli di stile con una forma più semplice, completa e potente rispetto ai CSS e a generare file CSS ottimizzati, aggregando le strutture definite anche in modo complesso. Per poter usare SASS è necessario installarlo tramite Ruby}
}

\newglossaryentry{server}
{
	name=server,
	description={Componente o sottosistema informatico di elaborazione e gestione del traffico di informazioni, che fornisce, a livello logico e fisico, un qualunque tipo di servizio ad altre componenti, dette client}
}
\newglossaryentry{Shiny}
{
	name=Shiny,
	description={Framework per applicazioni web per il linguaggio di programmazione R. Esso trasforma le analisi in applicazioni web interattive, senza avere conoscenze di HTML, CSS o JavaScript}
}

\newglossaryentry{SiriSDK}
{
	name=SiriSDK,
	description={Sistema di comando vocale sviluppato da Apple Inc. }
}

\newglossaryentry{Skype}
{
	name=Skype,
	description={Software gratuito di messaggistica istantanea, chiamate e videochiamate con protocollo VoIP senza costi, semplicemente sfruttando la propria connessione internet. La versione SkypeOut permette di effettuare telefonate anche a numeri fissi}
}

\newglossaryentry{SLA}
{
	name=SLA,
	description={Acronimo per Service Level Agreement, letteralmente contratto di servizio, è un impegno ufficiale che prevale tra un fornitore di servizi e un cliente, i quali concordano alcuni aspetti particolari del servizio come qualità, disponibilità, responsabilità, etc}
}

\newglossaryentry{Slack}
{
	name=Slack,
	description={Strumento per la comunicazione interna e il coordinamento di attività di un'azienda o di un team di sviluppo. \MakeUppercase{è} disponibile per la maggior parte dei sistemi operativi, quali Windows, Linux, Mac OS X, come applicazione web o applicazione desktop, mentre su dispositivi mobili come app per iOS, Android e Windows Phone}
}

\newglossaryentry{Smartsheet}
{
	name=Smartsheet,
	description={strumento di gestione e collaborazione per aziende di tutte le dimensioni, che permette la condivisione dei progetti, l'importazione da fogli di calcolo Excel, Project o Google e la generazione di diagrammi Gantt a partire da tabelle simil-Excel}
}

\newglossaryentry{SOA}
{
	name=SOA,
	description={SOA, acronimo di Service-oriented architecture, è un design pattern architetturale orientato agli oggetti}
}

\newglossaryentry{SPICE}
{
	name=SPICE,
	description={Acronimo di Software Process Improvement and Capability Determination è un termine utilizzato per indicare lo standard ISO/IEC 15504}
}

\newglossaryentry{SQL}
{
	name=SQL,
	description={acronimo per Structured Query Language, è un linguaggio standardizzato per database basati sul modello relazionale (RDBMS). Non è semplicemente un linguaggio di interrogazione, ma si occupa anche della creazione, gestione e amministrazione del database}
}

\newglossaryentry{SQLFiddle}
{
	name=SQLFiddle,
	description={SQL Fiddle è un simulatore online di ambienti database. L'utente può scegliere l'ambiente, generare uno schema test, inserire query ed eseguirle e visualizzare i risultati. La simulazione create può essere condivisa con altri user attraverso un link}
}

\newglossaryentry{Sublime Text}
{
	name=Sublime Text,
	description={un editor di testo sofisticato per lo sviluppo di codice, con un'interfaccia pulita e intuitiva. Una delle sue peculiarit\`{a} \`{e} la capacità di adattarsi a diversi linguaggi e a diverse metodologie di sviluppo, grazie soprattutto alla vasta scelta di plug-in disponibili}
}

\newglossaryentry{subtask}
{
	name=subtask,
	description={sottoattività necessaria per il completamento dell'attività da cui ne deriva}
}

\newglossaryentry{Swift}
{
	name=Swift,
	description={linguaggio di programmazione multiparadigma, compilato, sviluppato da Apple Inc}
}
