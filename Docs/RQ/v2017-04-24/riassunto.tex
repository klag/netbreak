\section{Introduzione}

	\begin{itemize}
		\item Tipologia: riunione informativa;
		\item Redatto da: \AS;
		\item Data: 24 aprile 2017;
		\item Luogo: Skype;
		\item Ora inizio: 10.30;
		\item Ora fine: 12.30;
		\item Presenti: \DS, \NS, \MC, \DAN, \AS, \AN;	
		\item Assenti: nessuno.
	\end{itemize}

\section{Riassunto}
Il giorno lunedì 24 aprile, il team \gruppo\ ed il proponente si confrontano via Skype per valutare l'avanzamento del progetto.\\
Il problema della generazione della documentazione è stato risolto con l'ausilio del proponente, ed è stato affrontato l'ambito sicurezza delle API Key.\\
Inizialmente, il team aveva pensato ad una soluzione che richiedeva l'utilizzo di algoritmi di criptaggio a chiave pubblica, con lo svantaggio di peggiorare le performance,  appesantendo l'API Gateway, responsabile del controllo della validità di una API Key al momento del suo inserimento per utilizzo lato client.\\
Il proponente, al momento, suggerisce una soluzione "non protetta", in stile API Google, dove al massimo le API Key potranno essere codificate con qualche funzione di hash, giusto perchè non appaiano in chiaro.

\section{Tracciamento decisioni}
Di seguito, vengono evidenziate le principali decisioni prese durante la riunione esterna del 24 aprile 2017.

\begin{table}[H]
	\begin{center}
		\begin{tabular}{|p{3cm}| p{11cm}|}
			\hline
			\textbf{Identificativo}	& \textbf{Descrizione} \\
			\hline
			VE\_16	&  Risolto il problema della generazione automatica della documentazione \\
			\hline
			VE\_17 & Secondo il proponente, per il momento, le API Key non richiedono alcun livello di protezione e possono essere trasmesse senza controlli o criptaggi vari \\
			\hline
		\end{tabular}
		\caption{Tracciamento decisioni riunione 24 aprile 2017}
	\end{center}
\end{table}