\newpage
\subsection{Requisiti di qualità}
\begin{longtable}{|c|m{8cm}|c|}

\hline \rowcolor{Gray}
\textbf{Codice requisito} & \textbf{Descrizione} & \textbf{Fonte} \\
\hline
\endhead

\hypertarget{RQO1}{RQO1} & Per ogni servizio, deve essere fornita la descrizione di ogni servizio e delle singole API, l'interfaccia delle API, lo schema design relativo all'eventuale base di dati associata & \makecell*{Capitolato} \\
\hline

\hypertarget{RQO2}{RQO2} & Deve essere fornito il sequence chart diagram delle interazioni che prevedono il coinvolgimento di più microservizi & \makecell*{Capitolato} \\
\hline

\hypertarget{RQO3}{RQO3} & Devono essere forniti gli algoritmi delle policy per l'utilizzo delle API & \makecell*{Capitolato} \\
\hline

\hypertarget{RQO4}{RQO4} & Deve essere fornito l'algoritmo di generazione delle API key & \makecell*{Capitolato} \\
\hline

\hypertarget{RQO5}{RQO5} & Il prodotto finale deve superare i test forniti da ItalianaSoftware & \makecell*{Capitolato} \\
\hline

\hypertarget{RQO6}{RQO6} & Il prodotto finale deve essere depositato su una repository git &\makecell*{Capitolato} \\
\hline

\hypertarget{RQO7}{RQO7} & Deve essere stilato breve report tecnico che evidenzi gli aspetti positivi e negativi di un'architettura a microservizi &\makecell*{Capitolato} \\
\hline

\hypertarget{RQD8}{RQD8} & Deve essere redatto un manuale utentes per l'installazione e l'avvio dell'applicazione web &\makecell*{Interno} \\
\hline

\caption{Tabella dei requisiti di qualità}
\end{longtable}
