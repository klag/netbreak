\section{Introduzione}
Lo scopo di questo documento è quello di fornire una rapida consultazione riepilogativa di tutte le scelte significative prese durante le riunioni esterne del team \gruppo, come spiegato nel documento \textsc{Norme Di Progetto 3\_0\_0.pdf}.\\
Per facilitarne la comprensione, viene presentata una tabella composta di tre colonne, una per identificare il verbale esterno al quale si riferisce, una per l'identificativo e l'ultima per una breve descrizione.

\section{Tracciamento Verbali Esterni}
Di seguito, vengono raccolte tutte le principali decisioni prese durante le varie riunioni esterne sostenute dal team nel periodo che ha preceduto la \RQ.

\begin{table}[H]
	\begin{center}
		\begin{tabular}{|p{5.5cm} |p{2.5cm}| p{7cm}|}
			\hline
			\textbf{Verbale} & \textbf{Id Decisione}	& \textbf{Descrizione} \\
			\hline
			Verbale Esterno 2017-03-17 & VE\_01	& Viene richiesto al proponente se è possibile includere/creare dinamicamente interfaccia e outputPort in Jolie	\\
			\hline
			Verbale Esterno 2017-03-17 & VE\_02 & Visione della documentazione ufficiale Jolie suggerita dal proponente \\
			\hline
			Verbale Esterno 2017-03-17 & VE\_03 & Presa di accordi col proponente secondo le sue dirette specifiche in merito all'API Gateway \\
			\hline
			Verbale Esterno 2017-03-17 & VE\_04 & Dopo la chiamata Skype, i progettisti del team si accordano secondo le specifiche appena fornite \\
			\hline
			Verbale Esterno 2017-04-06 & VE\_05	& Viene sottoposto al proponente il primo prototipo di API Gateway \\
			\hline
			Verbale Esterno 2017-04-06 & VE\_06 &  Il proponente suggerisce una soluzione che utilizzi aggregation e courier, combinate con un servizio di redirection dinamica \\
			\hline
			Verbale Esterno 2017-04-06 & VE\_07 & Con la soluzione proposta sarà possibile estendere il gateway, fornendone una versione distribuita\\
			\hline
			Verbale Esterno 2017-04-10 & VE\_08	& Viene sottoposto al proponente il secondo prototipo di API Gateway \\
			\hline
			Verbale Esterno 2017-04-10 & VE\_09 & Vengono concordate con il proponente le imminenti modifiche che i programmatori andranno ad apportare alla soluzione proposta, al fine di consolidarla \\
			\hline
			Verbale Esterno 2017-04-10 & VE\_10 & Valutazione dei metodi di deployment di servizi Jolie\\
			\hline
			Verbale Esterno 2017-04-13 & VE\_11	&  Il proponente concorda che è corretto assumere 1 servizio = n interfacce = n outputPort \\
			\hline
			Verbale Esterno 2017-04-13 & VE\_12 &  Scelta di JolieDoc come tool per l'inserimento della documentazione di un microservizio \\
			\hline
			Verbale Esterno 2017-04-13 & VE\_13 & La piattaforma \progetto\ non sarà responsabile del deployment dei microservizi, ma avrà il compito di collezionare microservizi e tenere traccia dei relativi dati di utilizzo (data monitoring) \\
			\hline
			Verbale Esterno 2017-04-19 & VE\_14	&  Marketplace Mashape come fonte di ispirazione per il front-end dell'applicazione web \\
			\hline
			Verbale Esterno 2017-04-19 & VE\_15 &  Scelta del tool \textit{jolie2surface} e dei javaservice \textit{MetaJolie} e \textit{Parser} per la documentazione di una API\\
			\hline
			Verbale Esterno 2017-04-24 & VE\_16	&  Risolto il problema della generazione automatica della documentazione \\
			\hline
			Verbale Esterno 2017-04-24 & VE\_17 & Secondo il proponente, per il momento, le API Key non richiedono alcun livello di protezione e possono essere trasmesse senza controlli o criptaggi vari \\
			\hline
		\end{tabular}
		\caption{Tracciamento Verbali Esterni RQ}
	\end{center}
\end{table}