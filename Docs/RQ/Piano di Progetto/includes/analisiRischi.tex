\newpage
\section{Analisi dei rischi}
In questa sezione del documento vengono elencati i potenziali rischi che potrebbero verificarsi durante la realizzazione del prodotto API Market e la metodologia adottata per la loro identificazione.

\subsection{Metodologia}

La procedura che il gruppo \textit{\gruppo} intende utilizzare per la gestione dei rischi è composta dalle seguenti fasi:
\begin{itemize}
	\item \textbf{Identificazione:} vengono individuati tutti i potenziali rischi che possono presentarsi durante lo sviluppo del progetto, al fine di studiarne la loro natura. Essi, infatti, possono essere di tre tipi:
	\begin{itemize}
		\item \textbf{Progetto:} relativi a pianificazione, strumenti e risorse;
		\item \textbf{Prodotto:} relativi a conformità e aspettative del committente;
		\item \textbf{Mercato:} relativi a costi e concorrenza.
	\end{itemize}
	\item \textbf{Analisi:} per ogni rischio, si studiano le probabilità di avvenimento e le possibili conseguenze, al fine di capirne criticità e grado di incidenza sul progetto;
	\item \textbf{Pianificazione:} vengono istituiti dei metodi per prevenire i rischi individuati e definiti dei piani alternativi per la loro gestione.
	\item \textbf{Controllo:} ogni rischio viene costantemente monitorato al fine di mitigarne gli effetti. Alcune possibili attività possono essere:
		\begin{itemize}
		\item \textbf{Verifica del livello di rischio};
		\item \textbf{Riconoscimento, trattamento e aggiornamento delle strategie}.
		\end{itemize}
	\end{itemize}
\MakeUppercase{è} inoltre di fondamentale importanza riportare periodicamente ogni rischio serio all'attenzione del \textit{\RdP}.

\subsection{Fattori di rischio}

Per ogni rischio viene fornito il seguente elenco di informazioni, necessario per comprenderne la natura:
\begin{itemize}
	\item \textbf{Nome:} identificativo per discriminare l'ambito del rischio;
	\item \textbf{Descrizione:} breve descrizione dello scenario con cui si presenta il rischio individuato;
	\item \textbf{Occorrenza:} indica la possibilità che si verifichi il rischio;
	\item \textbf{Pericolosità:} indica il grado di pericolosità del rischio;
	\item \textbf{Riconoscimento:} fornisce un metodo che permette di riconoscere il rischio;
	\item \textbf{Trattamento:} fornisce una soluzione affinchè si riducano ulteriormente le possibilità di occorrenza.
\end{itemize}

Inoltre vengono presentate le azioni correttive intraprese atte a mitigare il verificarsi delle situazioni di rischio, se occorse. Le azioni correttive vengono presentate con il seguente elenco
\begin{itemize}
	\item\textbf{Descrizione:} breve descrizione dello scenario con cui si presentato il rischio;
	\item\textbf{Azioni correttive:} azioni intraprese per mitigare il rischio presentato.
\end{itemize}

\textbf{N.B.:} le voci \textit{Occorrenza} e \textit{Pericolosità} possono assumere i valori \{1, 2, 3\}, che corrispondono rispettivamente a livello basso, medio e alto.\\\\
Un rischio può dipendere da diversi fattori:
\begin{itemize}
	\item Tecnologie;
	\item Rapporti personali;
	\item Organizzazione del lavoro;
	\item Requisiti e rapporti con gli stakeholder;
	\item Tempi e costi.
\end{itemize}

\subsubsection{Tecnologie}

Nelle tabelle presentate di seguito, sono elencati e descritti i possibili scenari di rischi a livello tecnologico.

\paragraph{Tecnologie adottate}

\begin{table}[H]
	\begin{center}
		\begin{tabular}{|l | p{11cm}|}
			\hline
			\textbf{Descrizione}	& Lo studio e l'utilizzo delle tecnologie web per la realizzazione del prodotto richiesto, può portare delle difficoltà al momento dell'integrazione con la tecnologia a microservizi. Inoltre, è possibile fare affidamento sul committente per problemi e/o incomprensioni riguardanti \textit{Jolie\ped{G}} e le sue features. \\
			\hline
			\textbf{Occorrenza}	&	\textbf{2}	\\
			\hline
			\textbf{Pericolosità}	&	\textbf{2}	\\
			\hline
			\textbf{Riconoscimento}	&	Il \textit{\RdP}\ deve verificare il grado di preparazione di ogni componente del gruppo in merito alle tecnologie scelte.	\\
			\hline
			\textbf{Trattamento}	&	Ogni membro del gruppo ha il compito di studiare autonomamente tutte le tecnologie necessarie alla realizzazione del prodotto, facendo uso del materiale e dei documenti forniti dal\textit{\RdP}.	\\
			\hline
		\end{tabular}
	\caption{Tabella dei rischi riguardante le tecnologie adottate}
	\end{center}
\end{table}
\subparagraph{Mitigazione rischio}

\begin{table}[H]
	\begin{center}
		\begin{tabular}{|l | p{11cm}|}
			\hline
			\textbf{Descrizione}	& I problemi sono emersi a causa di una scarsa conoscenza da parte del gruppo
			delle tecnologie e approcci richiesti per un architettura e un applicativo a microservizi. \\
			\hline
			\textbf{Numero occorrenze} & 2 \\
			\hline
			\textbf{Azioni correttive}	&	Il \textit{Responsabile di Progetto} ha preso coscienza del livello di comprensione per tali tematiche, ed è stata organizzata una serie di apposite riunioni per discutere dell'approccio tecnologico da adottare. Inoltre, tramite una più fitta comunicazione con il
			\textit{Proponente} si son chiariti i punti meno chiari.	\\
			\hline
		\end{tabular}
		\caption{Tabella relativa alla mitigazione dei rischi per le tecnologie adottate}
	\end{center}
\end{table}

\paragraph{Guasti hardware}

\begin{table}[H]
	\begin{center}
		\begin{tabular}{|l | p{11cm}|}
			\hline
			\textbf{Descrizione}	& Ogni componente del gruppo è dotato di computer portatili non professionali, quindi bisogna prendere in considerazione il rischio di rottura di uno di questi. \\
			\hline
			\textbf{Occorrenza}	&	\textbf{1}	\\
			\hline
			\textbf{Pericolosità}	&	\textbf{2}	\\
			\hline
			\textbf{Riconoscimento}	&	Ogni membro del gruppo deve prestare attenzione verso i propri strumenti hardware di lavoro.	\\
			\hline
			\textbf{Trattamento}	&	Ogni membro del gruppo possiede un dispositivo di riserva, in modo da poter proseguire il lavoro in caso di guasti o malfunzionamenti hardware.	\\
			\hline
		\end{tabular}
		\caption{Tabella dei rischi riguardante i guasti hardware}
	\end{center}
\end{table}


\paragraph{Malfunzionamenti del server}

\begin{table}[H]
	\begin{center}
		\begin{tabular}{|l | p{11cm}|}
			\hline
			\textbf{Descrizione}	& Il \textit{server\ped{G}} destinato ad ospitare il progetto presenta dei malfunzionamenti, il che mette a rischio l'intero lavoro per il gruppo. \\
			\hline
			\textbf{Occorrenza}	&	\textbf{1}	\\
			\hline
			\textbf{Pericolosità}	&	\textbf{3}	\\
			\hline
			\textbf{Riconoscimento}	&	Il progetto risiede anche su un server locale, che funge da alternativa nel caso in cui il server principale presenti dei problemi.	\\
			\hline
			\textbf{Trattamento}	&	\MakeUppercase{è} compito dell'\textit{\Amm} risolvere il malfunzionamento nel minor tempo possibile e riportare allo stato funzionante il server principale.	\\
			\hline
		\end{tabular}
		\caption{Tabella dei rischi riguardante i malfunzionamenti del server}
	\end{center}
\end{table}


\subsubsection{Rapporti personali}

Di seguito, sono elencati e descritti i possibili scenari di rischi a livello del personale.

\paragraph{Problemi interni al team}

\begin{table}[H]
	\begin{center}
		\begin{tabular}{|l | p{11cm}|}
			\hline
			\textbf{Descrizione}	& Dato che, per ogni membro del gruppo, si tratta della prima esperienza di lavoro in un team di queste dimensioni, è importante non sottovalutare gli eventuali problemi di collaborazione. Essi, infatti, potrebbero portare instabilità interna, con conseguenti ritardi nella presentazione del progetto. \\
			\hline
			\textbf{Occorrenza}	&	\textbf{1}	\\
			\hline
			\textbf{Pericolosità}	&	\textbf{3}	\\
			\hline
			\textbf{Riconoscimento}	& \MakeUppercase{è} importante che ci sia un rapporto costante con il \textit{\RdP}, affinchè egli possa monitorare e gestire ogni tipo di problematica sorta.	\\
			\hline
			\textbf{Trattamento}	&	In caso di contrasti tra membri del gruppo, è compito del \textit{\RdP} affidare alle persone coinvolte, delle attività che non siano strettamente legate. Questo fa sì che non venga influenzato il clima di lavoro per gli altri componenti del gruppo.	\\
			\hline
		\end{tabular}
		\caption{Tabella dei rischi riguardante i problemi interni al team}
	\end{center}
\end{table}

\paragraph{Problemi personali individuali}

\begin{table}[H]
	\begin{center}
		\begin{tabular}{|l | p{11cm}|}
			\hline
			\textbf{Descrizione}	& Possono verificarsi problemi organizzativi dovuti a sovrapposizioni di impegni e necessità proprie di ogni membro del gruppo. Ad esempio, un componente del team è anche un lavoratore presso un'azienda, quindi, occorre prestare attenzione alla gestione di casi simili. \\
			\hline
			\textbf{Occorrenza}	&	\textbf{3}	\\
			\hline
			\textbf{Pericolosità}	&	\textbf{3}	\\
			\hline
			\textbf{Riconoscimento}	&	Per evitare rischi di disorganizzazione, occorre fornire preventivamente e tempestivamente al \textit{\RdP}, tutti gli impegni di ogni componente del gruppo. Nel caso specifico del membro lavoratore, quest'ultimo dovrà fornire costantemente un calendario aggiornato contenente tutti i suoi impegni lavorativi.	\\
			\hline
			\textbf{Trattamento}	&	Per ogni impegno notificato, il \textit{\RdP}\ avrà il compito di ripianificare parte delle attività da svolgere per sopperire alle mancanze lavorative. Nel caso del membro lavoratore, dovranno essergli forniti gli strumenti necessari, affinchè egli possa essere aggiornato sull'andamento dello sviluppo del progetto e non influenzi negativamente il lavoro del team.	\\
			\hline
		\end{tabular}
		\caption{Tabella dei rischi riguardante i problemi personali individuali}
	\end{center}
\end{table}

\subparagraph{Mitigazione problemi personali individuali}

\begin{table}[H]
	\begin{center}
		\begin{tabular}{|l | p{11cm}|}
			\hline
			\textbf{Descrizione}	& I problemi si sono verificati a causa della corrispondenza di impegni didattici
			(esami universitari) e lavorativi dei membri del gruppo. \\
			\hline
			\textbf{Numero occorrenze} & 3 \\
			\hline
			\textbf{Azioni correttive}	&	Per una più corretta pianificazione è stata rafforzata la comunicazione
			relativa agli impegni personali. Il \textit{Responsabile} inoltre verifica con maggior costanza lo
			stato di ciascun membro e ripianifica le eventuali attività su base giornaliera per far fronte
			ad ogni imprevisto.	\\
			\hline
		\end{tabular}
		\caption{Tabella relativa alla mitigazione dei rischi per problemi personali individuali}
	\end{center}
\end{table}

\subsubsection{Organizzazione del lavoro}

Di seguito, sono elencati e descritti i possibili scenari di rischi a livello organizzativo.

\paragraph{Pianificazione errata}

\begin{table}[H]
	\begin{center}
		\begin{tabular}{|l | p{11cm}|}
			\hline
			\textbf{Descrizione}	& Durante l'attività di Pianificazione, è possibile che i tempi per lo svolgimento di alcune attività vengano calcolati in modo errato. \\
			\hline
			\textbf{Occorrenza}	&	\textbf{2}	\\
			\hline
			\textbf{Pericolosità}	&	\textbf{3}	\\
			\hline
			\textbf{Riconoscimento}	&	Bisogna monitorare costantemente lo stato delle attività nel programma di project management, in modo da gestire eventuali ritardi nello sviluppo delle attività stesse.	\\
			\hline
			\textbf{Trattamento}	&	Per ogni attività, è previsto un periodo maggiore di quanto normalmente richiesto. Ciò consente ad un eventuale ritardo di non impattare negativamente sulla durata totale del progetto.	\\
			\hline
		\end{tabular}
		\caption{Tabella dei rischi riguardante una pianificazione errata}
	\end{center}
\end{table}

\subsubsection{Requisiti e rapporti con gli stakeholder}

Di seguito, sono elencati e descritti i possibili scenari di rischi a livello dei requisiti.

\paragraph{Incomprensioni sui requisiti}

\begin{table}[H]
	\begin{center}
		\begin{tabular}{|l | p{11cm}|}
			\hline
			\textbf{Descrizione}	& Alcuni requisiti individuati dagli \textit{\Anas} possono essere interpretati in modo errato oppure possono, a loro volta, implicare ulteriori requisiti. Inoltre, è possibile che alcuni requisiti vengano aggiunti, modificati o eliminati, a seconda degli accordi presi con il proponente. \\
			\hline
			\textbf{Occorrenza}	&	\textbf{2}	\\
			\hline
			\textbf{Pericolosità}	&	\textbf{3}	\\
			\hline
			\textbf{Riconoscimento}	&	Vengono fissati degli incontri con il proponente, in modo da concordare sulla visione del prodotto, al fine di fornire un prodotto conforme alla richieste. Ad ogni revisione prevista, i documenti relativi al progetto verranno consegnati e valutati dal committente. \\
			\hline
			\textbf{Trattamento}	&	E' indispensabile correggere tutti gli eventuali errori e/o le imprecisioni individuate dal committente in seguito ad ogni revisione.	\\
			\hline
		\end{tabular}
		\caption{Tabella dei rischi riguardante incomprensioni sui requisiti}
	\end{center}
\end{table}

\subparagraph{Mitigazione per incompresioni sui requisiti}

\begin{table}[H]
	\begin{center}
		\begin{tabular}{|l | p{11cm}|}
			\hline
			\textbf{Descrizione}	&  Il problema si è verificato in fase di progettazione. Alcuni aspetti, seppur
			marginali, del progetto erano infatti stati travisati e han richiesto tempestive correzioni.	\\
			\hline
			\textbf{Numero occorrenze} & 1 \\
			\hline
			\textbf{Azioni correttive}	&	\MakeUppercase{è} stata predisposta un opportuna sessione di lavoro con il \textit{Proponente} per chiarire i dubbi relativi al prodotto desiderato. Inoltre, è stata rafforzata la
			comunicazione con il secondo team al lavoro sul medesimo progetto, nonchè con lo stesso
			\textit{Proponente}.	\\
			\hline
		\end{tabular}
		\caption{Tabella relativa alla mitigazione dei rischi per incompresioni sui requisiti}
	\end{center}
\end{table}



\subsubsection{Tempi e costi}

Di seguito, sono elencati e descritti i possibili scenari di rischi a livello di valutazione dei costi.

\paragraph{Stime e previsioni}

\begin{table}[H]
	\begin{center}
		\begin{tabular}{|l | p{11cm}|}
			\hline
			\textbf{Descrizione}	& I tempi stabiliti nella pianificazione delle attività per lo svolgimento del progetto vengono sovrastimate o sottostimate. Ciò può comportare una variazione sul costo preventivo presentato. \\
			\hline
			\textbf{Occorrenza}	&	\textbf{2}	\\
			\hline
			\textbf{Pericolosità}	&	\textbf{2}	\\
			\hline
			\textbf{Riconoscimento}	& Un’attività si dice sottostimata se occupa molto più tempo di quello preventivato; invece, un'attività si dice sovrastimata, se occupa meno tempo rispetto a quello previsto. Il \textit{\RdP} deve controllare con attenzione il programma di project management ed intervenire tempestivamente per modificare la pianificazione e il rendiconto dei costi.	\\
			\hline
			\textbf{Trattamento}	&	Ogni qualvolta viene assegnata un'attività ad un membro del team, quest'ultimo ha l'obbligo di rispettare i tempi e le scadenza stabilite dal \textit{\RdP}.	\\
			\hline
		\end{tabular}
		\caption{Tabella dei rischi riguardante stime e previsioni}
	\end{center}
\end{table}

\subparagraph{Mitigazione per stime e previsioni}

\begin{table}[H]
	\begin{center}
		\begin{tabular}{|l | p{11cm}|}
			\hline
			\textbf{Descrizione}	& Per motivi per lo più legati alla poca chiarezza negli aspetti di tecnologie
			adottate, il lavoro di progettazione ha richiesto più ore del preventivato. \\
			\hline
			\textbf{Numero occorrenze} & 2 \\
			\hline
			\textbf{Azioni correttive}	&	Il \textit{Responsabile} ha effettuato una ripianificazione dei task, anche in seguito alle riunioni interne, per sopperire alle attività di Progettazione sottostimate.\\
			\hline
		\end{tabular}
		\caption{Tabella relativa alla mitigazione dei rischi per stime e previsioni}
	\end{center}
\end{table}


\begin{minipage}{\linewidth}
	
\subsubsection{Indici di rischio}

Viene elencata di seguito una tabella che riassume gli indici di rischio individuati ai punti precedenti, indicati con il valore \textbf{Alto, Medio, Basso}. Essi permettono una rapida consultazione di quali sono i rischi individuati che presentano la maggior incidenza, nonchè quelli che potenzialmente possono provocare i maggiori inconvenienti. 

\begin{table}[H]
	\begin{center}
		\begin{tabular}{|p{5cm} | p{5cm} | p{5cm}|}
			\hline
			\textbf{Tipo di rischio}	& \textbf{Occorrenza} & \textbf{Pericolosità}\\
			\hline
			Tecnologie adottate	&	Media 	& 	Media	\\
			\hline
			Guasti hardware	&	Bassa 	& 	Media	\\
			\hline
			Malfunzionamento del server	&	Bassa 	& 	Alta	\\
			\hline
			Problemi interni al team	&	Bassa 	& 	Alta	\\
			\hline
			Problemi personali individuali	&	Alta 	& 	Alta	\\
			\hline
			Pianificazione errata	&	Media 	& 	Alta	\\
			\hline
			Incomprensioni sui requisiti	&	Media 	& 	Alta	\\
			\hline
			Stime e previsioni	&	Media 	& 	Media	\\
			\hline
		\end{tabular}
		\caption{Tabella degli indici di rischio}
	\end{center}
\end{table}
\end{minipage}

\subsection{Analisi di occorrenza rischi}
In questa sezione viene presentato un resoconto dell'occorrenza dei rischi e della loro mitigazione, ovvero le misure correttive attuate per ridurre nuove occorrenze del rischio citato. La tabella sottostante analizza le situazioni verificatesi e l'eventuale impatto che esse hanno avuto ai fini del progetto. I dati analizzati in questa sede riguardano il periodo antecedente all'ultima approvazione del documento

\begin{table}[H]
	\begin{center}

		\begin{tabular}{| p{5cm} | p{5cm} | p{5cm} |}
			\hline
			\textbf{Tipo di rischio}	& \textbf{Occorrenze} & \textbf{Impatto}\\
			\hline
			Tecnologie adottate	&	2 	&  Medio	\\
			\hline
			Guasti hardware	&	0 	& 	\\
			\hline
			Malfunzionamento del server	&	0 	& 	\\
			\hline
			Problemi interni al team	&	0 	& 	\\
			\hline
			Problemi personali individuali	&	3 	&	Alto \\
			\hline
			Pianificazione errata	&	0 	& 	\\
			\hline
			Incomprensioni sui requisiti	&	1 	&	Medio	\\
			\hline
			Stime e previsioni	&	2 	&  Medio  \\
			\hline
		\end{tabular}
		\vline
		\caption{Tabella di occorrenza dei rischi elencati}
			\vline
	\end{center}
\end{table}

\begin{comment}

\subsubsection{Mitigazione rischi}

Di seguito sono presentate le azioni correttive intraprese atte a mitigare il verificarsi delle situazioni di rischio.

\paragraph{Tecnologie adottate}
\begin{itemize}
	\item \textbf{Descrizione}: i problemi sono emersi a causa di una scarsa conoscenza da parte del gruppo delle tecnologie e approcci richiesti per un architettura e un applicativo a microservizi
	\item \textbf{Azioni correttive}: Il \textit{\RdP} ha preso coscienza del livello di comprensione per tali tematiche, ed è stata organizzata una serie di apposite riunioni per discutere dell'approccio tecnologico da adottare. Inoltre, tramite una più fitta comunicazione con il \textit{Proponente} si son chiariti i punti meno chiari. 
\end{itemize}

\paragraph{Problemi personali individuali}
\begin{itemize}
	\item \textbf{Descrizione}: i problemi si sono verificati a causa della corrispondenza di impegni didattici (esami universitari) e lavorativi dei membri del gruppo.
	\item \textbf{Azioni correttive}: per una più corretta pianificazione è stata rafforzata la comunicazione relativa agli impegni personali. Il \textit{\Res} inoltre verifica con maggior costanza lo stato di ciascun membro e ripianifica le eventuali attività su base giornaliera per far fronte ad ogni imprevisto.
\end{itemize}

\paragraph{Incomprensione sui requisiti}
\begin{itemize}
	\item \textbf{Descrizione}: il problema si è verificato in fase di progettazione. Alcuni aspetti, seppur marginali, del progetto erano infatti stati travisati e han richiesto tempestive correzioni
	\item \textbf{Azioni correttive}: è stata predisposta un opportuna sessione di lavoro con il \textit{Proponente} per chiarire i dubbi relativi al prodotto desiderato. Inoltre, è stata rafforzata la comunicazione con il secondo team al lavoro sul medesimo progetto, nonchè con lo stesso \textit{Proponente}.
\end{itemize}

\paragraph{Stime e Previsioni}
\begin{itemize}
	\item \textbf{Descrizione}: Per motivi per lo più legati alla poca chiarezza negli aspetti di Tecnologie adottate, il lavoro di progettazione ha richiesto più ore del preventivato.
	\item \textbf{Azioni correttive}: Il \textit{\Res} ha effettuato una ripianificazione dei task, anche in seguito alle riunioni interne, per sopperire alle attività di Progettazione sottostimate.
\end{itemize}
\end{comment}
