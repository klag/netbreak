\newpage
\section{Resoconto Qualità di Processo}

In questa sezione del documento vengono descritti e analizzati gli esiti delle metriche previste per il controllo della qualità di processo, in base alla revisione di progetto.
La prima tabella contiene le metriche con associato esito applicate sull'intero sistema.
Le successive tabelle sono una per ogni metrica applicata ad un singolo modulo, metodo, funzione, procedura o classe che sia.

	\subsection{Revisione di Qualifica}
	
		\begin{longtable}{|>{\centering\arraybackslash}p{2cm}|>{\centering\arraybackslash}p{5cm}|>{\centering\arraybackslash}p{3cm}|>{\centering\arraybackslash}p{3cm}|}
			\hline
			\rowcolor{Gray}
			\textbf{Id} & \textbf{Metrica} & \textbf{Valore} & \textbf{Esito} \\
			\hline
			3.1.2.1 & \textit{Disponibilità \textit{NetBreakDB}} & xx & Superato\\
			\hline
			3.2.2.1 & \textit{Schedule Variance} & xx & Superato\\
			\hline
			3.2.2.2 & \textit{Budget Variance} & xx & Superato\\
			\hline
			3.3.2.1 & \textit{Rischi non preventivati} & xx & Superato\\
			\hline
			3.4.2.1 & \textit{Adempimento requisiti obbligatori} & xx & Superato\\
			\hline
			3.5.2.1 & \textit{Fan In} & xx & Superato\\
			\hline
			3.5.2.2 & \textit{Fan Out} & xx & Superato\\
			\hline
			3.8.2.3 & \textit{Linee di commento} & xx & Superato\\
			\hline
			3.9.2.1 & \textit{Test di Unità} & xx & Superato\\
			\hline
			3.9.2.2 & \textit{Test di Integrazione} & xx & Superato\\
			\hline
			3.9.2.3 & \textit{Test di Sistema} & xx & Superato\\
			\hline
			3.9.2.4 & \textit{Test superati} & xx & Superato\\
			\hline
			3.11.2.1 & \textit{Code Coverage} & xx & Superato\\
			\hline
		
			\caption{Resoconto esiti metriche - Qualità di processo}
		\end{longtable}

\textbf{Metriche per Software Construction Process}
\begin{itemize}
	\item 3.8.2.1  \textit{Complessità Ciclomatica}  per ogni metodo/funzione/modulo/classe
	\item 3.8.2.2  \textit{Numero di livelli di annidamento}  per ogni metodo
\end{itemize}


\textbf{Metriche per Software Detail Design Process}
\begin{itemize}
	\item 3.7.2.1 \textit{Numero di metodi per classe} per ogni classe
	\item 3.7.2.2  \textit{Numero di parametri per metodo} per ogni metodo
	\item 3.7.2.3  \textit{Numero di attributi per classe} per ogni classe
\end{itemize}