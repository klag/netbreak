\newpage
\section{Resoconto verifica documenti}

In questa sezione del documento vengono descritti e analizzati gli esiti delle attività di verifica svolte su tutti i documenti che vengono consegnati nelle varie revisioni di avanzamento del progetto.
L’analisi dei documenti mediante la tecnica \textit{Walkthrough} ha reso possibile individuare alcuni errori frequenti, che si sono aggiunti alla lista di controllo stilata ed aggiornata nelle varie revisioni, utile per applicare la tecnica \textit{Inspection} nelle future attività di verifica.\\
Il team, attraverso il software interno \textit{NetBreakDB}, è riuscito ad effettuare i tracciamenti per le componenti di interesse previste da ogni revisione (casi d’uso-requisiti, requisiti-fonti, requisiti-componenti, requisiti-classi, etc.).\\
Inoltre, questo software è stato utilizzato per generare le tabelle dei vari test e dei relativi tracciamenti con requisiti, componenti, classi e metodi.
	
	\subsection{Revisione dei Requisiti}
	Di seguito, sono riportati gli esiti delle verifiche sottoposte a tutti i documenti, per il calcolo dell’indice Gulpease.
	
		\begin{longtable}{|>{\centering\arraybackslash}p{5cm}|>{\centering\arraybackslash}p{5cm} | >{\centering\arraybackslash}p{5cm}|}
			\hline
			\rowcolor{Gray}
			\textbf{Documento} & \textbf{Indice Gulpease} & \textbf{Esito} \\
			\hline
			\textit{\NdP} & 49 & Superato\\
			\hline
			\textit{\PdP} & 50 & Superato \\
			\hline
			\textit{\PdQ} & 42 & Superato\\
			\hline
			\textit{\AdR} & 68 & Superato \\
			\hline
			\textit{\SdF} & 54 & Superato\\
			\hline
			\textit{\G}& 43 & Superato\\
			\hline
			\textit{Verbale Interno - 28/11/2016}		& 	60	&	Superato	\\
			\hline
			\textit{Verbale Interno - 01/12/2016}		& 	63	&	Superato	\\
			\hline
			\textit{Verbale Interno - 12/12/2016}		& 	61	&	Superato	\\
			\hline
			\textit{Verbale Esterno - 22/12/2016}		& 	59	&	Superato	\\
			\hline
			\textit{Verbale Interno - 28/12/2016}		& 	61	&	Superato	\\
			\hline
		
		\caption{Resoconto verifiche documenti RR}
	\end{longtable}

\newpage	
	\subsection{Revisione di Progettazione}
	Di seguito, sono riportati gli esiti delle verifiche sottoposte a tutti i documenti, per il calcolo dell’indice Gulpease.
	
			\begin{longtable}{|>{\centering\arraybackslash}p{5cm}|>{\centering\arraybackslash}p{5cm} | >{\centering\arraybackslash}p{5cm}|}
				\hline
				\rowcolor{Gray}
				\textbf{Documento} & \textbf{Indice Gulpease} & \textbf{Esito} \\
				\hline
				\textit{\ST} & 67  & Superato\\
				\hline
				\textit{\NdP} & 57  & Superato\\
				\hline
				\textit{\PdP} & 55 & Superato \\
				\hline
				\textit{\PdQ} & 54  & Superato\\
				\hline
				\textit{\AdR} & 70  & Superato \\
				\hline
				\textit{\G}& 49 & Superato\\
				\hline
				\textit{Verbale Interno - 27/01/2017}		& 	55	&	Superato	\\
				\hline
				\textit{Verbale Interno - 04/02/2017}		& 	64	&	Superato	\\
				\hline
				\textit{Verbale Interno - 13/02/2017}		& 	58	&	Superato	\\
				\hline
				\textit{Verbale Esterno - 16/02/2017}		& 	60	&	Superato	\\
				\hline
				\textit{Verbale Interno - 17/02/2017}		& 	57	&	Superato	\\
				\hline
				\textit{Verbale Interno - 21/02/2017}		& 	63	&	Superato	\\
				\hline
				\textit{Verbale Interno - 02/03/2017}		& 	61	&	Superato	\\
				\hline
				\textit{Verbale Interno - 03/03/2017}		& 	61	&	Superato	\\
				\hline
			
			\caption{Resoconto verifiche documenti RP}
		\end{longtable}

\newpage
	\subsection{Revisione di Qualifica}
	Di seguito, sono riportati gli esiti delle verifiche sottoposte a tutti i documenti, per il calcolo dell’indice Gulpease.
	
			\begin{longtable}{|>{\centering\arraybackslash}p{5cm}|>{\centering\arraybackslash}p{5cm} | >{\centering\arraybackslash}p{5cm}|}
				\hline
				\rowcolor{Gray}
				\textbf{Documento} & \textbf{Indice Gulpease} & \textbf{Esito} \\
				\hline
				\textit{\DDP} &   & Superato\\
				\hline
				\textit{\MU} &   & Superato\\
				\hline
				\textit{\ST} &   & Superato\\
				\hline
				\textit{\NdP} &   & Superato\\
				\hline
				\textit{\PdP} &  & Superato \\
				\hline
				\textit{\PdQ} &   & Superato\\
				\hline
				\textit{\AdR} &   & Superato \\
				\hline
				\textit{\G}&  & Superato\\
				\hline
				\textit{Tracciamento Verbali Interni RQ}		& 		&	Superato	\\
				\hline
				\textit{Tracciamento Verbali Esterni RQ}		& 		&	Superato	\\
				\hline
				\textit{Verbale Interno 2017-03-16}		& 		&	Superato	\\
				\hline
				\textit{Verbale Interno 2017-03-31}		& 		&	Superato	\\
				\hline
				\textit{Verbale Interno 2017-04-28}		& 		&	Superato	\\
				\hline
				\textit{Verbale Esterno 2017-03-17}		& 		&	Superato	\\
				\hline
				\textit{Verbale Esterno 2017-04-06}		& 		&	Superato	\\
				\hline
				\textit{Verbale Esterno 2017-04-10}		& 		&	Superato	\\
				\hline
				\textit{Verbale Esterno 2017-04-13}		& 		&	Superato	\\
				\hline
				\textit{Verbale Esterno 2017-04-19}		& 		&	Superato	\\
				\hline
				\textit{Verbale Esterno 2017-04-24}		& 		&	Superato	\\
				\hline

			\caption{Resoconto verifiche documenti RQ}
		\end{longtable}
