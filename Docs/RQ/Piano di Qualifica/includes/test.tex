\newpage
\section{Test}
Al fine di produrre del software di qualità, nelle successive attività di progetto ,il team ha l'obiettivo di strutturare dei test volti a verificare che il software prodotto rispecchi le funzionalità richieste.
Tutte le attività di testing prodotte devono poter essere ripetibili e deterministiche, al fine di poter fornire informazioni utili a intraprendere azioni correttive, nel caso si ottengano dei risultati diversi da quelli attesi.
Per avere un tracciamento dei test prodotti e dei risultati ottenuti, si è scelto di rappresentare delle tabelle di log di facile consultazione, le quali forniscono un'indicazione degli output delle attività di verifica, eventuali errori e/o risultati non coerenti con quanto fissato.

	\subsection{Test di unità}
	Questa tipologia di test serve a verificare il corretto comportamento dei singoli metodi o funzioni implementate.
	I test di unità saranno descritti nel modo seguente:
	\begin{center}
		\textbf{TU}[\textit{IdTest}]
	\end{center}
	\begin{center}
		dove \textbf{\textit{IdTest}} rappresenta il codice identificativo progressivo dell'unità presa in considerazione.
	\end{center}
	
	\subsection{Test di integrazione}
	Questa tipologia di test serve a verificare il corretto funzionamento delle singole componenti di sistema progettate durante l'attività di progettazione ad alto livello.
	Per questa tipologia di test, l'idea è quella di utilizzare un approccio top-down, il quale rispecchia la strategia incrementale.
	
	\subsection{Test di sistema}
	Questa tipologia di test serve a verificare il corretto comportamento e funzionamento dell’architettura.
	
	\subsection{Test di validazione}
	Questa tipologia di test serve a verificare che il prodotto soddisfi le richieste del proponente attraverso le funzionalità implementate.\\
	Per questo motivo, occorrerà simulare il comportamento generale dell'applicativo e dell'utente che interagisce con esso, attraverso delle macro azioni.
	I test di validazione saranno organizzati nel modo seguente:
	\begin{center}
		\textbf{TV}[\textit{TipologiaRequisito}][\textit{RilevanzaRequisito}][\textit{CodiceRequisito}]
	\end{center}
	dove:
	\begin{itemize}
		\item
		\textbf{TipologiaRequisito} può assumere valori tra:
		\begin{itemize}
			\item
			\textit{V} per i requisiti di vincolo;
			\item
			\textit{F} per i requisiti di funzionalità;
			\item
			\textit{Q} per i requisiti di qualità;
			\item
			\textit{P} per i requisiti prestazionali.
		\end{itemize}
		\item 
		\textbf{RilevanzaRequisito} può assumere valori tra:
		\begin{itemize}
			\item
			\textit{O} per i requisiti obbligatori;
			\item
			\textit{D} per i requisiti desiderabili;
			\item
			\textit{F} per i requisiti facoltativi.
		\end{itemize}
		\item
		\textbf{CodiceRequisito} assume un valore gerarchico che identifica il singolo requisito.
	\end{itemize}

	% TABELLA
	\normalsize
	\begin{longtable}[ht]{|c|>{}m{8cm}|c|}
		\hline 
		\textbf{Id Test} & \textbf{Descrizione} & \textbf{Stato}\\
		\hline
		\endhead
		\hypertarget{TVFO1}{TVFO1} & L’utente intende registrarsi alla piattaforma \progetto. All’utente è richiesto di:
		\begin{itemize}
			\item Trovarsi nella sezione apposita;
			\item Compilare il form di registrazione;
			\item Premere il pulsante di conferma;
			\item Verificare attraverso l’autenticazione che la registrazione sia avvenuta correttamente.
		\end{itemize}
		& \textit{Non Implementato}\\ \hline
		\hypertarget{TVFO2}{TVFO2} & L’utente intende autenticarsi alla piattaforma \progetto. All’utente è richiesto di:
		\begin{itemize}
			\item Trovarsi nella sezione apposita;
			\item Inserire le credenziali nell’apposito form;
			\item Premere il pulsante di autenticazione;
			\item Verificare che l’autenticazione sia effettivamente avvenuta.
		\end{itemize}
		& \textit{Non Implementato}\\ \hline
		\hypertarget{TVFO11}{TVFO11} & L’utente intende disconnettersi dalla piattaforma \progetto. All’utente è richiesto di:
		\begin{itemize}
			\item Essere autenticato;
			\item Trovarsi nella sezione apposita;
			\item Premere il pulsante di logout;
			\item Verificare che la disconnessione sia effettivamente avvenuta.
		\end{itemize}
		& \textit{Non Implementato}\\ \hline
		\hypertarget{TVFO10.1}{TVFO10.1} & L’utente autenticato intende gestire i propri dati. All’utente è richiesto di:
		\begin{itemize}
			\item Essere autenticato;
			\item Trovarsi nella sezione apposita;
			\item Modificare i campi dati consentiti;
			\item Premere il tasto conferma modifica;
			\item Visualizzare il profilo dell’utente modificato.
		\end{itemize}
		& \textit{Non Implementato}\\ \hline
		\hypertarget{TVFxxx}{TVFxxx} & L’utente autenticato  intende modificare la tipologia di utenza. All’utente è richiesto di:
		\begin{itemize}
			\item Essere autenticato;
			\item Trovarsi nella sezione apposita;
			\item Cambiare la tipologia di utenza;
			\item Verificare la modifica effettuata.
		\end{itemize} & \textit{Non Implementato}\\ \hline
		\hypertarget{TVFxxx}{TVFxxx} & L’utente autenticato  intende eliminare il proprio account. All’utente è richiesto di:
		\begin{itemize}
			\item Essere autenticato;
			\item Trovarsi nella sezione apposita;
			\item Premere il tasto di eliminazione account;
			\item Verificare che la disconnessione della sessione sia avvenuta;
			\item Verificare che l’autenticazione con le credenziali dell’utente eliminato provochi un errore.
		\end{itemize} & \textit{Non Implementato}\\ \hline
		\hypertarget{TVFO4}{TVFO4} & L’utente autenticato  intende ricercare una API sul marketplace. All’utente è richiesto di:
		\begin{itemize}
			\item Trovarsi nella sezione apposita;
			\item Ricercare una API digitando le keywords;
			\item Visualizzazione dei dati relativi alla API cercata.
		\end{itemize} & \textit{Non Implementato}\\ \hline
		\hypertarget{TVFO7}{TVFO7} & L’utente autenticato  intende acquistare l'API che sta visualizzando. All’utente è richiesto di:
		\begin{itemize}
			\item Essere autenticato;
			\item Trovarsi nella sezione apposita;
			\item Selezionare la policy di vendita;
			\item Confermare l'acquisto attraverso l'apposito pulsante;
			\item Verificare la transazione nel proprio storico.
		\end{itemize} & \textit{Non Implementato}\\ \hline
		\hypertarget{TVFxxx}{TVFxxx} & L’utente autenticato intende caricare una propria API sulla piattaforma \progetto. All’utente è richiesto di:
		\begin{itemize}
			\item Essere un utente sviluppatore autenticato;
			\item Trovarsi nella sezione apposita;
			\item Premere il pulsante "Inserisci API";
			\item Inserire i dati necessari alla pubblicazione della propria API;
			\item Premere il pulsante di conferma inserimento nuova API;
			\item Verificare che sia stata aggiunta al marketplace l'API.
		\end{itemize} & \textit{Non Implementato}\\ \hline
		\hypertarget{TVFxxx}{TVFxxx} & L’utente autenticato intende modificare i dati relativi ad una sua API caricata. All’utente è richiesto di:
		\begin{itemize}
			\item Essere un utente sviluppatore autenticato;
			\item Trovarsi nella sezione apposita;
			\item Premere il pulsante "Modifica API";
			\item Modificare i dati dell'API selezionata;
			\item Premere il pulsante di conferma modifica;
			\item Verificare che sia stata modificata l'API.
		\end{itemize} & \textit{Non Implementato}\\ \hline
		\hypertarget{TVFxxx}{TVFxxx} & L’utente autenticato intende visualizzare le API da lui caricate sulla piattaforma \progetto. All’utente è richiesto di:
		\begin{itemize}
			\item Essere un utente sviluppatore autenticato;
			\item Trovarsi nella sezione apposita;
			\item Verificare che vengano visualizzati tutte le proprie API inserite.
		\end{itemize} & \textit{Non Implementato}\\ \hline
		\hypertarget{TVFxxx}{TVFxxx} & L’utente autenticato intende visualizzare il numero di licenze attive per le API da lui caricate sulla piattaforma \progetto. All’utente è richiesto di:
		\begin{itemize}
			\item Essere un utente sviluppatore autenticato;
			\item Trovarsi nella sezione apposita;
			\item Verificare che vengano visualizzati i dati relativi alle licenze per le proprie API inserite.
		\end{itemize} & \textit{Non Implementato}\\ \hline
		\hypertarget{TVFxxx}{TVFxxx} & L’utente amministratore della piattaforma \progetto intende moderare un determinato utente registrato. All’utente è richiesto di:
		\begin{itemize}
			\item Essere un utente amministratore autenticato;
			\item Trovarsi nella sezione apposita;
			\item Selezionare l'utente che si vuole moderare;
			\item Inserire i dati del rapporto di moderazione;
			\item Verificare che l'utente sia stato effettivamente moderato.
		\end{itemize} & \textit{Non Implementato}\\ \hline
		\hypertarget{TVFO10.1.1}{TVFO10.1.1} & L’utente autenticato intende visualizzare il proprio profilo. All’utente è richiesto di:
		\begin{itemize}
			\item Essere autenticato;
			\item Trovarsi nella sezione apposita;
			\item Visualizzare il proprio profilo.
		\end{itemize} & \textit{Non Implementato}\\ \hline
		\hypertarget{TVFxxx}{TVFxxx} & L’utente autenticato intende visualizzare lo storico delle transazioni effettuate. All’utente è richiesto di:
		\begin{itemize}
			\item Essere autenticato;
			\item Trovarsi nella sezione apposita;
			\item Visualizzare lo storico delle transazioni.
		\end{itemize} & \textit{Non Implementato}\\ \hline
		\caption[Test di Validazione]{Test di Validazione}
		\label{tabella:test0}
	\end{longtable}
	\clearpage
	