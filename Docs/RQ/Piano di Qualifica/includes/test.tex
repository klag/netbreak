\newpage
\section{Test}
Al fine di produrre del software di qualità, nelle successive attività di progetto ,il team ha l'obiettivo di strutturare dei test volti a verificare che il software prodotto rispecchi le funzionalità richieste.
Tutte le attività di testing prodotte devono poter essere ripetibili e deterministiche, al fine di poter fornire informazioni utili a intraprendere azioni correttive, nel caso si ottengano dei risultati diversi da quelli attesi.
Per avere un tracciamento dei test prodotti e dei risultati ottenuti, si è scelto di rappresentare delle tabelle di log di facile consultazione, le quali forniscono un'indicazione degli output delle attività di verifica, eventuali errori e/o risultati non coerenti con quanto fissato.

	\subsection{Test di unità}
	Questa tipologia di test serve a verificare il corretto comportamento dei singoli metodi o funzioni implementate.
	I test di unità saranno descritti nel modo seguente:
	\begin{center}
		\textbf{TU}[\textit{IdTest}]
	\end{center}
	\begin{center}
		dove \textbf{\textit{IdTest}} rappresenta il codice identificativo progressivo dell'unità presa in considerazione.
	\end{center}
	
	\subsection{Test di integrazione}
	Questa tipologia di test serve a verificare il corretto funzionamento delle singole componenti di sistema progettate durante l'attività di progettazione ad alto livello.
	Per questa tipologia di test, l'idea è quella di utilizzare un approccio top-down, il quale rispecchia la strategia incrementale.
	
	\subsection{Test di sistema}
	Questa tipologia di test serve a verificare il corretto comportamento e funzionamento dell’architettura.
	I test di sistema saranno organizzati nel modo seguente:
	\begin{center}
		\textbf{TS}[\textit{TipologiaRequisito}][\textit{RilevanzaRequisito}][\textit{CodiceRequisito}]
	\end{center}
	dove:
	\begin{itemize}
		\item
		\textbf{TipologiaRequisito} può assumere valori tra:
		\begin{itemize}
			\item
			\textit{V} per i requisiti di vincolo;
			\item
			\textit{F} per i requisiti di funzionalità;
			\item
			\textit{Q} per i requisiti di qualità;
			\item
			\textit{P} per i requisiti prestazionali.
		\end{itemize}
		\item 
		\textbf{RilevanzaRequisito} può assumere valori tra:
		\begin{itemize}
			\item
			\textit{O} per i requisiti obbligatori;
			\item
			\textit{D} per i requisiti desiderabili;
			\item
			\textit{F} per i requisiti facoltativi.
		\end{itemize}
		\item
		\textbf{CodiceRequisito} assume un valore gerarchico che identifica il singolo requisito.
	\end{itemize}
	
	% TABELLA
	\normalsize
	\begin{longtable}{|>{\centering\arraybackslash}p{2.3cm}|>{\centering\arraybackslash}p{7.5cm} | >{\centering\arraybackslash}p{3.8cm}|}
		\hline \rowcolor{Gray}
		\textbf{Id Test} & \textbf{Descrizione} & \textbf{Stato}\\
		\hline
		\endhead
		\hypertarget{TSFO1}{TSFO1} & Viene verificato che il sistema registri correttamente un utente. & \textit{Non Implementato}\\ \hline
		\hypertarget{TSFO2}{TSFO2} & Viene verificato che il sistema riesca a far effettuare correttamente il login ad un utente attraverso \progetto. & \textit{Non Implementato}\\ \hline
		\hypertarget{TSFO11}{TSFO11} & Viene verificato che il sistema riesca a far effettuare correttamente il logout ad un utente.  & \textit{Non Implementato}\\ \hline
		\hypertarget{TSFO10.1}{TSFO10.1} & Viene verificato che il sistema permetta di gestire correttamente ad un utente le informazioni del proprio profilo. & \textit{Non Implementato}\\ \hline
		\hypertarget{TSFO10.1.2}{TSFO10.1.2} & Viene verificato che il sistema modifichi correttamente le informazioni di un utente. & \textit{Non Implementato}\\ \hline
		\hypertarget{TSFD10.1.2.9}{TSFD10.1.2.9} & Viene verificato che il sistema effettui correttamente l'upgrade dell'account di un utente. & \textit{Non Implementato}\\ \hline
		\hypertarget{TSFO4}{TSFO4} & Viene verificato che il sistema ricerchi correttamente una API. & \textit{Non Implementato}\\ \hline
		\hypertarget{TSFxxx}{TSFxxx} & Viene verificato che il sistema permetta il corretto inserimento di una nuova API da parte di uno sviluppatore. & \textit{Non Implementato}\\ \hline
		\hypertarget{TSFxxx}{TSFxxx} & Viene verificato che il sistema visualizzi correttamente le API inserite da uno sviluppatore. & \textit{Non Implementato}\\ \hline
		\hypertarget{TSFxxx}{TSFxxx} & Viene verificato che il sistema modifichi correttamente i dati di una API di uno sviluppatore. & \textit{Non Implementato}\\ \hline
		\hypertarget{TSFxxx}{TSFxxx} & Viene verificato che il sistema rimuova correttamente una API dal marketplace. & \textit{Non Implementato}\\ \hline
		\hypertarget{TSFxxx}{TSFxxx} & Viene verificato che il sistema visualizzi correttamente i dati relativi ad una API. & \textit{Non Implementato}\\ \hline
		\hypertarget{TSFD3}{TSFD3} & Viene verificato che il sistema permetta il recupero della password per un utente registrato ad \progetto. & \textit{Non Implementato}\\ \hline
		\hypertarget{TSFO4.3}{TSFO4.3} & Viene verificato che il sistema restituisca un insieme di risultati coerenti secondo i parametri di ricerca specificati da un utente. & \textit{Non Implementato}\\ \hline	
		\hypertarget{TSFO5}{TSFO5} & Viene verificato che il sistema visualizzi correttamente i dati relativi ad una API. & \textit{Non Implementato}\\ \hline
		\hypertarget{TSFO5.6}{TSFO5.6} & Viene verificato che il sistema consenta la consultazione della documentazione di una API. & \textit{Non Implementato}\\ \hline
		\hypertarget{TSFO5.7}{TSFO5.7} & Viene verificato che il sistema visualizzi correttamente i dati di utilizzo di una API. & \textit{Non Implementato}\\ \hline
		\hypertarget{TSFO6.2}{TSFO6.2} & Viene verificato che il sistema visualizzi correttamente la lista di API acquistate da un utente. & \textit{Non Implementato}\\ \hline
		\hypertarget{TSFO7}{TSFO7} & Viene verificato che il sistema consenta l'acquisto di una API, con conseguente rilascio di una API Key, per un utente. & \textit{Non Implementato}\\ \hline
		\hypertarget{TSFO7.5.3}{TSFO7.5.3} & Viene verificato che il sistema crei e visualizzi correttamente l'API Key associata ad una API, per cui è stata acquistata una licenza da un utente. & \textit{Non Implementato}\\ \hline
		\hypertarget{TSFO8.2}{TSFO8.2} & Viene verificato che il sistema visualizzi correttamente la lista delle API registrate da uno sviluppatore. & \textit{Non Implementato}\\ \hline
		\hypertarget{TSFO8.2.4}{TSFO8.2.4} & Viene verificato che il sistema modifichi correttamente le informazioni relative ad una API registrata da uno sviluppatore. & \textit{Non Implementato}\\ \hline
		\hypertarget{TSFO8.2.8}{TSFO8.2.8} & Viene verificato che il sistema visualizzi correttamente il guadagno netto, in base alla policy, di una API da lui inserita. & \textit{Non Implementato}\\ \hline
		\hypertarget{TSFO8.2.9}{TSFO8.2.9} & Viene verificato che il sistema elimini correttamente una API su richiesta dello sviluppatore autore. & \textit{Non Implementato}\\ \hline
		\hypertarget{TSFO9}{TSFO9} & Viene verificato che il sistema inserisca correttamente nel marketplace una nuova API. & \textit{Non Implementato}\\ \hline
		\hypertarget{TSFO10.1.1}{TSFO10.1.1} & Viene verificato che il sistema visualizzi correttamente le informazioni del profilo di un utente. & \textit{Non Implementato}\\ \hline
		\hypertarget{TSFO10.2}{TSFO10.2} & Viene verificato che il sistema permetta di gestire correttamente ad un utente il proprio conto. & \textit{Non Implementato}\\ \hline
		\hypertarget{TSFO10.2.1}{TSFO10.2.1} & Viene verificato che il sistema visualizzi correttamente il saldo attuale del conto di un utente. & \textit{Non Implementato}\\ \hline
		\hypertarget{TSFO10.2.2}{TSFO10.2.2} & Viene verificato che il sistema permetta ad un utente di ricaricare il saldo del proprio conto attraverso PayPal. & \textit{Non Implementato}\\ \hline
		\hypertarget{TSFO10.3}{TSFO10.3} & Viene verificato che il sistema visualizzi correttamente lo storico delle transazioni di un utente. & \textit{Non Implementato}\\ \hline
		\hypertarget{TSFO10.3.2}{TSFO10.3.2} & Viene verificato che il sistema visualizzi correttamente la lista delle transazioni concluse di un utente. & \textit{Non Implementato}\\ \hline
		\hypertarget{TSFO12.1.1.1}{TSFO12.1.1.1} & Viene verificato che il sistema visualizzi correttamente i dati avanzati di una API da parte di un amministratore. & \textit{Non Implementato}\\ \hline
		\hypertarget{TSFO12.1.1.2}{TSFO12.1.1.2} & Viene verificato che il sistema sospenda correttamente una API su richiesta di un amministratore. & \textit{Non Implementato}\\ \hline
		\hypertarget{TSFD12.1.1.3}{TSFD12.1.1.3} & Viene verificato che il sistema elimini correttamente una API su richiesta di un amministratore. & \textit{Non Implementato}\\ \hline
		\hypertarget{TSFO12.2.1.1}{TSFO12.2.1.1} & Viene verificato che il sistema permetta ad un amministratore di sospendere correttamente un utente. & \textit{Non Implementato}\\ \hline
		\hypertarget{TSFO12.2.1.2}{TSFO12.2.1.2} & Viene verificato che il sistema permetta ad un amministratore di sospendere correttamente i pagamenti veso un utente. & \textit{Non Implementato}\\ \hline
		\hypertarget{TSFO12.2.1.3}{TSFO12.2.1.3} & Viene verificato che il sistema permetta ad un amministratore di revocare correttamente la sospensione ad un utente sospeso. & \textit{Non Implementato}\\ \hline
		\hypertarget{TSFO12.2.1.5}{TSFO12.2.1.5} & Viene verificato che il sistema permetta ad un amministratore di eliminare correttamente un utente. & \textit{Non Implementato}\\ \hline		
		\hypertarget{TSVO1}{TSFO1} & Viene verificato che il sistema abbia un'architettura a microservizi. & \textit{Non Implementato}\\ \hline
		\hypertarget{TSVO2}{TSFO2} & Viene verificato che il sistema utilizzi il linguaggio \textit{Jolie} per l'API Gateway, il back-end dell'applicazione web e le interfacce dei microservizi. & \textit{Non Implementato}\\ \hline
		\hypertarget{TSVO3}{TSVO3} & Viene verificato che il sistema utilizzi il linguaggio di markup \textit{HTML5}, unito a fogli di stile in \textit{CSS3} e linguaggio di scripting \textit{JavaScript}. & \textit{Non Implementato}\\ \hline
		\hypertarget{TSVO4}{TSVO4} & Viene verificato che il sistema utilizzi il DBMS di tipo relazionale \textit{MySQL}. & \textit{Non Implementato}\\ \hline
		\hypertarget{TSVO5}{TSVO5} & Viene verificato che il sistema funzioni su \textit{Google Chrome} versione 55.0 o superiore & \textit{Non Implementato}\\ \hline
		\hypertarget{TSVO6}{TSVO6} & Viene verificato che il sistema funzioni su \textit{Mozilla Firefox} versione 51.0 o superiore & \textit{Non Implementato}\\ \hline
		\hypertarget{TSVO7}{TSVO7} & Viene verificato che il sistema funzioni su \textit{Safari} versione 10.0 o superiore & \textit{Non Implementato}\\ \hline
		\hypertarget{TSVD8}{TSVD8} & Viene verificato che il sistema funzioni su \textit{Opera} versione 42.0 o superiore & \textit{Non Implementato}\\ \hline
		\hypertarget{TSVD9}{TSVD9} & Viene verificato che il sistema funzioni su \textit{Internet Explorer} versione 11.0 o superiore & \textit{Non Implementato}\\ \hline
		\hypertarget{TSVO10}{TSVO10} & Viene verificato che il sistema funzioni su \textit{Microsoft Edge} versione 38.0 o superiore & \textit{Non Implementato}\\ \hline
		\hypertarget{TSVO11}{TSVO11} & Viene verificato che il sistema funzioni su \textit{Android Browser} versione 5.1 o superiore & \textit{Non Implementato}\\ \hline
		\hypertarget{TSVO12}{TSVO12} & Viene verificato che il sistema funzioni su \textit{Safari} per iOS 10 o versioni superiori & \textit{Non Implementato}\\ \hline
		\hypertarget{TSVF13}{TSVF13} & Viene verificato che il sistema funzioni su \textit{Google Chrome} per iOS versione 56.0 o superiore & \textit{Non Implementato}\\ \hline
		\hypertarget{TSVD14}{TSVD14} & Viene verificato che il sistema funzioni su \textit{Google Chrome} per Android versione 56.0 o superiore & \textit{Non Implementato}\\ \hline
		\caption[Test di Sistema]{Test di Sistema}
		\label{tabella:test1}
	\end{longtable}
	\clearpage
	
	\subsection{Test di validazione}
	Questa tipologia di test serve a verificare che il prodotto soddisfi le richieste del proponente attraverso le funzionalità implementate.\\
	Per questo motivo, occorrerà simulare il comportamento generale dell'applicativo e dell'utente che interagisce con esso, attraverso delle macro azioni.
	I test di validazione saranno organizzati nel modo seguente:
	\begin{center}
		\textbf{TV}[\textit{TipologiaRequisito}][\textit{RilevanzaRequisito}][\textit{CodiceRequisito}]
	\end{center}
	dove:
	\begin{itemize}
		\item
		\textbf{TipologiaRequisito} può assumere valori tra:
		\begin{itemize}
			\item
			\textit{V} per i requisiti di vincolo;
			\item
			\textit{F} per i requisiti di funzionalità;
			\item
			\textit{Q} per i requisiti di qualità;
			\item
			\textit{P} per i requisiti prestazionali.
		\end{itemize}
		\item 
		\textbf{RilevanzaRequisito} può assumere valori tra:
		\begin{itemize}
			\item
			\textit{O} per i requisiti obbligatori;
			\item
			\textit{D} per i requisiti desiderabili;
			\item
			\textit{F} per i requisiti facoltativi.
		\end{itemize}
		\item
		\textbf{CodiceRequisito} assume un valore gerarchico che identifica il singolo requisito.
	\end{itemize}

	% TABELLA
	\normalsize
	\begin{longtable}{|>{\centering\arraybackslash}p{2cm}|>{\centering\arraybackslash}p{7.5cm} | >{\centering\arraybackslash}p{4cm}|}
		\hline \rowcolor{Gray}
		\textbf{Id Test} & \textbf{Descrizione} & \textbf{Stato}\\
		\hline
		\endhead
		\hypertarget{TVFO1}{TVFO1} & L’utente intende registrarsi alla piattaforma \progetto. All’utente è richiesto di:
		\begin{itemize}
			\item Trovarsi nella sezione apposita;
			\item Compilare il form di registrazione;
			\item Premere il pulsante di conferma;
			\item Verificare attraverso l’autenticazione che la registrazione sia avvenuta correttamente.
		\end{itemize}
		& \textit{Non Implementato}\\ \hline
		\hypertarget{TVFO2}{TVFO2} & L’utente intende autenticarsi alla piattaforma \progetto. All’utente è richiesto di:
		\begin{itemize}
			\item Trovarsi nella sezione apposita;
			\item Inserire le credenziali nell’apposito form;
			\item Premere il pulsante di autenticazione;
			\item Verificare che l’autenticazione sia effettivamente avvenuta.
		\end{itemize}
		& \textit{Non Implementato}\\ \hline
		\hypertarget{TVFO4}{TVFO4} & L’utente autenticato  intende ricercare una API sul marketplace. All’utente è richiesto di:
		\begin{itemize}
			\item Trovarsi nella sezione apposita;
			\item Ricercare una API digitando le keywords;
			\item Visualizzazione dei risultati della ricerca, secondo le keywords specificate.
		\end{itemize} & \textit{Non Implementato}\\ \hline
		\hypertarget{TVFO5}{TVFO5} & L’utente autenticato  intende visualizzare le informazioni relative ad una API. All’utente è richiesto di:
		\begin{itemize}
			\item Trovarsi nella sezione apposita;
			\item Selezionare l'API di cui vuole visualizzare le informazioni;
			\item Visualizzazione dei dati relativi alla API selezionata.
		\end{itemize} & \textit{Non Implementato}\\ \hline
		\hypertarget{TVFO5.6}{TVFO5.6} & L’utente autenticato  intende consultare la documentazione relativa ad una API. All’utente è richiesto di:
		\begin{itemize}
			\item Trovarsi nella sezione apposita;
			\item Selezionare l'API di interesse;
			\item Visualizzazione della documentazione dell'API.
		\end{itemize} & \textit{Non Implementato}\\ \hline
		\hypertarget{TVFO7}{TVFO7} & L’utente autenticato intende acquistare l'API che sta visualizzando. All’utente è richiesto di:
		\begin{itemize}
			\item Essere autenticato;
			\item Trovarsi nella sezione apposita;
			\item Selezionare la policy di vendita;
			\item Confermare l'acquisto attraverso l'apposito pulsante;
			\item Verificare la transazione nel proprio storico.
		\end{itemize} & \textit{Non Implementato}\\ \hline
		\hypertarget{TVFO8.2}{TVFO8.2} & L’utente sviluppatore intende visualizzare l'elenco di API da lui caricate sulla piattaforma \progetto. All’utente è richiesto di:
		\begin{itemize}
			\item Essere autenticato;
			\item Trovarsi nella sezione apposita;
			\item Verificare che vengano visualizzati tutte le proprie API inserite.
		\end{itemize} & \textit{Non Implementato}\\ \hline
		\hypertarget{TVFO8.2.3}{TVFO8.2.3} & L’utente sviluppatore intende visualizzare il numero di licenze attive per le API da lui caricate sulla piattaforma \progetto. All’utente è richiesto di:
		\begin{itemize}
			\item Essere autenticato;
			\item Trovarsi nella sezione apposita;
			\item Selezionare una API dall'elenco visualizzato;
			\item Verificare che vengao il numero di licenze attive per la propria API registrata.
		\end{itemize} & \textit{Non Implementato}\\ \hline
		\hypertarget{TVFO8.2.4}{TVFO8.2.4} & L’utente sviluppatore intende modificare i dati relativi ad una sua API caricata. All’utente è richiesto di:
		\begin{itemize}
			\item Essere autenticato;
			\item Trovarsi nella sezione apposita;
			\item Premere il pulsante "Modifica API";
			\item Modificare i dati dell'API selezionata;
			\item Premere il pulsante di conferma modifica;
			\item Verificare che sia stata modificata l'API.
		\end{itemize} & \textit{Non Implementato}\\ \hline
		\hypertarget{TVFO9}{TVFO9} & L’utente sviluppatore intende caricare una propria API sulla piattaforma \progetto. All’utente è richiesto di:
		\begin{itemize}
			\item Essere autenticato;
			\item Trovarsi nella sezione apposita;
			\item Premere il pulsante "Inserisci API";
			\item Inserire i dati necessari alla pubblicazione della propria API;
			\item Premere il pulsante di conferma inserimento nuova API;
			\item Verificare che sia stata aggiunta al marketplace l'API.
		\end{itemize} & \textit{Non Implementato}\\ \hline
		\hypertarget{TVFO10.1.1}{TVFO10.1.1} & L’utente autenticato intende visualizzare il proprio profilo. All’utente è richiesto di:
		\begin{itemize}
			\item Essere autenticato;
			\item Trovarsi nella sezione apposita;
			\item Visualizzare il proprio profilo.
		\end{itemize} & \textit{Non Implementato}\\ \hline
		\hypertarget{TVFO10.1.2}{TVFO10.1.2} & L’utente autenticato intende modificare i propri dati. All’utente è richiesto di:
		\begin{itemize}
			\item Essere autenticato;
			\item Trovarsi nella sezione apposita;
			\item Modificare i campi dati consentiti;
			\item Premere il tasto "Conferma Modifiche";
			\item Visualizzare il profilo dell’utente modificato.
		\end{itemize}
		& \textit{Non Implementato}\\ \hline
		\hypertarget{TVFD10.1.2.9}{TVFD10.1.2.9} & L’utente autenticato  intende modificare la tipologia di utenza, effettuando un upgrade dell'account. All’utente è richiesto di:
		\begin{itemize}
			\item Essere autenticato;
			\item Trovarsi nella sezione apposita;
			\item Effettuare l'upgrade dell'account, passando da "cliente" a "sviluppatore";
			\item Verificare la modifica effettuata.
		\end{itemize} & \textit{Non Implementato}\\ \hline
		\hypertarget{TVFO10.2.1}{TVFO10.2.1} & L’utente intende visualizzare il saldo del proprio conto associato all'account. All’utente è richiesto di:
		\begin{itemize}
			\item Essere autenticato;
			\item Trovarsi nella sezione apposita;
			\item Premere il pulsante "Conto Virtuale";
			\item Visualizzazione del saldo attuale del proprio conto.
		\end{itemize} & \textit{Non Implementato}\\ \hline
		\hypertarget{TVFO10.3}{TVFO10.3} & L’utente autenticato intende visualizzare lo storico delle transazioni effettuate. All’utente è richiesto di:
		\begin{itemize}
			\item Essere autenticato;
			\item Trovarsi nella sezione apposita;
			\item Visualizzare lo storico delle transazioni.
		\end{itemize} & \textit{Non Implementato}\\ \hline		
		\hypertarget{TVFO11}{TVFO11} & L’utente intende disconnettersi dalla piattaforma \progetto. All’utente è richiesto di:
		\begin{itemize}
			\item Essere autenticato;
			\item Trovarsi nella sezione apposita;
			\item Premere il pulsante di logout;
			\item Verificare che la disconnessione sia effettivamente avvenuta.
		\end{itemize}
		& \textit{Non Implementato}\\ \hline
		\hypertarget{TVFO12.2.1}{TVFO12.2.1} & L’utente amministratore della piattaforma \progetto intende moderare un determinato utente registrato. All’utente è richiesto di:
		\begin{itemize}
			\item Essere autenticato;
			\item Trovarsi nella sezione apposita;
			\item Selezionare l'utente che si vuole moderare;
			\item Inserire i dati del rapporto di moderazione;
			\item Verificare che l'utente sia stato effettivamente moderato.
		\end{itemize} & \textit{Non Implementato}\\ \hline
		\caption[Test di Validazione]{Test di Validazione}
		\label{tabella:test0}
	\end{longtable}
	\clearpage
	