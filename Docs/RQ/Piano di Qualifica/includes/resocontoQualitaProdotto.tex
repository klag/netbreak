\newpage
\section{Resoconto Qualità di Prodotto}

In questa sezione del documento vengono descritti e analizzati gli esiti delle metriche previste per il controllo della qualità di prodotto, in base alla revisione di progetto.

	\subsection{Revisione di Qualifica}
	
		\begin{longtable}{|>{\centering\arraybackslash}p{2cm}|>{\centering\arraybackslash}p{5cm}|>{\centering\arraybackslash}p{3cm}|>{\centering\arraybackslash}p{3cm}|}
			\hline
			\rowcolor{Gray}
			\textbf{Id} & \textbf{Metrica} & \textbf{Valore} & \textbf{Esito} \\
			\hline
				2.1.1.2.1 & \textit{Completezza delle funzioni sviluppate} & xx & Superato\\
				\hline
				2.1.1.2.2 & \textit{Correttezza delle funzioni sviluppate} & xx & Superato\\
				\hline
				2.1.1.2.3 & \textit{Accuratezza rispetto alle aspettative} & xx & Superato\\
				\hline
				2.1.1.2.4 & \textit{Controllo degli accessi} & xx & Superato\\
				\hline
				2.1.2.2.1 & \textit{Chiamate a microservizi corrette} & xx & Superato\\
				\hline
				2.1.2.2.2 & \textit{Copertura dei test} & xx & Superato\\
				\hline
				2.1.2.2.3 & \textit{Controllo dei guasti} & xx & Superato\\
				\hline
				2.1.3.2.1 & \textit{Comprensibilità delle funzionalità offerte} & xx & Superato\\
				\hline
				2.1.3.2.2 & \textit{Controllo e monitoraggio delle operazioni} & xx & Superato\\
				\hline
				2.1.3.2.3 & \textit{Qualità della messaggistica} & xx & Superato\\
				\hline
				2.1.4.2.1 & \textit{Tempo di risposta} & xx & Superato\\
				\hline
				2.1.5.2.1 & \textit{Impatto delle modifiche} & xx & Superato\\
				\hline
				2.1.6.2.1 & \textit{Supporto differenti versioni dei browser} & xx & Superato\\
				\hline
			
			\caption{Resoconto esiti metriche - Qualità di prodotto}
		\end{longtable}