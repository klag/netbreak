\newpage
\section{Requisiti}
In questa sezione verranno presentati i requisiti individuati dal \textit{team\ped{G}} durante l'analisi del capitolato e dei casi d'uso, discussi con il proponente durante le riunioni esterne e decisi dai componenti nelle riunioni interne. Per facilitare la consultazione, i
requisiti saranno separati su più tabelle in base alla loro categoria. I requisiti saranno classificati per tipo e importanza e utilizzeranno la seguente sintassi:
\begin{center}
	R[Importanza][Tipo][Codice]
\end{center}
dove:
\begin{itemize}
	\item \textbf{Importanza}: può assumere uno tra i seguenti valori:
	\begin{itemize}
		\item O: requisito obbligatorio;
		\item D: requisito desiderabile;
		\item F: requisito facoltativo.
	\end{itemize}
	\item \textbf{Tipo}: può assumere uno tra i seguenti valori:
	\begin{itemize}
		\item F: funzionale;
		\item P: prestazionale;
		\item Q: qualità;
		\item V: vincolo.
	\end{itemize}
	\item \textbf{Codice}: è il codice gerarchico univoco di ogni requisito espresso in numeri.
\end{itemize} 
Per ogni requisito inoltre verranno riportate:
\begin{itemize}
	\item \textbf{Descrizione}: breve testo ma completo che andrà a descrivere il requisito in esame;
	\item \textbf{Fonte}: che potrà essere una tra le seguenti:
	\begin{itemize}
		\item Capitolato: requisito dedotto direttamente dall'analisi del capitolato C5;
		\item Verbale interno: requisito derivante da uno dei seguenti verbali interni:
		\begin{itemize}
		\item \textit{Verbale\_1\_Interno\_2015-12-03};
		\item \textit{Verbale\_2\_Interno\_2015-12-11};
		\item \textit{Verbale\_3\_Interno\_2015-12-29};
		\item \textit{Verbale\_4\_Interno\_2016-01-12}.
		\end{itemize}
		\item Verbale 2016-01-11: requisito esterno derivato dal verbale \textit{Verbale\_1\_Esterno\_2016-01-11};
		\item Interno: requisito identificato dagli \textit{\Anas};
		\item Caso d’uso: si tratta di un requisito emerso da un caso d’uso; viene riportato l’identificativo del caso d’uso associato.
	\end{itemize} 
\end{itemize}
%\input{sezioni/TabelleRequisiti/requisitiAccettati.tex}
\newpage

\section{Requisiti}

\subsection{Catalogazione dei requisiti}

Di seguito sono elencati tutti i requisiti rilevati e descritti nel documento. Essi saranno catalogati secondo la nomenclatura indicata nel documento \textsc{NormeDiProgetto 1\_0\_0.pdf} di cui si cita la sezione di seguito:

\begin{center}
	R[Tipologia][Rilevanza][Codice]
\end{center}
\textbf{Tipologia:} può assumere uno dei seguenti valori:
\begin{itemize}
	\item \textbf{V:} requisito di vincolo;
	\item \textbf{F:} requisito di funzionalità;
	\item \textbf{Q:} requisito di qualità;
	\item \textbf{P:} requisito prestazionale.
\end{itemize}
\textbf{Rilevanza:} può assumere uno dei seguenti valori, elencati in ordine di importanza:
\begin{itemize}
	\item \textbf{O:} requisito obbligatorio;
	\item \textbf{D:} requisito desiderabile;
	\item \textbf{F:} requisito facoltativo.
\end{itemize}
\textbf{Codice:} assume un numero sequenziale e univoco, necessario a catalogare e riconoscere ogni requisito progettuale.

\subsection{Requisiti di funzionalità}

\begin{longtable}{|c|p{8cm}|c|}
\caption{Tabella dei requisiti di funzionalità} \\

\hline
\thead*{\textbf{Codice requisito}} &\thead{\textbf{Descrizione}}  &\thead{\textbf{Fonte}} \\
\hline
\endhead

\hline
\endfoot
\hline
\endlastfoot

\hypertarget{RFO1}{RFO1} & L'utente non autenticato può registrarsi & \makecell*{Capitolato\\UC3} \\
\hline

\hypertarget{RFD1.1}{RFD1.1} & L'utente non autenticato può inserire il proprio nome & \makecell*{Interno\\UC3\\UC3.1} \\
\hline
\hypertarget{RFD1.2}{RFD1.2} & L'utente non autenticato può inserire il proprio cognome & \makecell*{Interno\\UC3\\UC3.2} \\
\hline
\hypertarget{RFO1.3}{RFO1.3} & L'utente non autenticato può inserire il proprio username & \makecell*{Interno\\UC3\\UC3.3} \\
\hline
\hypertarget{RFO1.4}{RFO1.4} & L'utente non autenticato può inserire la propria email & \makecell*{Interno\\UC3\\UC3.4} \\
\hline
\hypertarget{RFO1.5}{RFO1.5} & L'utente non autenticato può inserire la propria password & \makecell*{Interno\\UC3\\UC3.5} \\
\hline
\hypertarget{RFO1.6}{RFO1.6} & L'utente non autenticato può confermare l'inserimento della propria password & \makecell*{Interno\\UC3\\UC3.6} \\
\hline
\hypertarget{RFO1.7}{RFO1.7} & L'utente non autenticato può inserire il proprio avatar & \makecell*{Interno\\UC3\\UC3.7} \\
\hline
\hypertarget{RFO1.8}{RFO1.8} &  L'applicazione web mostra un messaggio di errore nel caso l'inserimento dell'avatar sia fallito & \makecell*{Interno\\UC3\\UC3.8} \\
\hline
\hypertarget{RFO1.9}{RFO1.9} & L'utente non autenticato può confermare i dati inseriti, confermando la propria registrazione & \makecell*{Interno\\UC3\\UC3.9} \\
\hline
\hypertarget{RFF1.10}{RFF1.10} & L'applicazione web mostra un messaggio di errore nel caso la registrazione sia fallita & \makecell*{Interno\\UC3\\UC3.10} \\
\hline

\hypertarget{RFO2}{RFO2} & L'utente non autenticato può effettuare il login & \makecell*{Capitolato\\UC4} \\
\hline

\hypertarget{RFO2.1}{RFO2.1} & L'utente non autenticato può effettuare il login tramite APIMarket & \makecell*{Capitolato\\UC4\\UC4.1} \\
\hline

\hypertarget{RFF2.1.1}{RFF2.1.1} & L'utente non autenticato può inserire uno username o una email & \makecell*{Interno\\UC4\\UC4.1\\UC4.1.1} \\
\hline
\hypertarget{RFF2.1.2}{RFF2.1.2} & L'utente non autenticato può inserire una password & \makecell*{Interno\\UC4\\UC4.1\\UC4.1.2} \\
\hline
\hypertarget{RFF2.1.3}{RFF2.1.3} & L'utente non autenticato può confermare i dati inseriti, effettuando il login & \makecell*{Interno\\UC4\\UC4.1\\UC4.1.3} \\
\hline
\hypertarget{RFO2.1.4}{RFO2.1.4} & L'applicazione web mostra un messaggio di errore nel caso il login tramite API Market sia fallito & \makecell*{Interno\\UC4\\UC4.1\\UC4.1.4} \\
\hline

\hypertarget{RFF2.2}{RFF2.2} & L'utente non autenticato può effettuare il login tramite Facebook & \makecell*{Interno\\UC4\\UC4.2\\UC4.2.1} \\
\hline
\hypertarget{RFF2.2.2}{RFF2.2.2} & L'applicazione web mostra un messaggio di errore nel caso il login tramite Facebook sia fallito & \makecell*{Interno\\UC4\\UC4.2\\UC4.2.2} \\
\hline

\hypertarget{RFF2.3}{RFF2.3} & L'utente non autenticato può effettuare il login tramite Twitter & \makecell*{Interno\\UC4\\UC4.3\\UC4.3.1} \\
\hline
\hypertarget{RFF2.3.2}{RFF2.3.2} & L'applicazione web mostra un messaggio di errore nel caso il login tramite Twitter sia fallito & \makecell*{Interno\\UC4\\UC4.3\\UC4.3.2} \\
\hline

\hypertarget{RFF2.4}{RFF2.4} & L'utente non autenticato può effettuare il login tramite LinkedIn & \makecell*{Interno\\UC4\\UC4.4\\UC4.4.1} \\
\hline
\hypertarget{RFF2.4.2}{RFF2.4.2} & L'applicazione web mostra un messaggio di errore nel caso il login tramite LinkedIn sia fallito & \makecell*{Interno\\UC4\\UC4.4\\UC4.4.2} \\
\hline

\hypertarget{RFF2.5}{RFF2.5} & L'utente non autenticato può effettuare il login tramite Google+ & \makecell*{Interno\\UC4\\UC4.5\\YC4.5.1} \\
\hline
\hypertarget{RFF2.5.2}{RFF2.5.2} & L'applicazione web mostra un messaggio di errore nel caso il login tramite Google+ sia fallito & \makecell*{Interno\\UC4\\UC4.5\\UC4.5.2} \\
\hline

\hypertarget{RFD3}{RFD3} & L'utente non autenticato può recuperare la propria password & \makecell*{Interno\\UC5} \\
\hline

\hypertarget{RFD3.1}{RFD3.1} & L'utente non autenticato può inserire la propria email & \makecell*{Interno\\UC5\\UC5.1} \\
\hline
\hypertarget{RFD3.2}{RFD3.2} & L'utente non autenticato può confermare l'email inserita, permettendo all'applicazione web di inviare un'email, con un link per il reset, a quell'indirizzo & \makecell*{Interno\\UC5\\UC5.2} \\
\hline
\hypertarget{RFD3.3}{RFD3.3} & L'applicazione web mostra un messaggio di errore nel caso l'email per il recupero password non sia valida & \makecell*{Interno\\UC5\\UC5.3} \\
\hline

\hline
\hypertarget{RFD3.4}{RFD3.4} & L'utente non autenticato puo' scegliere una nuova password per il proprio account & \makecell*{Interno\\UC5\\UC5.4} \\
\hline

\hypertarget{RFO4}{RFO4} & L'utente generico può eseguire la ricerca sulle API sulla base dell'inserimento di alcune keywords & \makecell*{Capitolato\\UC6} \\
\hline

\hypertarget{RFO4.1}{RFO4.1} & L'utente generico può inserire le keywords di ricerca & \makecell*{Capitolato\\UC6\\UC6.1} \\
\hline
\hypertarget{RFO4.2}{RFO4.2} & L'utente generico può confermare la ricerca  & \makecell*{Capitolato\\UC6\\UC6.2} \\
\hline

\hypertarget{RFD4.3}{RFD4.3} & L'utente generico può visualizzare i risultati ottenuti dall'applicazione web, dopo aver confermato i parametri di ricerca scelti & \makecell*{Capitolato\\UC6\\UC6.3} \\
\hline
\hypertarget{RFO4.3.1}{RFO4.3.1} & L'utente generico può visualizzare il nome dell'API & \makecell*{Capitolato\\UC6\\UC6.3\\UC6.3.1} \\
\hline
\hypertarget{RFO4.3.2}{RFO4.3.2} & L'utente generico può visualizzare l'autore dell'API & \makecell*{Capitolato\\UC6\\UC6.3\\UC6.3.2} \\
\hline

\hypertarget{RFD4.3.3}{RFD4.3.3} & L'utente generico può visualizzare la categoria dell'API & \makecell*{Capitolato\\UC6\\UC6.3\\UC6.3.3} \\
\hline
\hypertarget{RFD4.3.4}{RFD4.3.4} & L'utente generico può visualizzare il logo dell'API & \makecell*{Capitolato\\UC6\\UC6.3\\UC6.3.4} \\
\hline
\hypertarget{RFO4.3.5}{RFO4.3.5} & L'utente generico può visualizzare il link alla pagina di visualizzazione dell'API & \makecell*{Capitolato\\UC6\\UC6.3\\UC6.3.5} \\
\hline

\hypertarget{RFO5}{RFO5} & L'utente generico può visualizzare la pagina dei dati relativi all'API selezionata & \makecell*{Capitolato\\UC7} \\
\hline

\hypertarget{RFO5.1}{RFO5.1} & L'utente generico può visualizzare il nome dell'API & \makecell*{Capitolato\\UC7\\UC7.1} \\
\hline
\hypertarget{RFO5.2}{RFO5.2} & L'utente generico può visualizzare la descrizione dell'API & \makecell*{Capitolato\\UC7\\UC7.2} \\
\hline
\hypertarget{RFO5.3}{RFO5.3} & L'utente generico può visualizzare l'autore dell'API & \makecell*{Capitolato\\UC7\\UC7.3} \\
\hline
\hypertarget{RFD5.4}{RFD5.4} & L'utente generico può visualizzare la categoria dell'API & \makecell*{Capitolato\\UC7\\UC7.4} \\
\hline

\hypertarget{RFO5.5}{RFO5.5} & L'utente generico può visualizzare l'interfaccia dell'API & \makecell*{Capitolato\\UC7\\UC7.5} \\
\hline
\hypertarget{RFO5.5.1}{RFO5.5.1} & L'utente generico può visualizzare l'interfaccia testuale dell'API & \makecell*{Capitolato\\UC7\\UC7.5\\UC7.5.1} \\
\hline
\hypertarget{RFO5.5.2}{RFO5.5.2} & L'utente generico può visualizzare il link di download dell'interfaccia dell'API & \makecell*{Capitolato\\UC7\\UC7.5\\UC7.5.2} \\
\hline

\hypertarget{RFO5.6}{RFO5.6} & L'utente generico può consultare la documentazione dell'API & \makecell*{Capitolato\\UC7\\UC7.6} \\
\hline
\hypertarget{RFO5.6.1}{RFO5.6.1} & L'utente generico può consultare la versione PDF della documentazione dell'API & \makecell*{Capitolato\\UC7\\UC7.6\\UC7.6.1} \\
\hline
\hypertarget{RFD5.6.2}{RFD5.6.2} & L'utente generico può consultare la versione esterna della documentazione dell'API, seguendo un link dell'autore dell'API  & \makecell*{Capitolato\\UC7\\UC7.6\\UC7.6.2} \\
\hline

\hypertarget{RFO5.7}{RFO5.7} & L'utente generico può visualizzare i dati di utilizzo dell'API & \makecell*{Capitolato\\UC7\\UC7.7} \\
\hline
\hypertarget{RFD5.7.1}{RFD5.7.1} & L'utente generico può visualizzare il numero di licenze attive dell'API & \makecell*{Capitolato\\UC7\\UC7.7\\UC7.7.1} \\
\hline
\hypertarget{RFO5.7.2}{RFO5.7.2} & L'utente generico può visualizzare i dati della SLA dell'API & \makecell*{Capitolato\\UC7\\UC7.7\\UC7.7.2} \\
\hline

\hypertarget{RFO5.8}{RFO5.8} & L'utente generico può visualizzare il prezzo dell'API & \makecell*{Capitolato\\UC7\\UC7.8} \\
\hline
\hypertarget{RFO5.9}{RFO5.9} & L'utente generico può visualizzare la data dell'ultima modifica all'API & \makecell*{Capitolato\\UC7\\UC7.9} \\
\hline
\hypertarget{RFO5.10}{RFO5.10} & L'utente generico può visualizzare la versione dell'API & \makecell*{Capitolato\\UC7\\UC7.10} \\
\hline
\hypertarget{RFD5.11}{RFD5.11} & L'utente generico può visualizzare il logo dell'API & \makecell*{Capitolato\\UC7\\UC7.11} \\
\hline
\hypertarget{RFO5.12}{RFO5.12} & L'utente generico può visualizzare la policy di vendita dell'API & \makecell*{Capitolato\\UC7\\UC7.12} \\
\hline
\hypertarget{RFO5.13}{RFO5.13} & L'utente generico può visualizzare il link d'acquisto dell'API & \makecell*{Capitolato\\UC7\\UC7.13} \\
\hline

\hypertarget{RFO6}{RFO6} & Il cliente può visualizzare le API acquistate & \makecell*{Capitolato\\UC8} \\
\hline

\hypertarget{RFO6.1}{RFO6.1} & Il cliente può visualizzare il numero di API acquistate & \makecell*{Capitolato\\UC8\\UC8.1} \\
\hline

\hypertarget{RFO6.2}{RFO6.2} & Il cliente può visualizzare la lista delle API acquistate & \makecell*{Capitolato\\UC8\\UC8.2} \\
\hline

\hypertarget{RFO6.2.1}{RFO6.2.1} & Il cliente può visualizzare il nome dell'API & \makecell*{Capitolato\\UC8\\UC8.2\\UC8.2.1} \\
\hline
\hypertarget{RFO6.2.2}{RFO6.2.2} & Il cliente può visualizzare il link alla pagina di visualizzazione dell'API & \makecell*{Capitolato\\UC8\\UC8.2\\UC8.2.2} \\
\hline
\hypertarget{RFO6.2.3}{RFO6.2.3} & Il cliente può visualizzare l'autore dell'API & \makecell*{Capitolato\\UC8\\UC8.2\\UC8.2.3} \\
\hline
\hypertarget{RFO6.2.4}{RFO6.2.4} & Il cliente può visualizzare la policy di vendita dell'API & \makecell*{Capitolato\\UC8\\UC8.2\\UC8.2.4} \\
\hline
\hypertarget{RFO6.2.5}{RFO6.2.5} & Il cliente può visualizzare la scadenza della licenza dell'API & \makecell*{Capitolato\\UC8\\UC8.2\\UC8.2.5} \\
\hline
\hypertarget{RFD6.2.6}{RFD6.2.6} & Il cliente può visualizzare gli avvisi riguardanti l'API & \makecell*{Capitolato\\UC8\\UC8.2\\UC8.2.6} \\
\hline
\hypertarget{RFD6.2.7}{RFD6.2.7} & Il cliente può visualizzare il logo dell'API & \makecell*{Capitolato\\UC8\\UC8.2\\UC8.2.7} \\
\hline

\hypertarget{RFO7}{RFO7} & Il cliente può acquistare l'API & \makecell*{Capitolato\\UC9} \\
\hline

\hypertarget{RFO7.1}{RFO7.1} & Il cliente può visualizzare i dati d'acquisto dell'API & \makecell*{Capitolato\\UC9\\UC9.1} \\
\hline

\hypertarget{RFO7.1.1}{RFO7.1.1} & Il cliente può visualizzare il nome dell'API & \makecell*{Capitolato\\UC9\\UC9.1\\UC9.1.1} \\
\hline
\hypertarget{RFO7.1.2}{RFO7.1.2} & Il cliente può visualizzare l'autore dell'API & \makecell*{Capitolato\\UC9\\UC9.1\\UC9.1.2} \\
\hline
\hypertarget{RFO7.1.3}{RFO7.1.3} & Il cliente può visualizzare la policy di vendita dell'API & \makecell*{Capitolato\\UC9\\UC9.1\\UC9.1.3} \\
\hline

\hypertarget{RFO7.2}{RFO7.2} & Il cliente può scegliere il blocco d'acquisto dell'API & \makecell*{Capitolato\\UC9\\UC9.2} \\
\hline
\hypertarget{RFO7.3}{RFO7.3} & Il cliente può visualizzare una previsione del saldo finale & \makecell*{Capitolato\\UC9\\UC9.3} \\
\hline

\hypertarget{RFO7.4}{RFO7.4} & Il cliente può confermare l'acquisto dell'API & \makecell*{Capitolato\\UC9\\UC9.4} \\
\hline

\hypertarget{RFO7.5}{RFO7.5} & Il cliente può visualizzare il riepilogo dell'acquisto effettuato & \makecell*{Capitolato\\UC9\\UC9.5} \\
\hline

\hypertarget{RFD7.5.1}{RFD7.5.1} & Il cliente può visualizzare il ringraziamento per l'acquisto dell'API & \makecell*{Capitolato\\UC9\\UC9.5.1} \\
\hline
\hypertarget{RFO7.5.2}{RFO7.5.2} & Il cliente può visualizzare il nome dell'API acquistata & \makecell*{Capitolato\\UC9\\UC9.5.2} \\
\hline
\hypertarget{RFO7.5.3}{RFO7.5.3} & Il cliente può visualizzare la chiave API dell'API acquistata & \makecell*{Capitolato\\UC9\\UC9.5.3} \\
\hline

\hypertarget{RFO7.6}{RFO7.6} & L'applicazione web mostra un messaggio di errore nel caso l'acquisto dell'API sia fallito & \makecell*{Capitolato\\UC9\\UC9.6} \\
\hline

\hypertarget{RFO8}{RFO8} & Lo sviluppatore può visualizzare le API registrate & \makecell*{Capitolato\\UC10} \\
\hline

\hypertarget{RFO8.1}{RFO8.1} & Lo sviluppatore può visualizzare il numero di API registrate & \makecell*{Capitolato\\UC10\\UC10.1} \\
\hline

\hypertarget{RFO8.2}{RFO8.2} & Lo sviluppatore può visualizzare la lista delle API registrate & \makecell*{Capitolato\\UC10\\UC10.2} \\
\hline
\hypertarget{RFO8.2.1}{RFO8.2.1} & Lo sviluppatore può visualizzare il nome dell'API & \makecell*{Capitolato\\UC10\\UC10.2\\UC10.2.1} \\
\hline
\hypertarget{RFO8.2.2}{RFO8.2.2} & Lo sviluppatore può visualizzare il link alla pagina di visualizzazione dell'API & \makecell*{Capitolato\\UC10\\UC10.2\\UC10.2.2} \\
\hline
\hypertarget{RFO8.2.3}{RFO8.2.3} & Lo sviluppatore può visualizzare il numero di licenze attive dell'API & \makecell*{Capitolato\\UC10\\UC10.2\\UC10.2.3} \\
\hline

\hypertarget{RFO8.2.4}{RFO8.2.4} & Lo sviluppatore può modificare l'API registrata & \makecell*{Capitolato\\UC10\\UC10.2\\UC10.2.4} \\
\hline

\hypertarget{RFO8.2.4.1}{RFO8.2.4.1} & Lo sviluppatore può modificare il nome dell'API & \makecell*{Capitolato\\UC10\\UC10.2\\UC10.2.4\\UC10.2.4.1} \\
\hline
\hypertarget{RFO8.2.4.2}{RFO8.2.4.2} & Lo sviluppatore può modificare la descrizione dell'API & \makecell*{Capitolato\\UC10\\UC10.2\\UC10.2.4\\UC10.2.4.2} \\
\hline
\hypertarget{RFD8.2.4.3}{RFD8.2.4.3} & Lo sviluppatore può modificare la categoria dell'API & \makecell*{Capitolato\\UC10\\UC10.2\\UC10.2.4\\UC10.2.4.3} \\
\hline
\hypertarget{RFO8.2.4.4}{RFO8.2.4.4} & Lo sviluppatore può modificare l'interfaccia dell'API & \makecell*{Capitolato\\UC10\\UC10.2\\UC10.2.4\\UC10.2.4.4} \\
\hline
\hypertarget{RFO8.2.4.5}{RFO8.2.4.5} & Lo sviluppatore può modificare la documentazione PDF dell'API & \makecell*{Capitolato\\UC10\\UC10.2\\UC10.2.4\\UC10.2.4.5} \\
\hline
\hypertarget{RFD8.2.4.6}{RFD8.2.4.6} & Lo sviluppatore può modificare il link alla documentazione esterna dell'API & \makecell*{Capitolato\\UC10\\UC10.2\\UC10.2.4\\UC10.2.4.6} \\
\hline
\hypertarget{RFO8.2.4.7}{RFO8.2.4.7} & Lo sviluppatore può modificare il logo dell'API & \makecell*{Capitolato\\UC10\\UC10.2\\UC10.2.4\\UC10.2.4.7} \\
\hline
\hypertarget{RFO8.2.4.8}{RFO8.2.4.8} & Lo sviluppatore può confermare le modifiche all'API & \makecell*{Capitolato\\UC10\\UC10.2\\UC10.2.4\\UC10.2.4.8} \\
\hline

\hypertarget{RFO8.2.4.9}{RFO8.2.4.9} & L'applicazione web mostra un errore nel caso la modifica dell'API sia fallita & \makecell*{Capitolato\\UC10\\UC10.2\\UC10.2.4\\UC10.2.4.9} \\
\hline
\hypertarget{RFO8.2.4.10}{RFO8.2.4.10} & L'applicazione web mostra un errore nel caso la modifica alla documentazione PDF dell'API sia fallita & \makecell*{Capitolato\\UC10\\UC10.2\\UC10.2.4\\UC10.2.4.10} \\
\hline
\hypertarget{RFO8.2.4.11}{RFO8.2.4.11} & L'applicazione web mostra un errore nel caso la modifica al logo dell'API sia fallita & \makecell*{Capitolato\\UC10\\UC10.2\\UC10.2.4\\UC10.2.4.11} \\
\hline

\hypertarget{RFD8.2.5}{RFD8.2.5} & Lo sviluppatore può visualizzare gli avvisi riguardanti l'API registrata & \makecell*{Capitolato\\UC10\\UC10.2\\UC10.2.5} \\
\hline
\hypertarget{RFD8.2.6}{RFD8.2.6} & Lo sviluppatore può aggiornare gli avvisi riguardanti l'API registrata & \makecell*{Capitolato\\UC10\\UC10.2\\UC10.2.6} \\
\hline
\hypertarget{RFD8.2.7}{RFD8.2.7} & Lo sviluppatore può visualizzare il logo dell'API registrata & \makecell*{Capitolato\\UC10\\UC10.2\\UC10.2.7} \\
\hline
\hypertarget{RFO8.2.8}{RFO8.2.8} & Lo sviluppatore può visualizzare il guadagno netto, in base alla policy, dell'API registrata & \makecell*{Capitolato\\UC10\\UC10.2\\UC10.2.8} \\
\hline

\hypertarget{RFO8.2.9}{RFO8.2.9} & Lo sviluppatore può cancellare l'API registrata & \makecell*{Capitolato\\UC10\\UC10.2\\UC10.2.9} \\
\hline

\hypertarget{RFO8.2.9.1}{RFO8.2.9.1} & Lo sviluppatore può confermare la cancellazione dell'API registrata & \makecell*{Capitolato\\UC10\\UC10.2\\UC10.2.9.1} \\
\hline
\hypertarget{RFO8.2.9.2}{RFO8.2.9.2} & L'applicazione web mostra un errore nel caso la cancellazione dell'API sia fallita & \makecell*{Capitolato\\UC10\\UC10.2\\UC10.2.9.2} \\
\hline

\hypertarget{RFO9}{RFO9} & Lo sviluppatore può registrare una nuova API & \makecell*{Capitolato\\UC11} \\
\hline

\hypertarget{RFO9.1}{RFO9.1} & Lo sviluppatore può inserire il nome dell'API & \makecell*{Capitolato\\UC11\\UC11.1} \\
\hline
\hypertarget{RFO9,2}{RFO9.2} & Lo sviluppatore può inserire la descrizione dell'API & \makecell*{Capitolato\\UC11\\UC11.2} \\
\hline
\hypertarget{RFD9.3}{RFD9.3} & Lo sviluppatore può inserire la categoria dell'API & \makecell*{Capitolato\\UC11\\UC11.3} \\
\hline
\hypertarget{RFO9.4}{RFO9.4} & Lo sviluppatore può inserire l'interfaccia dell'API & \makecell*{Capitolato\\UC11\\UC11.4} \\
\hline
\hypertarget{RFO9.5}{RFO9.5} & Lo sviluppatore può inserire la documentazione PDF dell'API & \makecell*{Capitolato\\UC11\\UC11.5} \\
\hline
\hypertarget{RFD9.6}{RFD9.6} & L'applicazione web mostra un errore nel caso l'inserimento del PDF fallisca & \makecell*{Capitolato\\UC11\\UC11.6} \\
\hline
\hypertarget{RFD9.7}{RFD9.7} & Lo sviluppatore può inserire il link alla documentazione esterna dell'API & \makecell*{Capitolato\\UC11\\UC11.7} \\
\hline

\hypertarget{RFO9.8}{RFO9.8} & Lo sviluppatore può scegliere la policy di vendita dell'API & \makecell*{Capitolato\\UC11\\UC11.8} \\
\hline

\hypertarget{RFO9.8.1}{RFO9.8.1} & Lo sviluppatore può scegliere la policy di vendita per numero di chiamate & \makecell*{Capitolato\\UC11\\UC11.8.1} \\
\hline
\hypertarget{RFO9.8.2}{RFO9.8.2} & Lo sviluppatore può scegliere la policy di vendita per tempo di utilizzo & \makecell*{Capitolato\\UC11\\UC11.8.2} \\
\hline
\hypertarget{RFO9.8.3}{RFO9.8.3} & Lo sviluppatore può scegliere la policy di vendita per traffico di dati & \makecell*{Capitolato\\UC11\\UC11.8.3} \\
\hline

\hypertarget{RFO9.9}{RFO9.9} & Lo sviluppatore può il guadagno netto, in base alla policy di vendita, desiderato dell'API & \makecell*{Capitolato\\UC11\\UC11.9} \\
\hline
\hypertarget{RFD9.10}{RFD9.10} & Lo sviluppatore può inserire il logo dell'API & \makecell*{Capitolato\\UC11\\UC11.10} \\
\hline
\hypertarget{RFD9.11}{RFD9.11} & L'applicazione web mostra un errore nel caso l'inserimento del logo fallisca & \makecell*{Capitolato\\UC11\\UC11.11} \\
\hline
\hypertarget{RFO9.12}{RFO9.12} & Lo sviluppatore può confermare la registrazione della nuova API & \makecell*{Capitolato\\UC11\\UC11.12} \\
\hline
\hypertarget{RFD9.13}{RFD9.13} & L'applicazione web mostra un errore nel caso la registrazione della nuova API fallisca & \makecell*{Capitolato\\UC11\\UC11.13} \\
\hline

\hypertarget{RFO10}{RFO10} & Il cliente può visualizzare il menù del profilo utente & \makecell*{Capitolato\\UC12} \\
\hline

\hypertarget{RFO10.1}{RFO10.1} & Il cliente può gestire le informazioni personali & \makecell*{Capitolato\\UC12\\UC12.1} \\
\hline

\hypertarget{RFO10.1.1}{RFO10.1.1} & Il cliente può visualizzare le informazioni personali del profilo utente &\makecell*{Capitolato\\UC12\\UC12.1\\UC12.1.1} \\
\hline

\hypertarget{RFO10.1.1.1}{RFO10.1.1.1} & Il cliente può visualizzare il nome & \makecell*{Capitolato\\UC12\\UC12.1\\UC12.1.1\\UC12.1.1.1} \\
\hline

\hypertarget{RFO10.1.1.2}{RFO10.1.1.2} & Il cliente può visualizzare il cognome & \makecell*{Capitolato\\UC12\\UC12.1\\UC12.1.1\\UC12.1.1.2} \\
\hline

\hypertarget{RFO10.1.1.3}{RFO10.1.1.3} & Il cliente può visualizzare lo username & \makecell*{Capitolato\\UC12\\UC12.1\\UC12.1.1\\UC12.1.1.3} \\
\hline

\hypertarget{RFO10.1.1.4}{RFO10.1.1.4} & Il cliente può visualizzare l'email & \makecell*{Capitolato\\UC12\\UC12.1\\UC12.1.1\\UC12.1.1.4} \\
\hline

\hypertarget{RFO10.1.1.5}{RFO10.1.1.5} &  Il cliente può visualizzare l'immagine del profilo & \makecell*{Capitolato\\UC12\\UC12.1\\UC12.1.1\\UC12.1.1.5} \\
\hline

\hypertarget{RFO10.1.2}{RFO10.1.2} &  Il cliente può modificare le informazioni personali del profilo utente &\makecell*{Capitolato\\UC12\\UC12.1\\UC12.1.2} \\
\hline

\hypertarget{RFO10.1.2.1}{RFO10.1.2.1} &  Il cliente può modificare il nome & \makecell*{Capitolato\\UC12\\UC12.1\\UC12.1.2\\UC12.1.1.1} \\
\hline

\hypertarget{RFO10.1.2.2}{RFO10.1.2.2} &  Il cliente può modificare il cognome & \makecell*{Capitolato\\UC12\\UC12.1\\UC12.1.2\\UC12.1.2.2} \\
\hline

\hypertarget{RFO10.1.2.3}{RFO10.1.2.3} &  Il cliente può modificare l'username & \makecell*{Capitolato\\UC12\\UC12.1\\UC12.1.2\\UC12.1.2.3} \\
\hline

\hypertarget{RFO10.1.2.4}{RFO10.1.2.4} &  Il cliente può modificare l'email & \makecell*{Capitolato\\UC12\\UC12.1\\UC12.1.2\\UC12.1.2.4} \\
\hline

\hypertarget{RFO10.1.2.5}{RFO10.1.2.5} &  Il cliente può modificare l'immagine del profilo & \makecell*{Capitolato\\UC12\\UC12.1\\UC12.1.2\\UC12.1.2.5} \\
\hline

\hypertarget{RFO10.1.2.6}{RFO10.1.2.6} &  Il cliente può confermare le modifiche delle info del profilo & \makecell*{Capitolato\\UC12\\UC12.1\\UC12.1.2\\UC12.1.2.6} \\
\hline

\hypertarget{RFO10.1.2.7}{RFO10.1.2.7} &  L'applicazione web puo' mostrare un errore in seguito a un'illegalita' sorta nel modificare le info del profilo & \makecell*{Capitolato\\UC12\\UC12.1\\UC12.1.2\\UC12.1.2.7} \\
\hline

\hypertarget{RFO10.2}{RFO10.2} &  Il cliente può gestire il metodo di pagamento & \makecell*{Capitolato\\UC12\\UC12.2} \\
\hline

\hypertarget{RFO10.2.1}{RFO10.2.1} &  Il cliente può visualizzare il saldo & \makecell*{Capitolato\\UC12\\UC12.2\\UC12.2.1} \\
\hline

\hypertarget{RFO10.2.2}{RFO10.2.2} &  Il cliente può versare il saldo & \makecell*{Capitolato\\UC12\\UC12.2\\UC12.2.2} \\
\hline

\hypertarget{RFO10.2.2.1}{RFO10.2.2.1} &  Il cliente può selezionare la somma di accredito & \makecell*{Capitolato\\UC12\\UC12.2\\UC12.2.2\\UC12.2.2.1} \\
\hline

\hypertarget{RFO10.2.2.2}{RFO10.2.2.2} &  Il cliente può confermare l'accredito della somma selezionata & \makecell*{Capitolato\\UC12\\UC12.2\\UC12.2.2\\UC12.2.2.2} \\
\hline

\hypertarget{RFO10.2.2.3}{RFO10.2.2.3} &  L'applicazione web mostra un errore di fallimento ricarica & \makecell*{Capitolato\\UC12\\UC12.2\\UC12.2.2\\UC12.2.2.3} \\
\hline

\hypertarget{RFO10.2.3}{RFO10.2.3} &  Il cliente può prelevare dal saldo & \makecell*{Capitolato\\UC12\\UC12.2\\UC12.2.3} \\
\hline

\hypertarget{RFF10.2.3.1}{RFF1O.2.3.1} &  Il cliente può trasferire il saldo su un conto Paypal esterno & \makecell*{Capitolato\\UC12\\UC12.2\\UC12.2.3\\UC12.2.3.1} \\
\hline

\hypertarget{RFO11}{RFO11} & L'utente autenticato può effettuare il logout & \makecell*{Capitolato\\UC13} \\
\hline

\hypertarget{RFO12}{RFO12} &  L'amministratore API Market puo' amministrare l'applicazione web & \makecell*{Capitolato\\UC14} \\
\hline

\hypertarget{RFO12.1}{RFO12.1} &  L'amministratore API Market puo' visualizzare i dati di utilizzo avanzati & \makecell*{Capitolato\\UC14\\UC14.1} \\
\hline

\hypertarget{RFO12.1.1}{RFO12.1.1} &  L'amministratore API Market puo' visualizzare il numero di utenti attivi per ogni API & \makecell*{Capitolato\\UC14\\UC14.1\\UC14.1.1} \\
\hline

\hypertarget{RFO12.1.1.1}{RFO12.1.1.1} &  L'amministratore API Market puo' visualizzare il nome dell'API & \makecell*{Capitolato\\UC14\\UC14.1\\UC14.1.1\\UC14.1.1.1} \\
\hline

\hypertarget{RFO12.1.1.2}{RFO12.1.1.2} &  L'amministratore API Market puo' visualizzare la durata residua della licenza dell'API & \makecell*{Capitolato\\UC14\\UC14.1\\UC14.1.1\\UC14.1.1.2} \\
\hline

\hypertarget{RFO12.2}{RFO12.2} &  L'amministratore API Market puo'effettuare azioni sugli utenti& \makecell*{Capitolato\\UC14\\UC14.2} \\
\hline

\hypertarget{RFO12.2.1}{RFO12.2.1} &  L'amministratore API Market puo' ricercare un utente attraverso il suo username& \makecell*{Capitolato\\UC14\\UC14.2\\UC14.2.1} \\
\hline

\hypertarget{RFO12.2.2}{RFO12.2.2} &  L'amministratore API Market puo' sospendere un utente& \makecell*{Capitolato\\UC14\\UC14.2\\UC14.2.2} \\
\hline

\hypertarget{RFO12.2.2.1}{RFO12.2.2.1} &  L'amministratore API Market puo' scegliere la durata della sospensione dell'utente& \makecell*{Capitolato\\UC14\\UC14.2\\UC14.2.2\\UC14.2.2.1} \\
\hline

\hypertarget{RFO12.2.2.2}{RFO12.2.2.2} &  L'amministratore API Market puo' confermare la sospensione dell'utente& \makecell*{Capitolato\\UC14\\UC14.2\\UC14.2.2\\UC14.2.2.2} \\
\hline

\hypertarget{RFO12.2.3}{RFO12.2.3} &  L'amministratore API Market puo' sospendere i pagamenti di un utente& \makecell*{Capitolato\\UC14\\UC14.2\\UC14.2.3} \\
\hline

\hypertarget{RFO12.2.4}{RFO12.2.4} &  L'amministratore API Market puo' revocare la sospensione di un utente& \makecell*{Capitolato\\UC14\\UC14.2\\UC14.2.4} \\
\hline

\hypertarget{RFO12.2.5}{RFO12.2.5} &  L'amministratore API Market puo' revocare la sospensione dei pagamenti di un utente& \makecell*{Capitolato\\UC14\\UC14.2\\UC14.2.5} \\
\hline


\end{longtable}


\subsection{Requisiti di qualità}
\begin{longtable}{|c|m{8cm}|c|}
\caption{Tabella dei requisiti di qualità} \\

\hline
\thead*{\textbf{Codice requisito}} &\thead{\textbf{Descrizione}}  &\thead{\textbf{Fonte}} \\
\hline
\endhead

\hline
\endfoot
\hline
\endlastfoot

\hypertarget{RQO1}{RQO1} &  Nel caso di più gruppi per questo progetto, le parti del prodotto finale dovranno essere componibili ed integrabili tra loro & \makecell*{Capitolato} \\
\hline

\hypertarget{RQO2}{RQO2} & Per ogni servizio, deve essere fornita la descrizione di ogni servizio e delle singole API, l'interfaccia delle API, lo schema design relativo all'eventuale base di dati associata & \makecell*{Capitolato} \\
\hline

\hypertarget{RQO3}{RQO3} & Deve essere fornito il sequence chart diagram delle interazioni che prevedono il coinvolgimento di più microservizi & \makecell*{Capitolato} \\
\hline

\hypertarget{RQO4}{RQO4} & Devono essere forniti gli algoritmi delle policy per l'utilizzo delle API & \makecell*{Capitolato} \\
\hline

\hypertarget{RQO5}{RQO5} & Deve essere fornito l'algoritmo di generazione delle API key & \makecell*{Capitolato} \\
\hline

\hypertarget{RQO6}{RQO6} & Il prodotto finale deve superare i test forniti da ItalianaSoftware & \makecell*{Capitolato} \\
\hline

\hypertarget{RQO7}{RQO7} & Il prodotto finale deve essere depositato su una repository git &\makecell*{Capitolato} \\
\hline

\hypertarget{RQO8}{RQO8} & Deve essere stilato breve report tecnico che evidenzi gli aspetti positivi e negativi di un'architettura a microservizi &\makecell*{Capitolato} \\
\hline

\hypertarget{RQD9}{RQD9} & Deve essere redatto un manuale per l'installazione e l'avvio dell'applicazione web &\makecell*{Capitolato} \\
\hline

\end{longtable}

\newpage
\subsection{Requisiti di vincolo}
\begin{longtable}{|c|m{8cm}|c|}
\caption{Tabella dei requisiti di vincolo} \\

\hline
\thead*{\textbf{Codice requisito}} &\thead{\textbf{Descrizione}}  &\thead{\textbf{Fonte}} \\
\hline
\endhead

\hline
\endfoot
\hline
\endlastfoot

\hypertarget{RVO1}{RVO1} & Il sistema deve avere un'architettura a microservizi & \makecell*{Capitolato} \\
\hline

\hypertarget{RVO2}{RVO2} & Si utilizzerà il linguaggio Jolie per le interfacce dei microservizi, per l'API Gateway e possibilmente per le altre componenti dell'applicazione web  &\makecell*{Capitolato} \\
\hline

\hypertarget{RVO3}{RVO3} & Le componenti web verranno realizzate utilizzando Javascript, HTML, css3 &\makecell*{Capitolato} \\
\hline

\hypertarget{RVO4}{RVO4} & Come database può essere utilizzato sia SQL che NoSQL &\makecell*{Capitolato} \\
\hline

\end{longtable}

\newpage
\subsection{Tabella Fonte-Requisiti}
\normalsize
\begin{longtable}{|>{\centering}m{5cm}|m{5cm}<{\centering}|}
\hline 
\textbf{Fonte} & \textbf{ID Requisito}\\
\hline
\endhead
\hyperlink{Capitolato}{Capitolato}
& \hyperlink{RFO1}{RFO1}\\
& \hyperlink{RFO2}{RFO2}\\
& \hyperlink{RFO2.1}{RFO2.1}\\
& \hyperlink{RFO4}{RFO4}\\
& \hyperlink{RFO5}{RFO5}\\
& \hyperlink{RFO5.6}{RFO5.6}\\
& \hyperlink{RFO5.6.1}{RFO5.6.1}\\
& \hyperlink{RFO5.6.2}{RFO5.6.2}\\
& \hyperlink{RFO5.7}{RFO5.7}\\
& \hyperlink{RFO5.7.2}{RFO5.7.2}\\
& \hyperlink{RFO7}{RFO7}\\
& \hyperlink{RFO7.5.3}{RFO7.5.3}\\
& \hyperlink{RFO8}{RFO8}\\
& \hyperlink{RFO8.2.4}{RFO8.2.4}\\
& \hyperlink{RFO9}{RFO9}\\
& \hyperlink{RFO9.8}{RFO9.8}\\
& \hyperlink{RFO9.8.1}{RFO9.8.1}\\
& \hyperlink{RFO9.8.2}{RFO9.8.2}\\
& \hyperlink{RFO9.8.3}{RFO9.8.3}\\
& \hyperlink{RFO10}{RFO10}\\
& \hyperlink{RFO10.1}{RFO10.1}\\
& \hyperlink{RFO10.1.1}{RFO10.1.1}\\
& \hyperlink{RFO10.1.2}{RFO10.1.2}\\
& \hyperlink{RFO11}{RFO11}\\
& \hyperlink{RFO12}{RFO12}\\
& \hyperlink{RFO12.1.1.1}{RFO12.1.1.1}\\
& \hyperlink{RFO12.1.1.1.6}{RFO12.1.1.1.6}\\
& \hyperlink{RFO13}{RFO13}\\
& \hyperlink{RQO1}{RQO1}\\
& \hyperlink{RQO2}{RQO2}\\
& \hyperlink{RQO3}{RQO3}\\
& \hyperlink{RQO4}{RQO4}\\
& \hyperlink{RQO5}{RQO5}\\
& \hyperlink{RQO6}{RQO6}\\
& \hyperlink{RQO7}{RQO7}\\
& \hyperlink{RQD8}{RQD8}\\
& \hyperlink{RVO1}{RVO1}\\
& \hyperlink{RVO2}{RVO2}\\
& \hyperlink{RVO3}{RVO3}\\
& \hyperlink{RVO4}{RVO4}\\

\hline
\hyperlink{Interno}{Interno} 
& \hyperlink{RFO1.1}{RFO1.1}\\
& \hyperlink{RFO1.2}{RFO1.2}\\
& \hyperlink{RFO1.3}{RFO1.3}\\
& \hyperlink{RFO1.4}{RFO1.4}\\
& \hyperlink{RFO1.5}{RFO1.5}\\
& \hyperlink{RFO1.6}{RFO1.6}\\
& \hyperlink{RFD1.7}{RFD1.7}\\
& \hyperlink{RFO1.8}{RFO1.8}\\
& \hyperlink{RFO1.9}{RFO1.9}\\
& \hyperlink{RFO1.10}{RFO1.10}\\
& \hyperlink{RFO2.1}{RFO2.1}\\
& \hyperlink{RFO2.1.1}{RFO2.1.1}\\
& \hyperlink{RFO2.1.2}{RFO2.1.2}\\
& \hyperlink{RFO2.1.3}{RFO2.1.3}\\
& \hyperlink{RFO2.1.4}{RFO2.1.4}\\
& \hyperlink{RFF2.2}{RFF2.2}\\
& \hyperlink{RFF2.2.2}{RFF2.2.2}\\
& \hyperlink{RFF2.3}{RFF2.3}\\
& \hyperlink{RFF2.3.2}{RFF2.3.2}\\
& \hyperlink{RFF2.4}{RFF2.4}\\
& \hyperlink{RFF2.4.2}{RFF2.4.2}\\
& \hyperlink{RFF2.5}{RFF2.5}\\
& \hyperlink{RFF2.5.2}{RFF2.5.2}\\
& \hyperlink{RFD3}{RFD3}\\
& \hyperlink{RFD3.1}{RFD3.1}\\
& \hyperlink{RFD3.2}{RFD3.2}\\
& \hyperlink{RFD3.3}{RFD3.3}\\
& \hyperlink{RFD3.4}{RFD3.4}\\
& \hyperlink{RFO4}{RFO4}\\
& \hyperlink{RFO4.1}{RFO4.1}\\
& \hyperlink{RFO4.2}{RFO4.2}\\
& \hyperlink{RFO4.3}{RFO4.3}\\
& \hyperlink{RFO4.3.1}{RFO4.3.1}\\
& \hyperlink{RFO4.3.2}{RFO4.3.2}\\
& \hyperlink{RFO4.3.3}{RFO4.3.3}\\
& \hyperlink{RFD4.3.4}{RFD4.3.4}\\
& \hyperlink{RFO4.3.5}{RFO4.3.5}\\
& \hyperlink{RFO5.1}{RFO5.1}\\
& \hyperlink{RFO5.2}{RFO5.2}\\
& \hyperlink{RFO5.3}{RFO5.3}\\
& \hyperlink{RFO5.4}{RFO5.4}\\
& \hyperlink{RFO5.5}{RFO5.5}\\
& \hyperlink{RFO5.5.1}{RFO5.5.1}\\
& \hyperlink{RFO5.5.2}{RFO5.5.2}\\
& \hyperlink{RFO5.6}{RFO5.6}\\
& \hyperlink{RFD5.7.1}{RFD5.7.1}\\
& \hyperlink{RFO5.8}{RFO5.8}\\
& \hyperlink{RFO5.9}{RFO5.9}\\
& \hyperlink{RFO5.10}{RFO5.10}\\
& \hyperlink{RFD5.11}{RFD5.11}\\
& \hyperlink{RFO5.12}{RFO5.12}\\
& \hyperlink{RFO5.13}{RFO5.13}\\
& \hyperlink{RFO6.1}{RFO6.1}\\
& \hyperlink{RFO6.2.1}{RFO6.2.1}\\
& \hyperlink{RFO6.2.2}{RFO6.2.2}\\
& \hyperlink{RFO6.2.3}{RFO6.2.3}\\
& \hyperlink{RFO6.2.4}{RFO6.2.4}\\
& \hyperlink{RFO6.2.5}{RFO6.2.5}\\
& \hyperlink{RFD6.2.6}{RFD6.2.6}\\
& \hyperlink{RFD6.2.7}{RFD6.2.7}\\
& \hyperlink{RFO7.1}{RFO7.1}\\
& \hyperlink{RFO7.1.1}{RFO7.1.1}\\
& \hyperlink{RFO7.1.2}{RFO7.1.2}\\
& \hyperlink{RFO7.1.3}{RFO7.1.3}\\
& \hyperlink{RFO7.2}{RFO7.2}\\
& \hyperlink{RFO7.3}{RFO7.3}\\
& \hyperlink{RFO7.4}{RFO7.4}\\
& \hyperlink{RFO7.5}{RFO7.5}\\
& \hyperlink{RFD7.5.1}{RFD7.5.1}\\
& \hyperlink{RFO7.5.2}{RFO7.5.2}\\
& \hyperlink{RFO7.6}{RFO7.6}\\
& \hyperlink{RFO8.1}{RFO8.1}\\
& \hyperlink{RFO8.2}{RFO8.2}\\
& \hyperlink{RFO8.2.1}{RFO8.2.1}\\
& \hyperlink{RFO8.2.2}{RFO8.2.2}\\

& \hyperlink{RFO8.2.3}{RFO8.2.3}\\
& \hyperlink{RFO8.2.4}{RFO8.2.4}\\
& \hyperlink{RFO8.2.4.1}{RFO8.2.4.1}\\
& \hyperlink{RFO8.2.4.2}{RFO8.2.4.2}\\
& \hyperlink{RFO8.2.4.3}{RFO8.2.4.3}\\
& \hyperlink{RFO8.2.4.4}{RFO8.2.4.4}\\
& \hyperlink{RFO8.2.4.5}{RFO8.2.4.5}\\
& \hyperlink{RFO8.2.4.6}{RFO8.2.4.6}\\
& \hyperlink{RFO8.2.4.7}{RFO8.2.4.7}\\
& \hyperlink{RFO8.2.4.8}{RFO8.2.4.8}\\
& \hyperlink{RFO8.2.4.9}{RFO8.2.4.9}\\
& \hyperlink{RFO8.2.4.10}{RFO8.2.4.10}\\
& \hyperlink{RFO8.2.4.11}{RFO8.2.4.11}\\
& \hyperlink{RFD8.2.5}{RFD8.2.5}\\
& \hyperlink{RFD8.2.6}{RFD8.2.6}\\
& \hyperlink{RFD8.2.7}{RFD8.2.7}\\
& \hyperlink{RFO8.2.8}{RFO8.2.8}\\
& \hyperlink{RFO8.2.9}{RFO8.2.9}\\
& \hyperlink{RFO8.2.9.1}{RFO8.2.9.1}\\
& \hyperlink{RFO8.2.9.2}{RFO8.2.9.2}\\
& \hyperlink{RFO9.1}{RFO9.1}\\
& \hyperlink{RFO9.2}{RFO9.2}\\
& \hyperlink{RFD9.3}{RFD9.3}\\
& \hyperlink{RFO9.4}{RFO9.4}\\
& \hyperlink{RFO9.5}{RFO9.5}\\
& \hyperlink{RFD9.6}{RFD9.6}\\
& \hyperlink{RFO9.7}{RFO9.7}\\
& \hyperlink{RFD9.9}{RFD9.9}\\
& \hyperlink{RFD9.10}{RFD9.10}\\
& \hyperlink{RFD9.11}{RFD9.11}\\
& \hyperlink{RFO9.12}{RFO9.12}\\
& \hyperlink{RFD9.13}{RFD9.13}\\
& \hyperlink{RFO10.1.1.1}{RFO10.1.1.1}\\
& \hyperlink{RFO10.1.1.2}{RFO10.1.1.2}\\
& \hyperlink{RFO10.1.1.3}{RFO10.1.1.3}\\
& \hyperlink{RFO10.1.1.4}{RFO10.1.1.4}\\
& \hyperlink{RFO10.1.1.5}{RFO10.1.1.5}\\
& \hyperlink{RFD10.1.1.6}{RFD10.1.1.6}\\
& \hyperlink{RFO10.1.2.1}{RFO10.1.2.1}\\
& \hyperlink{RFO10.1.2.2}{RFO10.1.2.2}\\
& \hyperlink{RFO10.1.2.3}{RFO10.1.2.3}\\
& \hyperlink{RFO10.1.2.4}{RFO10.1.2.4}\\
& \hyperlink{RFO10.1.2.5}{RFO10.1.2.5}\\
& \hyperlink{RFD10.1.2.6}{RFD10.1.2.6}\\
& \hyperlink{RFO10.1.2.7}{RFO10.1.2.7}\\
& \hyperlink{RFO10.1.2.8}{RFO10.1.2.8}\\
& \hyperlink{RFO10.2.1}{RFO10.2.1}\\
& \hyperlink{RFO10.2.2}{RFO10.2.2}\\
& \hyperlink{RFO10.2.2.1}{RFO10.2.2.1}\\
& \hyperlink{RFO10.2.2.2}{RFO10.2.2.2}\\
& \hyperlink{RFO10.3}{RFO10.3}\\
& \hyperlink{RFO10.3.1}{RFO10.3.1}\\
& \hyperlink{RFO10.3.1.1}{RFO10.3.1.1}\\
& \hyperlink{RFO10.3.1.2}{RFO10.3.1.2}\\
& \hyperlink{RFO10.3.2}{RFO10.3.2}\\
& \hyperlink{RFO10.3.2.1}{RFO10.3.2.1}\\
& \hyperlink{RFO10.3.2.2}{RFO10.3.2.2}\\
& \hyperlink{RFO10.3.2.3}{RFO10.3.2.3}\\
& \hyperlink{RFO10.3.2.4}{RFO10.3.2.4}\\
& \hyperlink{RFO10.3.2.5}{RFO10.3.2.5}\\
& \hyperlink{RFO10.3.3}{RFO10.3.3}\\
& \hyperlink{RFO10.3.3.1}{RFO10.3.3.1}\\
& \hyperlink{RFO10.3.3.2}{RFO10.3.3.2}\\
& \hyperlink{RFO10.3.3.3}{RFO10.3.3.3}\\
& \hyperlink{RFO10.3.3.4}{RFO10.3.3.4}\\

& \hyperlink{RFO12.1}{RFO12.1}\\
& \hyperlink{RFO12.1.1}{RFO12.1.1}\\
& \hyperlink{RFO12.1.1.1.1}{RFO12.1.1.1.1}\\
& \hyperlink{RFO12.1.1.1.2}{RFO12.1.1.1.2}\\
& \hyperlink{RFD12.1.1.1.3}{RFD12.1.1.1.3}\\
& \hyperlink{RFD12.1.1.1.4}{RFD12.1.1.1.4}\\
& \hyperlink{RFO12.1.1.1.5}{RFO12.1.1.1.5}\\
& \hyperlink{RFO12.1.1.1.5.1}{RFO12.1.1.1.5.1}\\
& \hyperlink{RFO12.1.1.1.5.2}{RFO12.1.1.1.5.2}\\
& \hyperlink{RFO12.1.1.2}{RFO12.1.1.2}\\
& \hyperlink{RFO12.1.1.2.1}{RFO12.1.1.2.1}\\
& \hyperlink{RFO12.1.1.2.2}{RFO12.1.1.2.2}\\
& \hyperlink{RFO12.1.1.3}{RFO12.1.1.3}\\
& \hyperlink{RFO12.1.1.3.1}{RFO12.1.1.3.1}\\
& \hyperlink{RFO12.1.1.3.2}{RFO12.1.1.3.2}\\
& \hyperlink{RFO12.1.1.3.3}{RFO12.1.1.3.3}\\

& \hyperlink{RFO12.2}{RFO12.2}\\
& \hyperlink{RFO12.2.1}{RFO12.2.1}\\
& \hyperlink{RFO12.2.2}{RFO12.2.2}\\
& \hyperlink{RFO12.2.2.1}{RFO12.2.2.1}\\
& \hyperlink{RFO12.2.2.2}{RFO12.2.2.2}\\
& \hyperlink{RFO12.2.3}{RFO12.2.3}\\
& \hyperlink{RFO12.2.4}{RFO12.2.4}\\
& \hyperlink{RFO12.2.5}{RFO12.2.5}\\
& \hyperlink{RQD9}{RQD9}\\

\hline
\hyperlink{UC3}{UC3} 
& \hyperlink{RFO1}{RFO1}\\
& \hyperlink{RFO1.1}{RFO1.1}\\
& \hyperlink{RFO1.2}{RFO1.2}\\
& \hyperlink{RFO1.3}{RFO1.3}\\
& \hyperlink{RFO1.4}{RFO1.4}\\
& \hyperlink{RFO1.5}{RFO1.5}\\
& \hyperlink{RFO1.6}{RFO1.6}\\
& \hyperlink{RFD1.7}{RFD1.7}\\
& \hyperlink{RFO1.8}{RFO1.8}\\
& \hyperlink{RFO1.9}{RFO1.9}\\
& \hyperlink{RFF1.10}{RFF1.10}\\\hline

\hyperlink{UC3.1}{UC3.1} & \hyperlink{RFO1.1}{RFO1.1}\\\hline
\hyperlink{UC3.2}{UC3.2} & \hyperlink{RFO1.2}{RFO1.2}\\\hline
\hyperlink{UC3.3}{UC3.3} & \hyperlink{RFO1.3}{RFO1.3}\\\hline
\hyperlink{UC3.4}{UC3.4} & \hyperlink{RFO1.4}{RFO1.4}\\\hline
\hyperlink{UC3.5}{UC3.5} & \hyperlink{RFO1.5}{RFO1.5}\\\hline
\hyperlink{UC3.6}{UC3.6} & \hyperlink{RFO1.6}{RFO1.6}\\\hline
\hyperlink{UC3.7}{UC3.7} & \hyperlink{RFD1.7}{RFD1.7}\\\hline
\hyperlink{UC3.8}{UC3.8} & \hyperlink{RFF1.8}{RFF1.8}\\\hline
\hyperlink{UC3.9}{UC3.9} & \hyperlink{RFO1.9}{RFO1.9}\\\hline
\hyperlink{UC3.10}{UC3.10} & \hyperlink{RFF1.10}{RFF1.10}\\\hline

\hyperlink{UC4}{UC4} 
& \hyperlink{RFO2}{RFO2}\\
& \hyperlink{RFO2.1}{RFO2.1}\\
& \hyperlink{RFO2.1.1}{RFO2.1.1}\\
& \hyperlink{RFO2.1.2}{RFO2.1.2}\\
& \hyperlink{RFO2.1.3}{RFO2.1.3}\\
& \hyperlink{RFO2.1.4}{RFO2.1.4}\\
& \hyperlink{RFF2.2}{RFF2.2}\\
& \hyperlink{RFF2.2.2}{RFF2.2.2}\\
& \hyperlink{RFF2.3}{RFF2.3}\\
& \hyperlink{RFF2.3.2}{RFF2.3.2}\\
& \hyperlink{RFF2.4}{RFF2.4}\\
& \hyperlink{RFF2.4.2}{RFF2.4.2}\\
& \hyperlink{RFF2.5}{RFD2.5}\\
& \hyperlink{RFF2.5.2}{RFD2.5.2}\\\hline


\hyperlink{UC4.1}{UC4.1} 
& \hyperlink{RFO2.1}{RFO2.1}\\
& \hyperlink{RFO2.1.1}{RFO2.1.1}\\
& \hyperlink{RFO2.1.2}{RFO2.1.2}\\
& \hyperlink{RFO2.1.3}{RFO2.1.3}\\
& \hyperlink{RFF2.1.4}{RFF2.1.4}\\
& \hyperlink{RFF2.2}{RFF2.2}\\
& \hyperlink{RFF2.3}{RFF2.3}\\
& \hyperlink{RFF2.4}{RFF2.4}\\
& \hyperlink{RFF2.5}{RFD2.5}\\\hline

\hyperlink{UC4.1.1}{UC4.1.1} 
& \hyperlink{RFO2.1.1}{RFO2.1.1}\\ \hline
\hyperlink{UC4.1.2}{UC4.1.2} 
& \hyperlink{RFO2.1.2}{RFO2.1.2}\\ \hline
\hyperlink{UC4.1.3}{UC4.1.3} 
& \hyperlink{RFO2.1.3}{RFO2.1.3}\\ \hline
\hyperlink{UC4.1.4}{UC4.1.4} 
& \hyperlink{RFF2.1.4}{RFF2.1.4}\\ \hline

\hyperlink{UC4.2}{UC4.2} 
& \hyperlink{RFF2.2}{RFF2.2}\\
& \hyperlink{RFF2.2.2}{RFF2.2.2}\\ \hline
\hyperlink{UC4.3}{UC4.3}
& \hyperlink{RFF2.3}{RFF2.3}\\
& \hyperlink{RFF2.3.2}{RFF2.3.2}\\ \hline
\hyperlink{UC4.4}{UC4.4} 
& \hyperlink{RFF2.4}{RFF2.4}\\
& \hyperlink{RFF2.4.2}{RFF2.4.2}\\ \hline
\hyperlink{UC4.5}{UC4.5} 
& \hyperlink{RFF2.5}{RFF2.5}\\
& \hyperlink{RFF2.5.2}{RFF2.5.2}\\ \hline

\hyperlink{UC4.2.2}{UC4.2.2} & \hyperlink{RFF2.2.2}{RFF2.2.2}\\ \hline
\hyperlink{UC4.3.2}{UC4.3.2} & \hyperlink{RFF2.3.2}{RFF2.3.2}\\ \hline
\hyperlink{UC4.4.2}{UC4.4.2} & \hyperlink{RFF2.4.2}{RFF2.4.2}\\ \hline
\hyperlink{UC4.5.2}{UC4.5.2} & \hyperlink{RFF2.5.2}{RFF2.5.2}\\ \hline

\hyperlink{UC5}{UC5} 
& \hyperlink{RFD3}{RFD3}\\
& \hyperlink{RFD3.1}{RFD3.1}\\
& \hyperlink{RFD3.2}{RFD3.2}\\
& \hyperlink{RFD3.3}{RFD3.3}\\
& \hyperlink{RFD3.4}{RFD3.4}\\\hline

\hyperlink{UC5.1}{UC5.1} & \hyperlink{RFD3.1}{RFD3.1}\\ \hline
\hyperlink{UC5.2}{UC5.2} & \hyperlink{RFD3.2}{RFD3.2}\\ \hline
\hyperlink{UC5.3}{UC5.3} & \hyperlink{RFD3.3}{RFD3.3}\\ \hline
\hyperlink{UC5.4}{UC5.4} & \hyperlink{RFD3.4}{RFD3.4}\\ \hline

\hyperlink{UC6}{UC6} 
& \hyperlink{RFO4}{RFO4}\\
& \hyperlink{RFO4.1}{RFO4.1}\\
& \hyperlink{RFO4.2}{RFO4.2}\\
& \hyperlink{RFO4.3}{RFO4.3}\\
& \hyperlink{RFO4.3.1}{RFO4.3.1}\\
& \hyperlink{RFO4.3.2}{RFO4.3.2}\\
& \hyperlink{RFO4.3.3}{RFO4.3.3}\\
& \hyperlink{RFO4.3.4}{RFO4.3.4}\\
& \hyperlink{RFO4.3.5}{RFO4.3.5}\\\hline

\hyperlink{UC6.1}{UC6.1} 
& \hyperlink{RFO4.1}{RFO4.1}\\
& \hyperlink{RFO4.3}{RFO4.3}\\
& \hyperlink{RFO4.3.1}{RFO4.3.1}\\
& \hyperlink{RFO4.3.2}{RFO4.3.2}\\
& \hyperlink{RFO4.3.3}{RFO4.3.3}\\
& \hyperlink{RFD4.3.4}{RFD4.3.4}\\
& \hyperlink{RFO4.3.5}{RFO4.3.5}\\ \hline

\hyperlink{UC6.2}{UC6.2} 
& \hyperlink{RFO4.2}{RFO4.2}\\ \hline

\hyperlink{UC6.3}{UC6.3} 
& \hyperlink{RFO4.3}{RFO4.3}\\
& \hyperlink{RFO4.3.1}{RFO4.3.1}\\
& \hyperlink{RFO4.3.2}{RFO4.3.2}\\
& \hyperlink{RFO4.3.3}{RFO4.3.3}\\
& \hyperlink{RFD4.3.4}{RFD4.3.4}\\
& \hyperlink{RFO4.3.5}{RFO4.3.5}\\ \hline

\hyperlink{UC6.3.1}{UC6.3.1} & \hyperlink{RFO4.3.1}{RFO4.3.1}\\ \hline
\hyperlink{UC6.3.2}{UC6.3.2} & \hyperlink{RFO4.3.2}{RFO4.3.2}\\ \hline
\hyperlink{UC6.3.3}{UC6.3.3} & \hyperlink{RFO4.3.3}{RFO4.3.3}\\ \hline
\hyperlink{UC6.3.4}{UC6.3.4} & \hyperlink{RFD4.3.4}{RFD4.3.4}\\ \hline
\hyperlink{UC6.3.5}{UC6.3.5} & \hyperlink{RFO4.3.5}{RFO4.3.5}\\ \hline

\hyperlink{UC7}{UC7} 
& \hyperlink{RFO5}{RFO5}\\
& \hyperlink{RFO5.1}{RFO5.1}\\
& \hyperlink{RFO5.2}{RFO5.2}\\
& \hyperlink{RFO5.3}{RFO5.3}\\
& \hyperlink{RFO5.4}{RFO5.4}\\
& \hyperlink{RFO5.5}{RFO5.5}\\
& \hyperlink{RFO5.6}{RFO5.6}\\
& \hyperlink{RFO5.6.1}{RFO5.6.1}\\
& \hyperlink{RFO5.6.2}{RFO5.6.2}\\
& \hyperlink{RFO5.7}{RFO5.7}\\
& \hyperlink{RFO5.7.1}{RFO5.7.1}\\
& \hyperlink{RFO5.7.2}{RFO5.7.2}\\
& \hyperlink{RFO5.7.3}{RFO5.7.3}\\
& \hyperlink{RFO5.7.4}{RFO5.7.4}\\
& \hyperlink{RFO5.8}{RFO5.8}\\
& \hyperlink{RFO5.8.1}{RFO5.8.1}\\
& \hyperlink{RFO5.8.2}{RFO5.8.2}\\
& \hyperlink{RFO5.9}{RFO5.9}\\
& \hyperlink{RFO5.10}{RFO5.10}\\
& \hyperlink{RFD5.11}{RFD5.11}\\
& \hyperlink{RFO5.12}{RFO5.12}\\
& \hyperlink{RFO5.13}{RFO5.13}\\\hline

\hyperlink{UC7.1}{UC7.1} & \hyperlink{RFO5.1}{RFO5.1}\\\hline
\hyperlink{UC7.2}{UC7.2} & \hyperlink{RFO5.2}{RFO5.2}\\\hline
\hyperlink{UC7.3}{UC7.3} & \hyperlink{RFO5.3}{RFO5.3}\\\hline
\hyperlink{UC7.4}{UC7.4} & \hyperlink{RFO5.4}{RFO5.4}\\\hline
\hyperlink{UC7.5}{UC7.5}
& \hyperlink{RFO5.5}{RFO5.5}\\
& \hyperlink{RFO5.5.1}{RFO5.5.1}\\
& \hyperlink{RFO5.5.2}{RFO5.5.2}\\\hline
\hyperlink{UC7.6}{UC7.6} 
& \hyperlink{RFO5.6}{RFO5.6}\\
& \hyperlink{RFO5.6.1}{RFO5.6.1}\\
& \hyperlink{RFD5.6.2}{RFD5.6.2}\\\hline
\hyperlink{UC7.7}{UC7.7} & \hyperlink{RFO5.7}{RFO5.7}\\
& \hyperlink{RFD5.7.1}{RFD5.7.1}\\
& \hyperlink{RFO5.7.2}{RFO5.7.2}\\\hline
\hyperlink{UC7.8}{UC7.8} & \hyperlink{RFO5.8}{RFO5.8}\\\hline
\hyperlink{UC7.9}{UC7.9} & \hyperlink{RFO5.9}{RFO5.9}\\\hline
\hyperlink{UC7.10}{UC7.10} & \hyperlink{RFO5.10}{RFO5.10}\\\hline
\hyperlink{UC7.11}{UC7.11} & \hyperlink{RFD5.11}{RFD5.11}\\\hline
\hyperlink{UC7.12}{UC7.12} & \hyperlink{RFO5.12}{RFO5.12}\\\hline
\hyperlink{UC7.13}{UC7.13} & \hyperlink{RFO5.13}{RFO5.13}\\\hline

\hyperlink{UC7.5.1}{UC7.5.1} & \hyperlink{RFO5.5.1}{RFO5.5.1}\\\hline
\hyperlink{UC7.5.2}{UC7.5.2} & \hyperlink{RFO5.5.2}{RFO5.5.2}\\\hline

\hyperlink{UC7.6.1}{UC7.6.1} & \hyperlink{RFO5.6.1}{RFO5.6.1}\\\hline
\hyperlink{UC7.6.2}{UC7.6.2} & \hyperlink{RFD5.6.2}{RFD5.6.2}\\\hline

\hyperlink{UC7.7.1}{UC7.7.1} & \hyperlink{RFD5.7.1}{RFD5.7.1}\\\hline
\hyperlink{UC7.7.2}{UC7.7.2} & \hyperlink{RFO5.7.2}{RFO5.7.2}\\\hline

\hyperlink{UC8}{UC8} & \hyperlink{RFO6}{RFO6}\\
& \hyperlink{RFO6.1}{RFO6.1}\\
& \hyperlink{RFO6.2}{RFO6.2}\\
& \hyperlink{RFO6.2.1}{RFO6.2.1}\\
& \hyperlink{RFO6.2.2}{RFO6.2.2}\\
& \hyperlink{RFO6.2.3}{RFO6.2.3}\\
& \hyperlink{RFO6.2.4}{RFO6.2.4}\\
& \hyperlink{RFO6.2.5}{RFO6.2.5}\\
& \hyperlink{RFD6.2.6}{RFD6.2.6}\\
& \hyperlink{RFD6.2.7}{RFD6.2.7}\\\hline

\hyperlink{UC8.1}{UC8.1} & \hyperlink{RFO6.1}{RFO6.1}\\\hline

\hyperlink{UC8.2}{UC8.2} & \hyperlink{RFO6.2}{RFO6.2}\\
& \hyperlink{RFO6.2.1}{RFO6.2.1}\\
& \hyperlink{RFO6.2.2}{RFO6.2.2}\\
& \hyperlink{RFO6.2.3}{RFO6.2.3}\\
& \hyperlink{RFO6.2.4}{RFO6.2.4}\\
& \hyperlink{RFO6.2.5}{RFO6.2.5}\\
& \hyperlink{RFD6.2.6}{RFD6.2.6}\\
& \hyperlink{RFD6.2.7}{RFD6.2.7}\\\hline

\hyperlink{UC8.2.1}{UC8.2.1} & \hyperlink{RFO6.2.1}{RFO6.2.1}\\\hline
\hyperlink{UC8.2.2}{UC8.2.2} & \hyperlink{RFO6.2.2}{RFO6.2.2}\\\hline
\hyperlink{UC8.2.3}{UC8.2.3} & \hyperlink{RFO6.2.3}{RFO6.2.3}\\\hline
\hyperlink{UC8.2.4}{UC8.2.4} & \hyperlink{RFO6.2.4}{RFO6.2.4}\\\hline
\hyperlink{UC8.2.5}{UC8.2.5} & \hyperlink{RFO6.2.5}{RFO6.2.5}\\\hline
\hyperlink{UC8.2.6}{UC8.2.6} & \hyperlink{RFD6.2.6}{RFD6.2.6}\\\hline
\hyperlink{UC8.2.7}{UC8.2.7} & \hyperlink{RFD6.2.7}{RFD6.2.7}\\\hline

\hyperlink{UC9}{UC9} & \hyperlink{RFO7}{RFO7}\\
& \hyperlink{RFO7.1}{RFO7.1}\\
& \hyperlink{RFO7.1.1}{RFO7.1.1}\\
& \hyperlink{RFO7.1.2}{RFO7.1.2}\\
& \hyperlink{RFO7.1.3}{RFO7.1.3}\\
& \hyperlink{RFO7.2}{RFO7.2}\\
& \hyperlink{RFO7.3}{RFO7.3}\\
& \hyperlink{RFO7.4}{RFO7.4}\\
& \hyperlink{RFO7.5}{RFO7.5}\\
& \hyperlink{RFO7.6}{RFO7.6}\\\hline

\hyperlink{UC9.1}{UC9.1} 
& \hyperlink{RFO7.1}{RFO7.1}\\
& \hyperlink{RFO7.1.1}{RFO7.1.1}\\
& \hyperlink{RFO7.1.2}{RFO7.1.2}\\
& \hyperlink{RFO7.1.3}{RFO7.1.3}\\\hline

\hyperlink{UC9.2}{UC9.2} & \hyperlink{RFO7.2}{RFO7.2}\\\hline
\hyperlink{UC9.3}{UC9.3} & \hyperlink{RFO7.3}{RFO7.3}\\\hline
\hyperlink{UC9.4}{UC9.4} & \hyperlink{RFO7.4}{RFO7.4}\\\hline
\hyperlink{UC9.5}{UC9.5} 
& \hyperlink{RFO7.5}{RFO7.5}\\
& \hyperlink{RFO7.5.1}{RFO7.5.1}\\
& \hyperlink{RFO7.5.2}{RFO7.5.2}\\
& \hyperlink{RFO7.5.3}{RFO7.5.3}\\\hline
\hyperlink{UC9.6}{UC9.6} & \hyperlink{RFO7.6}{RFO7.6}\\\hline

\hyperlink{UC9.1.1}{UC9.1.1} & \hyperlink{RFO7.1.1}{RFO7.1.1}\\\hline
\hyperlink{UC9.1.2}{UC9.1.2} & \hyperlink{RFO7.1.2}{RFO7.1.2}\\\hline
\hyperlink{UC9.1.3}{UC9.1.3} & \hyperlink{RFO7.1.3}{RFO7.1.3}\\\hline

\hyperlink{UC9.5.1}{UC9.5.1} & \hyperlink{RFO7.5.1}{RFO7.5.1}\\\hline
\hyperlink{UC9.5.2}{UC9.5.2} & \hyperlink{RFO7.5.2}{RFO7.5.2}\\\hline
\hyperlink{UC9.5.3}{UC9.5.3} & \hyperlink{RFO7.5.3}{RFO7.5.3}\\\hline

\hyperlink{UC10}{UC10} 
& \hyperlink{RFO8}{RFO8}\\
& \hyperlink{RFO8.1}{RFO8.1}\\
& \hyperlink{RFO8.2}{RFO8.2}\\
& \hyperlink{RFO8.2.1}{RFO8.2.1}\\
& \hyperlink{RFO8.2.2}{RFO8.2.2}\\
& \hyperlink{RFO8.2.3}{RFO8.2.3}\\
& \hyperlink{RFO8.2.4}{RFO8.2.4}\\
& \hyperlink{RFO8.2.4.1}{RFO8.2.4.1}\\
& \hyperlink{RFO8.2.4.2}{RFO8.2.4.2}\\
& \hyperlink{RFD8.2.4.3}{RFD8.2.4.3}\\
& \hyperlink{RFO8.2.4.4}{RFO8.2.4.4}\\
& \hyperlink{RFO8.2.4.5}{RFO8.2.4.5}\\
& \hyperlink{RFD8.2.4.6}{RFD8.2.4.6}\\
& \hyperlink{RFO8.2.4.7}{RFO8.2.4.7}\\
& \hyperlink{RFO8.2.4.8}{RFO8.2.4.8}\\
& \hyperlink{RFO8.2.4.9}{RFO8.2.4.9}\\
& \hyperlink{RFO8.2.4.10}{RFO8.2.4.10}\\
& \hyperlink{RFD8.2.5}{RFD8.2.5}\\
& \hyperlink{RFD8.2.6}{RFD8.2.6}\\
& \hyperlink{RFD8.2.7}{RFD8.2.7}\\
& \hyperlink{RFO8.2.8}{RFO8.2.8}\\
& \hyperlink{RFO8.2.9}{RFO8.2.9}\\
& \hyperlink{RFO8.2.9.1}{RFO8.2.9.1}\\
& \hyperlink{RFO8.2.9.2}{RFO8.2.9.2}\\\hline

\hyperlink{UC10.1}{UC10.1} & \hyperlink{RFO8.1}{RFO8.1}\\\hline

\hyperlink{UC10.2}{UC10.2} 
& \hyperlink{RFO8.2}{RFO8.2}\\
& \hyperlink{RFO8.2.1}{RFO8.2.1}\\
& \hyperlink{RFO8.2.2}{RFO8.2.2}\\
& \hyperlink{RFO8.2.3}{RFO8.2.3}\\
& \hyperlink{RFO8.2.4}{RFO8.2.4}\\
& \hyperlink{RFO8.2.4.1}{RFO8.2.4.1}\\
& \hyperlink{RFO8.2.4.2}{RFO8.2.4.2}\\
& \hyperlink{RFD8.2.4.3}{RFD8.2.4.3}\\
& \hyperlink{RFO8.2.4.4}{RFO8.2.4.4}\\
& \hyperlink{RFO8.2.4.5}{RFO8.2.4.5}\\
& \hyperlink{RFD8.2.4.6}{RFD8.2.4.6}\\
& \hyperlink{RFO8.2.4.7}{RFO8.2.4.7}\\
& \hyperlink{RFO8.2.4.8}{RFO8.2.4.8}\\
& \hyperlink{RFO8.2.4.9}{RFO8.2.4.9}\\
& \hyperlink{RFO8.2.4.10}{RFO8.2.4.10}\\
& \hyperlink{RFD8.2.5}{RFD8.2.5}\\
& \hyperlink{RFD8.2.6}{RFD8.2.6}\\
& \hyperlink{RFD8.2.7}{RFD8.2.7}\\
& \hyperlink{RFO8.2.8}{RFO8.2.8}\\
& \hyperlink{RFO8.2.9}{RFO8.2.9}\\
& \hyperlink{RFO8.2.9.1}{RFO8.2.9.1}\\
& \hyperlink{RFO8.2.9.2}{RFO8.2.9.2}\\\hline

\hyperlink{UC10.2.1}{UC10.2.1} & \hyperlink{RFO8.2.1}{RFO8.2.1}\\\hline
\hyperlink{UC10.2.2}{UC10.2.2} & \hyperlink{RFO8.2.2}{RFO8.2.2}\\\hline
\hyperlink{UC10.2.3}{UC10.2.3} & \hyperlink{RFO8.2.3}{RFO8.2.3}\\\hline
\hyperlink{UC10.2.4}{UC10.2.4} & \hyperlink{RFO8.2.4}{RFO8.2.4}\\\hline

\hyperlink{UC10.2.4}{UC10.2.4} 
& \hyperlink{RFO8.2.4}{RFO8.2.4}\\
& \hyperlink{RFO8.2.4.1}{RFO8.2.4.1}\\
& \hyperlink{RFO8.2.4.2}{RFO8.2.4.2}\\
& \hyperlink{RFD8.2.4.3}{RFD8.2.4.3}\\
& \hyperlink{RFO8.2.4.4}{RFO8.2.4.4}\\
& \hyperlink{RFO8.2.4.5}{RFO8.2.4.5}\\
& \hyperlink{RFD8.2.4.6}{RFD8.2.4.6}\\
& \hyperlink{RFO8.2.4.7}{RFO8.2.4.7}\\
& \hyperlink{RFO8.2.4.8}{RFO8.2.4.8}\\
& \hyperlink{RFO8.2.4.9}{RFO8.2.4.9}\\
& \hyperlink{RFO8.2.4.10}{RFO8.2.4.10}\\\hline

\hyperlink{UC10.2.5}{UC10.2.5} & \hyperlink{RFD8.2.5}{RFD8.2.5}\\\hline
\hyperlink{UC10.2.6}{UC10.2.6} & \hyperlink{RFD8.2.6}{RFD8.2.6}\\\hline
\hyperlink{UC10.2.7}{UC10.2.7} & \hyperlink{RFD8.2.7}{RFD8.2.7}\\\hline
\hyperlink{UC10.2.8}{UC10.2.8} & \hyperlink{RFO8.2.8}{RFO8.2.8}\\\hline
\hyperlink{UC10.2.9}{UC10.2.9} & \hyperlink{RFO8.2.9}{RFO8.2.9}\\\hline

\hyperlink{UC10.2.4.1}{UC10.2.4.1} & \hyperlink{RFO8.2.4.1}{RFO8.2.4.1}\\\hline
\hyperlink{UC10.2.4.2}{UC10.2.4.2} & \hyperlink{RFO8.2.4.2}{RFO8.2.4.2}\\\hline
\hyperlink{UC10.2.4.3}{UC10.2.4.3} & \hyperlink{RFO8.2.4.3}{RFO8.2.4.3}\\\hline
\hyperlink{UC10.2.4.4}{UC10.2.4.4} & \hyperlink{RFO8.2.4.4}{RFO8.2.4.4}\\\hline
\hyperlink{UC10.2.4.5}{UC10.2.4.5} & \hyperlink{RFO8.2.4.5}{RFO8.2.4.5}\\\hline
\hyperlink{UC10.2.4.6}{UC10.2.4.6} & \hyperlink{RFO8.2.4.6}{RFO8.2.4.6}\\\hline
\hyperlink{UC10.2.4.7}{UC10.2.4.7} & \hyperlink{RFO8.2.4.7}{RFO8.2.4.7}\\\hline
\hyperlink{UC10.2.4.8}{UC10.2.4.8} & \hyperlink{RFO8.2.4.8}{RFO8.2.4.8}\\\hline
\hyperlink{UC10.2.4.9}{UC10.2.4.9} & \hyperlink{RFO8.2.4.9}{RFO8.2.4.9}\\\hline
\hyperlink{UC10.2.4.10}{UC10.2.4.10} & \hyperlink{RFO8.2.4.10}{RFO8.2.4.10}\\\hline

\hyperlink{UC10.2.9}{UC10.2.9}
& \hyperlink{RFO8.2.9}{RFO8.2.9}\\
& \hyperlink{RFO8.2.9.1}{RFO8.2.9.1}\\
& \hyperlink{RFO8.2.9.2}{RFO8.2.9.2}\\\hline

\hyperlink{UC10.2.9.1}{UC10.2.9.1} & \hyperlink{RFO8.2.9.1}{RFO8.2.9.1}\\\hline
\hyperlink{UC10.2.9.2}{UC10.2.9.2} & \hyperlink{RFO8.2.9.2}{RFO8.2.9.2}\\\hline

\hyperlink{UC11}{UC11} 
& \hyperlink{RFO9}{RFO9}\\
& \hyperlink{RFO9.1}{RFO9.1}\\
& \hyperlink{RFO9.2}{RFO9.2}\\
& \hyperlink{RFD9.3}{RFD9.3}\\
& \hyperlink{RFO9.4}{RFO9.4}\\
& \hyperlink{RFO9.5}{RFO9.5}\\
& \hyperlink{RFD9.6}{RFD9.6}\\
& \hyperlink{RFD9.7}{RFD9.7}\\
& \hyperlink{RFO9.8}{RFO9.8}\\
& \hyperlink{RFO9.8.1}{RFO9.8.1}\\
& \hyperlink{RFO9.8.2}{RFO9.8.2}\\
& \hyperlink{RFO9.8.3}{RFO9.8.3}\\
& \hyperlink{RFO9.9}{RFO9.9}\\
& \hyperlink{RFD9.10}{RFD9.10}\\
& \hyperlink{RFD9.11}{RFD9.11}\\
& \hyperlink{RFO9.12}{RFO9.12}\\
& \hyperlink{RFD9.13}{RFD9.13}\\\hline

\hyperlink{UC11.1}{UC11.1} & \hyperlink{RFO9.1}{RFO9.1}\\\hline
\hyperlink{UC11.2}{UC11.2} & \hyperlink{RFO9.2}{RFO9.2}\\\hline
\hyperlink{UC11.3}{UC11.3} & \hyperlink{RFD9.3}{RFD9.3}\\\hline
\hyperlink{UC11.4}{UC11.4} & \hyperlink{RFO9.4}{RFO9.4}\\\hline
\hyperlink{UC11.5}{UC11.5} & \hyperlink{RFO9.5}{RFO9.5}\\\hline
\hyperlink{UC11.6}{UC11.6} & \hyperlink{RFD9.6}{RFD9.6}\\\hline
\hyperlink{UC11.7}{UC11.7} & \hyperlink{RFD9.7}{RFD9.7}\\\hline
\hyperlink{UC11.8}{UC11.8} & \hyperlink{RFO9.8}{RFO9.8}\\\hline
\hyperlink{UC11.9}{UC11.9} & \hyperlink{RFD9.9}{RFD9.9}\\\hline
\hyperlink{UC11.10}{UC11.10} & \hyperlink{RFD9.10}{RFD9.10}\\\hline
\hyperlink{UC11.11}{UC11.11} & \hyperlink{RFD9.11}{RFD9.11}\\\hline
\hyperlink{UC11.12}{UC11.12} & \hyperlink{RFO9.12}{RFO9.12}\\\hline
\hyperlink{UC11.13}{UC11.13} & \hyperlink{RFD9.13}{RFD9.13}\\\hline

\hyperlink{UC11.8.1}{UC11.8.1} & \hyperlink{RFO9.8.1}{RFO9.8.1}\\\hline
\hyperlink{UC11.8.2}{UC11.8.2} & \hyperlink{RFO9.8.2}{RFO9.8.2}\\\hline
\hyperlink{UC11.8.3}{UC11.8.3} & \hyperlink{RFO9.8.3}{RFO9.8.3}\\\hline

\hyperlink{UC12}{UC12}
& \hyperlink{RFO10}{RFO10}\\
& \hyperlink{RFO10.1}{RFO10.1}\\
& \hyperlink{RFO10.1.1}{RFO10.1.1}\\
& \hyperlink{RFO10.1.1.1}{RFO10.1.1.1}\\
& \hyperlink{RFO10.1.1.2}{RFO10.1.1.2}\\
& \hyperlink{RFO10.1.1.3}{RFO10.1.1.3}\\
& \hyperlink{RFO10.1.1.4}{RFO10.1.1.4}\\
& \hyperlink{RFD10.1.1.5}{RFD10.1.1.5}\\
& \hyperlink{RFD10.1.1.6}{RFD10.1.1.6}\\
& \hyperlink{RFO10.1.2}{RFO10.1.2}\\
& \hyperlink{RFO10.1.2.1}{RFO10.1.2.1}\\
& \hyperlink{RFO10.1.2.2}{RFO10.1.2.2}\\
& \hyperlink{RFO10.1.2.3}{RFO10.1.2.3}\\
& \hyperlink{RFO10.1.2.4}{RFO10.1.2.4}\\
& \hyperlink{RFO10.1.2.5}{RFO10.1.2.5}\\
& \hyperlink{RFD10.1.2.6}{RFD10.1.2.6}\\
& \hyperlink{RFD10.1.2.7}{RFD10.1.2.7}\\
& \hyperlink{RFO10.1.2.8}{RFO10.1.2.8}\\
& \hyperlink{RFO10.1.2.9}{RFO10.1.2.9}\\
& \hyperlink{RFO10.2}{RFO10.2}\\
& \hyperlink{RFO10.2.1}{RFO10.2.1}\\
& \hyperlink{RFO10.2.2}{RFO10.2.2}\\
& \hyperlink{RFO10.2.2.1}{RFO10.2.2.1}\\
& \hyperlink{RFO10.2.2.2}{RFO10.2.2.2}\\
& \hyperlink{RFO10.3}{RFO10.3}\\
& \hyperlink{RFO10.3.1}{RFO10.3.1}\\
& \hyperlink{RFO10.3.1.1}{RFO10.3.1.1}\\
& \hyperlink{RFO10.3.1.2}{RFO10.3.1.2}\\
& \hyperlink{RFO10.3.2}{RFO10.3.2}\\
& \hyperlink{RFO10.3.2.1}{RFO10.3.2.1}\\
& \hyperlink{RFO10.3.2.2}{RFO10.3.2.2}\\
& \hyperlink{RFO10.3.2.3}{RFO10.3.2.3}\\
& \hyperlink{RFO10.3.2.4}{RFO10.3.2.4}\\
& \hyperlink{RFO10.3.2.5}{RFO10.3.2.5}\\
& \hyperlink{RFO10.3.3}{RFO10.3.3}\\
& \hyperlink{RFO10.3.3.1}{RFO10.3.3.1}\\
& \hyperlink{RFO10.3.3.2}{RFO10.3.3.2}\\
& \hyperlink{RFO10.3.3.3}{RFO10.3.3.3}\\
& \hyperlink{RFO10.3.3.4}{RFO10.3.3.4}\\\hline

\hyperlink{UC12.1}{UC12.1} 
& \hyperlink{RFO10.1}{RFO10.1}\\
& \hyperlink{RFO10.1.1}{RFO10.1.1}\\
& \hyperlink{RFO10.1.1.1}{RFO10.1.1.1}\\
& \hyperlink{RFO10.1.1.2}{RFO10.1.1.2}\\
& \hyperlink{RFO10.1.1.3}{RFO10.1.1.3}\\
& \hyperlink{RFO10.1.1.4}{RFO10.1.1.4}\\
& \hyperlink{RFD10.1.1.5}{RFD10.1.1.5}\\
& \hyperlink{RFD10.1.1.6}{RFD10.1.1.6}\\
& \hyperlink{RFO10.1.2}{RFO10.1.2}\\
& \hyperlink{RFO10.1.2.1}{RFO10.1.2.1}\\
& \hyperlink{RFO10.1.2.2}{RFO10.1.2.2}\\
& \hyperlink{RFO10.1.2.3}{RFO10.1.2.3}\\
& \hyperlink{RFO10.1.2.4}{RFO10.1.2.4}\\
& \hyperlink{RFO10.1.2.5}{RFO10.1.2.5}\\
& \hyperlink{RFD10.1.2.6}{RFD10.1.2.6}\\
& \hyperlink{RFD10.1.2.7}{RFD10.1.2.7}\\
& \hyperlink{RFO10.1.2.8}{RFO10.1.2.8}\\
& \hyperlink{RFO10.1.2.9}{RFO10.1.2.9}\\\hline

\hyperlink{UC12.1.1}{UC12.1.1} 
& \hyperlink{RFO10.1.1}{RFO10.1.1}\\
& \hyperlink{RFO10.1.1.1}{RFO10.1.1.1}\\
& \hyperlink{RFO10.1.1.2}{RFO10.1.1.2}\\
& \hyperlink{RFO10.1.1.3}{RFO10.1.1.3}\\
& \hyperlink{RFO10.1.1.4}{RFO10.1.1.4}\\
& \hyperlink{RFD10.1.1.5}{RFD10.1.1.5}\\
& \hyperlink{RFD10.1.1.6}{RFD10.1.1.6}\\\hline

\hyperlink{UC12.1.1.1}{UC12.1.1.1} & \hyperlink{RFO10.1.1.1}{RFO10.1.1.1}\\\hline
\hyperlink{UC12.1.1.2}{UC12.1.1.2} & \hyperlink{RFO10.1.1.2}{RFO10.1.1.2}\\\hline
\hyperlink{UC12.1.1.3}{UC12.1.1.3} & \hyperlink{RFO10.1.1.3}{RFO10.1.1.3}\\\hline
\hyperlink{UC12.1.1.4}{UC12.1.1.4} & \hyperlink{RFO10.1.1.4}{RFO10.1.1.4}\\\hline
\hyperlink{UC12.1.1.5}{UC12.1.1.5} & \hyperlink{RFO10.1.1.5}{RFO10.1.1.5}\\\hline
\hyperlink{UC12.1.1.6}{UC12.1.1.6} & \hyperlink{RFO10.1.1.6}{RFO10.1.1.6}\\\hline

\hyperlink{UC12.1.2}{UC12.1.2}
& \hyperlink{RFO10.1.2}{RFO10.1.2}\\
& \hyperlink{RFO10.1.2.1}{RFO10.1.2.1}\\
& \hyperlink{RFO10.1.2.2}{RFO10.1.2.2}\\
& \hyperlink{RFO10.1.2.3}{RFO10.1.2.3}\\
& \hyperlink{RFO10.1.2.4}{RFO10.1.2.4}\\
& \hyperlink{RFO10.1.2.5}{RFO10.1.2.5}\\
& \hyperlink{RFD10.1.2.6}{RFD10.1.2.6}\\
& \hyperlink{RFD10.1.2.7}{RFD10.1.2.7}\\
& \hyperlink{RFO10.1.2.8}{RFO10.1.2.8}\\
& \hyperlink{RFO10.1.2.9}{RFO10.1.2.9}\\\hline

\hyperlink{UC12.1.2.1}{UC12.1.2.1} & \hyperlink{RFO10.1.2.1}{RFO10.1.2.1}\\\hline
\hyperlink{UC12.1.2.2}{UC12.1.2.2} & \hyperlink{RFO10.1.2.2}{RFO10.1.2.2}\\\hline
\hyperlink{UC12.1.2.3}{UC12.1.2.3} & \hyperlink{RFO10.1.2.3}{RFO10.1.2.3}\\\hline
\hyperlink{UC12.1.2.4}{UC12.1.2.4} & \hyperlink{RFO10.1.2.4}{RFO10.1.2.4}\\\hline
\hyperlink{UC12.1.2.5}{UC12.1.2.5} & \hyperlink{RFO10.1.2.5}{RFO10.1.2.5}\\\hline
\hyperlink{UC12.1.2.6}{UC12.1.2.6} & \hyperlink{RFO10.1.2.6}{RFO10.1.2.6}\\\hline
\hyperlink{UC12.1.2.7}{UC12.1.2.7} & \hyperlink{RFO10.1.2.7}{RFO10.1.2.7}\\\hline
\hyperlink{UC12.1.2.8}{UC12.1.2.8} & \hyperlink{RFO10.1.2.8}{RFO10.1.2.8}\\\hline
\hyperlink{UC12.1.2.9}{UC12.1.2.9} & \hyperlink{RFO10.1.2.8}{RFO10.1.2.9}\\\hline

\hyperlink{UC12.2}{UC12.2} 
& \hyperlink{RFO10.2}{RFO10.2}\\
& \hyperlink{RFO10.2.1}{RFO10.2.1}\\
& \hyperlink{RFO10.2.2}{RFO10.2.2}\\
& \hyperlink{RFO10.2.2.1}{RFO10.2.2.1}\\
& \hyperlink{RFO10.2.2.2}{RFO10.2.2.2}\\\hline

\hyperlink{UC12.2.1}{UC12.2.1} & \hyperlink{RFO10.2.1}{RFO10.2.1}\\\hline
\hyperlink{UC12.2.2}{UC12.2.2} 
& \hyperlink{RFO10.2.2}{RFO10.2.2}\\
& \hyperlink{RFO10.2.2.1}{RFO10.2.2.1}\\
& \hyperlink{RFO10.2.2.2}{RFO10.2.2.2}\\\hline

\hyperlink{UC12.2.2.1}{UC12.2.2.1} & \hyperlink{RFO10.2.2.1}{RFO10.2.2.1}\\\hline
\hyperlink{UC12.2.2.2}{UC12.2.2.2} & \hyperlink{RFO10.2.2.2}{RFO10.2.2.2}\\\hline

\hyperlink{UC12.3}{UC12.3}
& \hyperlink{RFO10.3}{RFO10.3}\\
& \hyperlink{RFO10.3.1}{RFO10.3.1}\\
& \hyperlink{RFO10.3.1.1}{RFO10.3.1.1}\\
& \hyperlink{RFO10.3.1.2}{RFO10.3.1.2}\\
& \hyperlink{RFO10.3.2}{RFO10.3.2}\\
& \hyperlink{RFO10.3.2.1}{RFO10.3.2.1}\\
& \hyperlink{RFO10.3.2.2}{RFO10.3.2.2}\\
& \hyperlink{RFO10.3.2.3}{RFO10.3.2.3}\\
& \hyperlink{RFO10.3.2.4}{RFO10.3.2.4}\\
& \hyperlink{RFO10.3.2.5}{RFO10.3.2.5}\\
& \hyperlink{RFO10.3.3}{RFO10.3.3}\\
& \hyperlink{RFO10.3.3.1}{RFO10.3.3.1}\\
& \hyperlink{RFO10.3.3.2}{RFO10.3.3.2}\\
& \hyperlink{RFO10.3.3.3}{RFO10.3.3.3}\\
& \hyperlink{RFO10.3.3.4}{RFO10.3.3.4}\\\hline

\hyperlink{UC12.3.1}{UC12.3.1}
& \hyperlink{RFO10.3.1}{RFO10.3.1}\\
& \hyperlink{RFO10.3.1.1}{RFO10.3.1.1}\\
& \hyperlink{RFO10.3.1.2}{RFO10.3.1.2}\\\hline

\hyperlink{UC12.3.1.1}{UC12.3.1.1} & \hyperlink{RFO10.3.1.1}{RFO10.3.1.1}\\\hline
\hyperlink{UC12.3.1.2}{UC12.3.1.2} & \hyperlink{RFO10.3.1.2}{RFO10.3.1.2}\\\hline

\hyperlink{UC12.3.2}{UC12.3.2}
& \hyperlink{RFO10.3.2}{RFO10.3.2}\\
& \hyperlink{RFO10.3.2.1}{RFO10.3.2.1}\\
& \hyperlink{RFO10.3.2.2}{RFO10.3.2.2}\\
& \hyperlink{RFO10.3.2.3}{RFO10.3.2.3}\\
& \hyperlink{RFO10.3.2.4}{RFO10.3.2.4}\\
& \hyperlink{RFO10.3.2.5}{RFO10.3.2.5}\\\hline

\hyperlink{UC12.3.2.1}{UC12.3.2.1} & \hyperlink{RFO10.3.2.1}{RFO10.3.2.1}\\\hline
\hyperlink{UC12.3.2.2}{UC12.3.2.2} & \hyperlink{RFO10.3.2.2}{RFO10.3.2.2}\\\hline
\hyperlink{UC12.3.2.3}{UC12.3.2.3} & \hyperlink{RFO10.3.2.3}{RFO10.3.2.3}\\\hline
\hyperlink{UC12.3.2.4}{UC12.3.2.4} & \hyperlink{RFO10.3.2.4}{RFO10.3.2.4}\\\hline
\hyperlink{UC12.3.2.5}{UC12.3.2.5} & \hyperlink{RFO10.3.2.5}{RFO10.3.2.5}\\\hline

\hyperlink{UC12.3.3}{UC12.3.3}
& \hyperlink{RFO10.3.3}{RFO10.3.3}\\
& \hyperlink{RFO10.3.3.1}{RFO10.3.3.1}\\
& \hyperlink{RFO10.3.3.2}{RFO10.3.3.2}\\
& \hyperlink{RFO10.3.3.3}{RFO10.3.3.3}\\
& \hyperlink{RFO10.3.3.4}{RFO10.3.3.4}\\\hline

\hyperlink{UC12.3.3.1}{UC12.3.3.1} & \hyperlink{RFO10.3.3.1}{RFO10.3.3.1}\\\hline
\hyperlink{UC12.3.3.2}{UC12.3.3.2} & \hyperlink{RFO10.3.3.2}{RFO10.3.3.2}\\\hline
\hyperlink{UC12.3.3.3}{UC12.3.3.3} & \hyperlink{RFO10.3.3.3}{RFO10.3.3.3}\\\hline
\hyperlink{UC12.3.3.4}{UC12.3.3.4} & \hyperlink{RFO10.3.3.4}{RFO10.3.3.4}\\\hline

\hyperlink{UC13}{UC13} & \hyperlink{RFO11}{RFO11}\\\hline

\hyperlink{UC14}{UC14} 
& \hyperlink{RFO12}{RFO12}\\
& \hyperlink{RFO12.1}{RFO12.1}\\
& \hyperlink{RFO12.1.1}{RFO12.1.1}\\
& \hyperlink{RFO12.1.1.1}{RFO12.1.1.1}\\
& \hyperlink{RFO12.1.1.1.1}{RFO12.1.1.1.1}\\
& \hyperlink{RFO12.1.1.1.2}{RFO12.1.1.1.2}\\
& \hyperlink{RFD12.1.1.1.3}{RFD12.1.1.1.3}\\
& \hyperlink{RFD12.1.1.1.4}{RFD12.1.1.1.4}\\
& \hyperlink{RFO12.1.1.1.5}{RFO12.1.1.1.5}\\
& \hyperlink{RFO12.1.1.1.5.1}{RFO12.1.1.1.5.1}\\
& \hyperlink{RFF12.1.1.1.5.2}{RFF12.1.1.1.5.2}\\
& \hyperlink{RFO12.1.1.1.6}{RFO12.1.1.1.6}\\
& \hyperlink{RFO12.1.1.2}{RFO12.1.1.2}\\
& \hyperlink{RFO12.1.1.2.1}{RFO12.1.1.2.1}\\
& \hyperlink{RFO12.1.1.2.2}{RFO12.1.1.2.2}\\
& \hyperlink{RFD12.1.1.3}{RFD12.1.1.3}\\
& \hyperlink{RFD12.1.1.3.1}{RFD12.1.1.3.1}\\
& \hyperlink{RFD12.1.1.3.2}{RFD12.1.1.3.2}\\
& \hyperlink{RFD12.1.1.3.3}{RFD12.1.1.3.3}\\
& \hyperlink{RFO12.2}{RFO12.2}\\
& \hyperlink{RFO12.2.1}{RFO12.2.1}\\
& \hyperlink{RFO12.2.1.1}{RFO12.2.1.1}\\
& \hyperlink{RFO12.2.1.1.1}{RFO12.2.1.1.1}\\
& \hyperlink{RFO12.2.1.1.2}{RFO12.2.1.1.2}\\
& \hyperlink{RFO12.2.1.2}{RFO12.2.1.2}\\
& \hyperlink{RFO12.2.1.2.1}{RFO12.2.1.2.1}\\
& \hyperlink{RFO12.2.1.2.2}{RFO12.2.1.2.2}\\
& \hyperlink{RFO12.2.1.3}{RFO12.2.1.3}\\
& \hyperlink{RFO12.2.1.4}{RFO12.2.1.4}\\
& \hyperlink{RFO12.2.1.5}{RFO12.2.1.5}\\
& \hyperlink{RFO12.2.1.5.1}{RFO12.2.1.5.1}\\
& \hyperlink{RFO12.2.1.5.2}{RFO12.2.1.5.2}\\\hline

\hyperlink{UC14.1}{UC14.1}
& \hyperlink{RFO12.1}{RFO12.1}\\
& \hyperlink{RFO12.1.1}{RFO12.1.1}\\
& \hyperlink{RFO12.1.1.1}{RFO12.1.1.1}\\
& \hyperlink{RFO12.1.1.1.1}{RFO12.1.1.1.1}\\
& \hyperlink{RFO12.1.1.1.2}{RFO12.1.1.1.2}\\
& \hyperlink{RFD12.1.1.1.3}{RFD12.1.1.1.3}\\
& \hyperlink{RFD12.1.1.1.4}{RFD12.1.1.1.4}\\
& \hyperlink{RFO12.1.1.1.5}{RFO12.1.1.1.5}\\
& \hyperlink{RFO12.1.1.1.5.1}{RFO12.1.1.1.5.1}\\
& \hyperlink{RFF12.1.1.1.5.2}{RFF12.1.1.1.5.2}\\
& \hyperlink{RFO12.1.1.1.6}{RFO12.1.1.1.6}\\
& \hyperlink{RFO12.1.1.2}{RFO12.1.1.2}\\
& \hyperlink{RFO12.1.1.2.1}{RFO12.1.1.2.1}\\
& \hyperlink{RFO12.1.1.2.2}{RFO12.1.1.2.2}\\
& \hyperlink{RFD12.1.1.3}{RFD12.1.1.3}\\
& \hyperlink{RFD12.1.1.3.1}{RFD12.1.1.3.1}\\
& \hyperlink{RFD12.1.1.3.2}{RFD12.1.1.3.2}\\
& \hyperlink{RFD12.1.1.3.3}{RFD12.1.1.3.3}\\
& \hyperlink{RFO12.2}{RFO12.2}\\
& \hyperlink{RFO12.2.1}{RFO12.2.1}\\
& \hyperlink{RFO12.2.1.1}{RFO12.2.1.1}\\
& \hyperlink{RFO12.2.1.1.1}{RFO12.2.1.1.1}\\
& \hyperlink{RFO12.2.1.1.2}{RFO12.2.1.1.2}\\
& \hyperlink{RFO12.2.1.2}{RFO12.2.1.2}\\
& \hyperlink{RFO12.2.1.2.1}{RFO12.2.1.2.1}\\
& \hyperlink{RFO12.2.1.2.2}{RFO12.2.1.2.2}\\
& \hyperlink{RFO12.2.1.3}{RFO12.2.1.3}\\
& \hyperlink{RFO12.2.1.4}{RFO12.2.1.4}\\
& \hyperlink{RFO12.2.1.5}{RFO12.2.1.5}\\
& \hyperlink{RFO12.2.1.5.1}{RFO12.2.1.5.1}\\
& \hyperlink{RFO12.2.1.5.2}{RFO12.2.1.5.2}\\\hline

\hyperlink{UC14.1.1}{UC14.1.1} 
& \hyperlink{RFO12.1.1}{RFO12.1.1}\\
& \hyperlink{RFO12.1.1.1}{RFO12.1.1.1}\\
& \hyperlink{RFO12.1.1.1.1}{RFO12.1.1.1.1}\\
& \hyperlink{RFO12.1.1.1.2}{RFO12.1.1.1.2}\\
& \hyperlink{RFD12.1.1.1.3}{RFD12.1.1.1.3}\\
& \hyperlink{RFD12.1.1.1.4}{RFD12.1.1.1.4}\\
& \hyperlink{RFO12.1.1.1.5}{RFO12.1.1.1.5}\\
& \hyperlink{RFO12.1.1.1.5.1}{RFO12.1.1.1.5.1}\\
& \hyperlink{RFF12.1.1.1.5.2}{RFF12.1.1.1.5.2}\\
& \hyperlink{RFO12.1.1.1.6}{RFO12.1.1.1.6}\\
& \hyperlink{RFO12.1.1.2}{RFO12.1.1.2}\\
& \hyperlink{RFO12.1.1.2.1}{RFO12.1.1.2.1}\\
& \hyperlink{RFO12.1.1.2.2}{RFO12.1.1.2.2}\\
& \hyperlink{RFD12.1.1.3}{RFD12.1.1.3}\\
& \hyperlink{RFD12.1.1.3.1}{RFD12.1.1.3.1}\\
& \hyperlink{RFD12.1.1.3.2}{RFD12.1.1.3.2}\\
& \hyperlink{RFD12.1.1.3.3}{RFD12.1.1.3.3}\\\hline

\hyperlink{UC14.1.1.1}{UC14.1.1.1} 
& \hyperlink{RFO12.1.1.1}{RFO12.1.1.1}\\
& \hyperlink{RFO12.1.1.1.1}{RFO12.1.1.1.1}\\
& \hyperlink{RFO12.1.1.1.2}{RFO12.1.1.1.2}\\
& \hyperlink{RFD12.1.1.1.3}{RFD12.1.1.1.3}\\
& \hyperlink{RFD12.1.1.1.4}{RFD12.1.1.1.4}\\
& \hyperlink{RFO12.1.1.1.5}{RFO12.1.1.1.5}\\
& \hyperlink{RFO12.1.1.1.5.1}{RFO12.1.1.1.5.1}\\
& \hyperlink{RFF12.1.1.1.5.2}{RFF12.1.1.1.5.2}\\
& \hyperlink{RFO12.1.1.1.6}{RFO12.1.1.1.6}\\\hline

\hyperlink{UC14.1.1.1.1}{UC14.1.1.1.1} & \hyperlink{RFO12.1.1.1.1}{RFO12.1.1.1.1}\\\hline
\hyperlink{UC14.1.1.1.2}{UC14.1.1.1.2} & \hyperlink{RFO12.1.1.1.2}{RFO12.1.1.1.2}\\\hline
\hyperlink{UC14.1.1.1.3}{UC14.1.1.1.3} & \hyperlink{RFO12.1.1.1.3}{RFO12.1.1.1.3}\\\hline
\hyperlink{UC14.1.1.1.4}{UC14.1.1.1.4} & \hyperlink{RFO12.1.1.1.4}{RFO12.1.1.1.4}\\\hline
\hyperlink{UC14.1.1.1.5}{UC14.1.1.1.5}
& \hyperlink{RFO12.1.1.1.5}{RFO12.1.1.1.5}\\
& \hyperlink{RFO12.1.1.1.5.1}{RFO12.1.1.1.5.1}\\
& \hyperlink{RFF12.1.1.1.5.2}{RFF12.1.1.1.5.2}\\\hline
\hyperlink{UC14.1.1.1.6}{UC14.1.1.1.6} & \hyperlink{RFO12.1.1.1.6}{RFO12.1.1.1.6}\\\hline

\hyperlink{UC14.1.1.1.5.1}{UC14.1.1.1.5.1} & \hyperlink{RFO12.1.1.1.5.1}{RFO12.1.1.1.5.1}\\\hline
\hyperlink{UC14.1.1.1.5.2}{UC14.1.1.1.5.2} & \hyperlink{RFF12.1.1.1.5.2}{RFF12.1.1.1.5.2}\\\hline

\hyperlink{UC14.1.1.2}{UC14.1.1.2}
& \hyperlink{RFO12.1.1.2}{RFO12.1.1.2}\\
& \hyperlink{RFO12.1.1.2.1}{RFO12.1.1.2.1}\\
& \hyperlink{RFO12.1.1.2.2}{RFO12.1.1.2.2}\\\hline

\hyperlink{UC14.1.1.3}{UC14.1.1.3}
& \hyperlink{RFO12.1.1.3}{RFO12.1.1.3}\\
& \hyperlink{RFO12.1.1.3.1}{RFO12.1.1.3.1}\\
& \hyperlink{RFO12.1.1.3.2}{RFO12.1.1.3.2}\\
& \hyperlink{RFO12.1.1.3.3}{RFO12.1.1.3.3}\\\hline

\hyperlink{UC14.2}{UC14.2} 
& \hyperlink{RFO12.2}{RFO12.2}\\
& \hyperlink{RFO12.2.1}{RFO12.2.1}\\
& \hyperlink{RFO12.2.1.1}{RFO12.2.1.1}\\
& \hyperlink{RFO12.2.1.1.1}{RFO12.2.1.1.1}\\
& \hyperlink{RFO12.2.1.1.2}{RFO12.2.1.1.2}\\
& \hyperlink{RFO12.2.1.2}{RFO12.2.1.2}\\
& \hyperlink{RFO12.2.1.2.1}{RFO12.2.1.2.1}\\
& \hyperlink{RFO12.2.1.2.2}{RFO12.2.1.2.2}\\
& \hyperlink{RFO12.2.1.3}{RFO12.2.1.3}\\
& \hyperlink{RFO12.2.1.4}{RFO12.2.1.4}\\
& \hyperlink{RFO12.2.1.5}{RFO12.2.1.5}\\
& \hyperlink{RFO12.2.1.5.1}{RFO12.2.1.5.1}\\
& \hyperlink{RFO12.2.1.5.2}{RFO12.2.1.5.2}\\\hline

\hyperlink{UC14.2.1}{UC14.2.1} 
& \hyperlink{RFO12.2.1}{RFO12.2.1}\\
& \hyperlink{RFO12.2.1.1}{RFO12.2.1.1}\\
& \hyperlink{RFO12.2.1.1.1}{RFO12.2.1.1.1}\\
& \hyperlink{RFO12.2.1.1.2}{RFO12.2.1.1.2}\\
& \hyperlink{RFO12.2.1.2}{RFO12.2.1.2}\\
& \hyperlink{RFO12.2.1.2.1}{RFO12.2.1.2.1}\\
& \hyperlink{RFO12.2.1.2.2}{RFO12.2.1.2.2}\\
& \hyperlink{RFO12.2.1.3}{RFO12.2.1.3}\\
& \hyperlink{RFO12.2.1.4}{RFO12.2.1.4}\\
& \hyperlink{RFO12.2.1.5}{RFO12.2.1.5}\\
& \hyperlink{RFO12.2.1.5.1}{RFO12.2.1.5.1}\\
& \hyperlink{RFO12.2.1.5.2}{RFO12.2.1.5.2}\\\hline

\hyperlink{UC14.2.1.1}{UC14.2.1.1} 
& \hyperlink{RFO12.2.1.1}{RFO12.2.1.1}\\
& \hyperlink{RFO12.2.1.1.1}{RFO12.2.1.1.1}\\
& \hyperlink{RFO12.2.1.1.2}{RFO12.2.1.1.2}\\\hline

\hyperlink{UC14.2.1.1.1}{UC14.2.1.1.1} & \hyperlink{RFO12.2.1.1.1}{RFO12.2.1.1.1}\\\hline
\hyperlink{UC14.2.1.1.2}{UC14.2.1.1.2} & \hyperlink{RFO12.2.1.1.2}{RFO12.2.1.1.2}\\\hline

\hyperlink{UC14.2.1.2}{UC14.2.1.2}
& \hyperlink{RFO12.2.1.2}{RFO12.2.1.2}\\
& \hyperlink{RFO12.2.1.2.1}{RFO12.2.1.2.1}\\
& \hyperlink{RFO12.2.1.2.2}{RFO12.2.1.2.2}\\\hline

\hyperlink{UC14.2.1.2.1}{UC14.2.1.2.1} & \hyperlink{RFO12.2.1.2.1}{RFO12.2.1.2.1}\\\hline
\hyperlink{UC14.2.1.2.2}{UC14.2.1.2.2} & \hyperlink{RFO12.2.1.2.2}{RFO12.2.1.2.2}\\\hline

\hyperlink{UC14.2.1.3}{UC14.2.1.3} & \hyperlink{RFO12.2.1.3}{RFO12.2.1.3}\\\hline
\hyperlink{UC14.2.1.4}{UC14.2.1.4} & \hyperlink{RFO12.2.1.4}{RFO12.2.1.4}\\\hline

\hyperlink{UC14.2.1.5}{UC14.2.1.5}
& \hyperlink{RFO12.2.1.5}{RFO12.2.1.5}\\
& \hyperlink{RFO12.2.1.5.1}{RFO12.2.1.5.1}\\
& \hyperlink{RFO12.2.1.5.2}{RFO12.2.1.5.2}\\\hline

\hyperlink{UC14.2.1.5.1}{UC14.2.1.5.1} & \hyperlink{RFO12.2.1.5.1}{RFO12.2.1.5.1}\\\hline
\hyperlink{UC14.2.1.5.2}{UC14.2.1.5.2} & \hyperlink{RFO12.2.1.5.2}{RFO12.2.1.5.2}\\\hline


\end{longtable}
\newpage
\subsection{Tabella Requisiti-Fonti}
\normalsize
\begin{longtable}{|>{\centering}m{5cm}|m{5cm}<{\centering}|}
\hline 
\textbf{Id Requisito} & \textbf{Fonti}\\
\hline
\endhead


\hyperlink{RFO1}{RFO1}
& \hyperlink{Capitolato}{Capitolato}\\
& \hyperref[UC3]{UC3}\\ \hline

\hyperlink{RFO1.1}{RFO1.1} 
& \hyperlink{Interno}{Interno}\\
& \hyperref[UC3]{UC3}\\
& \hyperref[UC3.1]{UC3.1}\\ \hline

\hyperlink{RFD1.2}{RFO1.2} 
& \hyperlink{Interno}{Interno}\\
& \hyperref[UC3]{UC3}\\
& \hyperref[UC3.2]{UC3.2}\\ \hline

\hyperlink{RFO1.3}{RFO1.3} 
& \hyperlink{Interno}{Interno}\\
& \hyperref[UC3]{UC3}\\
& \hyperref[UC3.3]{UC3.3}\\ \hline

\hyperlink{RFO1.4}{RFO1.4} 
& \hyperlink{Interno}{Interno}\\
& \hyperref[UC3]{UC3}\\
& \hyperref[UC3.4]{UC3.4}\\ \hline

\hyperlink{RFO1.5}{RFO1.5} 
& \hyperlink{Interno}{Interno}\\
& \hyperref[UC3]{UC3}\\
& \hyperref[UC3.5]{UC3.5}\\ \hline

\hyperlink{RFO1.6}{RFO1.6} 
& \hyperlink{Interno}{Interno}\\
& \hyperref[UC3]{UC3}\\
& \hyperref[UC3.6]{UC3.6}\\ \hline

\hyperlink{RFD1.7}{RFD1.7} 
& \hyperlink{Interno}{Interno}\\
& \hyperref[UC3]{UC3}\\
& \hyperref[UC3.7]{UC3.7}\\ \hline

\hyperlink{RFO1.8}{RFO1.8} 
& \hyperlink{Interno}{Interno}\\
& \hyperref[UC3]{UC3}\\
& \hyperref[UC3.8]{UC3.8}\\ \hline

\hyperlink{RFO1.9}{RFO1.9} 
& \hyperlink{Interno}{Interno}\\
& \hyperref[UC3]{UC3}\\
& \hyperref[UC3.9]{UC3.9}\\ \hline

\hyperlink{RFO1.10}{RFO1.10} 
& \hyperlink{Interno}{Interno}\\
& \hyperref[UC3]{UC3}\\
& \hyperref[UC3.10]{UC3.10}\\ \hline

\hyperlink{RFO2}{RFO2} 
& \hyperlink{Capitolato}{Capitolato}\\
& \hyperref[UC4]{UC4}\\ \hline

\hyperlink{RFO2.1}{RFO2.1} 
& \hyperlink{Capitolato}{Capitolato}\\
& \hyperref[UC4]{UC4}\\
& \hyperref[UC4.1]{UC4.1}\\ \hline

\hyperlink{RFO2.1.1}{RFO2.1.1} 
& \hyperlink{Interno}{Interno}\\
& \hyperref[UC4]{UC4}\\
& \hyperref[UC4.1]{UC4.1}\\
& \hyperref[UC4.1.1]{UC4.1.1}\\ \hline

\hyperlink{RFO2.1.2}{RFO2.1.2} 
& \hyperlink{Interno}{Interno}\\
& \hyperref[UC4]{UC4}\\
& \hyperref[UC4.1]{UC4.1}\\
& \hyperref[UC4.1.2]{UC4.1.2}\\ \hline

\hyperlink{RFO2.1.3}{RFO2.1.3} 
& \hyperlink{Interno}{Interno}\\
& \hyperref[UC4]{UC4}\\
& \hyperref[UC4.1]{UC4.1}\\
& \hyperref[UC4.1.3]{UC4.1.3}\\ \hline

\hyperlink{RFF2.1.4}{RFO2.1.4} 
& \hyperlink{Interno}{Interno}\\
& \hyperref[UC4]{UC4}\\
& \hyperref[UC4.1]{UC4.1}\\
& \hyperref[UC4.1.4]{UC4.1.4}\\ \hline

\hyperlink{RFF2.2}{RFF2.2} 
& \hyperlink{Interno}{Interno}\\
& \hyperref[UC4]{UC4}\\
& \hyperref[UC4.2]{UC4.2}\\ \hline

\hyperlink{RFF2.2.2}{RFF2.2.2} 
& \hyperlink{Interno}{Interno}\\
& \hyperref[UC4]{UC4}\\
& \hyperref[UC4.2]{UC4.2}\\
& \hyperref[UC4.2.2]{UC4.2.2}\\ \hline

\hyperlink{RFF2.3}{RFF2.3} 
& \hyperlink{Interno}{Interno}\\
& \hyperref[UC4]{UC4}\\
& \hyperref[UC4.3]{UC4.3}\\ \hline

\hyperlink{RFF2.3.2}{RFF2.3.2} 
& \hyperlink{Interno}{Interno}\\
& \hyperref[UC4]{UC4}\\
& \hyperref[UC4.3]{UC4.3}\\
& \hyperref[UC4.3.2]{UC4.3.2}\\ \hline

\hyperlink{RFF2.4}{RFF2.4}
& \hyperlink{Interno}{Interno}\\
& \hyperref[UC4]{UC4}\\
& \hyperref[UC4.4]{UC4.4}\\ \hline

\hyperlink{RFF2.4.2}{RFF2.4.2} 
& \hyperlink{Interno}{Interno}\\
& \hyperref[UC4]{UC4}\\
& \hyperref[UC4.4]{UC4.4}\\
& \hyperref[UC4.4.2]{UC4.4.2}\\ \hline

\hyperlink{RFF2.5}{RFF2.5} 
& \hyperlink{Interno}{Interno}\\
& \hyperref[UC4]{UC4}\\
& \hyperref[UC4.5]{UC4.5}\\ \hline

\hyperlink{RFF2.5.2}{RFF2.5.2} 
& \hyperlink{Interno}{Interno}\\
& \hyperref[UC4]{UC4}\\
& \hyperref[UC4.5]{UC4.5}\\
& \hyperref[UC4.5.2]{UC4.5.2}\\ \hline

\hyperlink{RFD3}{RFD3} 
& \hyperlink{Interno}{Interno}\\
& \hyperref[UC4]{UC4}\\ \hline

\hyperlink{RFD3.1}{RFD3.1} 
& \hyperlink{Interno}{Interno}\\
& \hyperref[UC5]{UC5}\\
& \hyperref[UC5.1]{UC5.1}\\ \hline

\hyperlink{RFD3.2}{RFD3.2} 
& \hyperlink{Interno}{Interno}\\
& \hyperref[UC5]{UC5}\\
& \hyperref[UC5.2]{UC5.2}\\ \hline

\hyperlink{RFD3.3}{RFD3.3} 
& \hyperlink{Interno}{Interno}\\
& \hyperref[UC5]{UC5}\\
& \hyperref[UC5.3]{UC5.3}\\ \hline

\hyperlink{RFD3.4}{RFD3.4} 
& \hyperlink{Interno}{Interno}\\
& \hyperref[UC5]{UC5}\\
& \hyperref[UC5.4]{UC5.4}\\ \hline

\hyperlink{RFO4}{RFO4} 
& \hyperlink{Capitolato}{Capitolato}\\
& \hyperref[UC6]{UC6}\\ \hline

\hyperlink{RFO4.1}{RFO4.1} 
& \hyperlink{Interno}{Interno}\\
& \hyperref[UC6]{UC6}\\
& \hyperref[UC6.1]{UC6.1}\\ \hline

\hyperlink{RFO4.2}{RFO4.2} 
& \hyperlink{Interno}{Interno}\\
& \hyperref[UC6]{UC6}\\
& \hyperref[UC6.2]{UC6.2}\\ \hline

\hyperlink{RFD4.3}{RFD4.3} 
& \hyperlink{Interno}{Interno}\\
& \hyperref[UC6]{UC6}\\
& \hyperref[UC6.3]{UC6.3}\\ \hline

\hyperlink{RFO4.3.1}{RFO4.3.1} 
& \hyperlink{Interno}{Interno}\\
& \hyperref[UC6]{UC6}\\
& \hyperref[UC6.1]{UC6.1}\\
& \hyperref[UC6.3]{UC6.3}\\
& \hyperref[UC6.3.1]{UC6.3.1}\\ \hline

\hyperlink{RFO4.3.2}{RFO4.3.2} 
& \hyperlink{Interno}{Interno}\\
& \hyperref[UC6]{UC6}\\
& \hyperref[UC6.1]{UC6.1}\\
& \hyperref[UC6.3]{UC6.3}\\
& \hyperref[UC6.3.2]{UC6.3.2}\\ \hline

\hyperlink{RFO4.3.3}{RFO4.3.3} 
& \hyperlink{Interno}{Interno}\\
& \hyperref[UC6]{UC6}\\
& \hyperref[UC6.1]{UC6.1}\\
& \hyperref[UC6.3]{UC6.3}\\
& \hyperref[UC6.3.3]{UC6.3.3}\\ \hline

\hyperlink{RFD4.3.4}{RFD4.3.4} 
& \hyperlink{Interno}{Interno}\\
& \hyperref[UC6]{UC6}\\
& \hyperref[UC6.1]{UC6.1}\\
& \hyperref[UC6.3]{UC6.3}\\
& \hyperref[UC6.3.4]{UC6.3.4}\\ \hline

\hyperlink{RFO4.3.5}{RFO4.3.5} 
& \hyperlink{Interno}{Interno}\\
& \hyperref[UC6]{UC6}\\
& \hyperref[UC6.1]{UC6.1}\\
& \hyperref[UC6.3]{UC6.3}\\
& \hyperref[UC6.3.5]{UC6.3.5}\\ \hline

\hyperlink{RFO5}{RFO5} 
& \hyperlink{Capitolato}{Capitolato}\\
& \hyperref[UC7]{UC7}\\ \hline

\hyperlink{RFO5.1}{RFO5.1} 
& \hyperlink{Interno}{Interno}\\
& \hyperref[UC7]{UC7}\\
& \hyperref[UC7.1]{UC7.1}\\ \hline

\hyperlink{RFO5.2}{RFO5.2} 
& \hyperlink{Interno}{Interno}\\
& \hyperref[UC7]{UC7}\\
& \hyperref[UC7.2]{UC7.2}\\ \hline

\hyperlink{RFO5.3}{RFO5.3} 
& \hyperlink{Interno}{Interno}\\
& \hyperref[UC7]{UC7}\\
& \hyperref[UC7.3]{UC7.3}\\ \hline

\hyperlink{RFO5.4}{RFO5.4} 
& \hyperlink{Interno}{Interno}\\
& \hyperref[UC7]{UC7}\\
& \hyperref[UC7.4]{UC7.4}\\ \hline

\hyperlink{RFO5.5}{RFO5.5} 
& \hyperlink{Interno}{Interno}\\
& \hyperref[UC7]{UC7}\\
& \hyperref[UC7.5]{UC7.5}\\ \hline

\hyperlink{RFO5.5.1}{RFO5.5.1} 
& \hyperlink{Interno}{Interno}\\
& \hyperref[UC7]{UC7}\\
& \hyperref[UC7.5]{UC7.5}\\
& \hyperref[UC7.5.1]{UC7.5.1}\\ \hline

\hyperlink{RFO5.5.2}{RFO5.5.2} 
& \hyperlink{Interno}{Interno}\\
& \hyperref[UC7]{UC7}\\
& \hyperref[UC7.5]{UC7.5}\\
& \hyperref[UC7.5.2]{UC7.5.2}\\ \hline

\hyperlink{RFO5.6}{RFO5.6} 
& \hyperlink{Capitolato}{Capitolato}\\
& \hyperref[UC7]{UC7}\\
& \hyperref[UC7.6]{UC7.6}\\ \hline

\hyperlink{RFO5.6.1}{RFO5.6.1} 
& \hyperlink{Capitolato}{Capitolato}\\
& \hyperref[UC7]{UC7}\\
& \hyperref[UC7.6]{UC7.6}\\
& \hyperref[UC7.6.1]{UC7.6.1}\\ \hline

\hyperlink{RFO5.6.2}{RFO5.6.2} 
& \hyperlink{Capitolato}{Capitolato}\\
& \hyperref[UC7]{UC7}\\
& \hyperref[UC7.6]{UC7.6}\\
& \hyperref[UC7.6.2]{UC7.6.2}\\ \hline


\hyperlink{RFO5.7}{RFO5.7} 
& \hyperlink{Capitolato}{Capitolato}\\
& \hyperref[UC7]{UC7}\\
& \hyperref[UC7.6]{UC7.6}\\ \hline

\hyperlink{RFD5.7.1}{RFD5.7.1} 
& \hyperlink{Interno}{Interno}\\
& \hyperref[UC7]{UC7}\\
& \hyperref[UC7.7]{UC7.7}\\
& \hyperref[UC7.7.1]{UC7.7.1}\\ \hline

\hyperlink{RFO5.7.2}{RFO5.7.2} 
& \hyperlink{Capitolato}{Capitolato}\\
& \hyperref[UC7]{UC7}\\
& \hyperref[UC7.7]{UC7.7}\\
& \hyperref[UC7.7.2]{UC7.7.2}\\ \hline

\hyperlink{RFO5.8}{RFO5.8} 
& \hyperlink{Interno}{Interno}\\
& \hyperref[UC7]{UC7}\\
& \hyperref[UC7.8]{UC7.8}\\
 \hline
 
 \hyperlink{RFO5.9}{RFO5.9} 
 & \hyperlink{Interno}{Interno}\\
 & \hyperref[UC7]{UC7}\\
 & \hyperref[UC7.9]{UC7.9}\\\hline
 
\hyperlink{RFO5.10}{RFO5.10} 
& \hyperlink{Interno}{Interno}\\
 & \hyperref[UC7]{UC7}\\
 & \hyperref[UC7.10]{UC7.10}\\\hline
 
 \hyperlink{RFD5.11}{RFO5.11} 
& \hyperlink{Interno}{Interno}\\
 & \hyperref[UC7]{UC7}\\
 & \hyperref[UC7.11]{UC7.11}\\\hline
 
\hyperlink{RFO5.12}{RFO5.12} 
& \hyperlink{Interno}{Interno}\\
 & \hyperref[UC7]{UC7}\\
 & \hyperref[UC7.12]{UC7.12}\\\hline
 
 \hyperlink{RFO5.13}{RFO5.13} 
& \hyperlink{Interno}{Interno}\\
 & \hyperref[UC7]{UC7}\\
 & \hyperref[UC7.13]{UC7.13}\\\hline 
 
 \hyperlink{RFO6}{RFO6} 
 & \hyperlink{Capitolato}{Capitolato}\\
 & \hyperref[UC8]{UC8}\\\hline
 
 \hyperlink{RFO6.1}{RFO6.1} 
 & \hyperlink{Interno}{Interno}\\
 & \hyperref[UC8]{UC8}\\
 & \hyperref[UC8.1]{UC8.1}\\\hline
 
 \hyperlink{RFO6.2}{RFO6.2} 
 & \hyperlink{Capitolato}{Capitolato}\\
 & \hyperref[UC8]{UC8}\\
 & \hyperref[UC8.2]{UC8.2}\\\hline
 
 \hyperlink{RFO6.2.1}{RFO6.2.1} 
 & \hyperlink{Interno}{Interno}\\
 & \hyperref[UC8]{UC8}\\
 & \hyperref[UC8.2]{UC8.2}\\
  & \hyperref[UC8.2.1]{UC8.2.1}\\\hline
  
   \hyperlink{RFO6.2.2}{RFO6.2.2} 
   & \hyperlink{Interno}{Interno}\\
  & \hyperref[UC8]{UC8}\\
  & \hyperref[UC8.2]{UC8.2}\\
  & \hyperref[UC8.2.2]{UC8.2.2}\\\hline
  
   \hyperlink{RFO6.2.3}{RFO6.2.3} 
   & \hyperlink{Interno}{Interno}\\
  & \hyperref[UC8]{UC8}\\
  & \hyperref[UC8.2]{UC8.2}\\
  & \hyperref[UC8.2.3]{UC8.2.3}\\\hline
  
  \hyperlink{RFO6.2.4}{RFO6.2.4} 
  & \hyperlink{Interno}{Interno}\\
  & \hyperref[UC8]{UC8}\\
  & \hyperref[UC8.2]{UC8.2}\\
  & \hyperref[UC8.2.4]{UC8.2.4}\\\hline
  
   \hyperlink{RFO6.2.5}{RFO6.2.5} 
   & \hyperlink{Interno}{Interno}\\
  & \hyperref[UC8]{UC8}\\
  & \hyperref[UC8.2]{UC8.2}\\
  & \hyperref[UC8.2.5]{UC8.2.5}\\\hline
  
   \hyperlink{RFD6.2.6}{RFD6.2.6} 
   & \hyperlink{Interno}{Interno}\\
  & \hyperref[UC8]{UC8}\\
  & \hyperref[UC8.2]{UC8.2}\\
  & \hyperref[UC8.2.6]{UC8.2.6}\\\hline
  
  \hyperlink{RFD6.2.7}{RFD6.2.7} 
   & \hyperlink{Interno}{Interno}\\
  & \hyperref[UC8]{UC8}\\
  & \hyperref[UC8.2]{UC8.2}\\
  & \hyperref[UC8.2.7]{UC8.2.7}\\\hline
  
   \hyperlink{RFO7}{RFO7} 
   & \hyperlink{Capitolato}{Capitolato}\\
  & \hyperref[UC9]{UC9}\\\hline
  
    \hyperlink{RFO7.1}{RFO7.1} 
    & \hyperlink{Interno}{Interno}\\
  & \hyperref[UC9]{UC9}\\
   & \hyperref[UC9.1]{UC9.1}\\\hline
   
    \hyperlink{RFO7.1.1}{RFO7.1.1} 
    & \hyperlink{Interno}{Interno}\\
   & \hyperref[UC9]{UC9}\\
   & \hyperref[UC9.1]{UC9.1}\\
   & \hyperref[UC9.1.1]{UC9.1.1}\\\hline
   
   \hyperlink{RFO7.1.2}{RFO7.1.2} 
   & \hyperlink{Interno}{Interno}\\
   & \hyperref[UC9]{UC9}\\
   & \hyperref[UC9.1]{UC9.1}\\
   & \hyperref[UC9.1.2]{UC9.1.2}\\\hline
   
   \hyperlink{RFO7.1.3}{RFO7.1.3} 
   & \hyperlink{Interno}{Interno}\\
   & \hyperref[UC9]{UC9}\\
   & \hyperref[UC9.1]{UC9.1}\\
   & \hyperref[UC9.1.3]{UC9.1.3}\\\hline
   
   \hyperlink{RFO7.2}{RFO7.2} 
   & \hyperlink{Interno}{Interno}\\
   & \hyperref[UC9]{UC9}\\
   & \hyperref[UC9.2]{UC9.2}\\\hline
   
    \hyperlink{RFO7.3}{RFO7.3} 
    & \hyperlink{Interno}{Interno}\\
   & \hyperref[UC9]{UC9}\\
   & \hyperref[UC9.3]{UC9.3}\\\hline
   
   \hyperlink{RFO7.4}{RFO7.4} 
   & \hyperlink{Interno}{Interno}\\
   & \hyperref[UC9]{UC9}\\
   & \hyperref[UC9.4]{UC9.4}\\\hline
   
   \hyperlink{RFO7.5}{RFO7.5} 
   & \hyperlink{Interno}{Interno}\\
   & \hyperref[UC9]{UC9}\\
   & \hyperref[UC9.5]{UC9.5}\\\hline
   
   \hyperlink{RFD7.5.1}{RFD7.5.1} 
   & \hyperlink{Interno}{Interno}\\
   & \hyperref[UC9]{UC9}\\
   & \hyperref[UC9.5]{UC9.5}\\
   & \hyperref[UC9.5.1]{UC9.5.1}\\\hline
   
   \hyperlink{RFO7.5.2}{RFO7.5.2} 
   & \hyperlink{Interno}{Interno}\\
   & \hyperref[UC9]{UC9}\\
   & \hyperref[UC9.5]{UC9.5}\\
   & \hyperref[UC9.5.2]{UC9.5.2}\\\hline
   
   \hyperlink{RFO7.5.3}{RFO7.5.3} 
   & \hyperlink{Capitolato}{Capitolato}\\
   & \hyperref[UC9]{UC9}\\
   & \hyperref[UC9.5]{UC9.5}\\
   & \hyperref[UC9.5.3]{UC9.5.3}\\\hline
   
   \hyperlink{RFO7.6}{RFO7.6} 
   & \hyperlink{Interno}{Interno}\\
   & \hyperref[UC9]{UC9}\\
   & \hyperref[UC9.6]{UC9.6}\\\hline
   
   \hyperlink{RFO8}{RFO8} 
   & \hyperlink{Capitolato}{Capitolato}\\
   & \hyperref[UC10]{UC10}\\\hline
   
      \hyperlink{RFO8.1}{RFO8.1} 
      & \hyperlink{Interno}{Interno}\\
   & \hyperref[UC10]{UC10}\\
      & \hyperref[UC10.1]{UC10.1}\\\hline
  
      \hyperlink{RFO8.2}{RFO8.2} 
      & \hyperlink{Interno}{Interno}\\
  & \hyperref[UC10]{UC10}\\
  & \hyperref[UC10.2]{UC10.2}\\\hline
  
      \hyperlink{RFO8.2.1}{RFO8.2.1} 
      & \hyperlink{Interno}{Interno}\\
  & \hyperref[UC10]{UC10}\\
  & \hyperref[UC10.2]{UC10.2}\\
   & \hyperref[UC10.2.1]{UC10.2.1}\\\hline
   
  \hyperlink{RFO8.2.2}{RFO8.2.2} 
  & \hyperlink{Interno}{Interno}\\
  & \hyperref[UC10]{UC10}\\
  & \hyperref[UC10.2]{UC10.2}\\
  & \hyperref[UC10.2.2]{UC10.2.2}\\\hline
  
  \hyperlink{RFO8.2.3}{RFO8.2.3} 
  & \hyperlink{Interno}{Interno}\\
  & \hyperref[UC10]{UC10}\\
  & \hyperref[UC10.2]{UC10.2}\\
  & \hyperref[UC10.2.3]{UC10.2.3}\\\hline
  
  \hyperlink{RFO8.2.4}{RFO8.2.4} 
  & \hyperlink{Capitolato}{Capitolato}\\
  & \hyperref[UC10]{UC10}\\
  & \hyperref[UC10.2]{UC10.2}\\
  & \hyperref[UC10.2.4]{UC10.2.4}\\\hline
  
\hyperlink{RFO8.2.4.1}{RFO8.2.4.1} 
 & \hyperlink{Interno}{Interno}\\
 & \hyperref[UC10]{UC10}\\
 & \hyperref[UC10.2]{UC10.2}\\
 & \hyperref[UC10.2.4]{UC10.2.4}\\
 & \hyperref[UC10.2.4.1]{UC10.2.4.1}\\\hline
 
 \hyperlink{RFO8.2.4.2}{RFO8.2.4.2} 
 & \hyperlink{Interno}{Interno}\\
 & \hyperref[UC10]{UC10}\\
 & \hyperref[UC10.2]{UC10.2}\\
 & \hyperref[UC10.2.4]{UC10.2.4}\\
 & \hyperref[UC10.2.4.2]{UC10.2.4.2}\\\hline
 
 \hyperlink{RFO8.2.4.3}{RFO8.2.4.3} 
 & \hyperlink{Interno}{Interno}\\
 & \hyperref[UC10]{UC10}\\
 & \hyperref[UC10.2]{UC10.2}\\
 & \hyperref[UC10.2.4]{UC10.2.4}\\
 & \hyperref[UC10.2.4.3]{UC10.2.4.3}\\\hline
 
 \hyperlink{RFO8.2.4.4}{RFO8.2.4.4} 
 & \hyperlink{Interno}{Interno}\\
 & \hyperref[UC10]{UC10}\\
 & \hyperref[UC10.2]{UC10.2}\\
 & \hyperref[UC10.2.4]{UC10.2.4}\\
 & \hyperref[UC10.2.4.4]{UC10.2.4.4}\\\hline
 
 \hyperlink{RFO8.2.4.5}{RFO8.2.4.5} 
 & \hyperlink{Interno}{Interno}\\
 & \hyperref[UC10]{UC10}\\
 & \hyperref[UC10.2]{UC10.2}\\
 & \hyperref[UC10.2.4]{UC10.2.4}\\
 & \hyperref[UC10.2.4.5]{UC10.2.4.5}\\\hline
 
 \hyperlink{RFO8.2.4.6}{RFO8.2.4.6} 
 & \hyperlink{Interno}{Interno}\\
 & \hyperref[UC10]{UC10}\\
 & \hyperref[UC10.2]{UC10.2}\\
 & \hyperref[UC10.2.4]{UC10.2.4}\\
 & \hyperref[UC10.2.4.6]{UC10.2.4.6}\\\hline
 
 \hyperlink{RFO8.2.4.7}{RFO8.2.4.7} 
 & \hyperlink{Interno}{Interno}\\
 & \hyperref[UC10]{UC10}\\
 & \hyperref[UC10.2]{UC10.2}\\
 & \hyperref[UC10.2.4]{UC10.2.4}\\
 & \hyperref[UC10.2.4.7]{UC10.2.4.7}\\\hline
 
 \hyperlink{RFO8.2.4.8}{RFO8.2.4.8} 
 & \hyperlink{Interno}{Interno}\\
 & \hyperref[UC10]{UC10}\\
 & \hyperref[UC10.2]{UC10.2}\\
 & \hyperref[UC10.2.4]{UC10.2.4}\\
 & \hyperref[UC10.2.4.8]{UC10.2.4.8}\\\hline
 
 \hyperlink{RFO8.2.4.9}{RFO8.2.4.9} 
 & \hyperlink{Interno}{Interno}\\
 & \hyperref[UC10]{UC10}\\
 & \hyperref[UC10.2]{UC10.2}\\
 & \hyperref[UC10.2.4]{UC10.2.4}\\
 & \hyperref[UC10.2.4.9]{UC10.2.4.9}\\\hline
 
 \hyperlink{RFO8.2.4.10}{RFO8.2.4.10} 
 & \hyperlink{Interno}{Interno}\\
 & \hyperref[UC10]{UC10}\\
 & \hyperref[UC10.2]{UC10.2}\\
 & \hyperref[UC10.2.4]{UC10.2.4}\\
 & \hyperref[UC10.2.4.10]{UC10.2.4.10}\\\hline
  
 \hyperlink{RFD8.2.5}{RFD8.2.5} 
 & \hyperlink{Interno}{Interno}\\
 & \hyperref[UC10]{UC10}\\
 & \hyperref[UC10.2]{UC10.2}\\
 & \hyperref[UC10.2.5]{UC10.2.5}\\\hline
 
 \hyperlink{RFD8.2.6}{RFD8.2.6} 
 & \hyperlink{Interno}{Interno}\\
 & \hyperref[UC10]{UC10}\\
 & \hyperref[UC10.2]{UC10.2}\\
 & \hyperref[UC10.2.6]{UC10.2.6}\\\hline
 
 \hyperlink{RFD8.2.7}{RFD8.2.7} 
 & \hyperlink{Interno}{Interno}\\
 & \hyperref[UC10]{UC10}\\
 & \hyperref[UC10.2]{UC10.2}\\
 & \hyperref[UC10.2.7]{UC10.2.7}\\\hline
 
 \hyperlink{RFO8.2.8}{RFO8.2.8} 
 & \hyperlink{Interno}{Interno}\\
 & \hyperref[UC10]{UC10}\\
 & \hyperref[UC10.2]{UC10.2}\\
 & \hyperref[UC10.2.8]{UC10.2.8}\\\hline
 
 \hyperlink{RFO8.2.9}{RFO8.2.9} 
 & \hyperlink{Interno}{Interno}\\
 & \hyperref[UC10]{UC10}\\
 & \hyperref[UC10.2]{UC10.2}\\
 & \hyperref[UC10.2.9]{UC10.2.9}\\\hline
 
 \hyperlink{RFO8.2.9.1}{RFO8.2.9.1} 
 & \hyperlink{Interno}{Interno}\\
 & \hyperref[UC10]{UC10}\\
 & \hyperref[UC10.2]{UC10.2}\\
 & \hyperref[UC10.2.9]{UC10.2.9}\\
 & \hyperref[UC10.2.9.1]{UC10.2.9.1}\\\hline
 
\hyperlink{RFO8.2.9.2}{RFO8.2.9.2} 
 & \hyperlink{Interno}{Interno}\\
 & \hyperref[UC10]{UC10}\\
 & \hyperref[UC10.2]{UC10.2}\\
 & \hyperref[UC10.2.9]{UC10.2.9}\\
 & \hyperref[UC10.2.9.2]{UC10.2.9.2}\\\hline
 
  \hyperlink{RFO9}{RFO9} 
  & \hyperlink{Capitolato}{Capitolato}\\
 & \hyperref[UC11]{UC11}\\\hline
 
   \hyperlink{RFO9.1}{RFO9.1} 
   & \hyperlink{Interno}{Interno}\\
 & \hyperref[UC11]{UC11}\\
  & \hyperref[UC11.1]{UC11.1}\\\hline
  
    \hyperlink{RFO9.2}{RFO9.2} 
    & \hyperlink{Interno}{Interno}\\
  & \hyperref[UC11]{UC11}\\
  & \hyperref[UC11.2]{UC11.2}\\\hline
  
     \hyperlink{RFD9.3}{RFD9.3} 
     & \hyperlink{Interno}{Interno}\\
  & \hyperref[UC11]{UC11}\\
  & \hyperref[UC11.3]{UC11.3}\\\hline
  
      \hyperlink{RFO9.4}{RFO9.4} 
      & \hyperlink{Interno}{Interno}\\
  & \hyperref[UC11]{UC11}\\
  & \hyperref[UC11.4]{UC11.4}\\\hline
  
      \hyperlink{RFO9.5}{RFO9.5} 
      & \hyperlink{Interno}{Interno}\\
  & \hyperref[UC11]{UC11}\\
  & \hyperref[UC11.5]{UC11.5}\\\hline
  
      \hyperlink{RFD9.6}{RFD9.6} 
      & \hyperlink{Interno}{Interno}\\
  & \hyperref[UC11]{UC11}\\
  & \hyperref[UC11.6]{UC11.6}\\\hline
  
    \hyperlink{RFO9.7}{RFO9.7} 
    & \hyperlink{Interno}{Interno}\\
  & \hyperref[UC11]{UC11}\\
  & \hyperref[UC11.7]{UC11.7}\\\hline

  \hyperlink{RFO9.8}{RFO9.8} 
  & \hyperlink{Capitolato}{Capitolato}\\
& \hyperref[UC11]{UC11}\\
& \hyperref[UC11.8]{UC11.8}\\\hline

\hyperlink{RFO9.8.1}{RFO9.8.1} 
  & \hyperlink{Capitolato}{Capitolato}\\
& \hyperref[UC11]{UC11}\\
& \hyperref[UC11.8]{UC11.8}\\
& \hyperref[UC11.8.1]{UC11.8.1}\\\hline

\hyperlink{RFO9.8.2}{RFO9.8.2} 
  & \hyperlink{Capitolato}{Capitolato}\\
& \hyperref[UC11]{UC11}\\
& \hyperref[UC11.8]{UC11.8}\\
& \hyperref[UC11.8.2]{UC11.8.2}\\\hline

\hyperlink{RFO9.8.3}{RFO9.8.3} 
  & \hyperlink{Capitolato}{Capitolato}\\
& \hyperref[UC11]{UC11}\\
& \hyperref[UC11.8]{UC11.8}\\
& \hyperref[UC11.8.3]{UC11.8.3}\\\hline

  \hyperlink{RFD9.9}{RFD9.9} 
  & \hyperlink{Interno}{Interno}\\
& \hyperref[UC11]{UC11}\\
& \hyperref[UC11.9]{UC11.9}\\\hline

 \hyperlink{RFD9.10}{RFD9.10} 
  & \hyperlink{Interno}{Interno}\\
& \hyperref[UC11]{UC11}\\
& \hyperref[UC11.10]{UC11.10}\\\hline

 \hyperlink{RFD9.11}{RFD9.11} 
  & \hyperlink{Interno}{Interno}\\
& \hyperref[UC11]{UC11}\\
& \hyperref[UC11.11]{UC11.11}\\\hline

  \hyperlink{RFO9.12}{RFO9.12} 
  & \hyperlink{Interno}{Interno}\\
& \hyperref[UC11]{UC11}\\
& \hyperref[UC11.12]{UC11.12}\\\hline

  \hyperlink{RFD9.13}{RFD9.13} 
  & \hyperlink{Interno}{Interno}\\
& \hyperref[UC11]{UC11}\\
& \hyperref[UC11.13]{UC11.13}\\\hline

  \hyperlink{RFO10}{RFO10} 
  & \hyperlink{Capitolato}{Capitolato}\\
& \hyperref[UC12]{UC12}\\\hline

 \hyperlink{RFO10.1}{RFO10.1} 
  & \hyperlink{Capitolato}{Capitolato}\\
& \hyperref[UC12]{UC12}\\\hline

 \hyperlink{RFO10.1.1}{RFO10.1.1} 
  & \hyperlink{Capitolato}{Capitolato}\\
& \hyperref[UC12]{UC12}\\
& \hyperref[UC12.1]{UC12.1}\\
& \hyperref[UC12.1.1]{UC12.1.1}\\\hline

 \hyperlink{RFO10.1.1.1}{RFO10.1.1.1} 
 & \hyperlink{Interno}{Interno}\\
& \hyperref[UC12]{UC12}\\
& \hyperref[UC12.1]{UC12.1}\\
& \hyperref[UC12.1.1]{UC12.1.1}\\
& \hyperref[UC12.1.1.1]{UC12.1.1.1}\\\hline

 \hyperlink{RFO10.1.1.2}{RFO10.1.1.2} 
 & \hyperlink{Interno}{Interno}\\
& \hyperref[UC12]{UC12}\\
& \hyperref[UC12.1]{UC12.1}\\
& \hyperref[UC12.1.1]{UC12.1.1}\\
& \hyperref[UC12.1.1.2]{UC12.1.1.2}\\\hline

 \hyperlink{RFO10.1.1.3}{RFO10.1.1.3} 
 & \hyperlink{Interno}{Interno}\\
& \hyperref[UC12]{UC12}\\
& \hyperref[UC12.1]{UC12.1}\\
& \hyperref[UC12.1.1]{UC12.1.1}\\
& \hyperref[UC12.1.1.1]{UC12.1.1.1}\\\hline

 \hyperlink{RFO10.1.1.4}{RFO10.1.1.4} 
 & \hyperlink{Interno}{Interno}\\
& \hyperref[UC12]{UC12}\\
& \hyperref[UC12.1]{UC12.1}\\
& \hyperref[UC12.1.1]{UC12.1.1}\\
& \hyperref[UC12.1.1.4]{UC12.1.1.4}\\\hline

 \hyperlink{RFO10.1.1.5}{RFO10.1.1.5} 
 & \hyperlink{Interno}{Interno}\\
& \hyperref[UC12]{UC12}\\
& \hyperref[UC12.1]{UC12.1}\\
& \hyperref[UC12.1.1]{UC12.1.1}\\
& \hyperref[UC12.1.1.5]{UC12.1.1.5}\\\hline

 \hyperlink{RFD10.1.1.6}{RFD10.1.1.6} 
 & \hyperlink{Interno}{Interno}\\
& \hyperref[UC12]{UC12}\\
& \hyperref[UC12.1]{UC12.1}\\
& \hyperref[UC12.1.1]{UC12.1.1}\\
& \hyperref[UC12.1.1.6]{UC12.1.1.6}\\\hline

 \hyperlink{RFO10.1.2}{RFO10.1.2} 
  & \hyperlink{Capitolato}{Capitolato}\\
& \hyperref[UC12]{UC12}\\
& \hyperref[UC12.1]{UC12.1}\\
& \hyperref[UC12.1.2]{UC12.1.2}\\\hline

 \hyperlink{RFO10.1.2.1}{RFO10.1.2.1} 
 & \hyperlink{Interno}{Interno}\\
& \hyperref[UC12]{UC12}\\
& \hyperref[UC12.1]{UC12.1}\\
& \hyperref[UC12.1.2]{UC12.1.2}\\
& \hyperref[UC12.1.2.1]{UC12.1.2.1}\\\hline

 \hyperlink{RFO10.1.2.2}{RFO10.1.2.2} 
 & \hyperlink{Interno}{Interno}\\
& \hyperref[UC12]{UC12}\\
& \hyperref[UC12.1]{UC12.1}\\
& \hyperref[UC12.1.2]{UC12.1.2}\\
& \hyperref[UC12.1.2.2]{UC12.1.2.2}\\\hline

 \hyperlink{RFO10.1.2.3}{RFO10.1.2.3} & \hyperlink{Interno}{Interno}\\
& \hyperref[UC12]{UC12}\\
& \hyperref[UC12.1]{UC12.1}\\
& \hyperref[UC12.1.2]{UC12.1.2}\\
& \hyperref[UC12.1.2.3]{UC12.1.2.3}\\\hline

 \hyperlink{RFO10.1.2.4}{RFO10.1.2.4} 
 & \hyperlink{Interno}{Interno}\\
& \hyperref[UC12]{UC12}\\
& \hyperref[UC12.1]{UC12.1}\\
& \hyperref[UC12.1.2]{UC12.1.2}\\
& \hyperref[UC12.1.2.4]{UC12.1.2.4}\\\hline

 \hyperlink{RFO10.1.2.5}{RFO10.1.2.5} 
 & \hyperlink{Interno}{Interno}\\
& \hyperref[UC12]{UC12}\\
& \hyperref[UC12.1]{UC12.1}\\
& \hyperref[UC12.1.2]{UC12.1.2}\\
& \hyperref[UC12.1.2.5]{UC12.1.2.5}\\\hline

 \hyperlink{RFD10.1.2.6}{RFD10.1.2.6} 
 & \hyperlink{Interno}{Interno}\\
& \hyperref[UC12]{UC12}\\
& \hyperref[UC12.1]{UC12.1}\\
& \hyperref[UC12.1.2]{UC12.1.2}\\
& \hyperref[UC12.1.2.6]{UC12.1.2.6}\\\hline

 \hyperlink{RFD10.1.2.7}{RFD10.1.2.7} 
 & \hyperlink{Interno}{Interno}\\
& \hyperref[UC12]{UC12}\\
& \hyperref[UC12.1]{UC12.1}\\
& \hyperref[UC12.1.2]{UC12.1.2}\\
& \hyperref[UC12.1.2.7]{UC12.1.2.7}\\\hline

\hyperlink{RFO10.1.2.8}{RFO10.1.2.8} 
 & \hyperlink{Interno}{Interno}\\
& \hyperref[UC12]{UC12}\\
& \hyperref[UC12.1]{UC12.1}\\
& \hyperref[UC12.1.2]{UC12.1.2}\\
& \hyperref[UC12.1.2.8]{UC12.1.2.8}\\\hline

 \hyperlink{RFO10.2}{RFO10.2} 
 & \hyperlink{Capitolato}{Capitolato}\\\
& \hyperref[UC12]{UC12}\\
& \hyperref[UC12.2]{UC12.2}\\\hline

 \hyperlink{RFO10.2.1}{RFO10.2.1} 
 & \hyperlink{Interno}{Interno}\\
& \hyperref[UC12]{UC12}\\
& \hyperref[UC12.2]{UC12.2}\\
& \hyperref[UC12.2.1]{UC12.2.1}\\\hline

 \hyperlink{RFO10.2.2}{RFO10.2.2} 
 & \hyperlink{Interno}{Interno}\\
& \hyperref[UC12]{UC12}\\
& \hyperref[UC12.2]{UC12.2}\\
& \hyperref[UC12.2.2]{UC12.2.2}\\\hline

 \hyperlink{RFO10.2.2.1}{RFO10.2.2.1} 
 & \hyperlink{Interno}{Interno}\\
& \hyperref[UC12]{UC12}\\
& \hyperref[UC12.2]{UC12.2}\\
& \hyperref[UC12.2.2]{UC12.2.2}\\
& \hyperref[UC12.2.2.1]{UC12.2.2.1}\\\hline

 \hyperlink{RFO10.2.2.2}{RFO10.2.2.2} 
 & \hyperlink{Interno}{Interno}\\
& \hyperref[UC12]{UC12}\\
& \hyperref[UC12.2]{UC12.2}\\
& \hyperref[UC12.2.2]{UC12.2.2}\\
& \hyperref[UC12.2.2.2]{UC12.2.2.2}\\\hline

 \hyperlink{RFO10.3}{RFO10.3} 
& \hyperlink{Interno}{Interno}\\
& \hyperref[UC12]{UC12}\\
& \hyperref[UC12.3]{UC12.3}\\\hline

 \hyperlink{RFO10.3.1}{RFO10.3.1} 
& \hyperlink{Interno}{Interno}\\
& \hyperref[UC12]{UC12}\\
& \hyperref[UC12.3]{UC12.3}\\
& \hyperref[UC12.3.1]{UC12.3.1}\\\hline

 \hyperlink{RFO10.3.1.1}{RFO10.3.1.1} 
& \hyperlink{Interno}{Interno}\\
& \hyperref[UC12]{UC12}\\
& \hyperref[UC12.3]{UC12.3}\\
& \hyperref[UC12.3.1]{UC12.3.1}\\
& \hyperref[UC12.3.1.1]{UC12.3.1.1}\\\hline

 \hyperlink{RFO10.3.1.2}{RFO10.3.1.2} 
& \hyperlink{Interno}{Interno}\\
& \hyperref[UC12]{UC12}\\
& \hyperref[UC12.3]{UC12.3}\\
& \hyperref[UC12.3.1]{UC12.3.1}\\
& \hyperref[UC12.3.1.2]{UC12.3.1.2}\\\hline

 \hyperlink{RFO10.3.2}{RFO10.3.2} 
& \hyperlink{Interno}{Interno}\\
& \hyperref[UC12]{UC12}\\
& \hyperref[UC12.3]{UC12.3}\\
& \hyperref[UC12.3.2]{UC12.3.2}\\\hline

 \hyperlink{RFO10.3.2.1}{RFO10.3.2.1} 
& \hyperlink{Interno}{Interno}\\
& \hyperref[UC12]{UC12}\\
& \hyperref[UC12.3]{UC12.3}\\
& \hyperref[UC12.3.2]{UC12.3.2}\\
& \hyperref[UC12.3.2.1]{UC12.3.2.1}\\\hline

 \hyperlink{RFO10.3.2.2}{RFO10.3.2.2} 
& \hyperlink{Interno}{Interno}\\
& \hyperref[UC12]{UC12}\\
& \hyperref[UC12.3]{UC12.3}\\
& \hyperref[UC12.3.2]{UC12.3.2}\\
& \hyperref[UC12.3.2.2]{UC12.3.2.2}\\\hline

 \hyperlink{RFO10.3.2.3}{RFO10.3.2.3} 
& \hyperlink{Interno}{Interno}\\
& \hyperref[UC12]{UC12}\\
& \hyperref[UC12.3]{UC12.3}\\
& \hyperref[UC12.3.2]{UC12.3.2}\\
& \hyperref[UC12.3.2.1]{UC12.3.2.3}\\\hline

 \hyperlink{RFO10.3.2.4}{RFO10.3.2.4} 
& \hyperlink{Interno}{Interno}\\
& \hyperref[UC12]{UC12}\\
& \hyperref[UC12.3]{UC12.3}\\
& \hyperref[UC12.3.2]{UC12.3.2}\\
& \hyperref[UC12.3.2.4]{UC12.3.2.4}\\\hline

 \hyperlink{RFO10.3.2.5}{RFO10.3.2.5} 
& \hyperlink{Interno}{Interno}\\
& \hyperref[UC12]{UC12}\\
& \hyperref[UC12.3]{UC12.3}\\
& \hyperref[UC12.3.2]{UC12.3.2}\\
& \hyperref[UC12.3.2.5]{UC12.3.2.5}\\\hline

 \hyperlink{RFO10.3.3}{RFO10.3.3} 
& \hyperlink{Interno}{Interno}\\
& \hyperref[UC12]{UC12}\\
& \hyperref[UC12.3]{UC12.3}\\
& \hyperref[UC12.3.3]{UC12.3.3}\\\hline

 \hyperlink{RFO10.3.3.1}{RFO10.3.3.1} 
& \hyperlink{Interno}{Interno}\\
& \hyperref[UC12]{UC12}\\
& \hyperref[UC12.3]{UC12.3}\\
& \hyperref[UC12.3.3]{UC12.3.3}\\
& \hyperref[UC12.3.3.1]{UC12.3.3.1}\\\hline

 \hyperlink{RFO10.3.3.2}{RFO10.3.3.2} 
& \hyperlink{Interno}{Interno}\\
& \hyperref[UC12]{UC12}\\
& \hyperref[UC12.3]{UC12.3}\\
& \hyperref[UC12.3.3]{UC12.3.3}\\
& \hyperref[UC12.3.3.2]{UC12.3.3.2}\\\hline

 \hyperlink{RFO10.3.3.3}{RFO10.3.3.3} 
& \hyperlink{Interno}{Interno}\\
& \hyperref[UC12]{UC12}\\
& \hyperref[UC12.3]{UC12.3}\\
& \hyperref[UC12.3.3]{UC12.3.3}\\
& \hyperref[UC12.3.3.3]{UC12.3.3.3}\\\hline

 \hyperlink{RFO10.3.3.4}{RFO10.3.3.4} 
& \hyperlink{Interno}{Interno}\\
& \hyperref[UC12]{UC12}\\
& \hyperref[UC12.3]{UC12.3}\\
& \hyperref[UC12.3.3]{UC12.3.3}\\
& \hyperref[UC12.3.3.4]{UC12.3.3.4}\\\hline

 \hyperlink{RFO11}{RFO11} 
  & \hyperlink{Capitolato}{Capitolato}\\
& \hyperref[UC13]{UC13}\\\hline

\hyperlink{RFO11.1}{RFO11.1} 
& \hyperlink{Interno}{Interno}\\
& \hyperref[UC13]{UC13}\\
& \hyperref[UC13.1]{UC13.1}\\\hline

 \hyperlink{RFO12}{RFO12} 
  & \hyperlink{Capitolato}{Capitolato}\\
& \hyperref[UC14]{UC14}\\\hline

 \hyperlink{RFO12.1}{RFO12.1} 
 & \hyperlink{Interno}{Interno}\\
& \hyperref[UC14]{UC14}\\
& \hyperref[UC14.1]{UC14.1}\\\hline

 \hyperlink{RFO12.1.1}{RFO12.1.1} 
 & \hyperlink{Interno}{Interno}\\
& \hyperref[UC14]{UC14}\\
& \hyperref[UC14.1]{UC14.1}\\
& \hyperref[UC14.1.1]{UC14.1.1}\\\hline

 \hyperlink{RFO12.1.1.1}{RFO12.1.1.1} 
 & \hyperlink{Interno}{Interno}\\
& \hyperref[UC14]{UC14}\\
& \hyperref[UC14.1]{UC14.1}\\
& \hyperref[UC14.1.1]{UC14.1.1}\\
& \hyperref[UC14.1.1.1]{UC14.1.1.1}\\\hline

 \hyperlink{RFO12.1.1.1.1}{RFO12.1.1.1.1} 
 & \hyperlink{Interno}{Interno}\\
& \hyperref[UC14]{UC14}\\
& \hyperref[UC14.1]{UC14.1}\\
& \hyperref[UC14.1.1]{UC14.1.1}\\
& \hyperref[UC14.1.1.1]{UC14.1.1.1}\\
& \hyperref[UC14.1.1.1.1]{UC14.1.1.1.1}\\\hline

 \hyperlink{RFO12.1.1.1.2}{RFO12.1.1.1.2} 
 & \hyperlink{Interno}{Interno}\\
& \hyperref[UC14]{UC14}\\
& \hyperref[UC14.1]{UC14.1}\\
& \hyperref[UC14.1.1]{UC14.1.1}\\
& \hyperref[UC14.1.1.1]{UC14.1.1.1}\\
& \hyperref[UC14.1.1.1.2]{UC14.1.1.1.2}\\\hline

 \hyperlink{RFD12.1.1.1.3}{RFD12.1.1.1.3} 
 & \hyperlink{Interno}{Interno}\\
& \hyperref[UC14]{UC14}\\
& \hyperref[UC14.1]{UC14.1}\\
& \hyperref[UC14.1.1]{UC14.1.1}\\
& \hyperref[UC14.1.1.1]{UC14.1.1.1}\\
& \hyperref[UC14.1.1.1.3]{UC14.1.1.1.3}\\\hline

 \hyperlink{RFD12.1.1.1.4}{RFD12.1.1.1.4} 
 & \hyperlink{Interno}{Interno}\\
& \hyperref[UC14]{UC14}\\
& \hyperref[UC14.1]{UC14.1}\\
& \hyperref[UC14.1.1]{UC14.1.1}\\
& \hyperref[UC14.1.1.1]{UC14.1.1.1}\\
& \hyperref[UC14.1.1.1.4]{UC14.1.1.1.4}\\\hline

 \hyperlink{RFO12.1.1.1.5}{RFO12.1.1.1.5} 
 & \hyperlink{Interno}{Interno}\\
& \hyperref[UC14]{UC14}\\
& \hyperref[UC14.1]{UC14.1}\\
& \hyperref[UC14.1.1]{UC14.1.1}\\
& \hyperref[UC14.1.1.1]{UC14.1.1.1}\\
& \hyperref[UC14.1.1.1.5]{UC14.1.1.1.5}\\\hline

 \hyperlink{RFO12.1.1.1.5.1}{RFO12.1.1.1.5.1} 
 & \hyperlink{Interno}{Interno}\\
& \hyperref[UC14]{UC14}\\
& \hyperref[UC14.1]{UC14.1}\\
& \hyperref[UC14.1.1]{UC14.1.1}\\
& \hyperref[UC14.1.1.1]{UC14.1.1.1}\\
& \hyperref[UC14.1.1.1.5]{UC14.1.1.1.5}\\
& \hyperref[UC14.1.1.1.5.1]{UC14.1.1.1.5.1}\\\hline

 \hyperlink{RFO12.1.1.1.5.2}{RFO12.1.1.1.5.2} 
 & \hyperlink{Interno}{Interno}\\
& \hyperref[UC14]{UC14}\\
& \hyperref[UC14.1]{UC14.1}\\
& \hyperref[UC14.1.1]{UC14.1.1}\\
& \hyperref[UC14.1.1.1]{UC14.1.1.1}\\
& \hyperref[UC14.1.1.1.5]{UC14.1.1.1.5}\\
& \hyperref[UC14.1.1.1.5.2]{UC14.1.1.1.5.2}\\\hline

 \hyperlink{RFO12.1.1.1.6}{RFO12.1.1.1.6} 
  & \hyperlink{Capitolato}{Capitolato}\\
& \hyperref[UC14]{UC14}\\
& \hyperref[UC14.1]{UC14.1}\\
& \hyperref[UC14.1.1]{UC14.1.1}\\
& \hyperref[UC14.1.1.1]{UC14.1.1.1}\\
& \hyperref[UC14.1.1.1.6]{UC14.1.1.1.6}\\\hline

 \hyperlink{RFO12.1.1.2}{RFO12.1.1.2} 
 & \hyperlink{Interno}{Interno}\\
& \hyperref[UC14]{UC14}\\
& \hyperref[UC14.1]{UC14.1}\\
& \hyperref[UC14.1.1]{UC14.1.1}\\
& \hyperref[UC14.1.1.2]{UC14.1.1.2}\\\hline

 \hyperlink{RFO12.1.1.2.1}{RFO12.1.1.2.1} 
 & \hyperlink{Interno}{Interno}\\
& \hyperref[UC14]{UC14}\\
& \hyperref[UC14.1]{UC14.1}\\
& \hyperref[UC14.1.1]{UC14.1.1}\\
& \hyperref[UC14.1.1.2]{UC14.1.1.2}\\
& \hyperref[UC14.1.1.2.1]{UC14.1.1.2.1}\\\hline

\hyperlink{RFO12.1.1.2.2}{RFO12.1.1.2.2} 
 & \hyperlink{Interno}{Interno}\\
& \hyperref[UC14]{UC14}\\
& \hyperref[UC14.1]{UC14.1}\\
& \hyperref[UC14.1.1]{UC14.1.1}\\
& \hyperref[UC14.1.1.2]{UC14.1.1.2}\\
& \hyperref[UC14.1.1.2.2]{UC14.1.1.2.2}\\\hline

 \hyperlink{RFO12.1.1.3}{RFO12.1.1.3} 
 & \hyperlink{Interno}{Interno}\\
& \hyperref[UC14]{UC14}\\
& \hyperref[UC14.1]{UC14.1}\\
& \hyperref[UC14.1.1]{UC14.1.1}\\
& \hyperref[UC14.1.1.3]{UC14.1.1.3}\\\hline

 \hyperlink{RFO12.1.1.3.1}{RFO12.1.1.3.1} 
 & \hyperlink{Interno}{Interno}\\
& \hyperref[UC14]{UC14}\\
& \hyperref[UC14.1]{UC14.1}\\
& \hyperref[UC14.1.1]{UC14.1.1}\\
& \hyperref[UC14.1.1.3]{UC14.1.1.3}\\
& \hyperref[UC14.1.1.3.1]{UC14.1.1.3.1}\\\hline

 \hyperlink{RFO12.1.1.3.2}{RFO12.1.1.3.2} 
 & \hyperlink{Interno}{Interno}\\
& \hyperref[UC14]{UC14}\\
& \hyperref[UC14.1]{UC14.1}\\
& \hyperref[UC14.1.1]{UC14.1.1}\\
& \hyperref[UC14.1.1.3]{UC14.1.1.3}\\
& \hyperref[UC14.1.1.3.2]{UC14.1.1.3.2}\\\hline

 \hyperlink{RFO12.1.1.3.3}{RFO12.1.1.3.3} 
 & \hyperlink{Interno}{Interno}\\
& \hyperref[UC14]{UC14}\\
& \hyperref[UC14.1]{UC14.1}\\
& \hyperref[UC14.1.1]{UC14.1.1}\\
& \hyperref[UC14.1.1.3]{UC14.1.1.3}\\
& \hyperref[UC14.1.1.3.3]{UC14.1.1.3.3}\\\hline

 \hyperlink{RFO12.2}{RFO12.2} 
 & \hyperlink{Interno}{Interno}\\
& \hyperref[UC14]{UC14}\\
& \hyperref[UC14.2]{UC14.2}\\\hline

 \hyperlink{RFO12.2.1}{RFO12.2.1} 
 & \hyperlink{Interno}{Interno}\\
& \hyperref[UC14]{UC14}\\
& \hyperref[UC14.2]{UC14.2}\\
& \hyperref[UC14.2.1]{UC14.2.1}\\\hline

\hyperlink{RFO12.2.1.1}{RFO12.2.1.1} 
 & \hyperlink{Interno}{Interno}\\
& \hyperref[UC14]{UC14}\\
& \hyperref[UC14.2]{UC14.2}\\
& \hyperref[UC14.2.1]{UC14.2.1}\\
& \hyperref[UC14.2.1.1]{UC14.2.1.1}\\\hline

\hyperlink{RFO12.2.1.1.1}{RFO12.2.1.1.1} 
 & \hyperlink{Interno}{Interno}\\
& \hyperref[UC14]{UC14}\\
& \hyperref[UC14.2]{UC14.2}\\
& \hyperref[UC14.2.1.1]{UC14.2.1.1}\\
& \hyperref[UC14.2.1.1.1]{UC14.2.1.1.1}\\\hline

\hyperlink{RFO12.2.1.1.2}{RFO12.2.1.1.2} 
 & \hyperlink{Interno}{Interno}\\
& \hyperref[UC14]{UC14}\\
& \hyperref[UC14.2]{UC14.2}\\
& \hyperref[UC14.2.1.1]{UC14.2.1.1}\\
& \hyperref[UC14.2.1.1.2]{UC14.2.1.1.2}\\\hline

 \hyperlink{RFO12.2.1.2}{RFO12.2.1.2} 
 & \hyperlink{Interno}{Interno}\\
& \hyperref[UC14]{UC14}\\
& \hyperref[UC14.2]{UC14.2}\\
& \hyperref[UC14.2.1]{UC14.2.1}\\
& \hyperref[UC14.2.1.2]{UC14.2.1.2}\\\hline

 \hyperlink{RFO12.2.1.2.1}{RFO12.2.1.2.1} 
 & \hyperlink{Interno}{Interno}\\
& \hyperref[UC14]{UC14}\\
& \hyperref[UC14.2]{UC14.2}\\
& \hyperref[UC14.2.1]{UC14.2.1}\\
& \hyperref[UC14.2.1.2]{UC14.2.1.2}\\
& \hyperref[UC14.2.1.2.1]{UC14.2.1.2.1}\\\hline

 \hyperlink{RFO12.2.1.3}{RFO12.2.1.3} 
 & \hyperlink{Interno}{Interno}\\
& \hyperref[UC14]{UC14}\\
& \hyperref[UC14.2]{UC14.2}\\
& \hyperref[UC14.2.1]{UC14.2.1}\\
& \hyperref[UC14.2.1.3]{UC14.2.1.3}\\\hline

  \hyperlink{RFO12.2.1.4}{RFO12.2.1.4} 
 & \hyperlink{Interno}{Interno}\\
& \hyperref[UC14]{UC14}\\
& \hyperref[UC14.2]{UC14.2}\\
& \hyperref[UC14.2.1]{UC14.2.1}\\
& \hyperref[UC14.2.1.4]{UC14.2.1.4}\\\hline

 \hyperlink{RFO12.2.1.5}{RFO12.2.1.5} 
 & \hyperlink{Interno}{Interno}\\
& \hyperref[UC14]{UC14}\\
& \hyperref[UC14.2]{UC14.2}\\
& \hyperref[UC14.2.1]{UC14.2.1}\\
& \hyperref[UC14.2.1.5]{UC14.2.1.5}\\\hline

 \hyperlink{RFO12.2.1.5.1}{RFO12.2.1.5.1} 
 & \hyperlink{Interno}{Interno}\\
& \hyperref[UC14]{UC14}\\
& \hyperref[UC14.2]{UC14.2}\\
& \hyperref[UC14.2.1]{UC14.2.1}\\
& \hyperref[UC14.2.1.5]{UC14.2.1.5}\\
& \hyperref[UC14.2.1.5.1]{UC14.2.1.5.1}\\\hline

 \hyperlink{RFO12.2.1.5.2}{RFO12.2.1.5.2} 
 & \hyperlink{Interno}{Interno}\\
& \hyperref[UC14]{UC14}\\
& \hyperref[UC14.2]{UC14.2}\\
& \hyperref[UC14.2.1]{UC14.2.1}\\
& \hyperref[UC14.2.1.5]{UC14.2.1.5}\\
& \hyperref[UC14.2.1.5.2]{UC14.2.1.5.2}\\\hline

\hyperlink{RFO13}{RFO13} & \hyperlink{Capitolato}{Capitolato}\\
\hline

 \hyperlink{RQO1}{RQO1} & \hyperlink{Capitolato}{Capitolato}\\
\hline

 \hyperlink{RQO2}{RQO2} & \hyperlink{Capitolato}{Capitolato}\\
\hline

 \hyperlink{RQO3}{RQO3} & \hyperlink{Capitolato}{Capitolato}\\
\hline

 \hyperlink{RQO4}{RQO4} & \hyperlink{Capitolato}{Capitolato}\\
\hline

 \hyperlink{RQO5}{RQO5} & \hyperlink{Capitolato}{Capitolato}\\
\hline

 \hyperlink{RQO6}{RQO6} & \hyperlink{Capitolato}{Capitolato}\\
\hline

 \hyperlink{RQO7}{RQO7} & \hyperlink{Capitolato}{Capitolato}\\
\hline

 \hyperlink{RQD8}{RQD8} & \hyperlink{Capitolato}{Capitolato}\\
\hline

 \hyperlink{RVO1}{RVO1} & \hyperlink{Capitolato}{Capitolato}\\
\hline

 \hyperlink{RVO2}{RVO2} & \hyperlink{Capitolato}{Capitolato}\\
\hline

 \hyperlink{RVO3}{RVO3} & \hyperlink{Capitolato}{Capitolato}\\
\hline

 \hyperlink{RVO4}{RVO4} & \hyperlink{Capitolato}{Capitolato}\\
\hline

\caption[Tracciamento Requisiti-Fonti]{Tracciamento Requisiti-Fonti}
\label{tabella:requi-fonti}
\end{longtable}
\clearpage

\input{sezioni/TabelleRequisiti/tracciamentoFontiRequisiti.tex}
\input{sezioni/TabelleRequisiti/tracciamentoRequisitiFonti.tex}
\input{sezioni/TabelleRequisiti/riepilogoRequisiti.tex}