\subsection{Requisiti di qualità}
\begin{longtable}{|c|m{8cm}|c|}
\caption{Tabella dei requisiti di qualità} \\

\hline
\thead*{\textbf{Codice requisito}} &\thead{\textbf{Descrizione}}  &\thead{\textbf{Fonte}} \\
\hline
\endhead

\hline
\endfoot
\hline
\endlastfoot

\hypertarget{RQO1}{RQO1} &  Nel caso di più gruppi per questo progetto, le parti del prodotto finale dovranno essere componibili ed integrabili tra loro & \makecell*{Capitolato} \\
\hline

\hypertarget{RQO2}{RQO2} & Per ogni servizio, deve essere fornita la descrizione di ogni servizio e delle singole API, l'interfaccia delle API, lo schema design relativo all'eventuale base di dati associata & \makecell*{Capitolato} \\
\hline

\hypertarget{RQO3}{RQO3} & Deve essere fornito il sequence chart diagram delle interazioni che prevedono il coinvolgimento di più microservizi & \makecell*{Capitolato} \\
\hline

\hypertarget{RQO4}{RQO4} & Devono essere forniti gli algoritmi delle policy per l'utilizzo delle API & \makecell*{Capitolato} \\
\hline

\hypertarget{RQO5}{RQO5} & Deve essere fornito l'algoritmo di generazione delle API key & \makecell*{Capitolato} \\
\hline

\hypertarget{RQO6}{RQO6} & Il prodotto finale deve superare i test forniti da ItalianaSoftware & \makecell*{Capitolato} \\
\hline

\hypertarget{RQO7}{RQO7} & Il prodotto finale deve essere depositato su una repository git &\makecell*{Capitolato} \\
\hline

\hypertarget{RQO8}{RQO8} & Deve essere stilato breve report tecnico che evidenzi gli aspetti positivi e negativi di un'architettura a microservizi  &\makecell*{Capitolato} \\
\hline

\end{longtable}
