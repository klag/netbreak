\newpage
\section{Introduzione}

\subsection{Scopo del documento}
Lo scopo del presente documento è la definizione ad alto livello della struttura che dovrà avere il prodotto \textit{API Market\ped{G}}. Verranno presentati la gerarchia dei packages, le principali classi contenute in essi, le interazioni fra esse, e tutti i design pattern utilizzati. Il documento servirà come guida per i \textit{\Progrs} del team \textit{\gruppo} durante le successive attività di Codifica.

\subsection{Scopo del prodotto}
Lo scopo del prodotto è la realizzazione di un \progetto\ per l'acquisto e la vendita di \textit{microservizi\ped{G}}. Il sistema offrirà la possibilità di registrare nuove \textit{API\ped{G}} per la vendita, permetterà la consultazione e la ricerca di API ai potenziali acquirenti, gestendo i permessi di accesso ed utilizzo tramite creazione e controllo di relative \textit{API key\ped{G}}. Il sistema, oltre alla web app stessa, sarà corredato di un \textit{API Gateway\ped{G}} per la gestione delle richieste e il controllo delle chiavi, e fornirà funzionalità avanzate di statistiche per il gestore della piattaforma e per i fornitori dei microservizi.

\subsection{Riferimenti normativi}
\begin{itemize}
	\item \textsc{NormeDiProgetto 2\_0\_0.pdf}
	\item \textsc{AnalisiDeiRequisiti 2\_0\_0.pdf}
\end{itemize}

\subsection{Riferimenti informativi}
\begin{itemize}
	\item \textbf{Ingegneria del software - Ian Sommerville - 8a edizione:}\\
	Porzione dedicata alla Progettazione Architetturale (Capitolo 11)
	\item \textbf{Slides del corso - Design patterns strutturali}\\
	\url{http://www.math.unipd.it/~tullio/IS-1/2016/Dispense/E04.pdf}
	\item \textbf{Slides del corso - Design patterns creazionali}\\
	\url{http://www.math.unipd.it/~tullio/IS-1/2016/Dispense/E05.pdf}
	\item \textbf{Slides del corso - Design patterns architetturali}\\
	\url{http://www.math.unipd.it/~tullio/IS-1/2016/Dispense/E08.pdf}\\
	\url{http://www.math.unipd.it/~tullio/IS-1/2016/Dispense/E09.pdf}
	\item \textbf{Slides del corso - Diagrammi dei packages}\\
	\url{http://www.math.unipd.it/~tullio/IS-1/2016/Dispense/E02b.pdf}
	\item \textbf{Slides del corso - Diagrammi delle classi}\\
	\url{http://www.math.unipd.it/~tullio/IS-1/2016/Dispense/E02a.pdf}
	\item \textbf{Slides del corso - Diagrammi di sequenza}\\
	\url{http://www.math.unipd.it/~tullio/IS-1/2016/Dispense/E03a.pdf}
	\item \textbf{Slides del corso - Diagrammi di attività}\\
	\url{http://www.math.unipd.it/~tullio/IS-1/2016/Dispense/E03b.pdf}
	\item \textbf{Risorsa online sui microservizi}\\
	\url{http://microservices.io}
	\item \textbf{Design pattern a microservizi}\\
	\url{http://microservices.io/patterns/microservices.html}
	\item \textbf{HTML5 W3C}\\
	\url{https://www.w3.org/TR/html5/}
	\item \textbf{Jolie Programming Language}\\
	\url{http://www.jolie-lang.org/}
	\item \textbf{Java Programming Language}\\
	\url{https://www.oracle.com/it/java/index.html}
	\item \textbf{Bootstrap CSS Framework}\\
	\url{http://getbootstrap.com/}
	\item \textbf{JQuery Javascript Library}\\
	\url{https://jquery.com/}
	\item \textbf{Angular 2}\\
	\url{https://www.angularjs.org/}
	\item \textbf{PostgreSQL Open-source Database}\\
	\url{https://www.postgresql.org/}
	\item \textbf{Meteor Framework}\\
	\url{https://www.meteor.com/}
\end{itemize}

\subsection{Glossario}
Per semplificare la consultazione e disambiguare alcune terminologie tecniche, le voci indicate con la lettera \textit{G} a pedice sono descritte approfonditamente nel documento \textsc{Glossario 2\_0\_0.pdf} e specificate solo alla prima occorrenza all'interno del suddetto documento.