

\begin{figure}[H]
	\centering
	\includegraphics
	[width=0.7\linewidth]
	{UML/DiagrammiPackage/Model.png}
	\caption{Package APIM::BackEnd::Model}
\end{figure}

Il package Model contiene i package utilizzati per rappresentare i dati e l'implementazione della logica di bussiness e di validazione.
\begin{itemize}
	\item \textbf{Pubblisher}: questo package contiene le classi per la pubblicazione dei dati.
\end{itemize}

\begin{itemize}
	\item \textbf{WebServices}: questo package contiene le classi per la comunicazione col database.
\end{itemize}


\subsubsection{Publisher}
\begin{figure}[H]
	\centering
	\includegraphics
	[width=0.7\linewidth]
	{UML/DiagrammiPackage/Publisher.png}
	\caption{Package APIM::BackEnd::Model::Publisher}
\end{figure}

Il package Publisher contiene le classi necessarie ad esporre il modello dati al frontend.

\paragraph{APISubscriber}
\begin{itemize}
	\item \textbf{Funzione del componente}: esegue la pubblicazione delle API.
	\item \textbf{Attivita’ svolte e dati trattati}: permette l'accesso alla collezione di API.
\end{itemize}

\paragraph{TransactionSubscriber}
\begin{itemize}
	\item \textbf{Funzione del componente}: esegue la pubblicazione delle transazioni.
	\item \textbf{Attivita’ svolte e dati trattati}: permette l'accesso alla collezione di transazioni.
\end{itemize}

\paragraph{UserSubscriber}
\begin{itemize}
	\item \textbf{Funzione del componente}: esegue la pubblicazione degli utenti.
	\item \textbf{Attivita’ svolte e dati trattati}: permette l'accesso alla collezione di utenti.
\end{itemize}



\subsubsection{WebServices}

\begin{figure}[H]
	\centering
	\includegraphics
	[width=0.7\linewidth]
	{UML/DiagrammiPackage/WebServices.png}
	\caption{Package APIM::BackEnd::Model::WebServices}
\end{figure}

Il package WebServices contiene i microservizi utilizzati per mettere in comunicazione il database col Model.

\paragraph{UserService}
\begin{itemize}
	\item \textbf{Funzione del componente}: esegue le funzioni di inserimento, modifica e rimozione di utenti.
	\item \textbf{Relazioni d’uso di altri componenti}: si interfaccia con la ViewModel e il database per fornire le funzioni di inserimento, modifica e rimozione di utenti.
\end{itemize}

\paragraph{APIService}
\begin{itemize}
	\item \textbf{Funzione del componente}: esegue le funzioni di inserimento, modifica e rimozione di API.
\item \textbf{Relazioni d’uso di altri componenti}: si interfaccia con la ViewModel e il database per fornire le funzioni di inserimento, modifica e rimozione di API.
\end{itemize}

\paragraph{TransactionService}
\begin{itemize}
	\item \textbf{Funzione del componente}: esegue le funzioni di inserimento, modifica e rimozione di transazioni.
\item \textbf{Relazioni d’uso di altri componenti}: si interfaccia con la ViewModel e il database per fornire le funzioni di inserimento, modifica e rimozione di transazioni.
\end{itemize}


\paragraph{SLAService}
\begin{itemize}
	\item \textbf{Funzione del componente}: esegue le funzioni di inserimento, modifica e rimozione di log di SLA.
\item \textbf{Relazioni d’uso di altri componenti}: si interfaccia con la ViewModel e il database per fornire le funzioni di inserimento, modifica e rimozione di log di SLA.
\end{itemize}

\paragraph{KeyManagerService}
\begin{itemize}
	\item \textbf{Funzione del componente}: esegue le funzioni di inserimento, modifica e rimozione di API key.
\item \textbf{Relazioni d’uso di altri componenti}: si interfaccia con la ViewModel e il database per fornire le funzioni di inserimento, modifica e rimozione di API key.
\end{itemize}

