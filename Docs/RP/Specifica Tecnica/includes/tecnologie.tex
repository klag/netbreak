\newpage
\section{Tecnologie utilizzate}
In questa sezione, vengono descritte le tecnologie utilizzate per la realizzazione della piattaforma. Verranno elencati i pregi e i difetti individuati durante l'analisi, e le motivazioni che hanno spinto il \textit{gruppo NetBreak} a intraprendere tali scelte progettuali e tecnologiche.

\subsection{Angular 2}
\begin{itemize}
	\item \textbf{Descrizione}: AngularJS è un framework Javascript per lo sviluppo di applicazioni web Client-side. Può essere aggiunto alle pagine HTML tramite opportune inclusioni. E' distribuito come un file JavaScript
	\item \textbf{Utilizzo}: Questa tecnologia verrà utilizzata per rendere le pagine HTML reattive e rispondenti alle richieste in real-time.
	\item \textbf{Vantaggi}: 
	\begin{itemize}
		\item Interazioni in tempo reale con l'utente per pagine dinamiche single-page
		\item Facilita e snellisce lo sviluppo
		\item Risulta facile da testare
		\item Estende HTML
		\item Offre la possibilità di un data-binding bidirezionale
	\end{itemize}
	\item \textbf{Svantaggi}:
	\begin{itemize}
		\item Semplice da imparare ma con una curva di apprendimento esponenziale
		\item Potrebbe essere complicato un debug degli errori
		\item Documentazione non particolarmente sviluppata rispetto a tecnologie più consolidate
	\end{itemize}
\end{itemize}

\subsection{Bootstrap 3}
\begin{itemize}
	\item \textbf{Descrizione}: Bootstrap 3 è un framework CSS3 che viene utilizzato per semplificare la definizione di un interfaccia grafica sulla struttura HTML.
	\item \textbf{Utilizzo}: Questa tecnologia verrà utilizzata per la realizzazione dell'interfaccia grafica per il prodotto API Market.
	\item \textbf{Vantaggi}:
	\begin{itemize}
		\item Supporta la maggior parte dei browser
		\item Permette un ampio riuso del codice
		\item E' basato su SASS, un diffuso pre-processore per il linguaggio CSS
		\item La documentazione è vasta e completa, ed è ben supportato dalla community
		\item Leggero e facilmente personalizzabile
		\item Pensato per un modello responsive, con griglie e breakpoint predisposti
	\end{itemize}
	\item \textbf{Svantaggi}:
	\begin{itemize}
		\item La scrittura dello stile può risultare verbosa e può generare HTML poco elegante
		\item I siti realizzati con questo framework, se non personalizzati, risultano molto similari tra loro
	\end{itemize}
\end{itemize}

\subsection{CSS3}
\begin{itemize}
	\item \textbf{Descrizione}: CSS3 è la principale e più aggiornata tecnologia, e viene utilizzato per la descrizione della componente grafica di un sito.
	\item \textbf{Utilizzo}: Questa tecnologia viene utilizzata per la personalizzazione dell'aspetto grafico precedentemente definito tramite Bootstrap
	\item \textbf{Vantaggi}:
	\begin{itemize}
		\item Rappresenta l'ultima versione stabile di questo linguaggio
		\item Aiuta nella creazione di pagine con lo stesso stile e formato
		\item E' supportato ampiamente dalla maggior parte dei browser
		\item Risulta facile da apprendere
	\end{itemize}
	\item \textbf{Svantaggi}:
	\begin{itemize}
		\item L'uso di questa tecnologia può rendere caotica la creazione di siti utilizzando software di terze parti
	\end{itemize}
\end{itemize}

\subsection{HTML5}
\begin{itemize}
	\item \textbf{Descrizione}: HTML è il linguaggio di markup principale per la definizione di una struttura per un sito
	\item \textbf{Utilizzo}: Questa tecnologia verrà utilizzata per descrivere la struttura delle pagine front-end
	\item \textbf{Vantaggi}:
	\begin{itemize}
		\item L'ultima versione adottata ha introdotto significativi miglioramenti riguardo la pulizia del codice
		\item Sono state introdotte funzionalità avanzate, assenti nelle versioni precedenti
		\item La semantica è stata standardizzata in maniera più rigorosa
	\end{itemize}
	\item \textbf{Svantaggi}:
	\begin{itemize}
		\item Il supporto all'ultima versione di questo standard potrebbe non essere completamente definita in browser non aggiornati
	\end{itemize}
\end{itemize}

\subsection{Java}
\begin{itemize}
	\item \textbf{Descrizione}:
	\item \textbf{Utilizzo}:
	\item \textbf{Vantaggi}:
	\item \textbf{Svantaggi}:
\end{itemize}

\subsection{Javascript ES6}
\begin{itemize}
	\item \textbf{Descrizione}:
	\item \textbf{Utilizzo}:
	\item \textbf{Vantaggi}:
	\item \textbf{Svantaggi}:
\end{itemize}

\subsection{Jolie}
\begin{itemize}
	\item \textbf{Descrizione}:
	\item \textbf{Utilizzo}:
	\item \textbf{Vantaggi}:
	\item \textbf{Svantaggi}:
\end{itemize}

\subsection{JQuery}
\begin{itemize}
	\item \textbf{Descrizione}:
	\item \textbf{Utilizzo}:
	\item \textbf{Vantaggi}:
	\item \textbf{Svantaggi}:
\end{itemize}

\subsection{Leonardo}
\begin{itemize}
	\item \textbf{Descrizione}:
	\item \textbf{Utilizzo}:
	\item \textbf{Vantaggi}:
	\item \textbf{Svantaggi}:
\end{itemize}

\subsection{MeteorJS}
\begin{itemize}
	\item \textbf{Descrizione}:
	\item \textbf{Utilizzo}:
	\item \textbf{Vantaggi}:
	\item \textbf{Svantaggi}:
\end{itemize}

\subsection{PostgreSQL}
\begin{itemize}
	\item \textbf{Descrizione}:
	\item \textbf{Utilizzo}:
	\item \textbf{Vantaggi}:
	\item \textbf{Svantaggi}:
\end{itemize}



