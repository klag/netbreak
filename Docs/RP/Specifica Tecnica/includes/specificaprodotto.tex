\newpage
\section{Specifica del prodotto}
In questa sezione verrà descritta in maniera macroscopica l'architettura che verrà adottata nella progettazione. Per semplificare la comprensione, verrà utilizzato un approccio top-down partendo dalle linee generali fino a giungere ad una descrizione in dettaglio delle componenti.

\subsection{Architettura generale}
Il software API Market si caratterizza dalle comuni applicazioni per l'approccio a Microservizi adottato nell'implementazione. In maniera differente da un applicazione monolitica, ogni componente viene realizzato come microservizio a sè stante. L'aggregazione di questi microservizi definisce l'applicazione finale e ne realizza il comportamento definitivo. Esulando dall'approccio a microservizi, il sistema sarà realizzato al pari di una classica applicazione \textit{Client-Server}, dove il lato Front-end (Client) si occuperà di fornire all'utente finale la piattaforma su cui poter interagire, mentre il lato Back-end (Server) gestirà e salverà i dati, e si occuperà della gestione delle chiamate API (Tramite l'opportuna componente \textit{API Gateway}). La base di dati utilizzata si occupa della raccolta di dati sensibili dell'utenza, gestione delle applicazioni e chiavi e gestione dei dati statistici.

Il sistema verrà realizzato, per il lato back-end, tramite microservizi con interfaccia Jolie che verranno implementati con codice Java. Questo approccio, sebbene la tematica sia di recente sviluppo, risulta molto gettonato dalle grandi realtà che si sono affacciate a questo approccio (come ad esempio l'AWS Service registry "Eureka" usato dalla piattaforma Netflix). Per il lato front-end invece, l'applicazione verrà realizzata tramite chiamate AJAX per l'aggiornamento dinamico della pagina. Tale metodologia di sviluppo verrà implementata tramite HTML e Javascript (tramite opportune librerie JQuery e framework Angular 2).

\subsubsection{Pattern architetturale front-end}

\subsubsection{Pattern architetturale back-end}

