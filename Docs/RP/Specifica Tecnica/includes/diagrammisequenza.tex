\newpage
\section{Diagrammi di sequenza}
\subsection{Introduzione}
Un diagramma di sequenza è un diagramma previsto dall'UML utilizzato per descrivere uno scenario.
Uno scenario è una determinata sequenza di azioni in cui tutte le scelte sono state già effettuate; in pratica, nel diagramma non compaiono nè scelte, né flussi alternativi.
Il gruppo ha deciso di prendere in esame la situazione di chiamata di un microservizio passante per l'API Gateway, suddividendo la sequenza in due parti.

\subsection{Diagramma di sequenza relativa alla chiamata di una API}
\subsubsection{Diagramma di sequenza di gestione della chiamata}
\begin{figure}[h]
	\centering
	\includegraphics[width=1.0\linewidth]{"IMG/Sequence Diagram0"}
	\caption{Diagramma di sequenza di gestione della chiamata}
\end{figure}

\begin{itemize}
	\item \textbf{Precondizioni}: il client ha effettuato una chiamata ad un'API;
	\item \textbf{Postcondizioni}: la richiesta arriva all'\textit{API Gateway Manager} che la processa;
	\item \textbf{Descrizione}: il client effettua una chiamata, creando un oggetto \textit{APICall}. L'oggetto \textit{APICall} viene aggiunto alla coda \textit{QueueAPICall} in attesa di essere prelevato dall'\textit{API Gateway Manager} per la sua esecuzione. Quando la chiamata verrà processata i suoi dati verranno salvati nel \textit{Buffer}.
\end{itemize}

\newpage
\subsubsection{Diagramma di sequenza dell'API Gateway Manager}
\begin{figure}[h]
	\centering
	\includegraphics[width=1.0\linewidth]{"IMG/Sequence Diagram1"}
	\caption{Diagramma di sequenza dell'API Gateway Manager}
\end{figure}


\begin{itemize}
	\item \textbf{Precondizioni}: la richiesta arriva all'\textit{API Gateway Manager} che la processa;
	\item \textbf{Postcondizioni}: il client riceve la risposta alla sua chiamata e i dati relativi alla SLA, se abilitata, vengono salvati nel database;
	\item \textbf{Descrizione}: l'\textit{API Gateway Manager }invoca i metodi del \textit{Verifier} per eseguire i controlli necessari, al fine di verificare l'autenticità della chiamata. All'interno del \textit{Verifier} viene deciso, in modo casuale, se effettuare il log della SLA.
	Conclusi questi controlli preliminari con successo, il \textit{Manager} invoca i metodi di \textit{RequestForwarder} che si occupano di recuperare l'indirizzo dell'API invocata e salvarlo nel \textit{Service Registry}, effettuare un salvataggio relativo alla SLA e concludere la sua esecuzione con la chiamata all'\textit{API target}. Il \textit{Manager} invoca i metodi del \textit{ResponseForwarder}, il quale si occupa di effettuare il salvataggio relativo alla SLA, diminuire il campo \textit{Residue} nel database e inviare la risposta al client chiamante. Infine, il \textit{Manager} si occupa di salvare il record della SLA nel database.  
\end{itemize}
