\newpage

\section{Processi organizzativi}

	\subsection{Gestione Organizzativa}
		
		\subsubsection{Comunicazioni}
		Al fine di ottenere comunicazioni chiare e concise il \textit{\RdP} ha l'obbligo di gestirle in modo strutturato, utilizzando la forma che meglio si adatta alla situazione. Le comunicazioni possono essere interne o esterne al team.
			
			\paragraph{Comunicazioni interne}
			\subparagraph{Descrizione}
			Le comunicazioni interne sono uno strumento ad uso esclusivo dei membri del gruppo e possono essere in stile informale o formale, in forma scritta o orale. 
				\begin{itemize}
					\item \textbf{Comunicazione formale scritta:} è un'informazione rilevante per i membri del gruppo, deve essere utilizzata quando è di fondamentale importanza per la corretta gestione del progetto;
					\item \textbf{Comunicazione formale orale:} questa modalità sarà utilizzata durante le presentazioni o in occasione di incontri con il proponente e/o il committente. Se la comunicazione con questi ultimi ha particolare importanza per lo svolgimento del progetto, deve essere scritta e diventare un'informazione rilevante per i membri del team.
					\item \textbf{Comunicazione informale scritta:} scambio di messaggi tra i membri del team mediante l'utilizzo dell'applicazione di instant messaging \textit{Telegram\ped{G}}, sulla quale è presente la chat del gruppo, utile per chiarimenti o precisazioni.
					\item \textbf{Comunicazione informale orale:} colloquio che avviene tra i membri del team e possono riguardare opinioni e informazioni sulle attività del progetto. 
				\end{itemize}

			\paragraph{Comunicazioni esterne}
			Per tutte le comunicazioni esterne, il gruppo \textit{\gruppo} si è dotato di una email ufficiale, necessaria per gestire le comunicazioni con committente e proponente. La gestione di tale casella di posta elettronica è affidata al \textit{\RdP}, il quale ha il compito di tenere informati i membri del gruppo riguardo le comunicazioni importanti. La casella di posta elettronica è:
			\begin{center}
				\url{netbreakswe@gmail.com} 
			\end{center}
		Le email ufficiali devono rispettare alcune linee guida, esposte di seguito.
		\begin{itemize}
			\item \textbf{Destinatario:} indica il destinatario dell'email e se non è già presente nei contatti, è necessario salvarlo, tramite la funzione automatica di \textit{Gmail\ped{G}};
			\item \textbf{Oggetto:} l'oggetto dell'email deve esprimere in modo chiaro e breve il contenuto dell'email;
			\item \textbf{Corpo:} il corpo del messaggio è composto dal contenuto informativo che il mittente vuol comunicare ai destinatari;
			\item \textbf{Allegati:} un allegato è un file di testo, audio o video inviato insieme a una email. La dimensione del file influisce sulla dimensione totale dell'email e non è possibile inviare allegati sopra determinate dimensioni espresse in megabyte.
		\end{itemize}
		\paragraph{Coordinamento con secondo gruppo}
			Il proponente richiede una collaborazione tra i gruppi intenzionati a sviluppare il progetto proposto. In particolare, \proponente\ richiede un'intesa sull'architettura di massima del progetto. Per agevolare la comunicazione tra i gruppi è stato creato un gruppo su Telegram per il coordinamento del capitolato C1.
			
		\subsubsection{Riunioni}
		Le riunioni sono un elemento essenziale per il corretto svolgimento del progetto. Il loro compito è quello di permettere a tutti i membri del gruppo di confrontarsi tra loro, nel caso di riunioni interne, e con il proponente e il committente, nel caso di riunione esterne. La corretta gestione dell'andamento di una riunione permette di risparmiare tempo e sfruttare al meglio le risorse umane.
		
			\paragraph{Riunioni interne}
			Il \textit{\RdP} ha la facoltà di decidere e convocare una riunione del gruppo. Tale convocazione deve essere inviata tramite posta elettronica a tutti i membri del gruppo, con almeno due giorni di anticipo, e deve contenere l'ordine del giorno. Tutti i membri del gruppo hanno il dovere di confermare la loro presenza entro ventiquattro ore dalla ricezione della comunicazione. Se impossibilitati a partecipare, devono fornire al \textit{\RdP} una valida motivazione alla loro assenza. Il \textit{\RdP}, inoltre, ha il compito di dirigere la riunione, tenendo fede all'ordine del giorno descritto nella mail di invito ed evitare perdite di tempo. Tutti i partecipanti devono presentarsi alla riunione preparati, aver rispettato le scadenze decise nella precedente riunione e caricato il materiale assegnato sulla piattaforma di web storage indicata. Le scadenze sono visibili sull'applicazione multi piattaforma \textit{Asana\ped{G}}.
			All’inizio di ogni riunione il \textit{\RdP} nominerà un segretario, che avrà il compito di tenere la minuta dell'incontro, stilare un verbale informale a disposizione esclusivamente del gruppo e creare le attività su Asana con relative scadenze concordate.  
			Le riunioni devono avere un preciso scopo e in base ad esso possono avvenire tramite \textit{Skype\ped{G}} oppure devono essere svolte di persona. Le riunioni si dividono in quattro tipi:
			\begin{itemize}
				\item \textbf{informative:} il \textit{\RdP} informa il gruppo riguardo informazioni di carattere generale, comunicazioni ricevute da committente e/o proponente, prossime scadenze e piano di lavoro per il periodo attuale. Queste riunioni possono essere tenute su Skype;
				\item \textbf{valutative:} il \textit{\RdP} convoca la riunione per valutare insieme ai membri del team alcuni aspetti dello svolgimento del progetto in seguito a eventi particolari, quali nuove scadenze da rispettare oppure comunicazioni rilevanti da proponente e/o committente. \MakeUppercase{è} possibile svolgerle su Skype;
				\item \textbf{decisionali:} queste riunioni non posso avvenire tramite Skype, ma richiedono un incontro fisico tra i componenti. In queste riunioni, si prendono decisioni rilevanti per lo svolgimento del progetto e, quindi, la presenza fisica dei membri è necessaria per una corretta decisione;
				\item \textbf{di progettazione:} queste riunioni richiedono un incontro fisico del gruppo e riguardano l'attività di progettazione del prodotto \progetto. 
			\end{itemize}
		
			\paragraph{Riunioni esterne}
			Questo tipo di riunioni è fondamentale per il corretto svolgimento e progessivo andamento del progetto.
			Il \textit{\RdP} è una figura chiave in questi incontri, in quanto deve svolgere un numero di compiti notevole.
			Nei giorni precedenti alla riunione, sarà compito del \textit{\RdP} stilare un documento condiviso con l'ordine del giorno, le domande e/o i chiarimenti da esporre. Se la riunione è con il proponente e richiede di incontrare entrambi i gruppi al lavoro sul capitolato scelto, il \textit{\RdP} deve contattare il responsabile del secondo gruppo per accordarsi sullo svolgimento dell'incontro. Prima di ogni riunione, il \textit{\RdP} nominerà un segretario, incaricato di prendere nota dei temi discussi e redigere un verbale, evidenziando gli aspetti più rilevanti e/o le modifiche da apportare. Tale documento è riservato ad un uso esclusivo del team.
			A questi incontri, parteciperanno tutti i membri del gruppo, salvo impedimenti di rilievo, e sarà compito del \textit{\RdP} coordinare e gestire l'interazione tra il gruppo e l'interlocutore. Durante la riunione, i membri del team potranno effettuare domande e/o chiedere chiarimenti. 
			
			\paragraph{Verbali}
			Il gruppo ha deciso di redigere dei verbali per ogni riunione che viene effettuata. I verbali sono un documento interno al gruppo, utile per tenere traccia degli argomenti discussi e delle decisioni prese. Il team ha deciso di dividere i verbali in due categorie e di indicare all'inizio di ogni verbale se la riunione è privata, ovvero sono presenti solo i membri del gruppo, oppure se è presente anche un membro esterno (ad esempio, proponente o committente). All'inizio della riunione, il \textit{\RdP} nomina un segretario, il quale ha il compito di procedere alla stesura del documento. Il documento contiene un abstract che indica il motivo della convocazione della riunione, un elenco puntato con informazioni sulla riunione e un riassunto dei temi trattati. L'elenco contiene le seguenti informazioni:
			\begin{itemize}
				\item Tipologia verbale;
				\item Generalità del segretario;
				\item Data di svolgimento della riunione;
				\item Ordine del giorno;
				\item Luogo di svolgimento della riunione. Le riunioni possono avvenire sia fisicamente che virtualmente, tramite Skype. Quest'ultimo metodo sarà utilizzato per brevi comunicazioni e aggiornamenti sullo svolgimento del progetto;
				\item Orario inizio riunione;
				\item Orario fine riunione;
				\item Membri presenti e relativo ruolo ricoperto;
				\item Membri assenti alla riunione;
				\item Riassunto riunione.
			\end{itemize}

	\subsection{Ruoli e mansioni}
	Nel corso del progetto, ogni membro del team ricoprirà diverse mansioni. I ruoli sono standard, stabiliti con criterio e con mansioni mutuamente esclusive. Lo scopo di questa sezione è descrivere tutti i possibili ruoli che un membro del team può assumere durante lo svolgimento del progetto. Essi verranno organizzati in modo tale da:
	\begin{itemize}
		\item Evitare conflitti di interesse, quali concomitanza dei ruoli di stesura e verifica di uno stesso documento;
		\item Ruotare i ruoli per permettere a ciascun membro di assumere qualsiasi posizione durante il corso del progetto;
		\item Permettere ad una persona di assumere più mansioni contemporaneamente.
	\end{itemize}

		\subsubsection{\RdP}
		Il \textit{\RdP} organizza il lavoro interno del team, occupandosi della pianificazione delle attività. Egli mantiene i contatti con tutti gli enti esterni, gestisce le risorse, i rischi e la documentazione. Di quest'ultima, è l'unico incaricato di apporre l'approvazione definitiva. Le ulteriori mansioni organizzative del \textit{\RdP} sono quelle di garantire il rispetto dei ruoli ed assicurarsi che le attività assegnate vengano svolte scrupolosamente nei tempi stabiliti secondo i canoni stabiliti nel documento \textsc{NormeDiProgetto 2\_0\_0.pdf}. I documenti di sua diretta responsabilità sono \textsc{PianoDiProgetto 2\_0\_0.pdf} e \textsc{PianoDiQualifica 2\_0\_0.pdf}.

		\subsubsection{\Amm}
		L'\textit{\Amm} svolge una attività di controllo sull'ambiente di lavoro come mansione principale. Altre dirette responsabilità dell'\textit{\Amm} sono relative al versionamento di prodotto e alla documentazione. Il documento di sua diretta responsabilità è \textsc{NormeDiProgetto 2\_0\_0.pdf}, ma collabora anche alla redazione del documento \textsc{PianoDiProgetto 2\_0\_0.pdf}.
		
		\subsubsection{\Ana}
		La figura di \textit{\Ana} ha il ruolo di esaminare caratteristiche e problematiche del prodotto. Si occupa, dunque, dell'attività di analisi per ogni aspetto del progetto. Si occupa direttamente dei documenti \textsc{StudioDiFattibiltà 1\_0\_0.pdf} e \textsc{AnalisiDeiRequisiti 2\_0\_0.pdf}, mentre collabora alla stesura del documento \textsc{PianoDiQualifica 2\_0\_0.pdf}.

		\subsubsection{\Prog}
		Il ruolo di \textit{\Prog} è una mansione che ha il compito primario di realizzare una produzione astratta del progetto, effettuando le scelte necessarie a mantenere il prodotto modulare, di facile manutenzione e ampliamento. Il \textit{\Prog} si occupa direttamente della stesura dei documenti \textsc{SpecificaTecnica 1\_0\_0.pdf}, \textit{DefinizioneDiProdotto} e della parte di metriche del documento \textsc{PianoDiQualifica 2\_0\_0.pdf}.
		
		\subsubsection{\Progr}
		Il \textit{\Progr} svolge l'attività di produzione di codice che implementa la soluzione finale progettata. Egli ha il compito di seguire in maniera attenta la progettazione pregressa, effettuando la codifica seguendo le convenzioni descritte nel documento \textsc{NormeDiProgetto 2\_0\_0.pdf}. Si occupa, inoltre, di predisporre i test per le successive attività di Verifica, e di documentare quanto viene prodotto.
		Il \textit{\Progr} si occupa direttamente della stesura del documento \textsc{Manuale Utente} e della documentazione relativa al codice.
		
		\subsubsection{\Ver}
		Il \textit{\Ver} svolge l'attività di Verifica di tutta la documentazione prodotta. Si attiene scrupolosamente a \textsc{NormeDiProgetto 2\_0\_0.pdf}, al fine di assicurarsi che i documenti prodotti siano conformi agli standard e alle convenzioni prefissate. Inoltre, si occupa di redigere la sezione del documento \textsc{PianoDiQualifica 2\_0\_0.pdf}, il quale illustra il resoconto delle verifiche effettuate sui vari documenti prodotti.

	\subsection{Strumenti}

		\subsubsection{GitHub}
		Per il controllo di versione e la gestione semplificata dei file di progetto e della documentazione, si è deciso di utilizzare \textbf{GitHub}. La scelta è ricaduta su questo strumento, poiché altre piattaforme prese in considerazione non consentono la collaborazione di sei membri o più in modo gratuito. Il team lavorerà, dunque, su un repository pubblico. I membri del team potranno, tramite questa soluzione, collaborare simultaneamente, sincronizzando le versioni in tempo reale o in un secondo momento, qualora mancasse la connessione. Infatti, è possibile lavorare sulla copia locale della repository e caricare le modifiche una volta ristabilita la connessione con il server centrale. Durante la produzione dei documenti tramite TeXstudio, vengono generati diversi file, non utili ai fini del progetto e la creazione sulla repository di un file \textit{.gitignore}, il quale evita il caricamento di file locali con determinate estensioni, aiutando a mantenere la repository ordinata e di facile fruizione. GitHub offre, inoltre, una sezione wiki per la gestione di eventuale documentazione, e un sistema di issue tracking sufficientemente performante ai fini del progetto. 
		La struttura base della repository contenente le cartelle con i relativi documenti, sarà la seguente:
		\begin{itemize}
			\item RR;
			\item RP;
			\item RQ;
			\item RA;
			\item template;
		\end{itemize}

		\subsubsection{Google Drive}
		Per lo storage di ulteriori file, documenti e per tutto ciò che non è strettamente correlato alla produzione e consegna del progetto, sarà utilizzato il sistema di cloud storage \textbf{\textit{Google Drive\ped{G}}}. Ciò permette l'accesso dei file a tutti i membri e la possibilità di modificarli, pur non avendo una gestione capillare del versionamento, che, per l'appunto, è affidata a GitHub.

		\subsubsection{Asana}
		Per permettere una più semplice suddivisione del lavoro e una conseguente assegnazione delle mansioni, si è scelto di utilizzare il software \textbf{Asana}. Esso permette la creazione, modifica e assegnazione di \textit{task\ped{G}} e \textit{subtask\ped{G}}. Per ognuno di essi, va indicato un assegnatario e una data di scadenza. La creazione di task è limitata alla sola figura di \textit{\RdP}, che coordina il lavoro tra i membri del gruppo, evitando squilibri e conflitti di interessi. I task devono essere quanto più specifici possibile, e devono descrivere nel dettaglio i ruoli che l'assegnatario andrà a svolgere. Ogni componente del gruppo è tenuto a verificare le mansioni assegnate con frequenza giornaliera, ad aggiornare lo stato dei propri task e notificare l'eventuale completamento, e a esporre o rispondere a dubbi e interrogativi sollevati inerenti all'attività assegnata, tramite l'apposita conversazione disponibile per ogni singolo task.