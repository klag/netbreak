\newpage

\section{Processi di supporto}

	\subsection{Documentazione}
	In questa sezione del documento verrà spiegata la struttura, classificazione e metodo di revisione dei documenti del gruppo \textit{\gruppo}.
	
		\subsubsection{Nomenclatura e versione del documento}
		Ogni documento deve rispettare una denominazione comune, il più possibile chiara, affinché venga individuato facilmente il nome e la versione del documento. Il gruppo ha scelto di utilizzare la seguente nomenclatura:
		\begin{center}
			\textit{NomeDelDocumento X\_Y\_Z.pdf}
		\end{center}
		All'interno del nome, \textit{X\_Y\_Z} indica la versione del documento nel seguente modo:
		\begin{itemize}
			\item \textit{X}: indica il numero di pubblicazioni del documento. Questo indice è incrementato esclusivamente dal \textit{\RdP} in seguito alla sua approvazione finale. L’incremento di tale indice, azzera automaticamente gli indici Y e Z;
			\item \textit{Y}: indica il numero di verifiche e viene incrementato esclusivamente dai \textit{\Vers}. L'incremento di tale indice, azzera automaticamente l'indice Z;
			\item \textit{Z}: indica il numero di aggiornamenti minori effettuati prima o in seguito a una verifica o approvazione. Viene incrementato progressivamente e viene azzerato solo in seguito a una modifica degli indici X e Y.
		\end{itemize}
		Ogni modifica della versione del documento deve riflettersi nel changelog, nel nome del documento e nel frontespizio in corrispondenza della voce \textit{Versione}.
	
		\subsubsection{Classificazione del documento}
		Per tutti i documenti redatti dal gruppo occorre sempre specificare una classe, che può essere
		\begin{itemize}
		\item\textbf{Interna}: consultazione limitata al solo gruppo;
		\item \textbf{Esterna}: destinato ad una consultazione esterna al gruppo, quindi al proponente e committente.
		\end{itemize} 
	
		\subsubsection{Ciclo di vita del documento}
		Un documento può trovarsi in tre stati:
		\begin{itemize}
			\item \textbf{Bozza};
			\item \textbf{In attesa di verifica};
			\item \textbf{Approvato}.
		\end{itemize}
		
		i documenti in \textbf{Bozza} sono quelli in fase di stesura da parte del redattore. Ultimata la stesura il documento passa textbf{In attesa di verifica}, ovvero passano in mano al \Ver. Infine i documenti vengono consegnati al \RdP, il quale ha il compito di approvarli in via definitiva.
		
		\subsubsection{Strumenti di sviluppo del documento}
		Per la stesura dei documenti è stato scelto di utilizzare \textbf{\textit{\LaTeX\ped{G}}}, un linguaggio di markup utilizzato per la stesura di documenti scientifici. \textit{\LaTeX\ped{G}} è un linguaggio semplice e modulare  ed in grado di evitare possibili conflitti provenienti dall'utilizzo di software e piattaforme differenti.
	
		\subsubsection{Struttura del documento}
		Ogni documento redatto dal gruppo avrà una struttura chiara e di facile comprensione per chiunque sia il lettore. Una struttura semplice aiuta nella lettura e nell'individuare immediatamente di cosa si parla in quel documento e da chi è stato redatto, verificato e approvato.
		Di seguito, viene riportata la struttura utilizzata nei documenti:

			\paragraph{Frontespizio}
			\begin{itemize}
				\item{Nome del gruppo;}
				\item{Nome del progetto;}
				\item{Logo del gruppo;}
				\item{Nome del documento;}
				\item{Data di creazione;}
				\item{Ultima modifica;}
				\item{Versione;}
				\item{Stato del documento;}
				\item{Nome e cognome del/i redattore/i;}
				\item{Nome e cognome del/dei \textit{\Ver/i};}
				\item{Nome e cognome del \textit{\RdP} che ha approvato il documento;}
				\item{Classificazione del documento;}
				\item{Distribuzione;}
				\item{Destinatari del documento;}
				\item{Email di riferimento;}
				\item{Abstract del documento.}
			\end{itemize}
	
			\paragraph{Changelog}
			La seconda sezione del documento è una lista che raccoglie e tiene traccia di tutte le modifiche effettuate al documento, dalla sua creazione alla sua approvazione finale. Le modifiche sono indicate secondo questo schema:
			\begin{itemize}
				\item\textbf{Descrizione}: indica le aggiunte e/o modifiche effettuate al documento;
				\item\textbf{Autore e ruolo}: indica l'autore e il ruolo di chi ha effettuato la modifica;
				\item\textbf{Data e versione}: indica la data di modifica e la nuova versione del documento in questione.
			\end{itemize}	

			\paragraph{Indice}
			Ogni documento ad eccezione dei verbali, deve possedere un indice contenente tutte le sezioni presenti nel documento. l'indice è ordinato nella sequenza in cui appaiono i capitoli, le sezioni e sottosezioni e contiene il titolo di ognuna. 
			se nel documento sono presenti immagine o tabella  devono essere indicate con il relativo indice.

			\paragraph{Intestazione e piè di pagina}
			Tutte le pagine del documento, eccetto la prima, hanno un'intestazione e un piè di pagina. L'intestazione è suddivisa in due parti:
			\begin{itemize}
				\item Nell'angolo sinistro, sono indicati il nome del progetto e del documento in questione;
				\item Nell'angolo destro, è presente il nome del capitolo contente la sezione che si sta consultando, ad esclusione del changelog.
			\end{itemize}
			Il piè di pagina, invece, è strutturato nel seguente modo:
			\begin{itemize}
				\item Nell'angolo sinistro, vengono indicati nome e indirizzo email ufficiale del gruppo;
				\item Nell'angolo destro, è presente una numerazione in numeri romani per le pagine relative a changelog e indice. Per tutte le altre pagine, invece, viene utilizzata una numerazione in numeri arabi, che indica il numero di pagina attuale e il numero totale di pagine del documento.
			\end{itemize} 

			\paragraph{Norme tipografiche}
			Questa sezione del documento contiene i criteri riguardanti l'ortografia e la tipografia, utilizzate nel corso dello sviluppo della documentazione del progetto.

				\subparagraph{Stili di testo e punteggiatura}
				\begin{itemize}
					\item \textbf{Grassetto}: deve essere utilizzato per parole importanti, all’interno di frasi o elenchi puntati;
					\item \textit{Corsivo}: deve essere utilizzato per:
					\begin{itemize}
						\item Citazioni;
						\item Parole del glossario (unitamente ad una G maiuscola in pedice);
						\item Riferimenti ai ruoli;
						\item Riferimento all'intero gruppo.
					\end{itemize}
					\item MAIUSCOLO: viene utilizzato unicamente per scrivere acronimi e macro \LaTeX\ presenti nei documenti;
					\item \textsc{maiuscoletto}: viene utilizzato per i riferimenti ad altri documenti;
					\item \LaTeX: viene usato il comando \textbackslash{LaTeX} per ogni occorrenza del termine \LaTeX.
					\item Punteggiatura: ogni simbolo di punteggiatura è seguito da uno spazio, ad eccezione del punto, del punto interrogativo e del punto esclamativo, che sono seguiti da uno spazio e lettera maiuscola. Ogni voce di un elenco puntato deve terminare con punto e virgola, ad eccezione dell'ultima che termina con un punto. 
					\end{itemize}
	
				\subparagraph{Formato data}
				All'interno di ogni documento, tutte le date seguiranno lo standard \textit{\textbf{ISO 8601:2004\ped{G}}}:
				\begin{center}
					\textbf{YYYY-MM-DD}
				\end{center}
				Dove:
				\begin{itemize}
					\item \textbf{YYYY}: indica l'anno;
					\item \textbf{MM}: indica il mese;
					\item \textbf{DD}: indica il giorno.
				\end{itemize}
		
		\subsubsection{Documenti da consegnare}
			\paragraph{\SdF}
			\begin{itemize}
				\item \textbf{Classificazione}: interno;
				\item \textbf{Destinazione}: gruppo e committente;
				\item \textbf{Contenuto}: questo documento contiene lo studio effettuato dal gruppo \textit{\gruppo} su tutti i capitolati e, per ognuno di essi, le motivazioni che hanno portato alla relativa scelta o rifiuto.
			\end{itemize}

			\paragraph{\NdP}
			\begin{itemize}
				\item \textbf{Classificazione}: interno;
				\item \textbf{Destinazione}: gruppo e committente;
				\item \textbf{Contenuto}: lo scopo del documento è raccogliere tutte le convenzioni, gli strumenti e le regole che il gruppo \textit{\gruppo} adotterà durante l'intera realizzazione del progetto. 
			\end{itemize}	

			\paragraph{\AdR}
			\begin{itemize}
				\item \textbf{Classificazione}: esterno;
				\item \textbf{Destinazione}: gruppo, committente e proponente;
				\item \textbf{Contenuto}: questo documento si prefigge lo scopo di dare una visione generale dei requisiti essenziali del progetto e dei relativi casi d'uso.
			\end{itemize}

			\paragraph{\PdP}
			\begin{itemize}
				\item \textbf{Classificazione}: esterno;
				\item \textbf{Destinazione}: gruppo, committente e proponente;
				\item \textbf{Contenuto}: questo documento descrive come il gruppo \textit{\gruppo} ha impiegato tempo e risorse umane, ma anche la pianificazione delle stesse, per le attività future previste per la realizzazione del prodotto richiesto dal progetto.
			\end{itemize}

			\paragraph{\PdQ}
			\begin{itemize}
				\item \textbf{Classificazione}: esterno;
				\item \textbf{Destinazione}: gruppo, committente e proponente;
				\item \textbf{Contenuto}: questo documento descrive come il gruppo \textit{\gruppo} intende raggiungere gli obiettivi di qualità prefissati all'inizio del progetto.
			\end{itemize}

			\paragraph{\G}
			\begin{itemize}
				\item \textbf{Classificazione}: esterno;
				\item \textbf{Destinazione}: gruppo, committente e proponente;
				\item \textbf{Contenuto}: questo documento ha lo scopo di fornire una definizione di tutti i termini tecnici e acronimi, al fine di rendere la lettura comprensibile a tutti i destinatari della documentazione fornita.
			\end{itemize}
		\paragraph{\ST}
		\begin{itemize}
			\item \textbf{Classificazione}: esterno;
			\item \textbf{Destinazione}: gruppo, committente e proponente;
			\item \textbf{Contenuto}: questo documento ha lo scopo di definire nel dettaglio le singole componenti del sistema partendo dai casi d'uso definiti nel documento di Analisi dei Requisiti.
		\end{itemize}

	\subsection{Verifica e Validazione}
	Il processo di Verifica e Validazione ha lo scopo di controllare ed assicurare che documentazione e software prodotti rispecchino quanto previsto dai requisiti. In particolare, la verifica serve a stabilire che il prodotto rispetti le specifiche, mentre in fase di validazione ci si accerta che le specifiche siano rispettate nel modo più consono.

		\subsubsection{Analisi}

			\paragraph{Analisi statica}
			L'analisi statica è applicata ai documenti di testo, e consiste nel trovare errori sintattici ed ortografici. Una prima verifica viene effettuata dai \textit{\Vers} e, successivamente, dal \textit{\RdP}. Le due tecniche scelte sono la \textbf{\textit{Formal Walkthrough\ped{G}}} e la \textbf{\textit{Fagan Inspection\ped{G}}}.
			\begin{itemize}
				\item \textbf{\textit{Formal Walkthrough\ped{G}}}: consiste nell'individuare quanti più errori di sintassi e di ortografia possibili, senza concentrarsi su particolari e specifici errori. Questa tecnica sarà utilizzata dai \textit{\Vers}, i quali trascriveranno gli errori più frequenti in una apposita lista, necessaria per la tecnica di \textit{Fagan Inspection\ped{G}};
				\item \textbf{\textit{Fagan Inspection\ped{G}}}: si basa su una lettura attenta dei documenti, basandosi sulla lista degli errori stilata durante la \textit{Formal Walkthrough\ped{G}}. Questo processo acquisirà rilevanza in concomitanza con l'incremento della lista di possibili errori stilata dai \textit{\Vers}.
			\end{itemize}

			\paragraph{Analisi dinamica}
			L'analisi dinamica è applicata esclusivamente sul software prodotto, in quanto consiste nell'effettuare test per verificare il corretto funzionamento dell'applicativo.
	
			
		\subsubsection{Livelli dei test}
		Il sistema adottato per il testing prende atto della struttura comune a tutto il software, decomponendo quindi questa fase partendo da elementi semplici fino a giungere al test dell'intero sistema, seguendo opportune strategie di integrazione.
		
		\paragraph{Test dei moduli}
		Questa componente di test è atta ad analizzare ogni piccola unità software singolarmente. Così facendo, si minimizzano gli errori a partire da ogni singola componente. Testando i singoli moduli, si rende necessaria la presenza di componenti fittizie per simulare parti di sistema incomplete e altrimenti non testabili singolarmente. Il codice utilizzato per denotare questa fase di test sarà:
		
		\begin{center}
			TM[Identificativo test]
		\end{center}
	
		\paragraph{Test di integrazione}
		I test denominati di integrazione riguardano insiemi di moduli, che vengono testati una volta assemblati. Alternando passi di integrazione e di controllo, come in questo caso, significa necessariamente predisporre test su componenti ancora incomplete. Il codice utilizzato per denotare questa fase di test sarà:
		
		\begin{center}
			TI[Identificativo test]
		\end{center}
		
		\paragraph{Test sul sistema}
		I test sul sistema valutano ogni caratteristica di qualità di prodotto nella sua completezza. Il prodotto, giunto al periodo di testing sul sistema, si ritiene giunto ad una realizzazione definitiva. I test riguardano numerosi aspetti cruciali, e si possono riassumere nelle seguenti categorie:
		
		\begin{itemize}
			\item \textbf{Facility test}: Vengono testate le funzionalità del prodotto, affinché svolgano le attività preposte in modo corretto
			\item \textbf{Compatibility test}: Si testa la compatibilità del prodotto con software di terze parti che interagirà con il sistema. Nel nostro caso specifico, trattasi ad esempio dei web browser che verranno utilizzati dagli utilizzatori finali
			\item \textbf{Security test}: Viene valutata la robustezza del sistema in materia di sicurezza
			\item \textbf{Performance test}: Si analizza la reattività e le prestazioni del sistema, anche in situazioni di particolare carico
		\end{itemize}
	
		Il codice utilizzato per denotare questa fase di test sarà:
		
		\begin{center}
			TS[Identificativo test]
		\end{center}
	
		\paragraph{Test di regressione}
		Durante lo sviluppo, qualora si renda necessaria la modifica di alcune componenti, il test di regressione consiste nel rieseguire tutti i test riguardanti tale componente aggiornata. Il codice utilizzato per denotare questa fase di test sarà:
		
		\begin{center}
			TR[Identificativo test]
		\end{center}
	
		\paragraph{Test di accettazione}
		Esso è il controllo ultimo e definitivo che il prodotto realizzato dal fornitore sia conforme alla richiesta del proponente. Il test è effettuato in presenza del proponente, ed in caso di esito positivo il software può essere rilasciato.
		
		Il codice utilizzato per denotare questa fase di test sarà:
		
		\begin{center}
			TA[Identificativo test]
		\end{center}
	
		\subsubsection{Strumenti utilizzati}
		\begin{itemize}
			\item \textbf{\textit{W3C Markup Validator Service\ped{G}}}: validatore online di codice \textit{HTML\ped{G}}, utile per trovare eventuali errori nel codice. L'indirizzo web di riferimento è: \url https://validator.w3.org/;
			\item \textbf{\textit{CSSLint\ped{G}}}: validatore online di codice \textit{CSS\ped{G}}, utile per trovare eventuali errori nel codice. L'indirizzo web di riferimento è: \url http://csslint.net/;
			\item \textbf{\textit{SQLFiddle\ped{G}}}: validatore online di codice \textit{MySQL\ped{G}}. Utile per verificare la consistenza del codice del database.\MakeUppercase{è} compatibile anche con \textit{PostgreSQL\ped{G}} L'indirizzo web di riferimento è: \url http://sqlfiddle.com/;
			\item \textbf{Strumenti per sviluppatori Google Chrome}: è uno strumento atto a controllare l'utilizzo di risorse del prodotto, l'usabilità su differenti dispositivi, nonchè simulare situazioni di carico della rete. L'indirizzo web di riferimento è: \url https://www.google.it/chrome/
		\end{itemize}