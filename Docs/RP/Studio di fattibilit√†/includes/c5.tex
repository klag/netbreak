\newpage
\section{Capitolato C5}

\subsection{Descrizione}

Il capitolato proposto da Red Babel tratta la creazione di un \textit{framework\ped{G}} Monolith, implementato come package di \textit{Rocket.chat\ped{G}}. Lo scopo di questo
\textit{framework\ped{G}} è fornire agli sviluppatori uno strumento per l'implementazione di bolle interattive all'interno del sistema di messaggistica \textit{Rocket.chat\ped{G}}.
Queste bolle interattive nascono dalla neccessità sempre più frequente di trasmettere informazioni in modo più evoluto e immediato.
Sono stati individuati tre tipi di bolle interattive, la cui implementazione è richiesta obbligatoriamente:

\begin{itemize}
	\item \textbf{Rich media bubble:} rappresentano quei contenuti come link a video
	o a pagine web, dei quali è utile mostrarne il contenuto direttamente all'interno di una bolla nel sistema di messaggistica;
	\item \textbf{Self-updating bubble:} rappresentano quei contenuti dei quali è interessante
	consultare lo stato anche a distanza di tempo dalla condivisone, come
	può essere la variazione di prezzo di un articolo o le previsioni del tempo;
	\item \textbf{Editing bubble:} il contenuto più comune, appartenente a questo tipo di bolle, è il sondaggio. In questo caso, un utente può creare un sondaggio
	composto da un quesito e da alcune riposte, mentre gli altri utenti hanno la possibilità di scegliere una delle possibili risposte.
\end{itemize}


\subsection{Dominio}

Questo \textit{framework\ped{G}} si offre a tutti gli sviluppatori che necessitano di introdurre un sistema più avanzato per lo scambio di dati, come può essere un'azienda che vuole offrire una \textit{customer communication dashboard\ped{G}} ai suoi clienti come strumento di assistenza o un bot che risponda in modo automatico alle domande più frequenti. 

\subsection{Tecnologie}

Per la realizzazione di questo capitolato è necessario l'utilizzo delle seguenti tecnologie:
\begin{itemize}
	\item \textbf{\textit{JavaScript ES6\ped{G}}}, per lo sviluppo della parte \textit{back-end\ped{G}};
	\item \textbf{\textit{Angular2\ped{G}}}, per la parte \textit{front-end\ped{G}};
	\item \textbf{\textit{HTML5\ped{G}}}, \textbf{\textit{CSS3\ped{G}}} e \textbf{\textit{SASS\ped{G}}}, per la struttura e l'aspetto grafico;
\end{itemize}

\subsection{Aspetti critici}

Questo package deve essere facilmente installabile in un \textit{server\ped{G}} dove
risiede un'istanza di \textit{Rocket.chat\ped{G}}, ad esempio attraverso un package
manager. Le \textit{API\ped{G}} che il \textit{framework\ped{G}} deve fornire dovranno dare la possibilità
allo sviluppatore di implementare una grande varietà di bolle interattive,
che potranno essere poi inserite in un contesto più complesso come, ad esempio, una \textit{customer communication dashboard\ped{G}}. Questo potrà generare delle difficoltà
nel riconoscimento del tipo di contenuto da visualizzare e nel modo
in cui dovrà essere visualizzato. 

\subsection{Considerazioni conclusive}

Il capitolato tratta un argomento sicuramente interessante ed attuale,
vista l'enorme mole di dati che ogni giorno viene scambiata tra utenti
di tutto il mondo. In questo caso specifico, non riteniamo così interessante
l'implementazione di tale \textit{framework\ped{G}}, poichè grandi colossi mondiali, quali \textit{Telegram\ped{G}}, Whatsapp ed Apple, stanno implementando queste funzioni con ottimi risultati e in continuo miglioramento.
