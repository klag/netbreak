\newpage
\section{Capitolato C1 (Scelto)}

\subsection{Descrizione}

Il capitolato proposto da \proponente\ riguarda la creazione di una \textit{web app\ped{G}} per la gestione di microservizi. Le funzionalità principali che dovranno essere fornite saranno la vendita e l'acquisto di microservizi tramite un apposito \textit{marketplace\ped{G}}. Per ogni API presente sulla piattaforma, il proprietario dovrà fornirne l'interfaccia pubblica. Il progetto non richiede soltanto di gestire la compravendita di API attraverso il marketplace, ma anche di integrare delle funzioni di controllo tramite un API Gateway. Quest'ultimo dovrà essere in grado di effettuare un'analisi statistica dei dati di utilizzo più rilevanti e regolare l'accesso ai microservizi registrati, tramite opportune API key, impedendone l'utilizzo a coloro che non sono in possesso di una chiave valida. La creazione della web app sarà affrontata tramite l'utilizzo delle consuete tecnologie per lo sviluppo web lato \textit{front-end\ped{G}} e \textit{back-end\ped{G}}. La peculiarità del progetto è la realizzazione dell'API Gateway in linguaggio \textit{Jolie\ped{G}}, come da richiesta del proponente. Jolie, infatti, rappresenta un nuovo emegente linguaggio di programmazione open-source orientato ai microservizi.

\subsection{Dominio applicativo}

Lo scopo di questo prodotto è fornire la possibilità a sviluppatori e utilizzatori di avere a propria disposizione un valido strumento per l'acquisto e la vendita regolamentata di microservizi. Questo garantisce degli standard qualitativi al cliente finale, che può valutare e scegliere ciò che più rispetta le proprie esigenze. Il bacino di utenza riguarda, dunque, tutte le aziende e gli sviluppatori che necessitano di utilizzare dei microservizi e i creatori stessi che li mettono a disposizione sul mercato.

\subsection{Tecnologie}

Per la realizzazione di questo capitolato sono necessarie conoscenze di base per lo sviluppo di applicativi web. Nel nostro particolare caso la scelta può ricadere su:
\begin{itemize}
	\item \textbf{\textit{HTML5\ped{G}}} e \textbf{\textit{CSS3\ped{G}}}, per la struttura e l'aspetto grafico dell'applicazione web;
	\item \textbf{\textit{Bootstrap 3\ped{G}}}, come \textit{framework\ped{G}} \textit{CSS\ped{G}};
	\item \textbf{\textit{JavaScript\ped{G}}} e \textbf{\textit{jQuery\ped{G}}}, per la parte comportamentale front-end;
	\item \textbf{\textit{PHP 7\ped{G}}}, per le funzionalità back-end;
	\item \textbf{\textit{Oracle MySQL\ped{G}}}, come \textit{database SQL\ped{G}};
	\item \textbf{\textit{Jolie\ped{G}}}, per la realizzazione dell'API Gateway.
\end{itemize}

\subsection{Aspetti critici}

L'aspetto cruciale nella realizzazione del progetto riguarda, secondo la nostra analisi, la corretta implementazione dell'API Gateway, in modo da permettere le funzioni richieste dal committente. La volontà di introdurre uno \textit{SLA\ped{G}} per i servizi, e la conseguente necessità di gestione avanzata delle statistiche tramite API Gateway, evidenziano come questa parte del progetto sia in realtà il punto su cui è necessario prestare la massima attenzione.

\subsection{Considerazioni conclusive}

Il capitolato è stato designato come il più allettante da parte del gruppo per numerosi aspetti. Innanzitutto, si manifesta l'interesse condiviso da parte di tutti i componenti per le tecnologie web. Allo stesso tempo, acquisire un nuovo linguaggio
di programmazione per questa sfida risulta un'esperienza
formativa e interessante: di fatto, oggigiorno, queste nuove tecnologie vengono sempre più utilizzate, anche da grosse aziende (vedi Amazon, Netflix, Google, Facebook, etc.). Infine, il gruppo ha manifestato un forte interesse nell'intraprendere un'attività lavorativa con un'azienda emergente, nata proprio da un progetto universitario.