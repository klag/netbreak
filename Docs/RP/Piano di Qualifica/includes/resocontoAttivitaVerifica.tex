\newpage
\section{Resoconto attività di verifica}
In questa sezione del documento vengono descritti e analizzati gli esiti delle attività di verifica svolte su tutti i documenti che vengono consegnati nelle varie revisioni di avanzamento del progetto.
	
	\subsection{Revisione dei Requisiti}
			
		\subsubsection{Tracciamento}
		Il \textit{team\ped{G}} ha deciso di utilizzare il software interno NetBreakDB, in modo da facilitare il tracciamento casi d’uso-requisiti e requisiti-fonti.
			
		\subsubsection{Analisi statica dei documenti}
		L’analisi dei documenti mediante la tecnica \textit{Walkthrough\ped{G}} ha reso possibile individuare alcuni errori frequenti, dai quali è stata stilata una lista di controllo alla quale verrà applicata l’\textit{Inspection\ped{G}} per le future attività di verifica.
			
		\subsubsection{Esiti verifiche}
		Di seguito, sono riportati gli esiti delle verifiche sottoposte a tutti i documenti, per il calcolo dell’\textit{Indice Gulpease\ped{G}}.
	
	
		\begin{table}[H]
		\begin{longtable}{>{\centering\arraybackslash}p{5cm}|>{\centering\arraybackslash}p{5cm} | >{\centering\arraybackslash}p{5cm}}
			\hline
			\rowcolor{Gray}
			\textbf{Documento} & \textbf{Indice Gulpease} & \textbf{Esito} \\
			\hline
			\textit{\AdR} & 88,13& Superato\\
			\hline
			\textit{\G} & 43,32& Superato \\
			\hline
			\textit{\NdP} & 49,13& Superato \\
			\hline
			\textit{\PdP}& 50,12& Superato\\
			\hline
			\textit{\PdQ} & 42,15& Superato\\
			\hline
			\textit{\SdF} & 54,12& Superato\\
			\hline
			\textit{Verbale 1 - 28 novembre 2016}		& 		&	Superato	\\
			\hline
			\textit{Verbale 2 - 01 dicembre 2016}		& 		&	Superato	\\
			\hline
			\textit{Verbale 3 - 12 dicembre 2016}		& 		&	Superato	\\
			\hline
			\textit{Verbale 4 - 22 dicembre 2016}		& 		&	Superato	\\
			\hline
			\textit{Verbale 5 - 28 dicembre 2016}		& 		&	Superato	\\
			\hline
		\end{longtable}
		\caption{Resoconto verifiche documenti - Revisione dei Requisiti}
	\end{table}
	
	\subsection{Revisione di Progettazione}
	
	\subsubsection{Tracciamento}
	Il \textit{team\ped{G}} ha deciso di utilizzare il software interno NetBreakDB, in modo da facilitare il tracciamento requisiti-componenti e requisiti-classi.
	
	\subsubsection{Analisi statica dei documenti}
	L’analisi dei documenti mediante la tecnica \textit{Walkthrough\ped{G}} ha reso possibile individuare alcuni errori frequenti, dai quali è stata stilata una lista di controllo alla quale verrà applicata l’\textit{Inspection\ped{G}} per le future attività di verifica.
	
	\subsubsection{Esiti verifiche}
	Di seguito, sono riportati gli esiti delle verifiche sottoposte a tutti i documenti, per il calcolo dell’\textit{Indice Gulpease\ped{G}}.


	\begin{table}[H]
		\begin{longtable}{>{\centering\arraybackslash}p{5cm}|>{\centering\arraybackslash}p{5cm} | >{\centering\arraybackslash}p{5cm}}
			\hline
			\rowcolor{Gray}
			\textbf{Documento} & \textbf{Indice Gulpease} & \textbf{Esito} \\
			\hline
			\textit{\ST} & & Superato \\
			\hline
			\textit{\AdR} & & Superato\\
			\hline
			\textit{\G} & & Superato \\
			\hline
			\textit{\NdP} & & Superato \\
			\hline
			\textit{\PdP}& & Superato\\
			\hline
			\textit{\PdQ} & & Superato\\
			\hline
			\textit{\SdF} & & Superato\\
		\end{longtable}
		\caption{Resoconto verifiche documenti - Revisione di Progettazione}
	\end{table}