\newpage
\section{Resoconto attività di verifica}
In questa sezione del documento vengono descritti e analizzati gli esiti delle attività di verifica svolte su tutti i documenti che vengono consegnati nelle varie revisioni di avanzamento del progetto.
	
	\subsection{Revisione dei Requisiti}
			
		\subsubsection{Tracciamento}
		Il team ha deciso di utilizzare il software interno \textit{NetBreakDB}, in modo da facilitare il tracciamento casi d’uso-requisiti e requisiti-fonti.
			
		\subsubsection{Analisi statica dei documenti}
		L’analisi dei documenti mediante la tecnica \textit{Walkthrough} ha reso possibile individuare alcuni errori frequenti, dai quali è stata stilata una lista di controllo alla quale verrà applicata l’\textit{Inspection} per le future attività di verifica.
			
		\subsubsection{Esiti verifiche}
		Di seguito, sono riportati gli esiti delle verifiche sottoposte a tutti i documenti, per il calcolo dell’indice Gulpease.
	
	
		\begin{table}[H]
		\begin{longtable}{>{\centering\arraybackslash}p{5cm}|>{\centering\arraybackslash}p{5cm} | >{\centering\arraybackslash}p{5cm}}
			\hline
			\rowcolor{Gray}
			\textbf{Documento} & \textbf{Indice Gulpease} & \textbf{Esito} \\
			\hline
			\textit{\NdP} & 49 & Superato\\
			\hline
			\textit{\PdP} & 50 & Superato \\
			\hline
			\textit{\PdQ} & 42 & Superato\\
			\hline
			\textit{\AdR} & 68 & Superato \\
			\hline
			\textit{\SdF} & 54 & Superato\\
			\hline
			\textit{\G}& 43 & Superato\\
			\hline
			\textit{Verbale Interno - 28/11/2016}		& 	60	&	Superato	\\
			\hline
			\textit{Verbale Interno - 01/12/2016}		& 	63	&	Superato	\\
			\hline
			\textit{Verbale Interno - 12/12/2016}		& 	61	&	Superato	\\
			\hline
			\textit{Verbale Esterno - 22/12/2016}		& 	59	&	Superato	\\
			\hline
			\textit{Verbale Interno - 28/12/2016}		& 	61	&	Superato	\\
			\hline
		\end{longtable}
		\caption{Resoconto verifiche documenti - Revisione dei Requisiti}
	\end{table}
	
	\subsection{Revisione di Progettazione}
	
	\subsubsection{Tracciamento}
	Il team ha deciso di utilizzare il software interno \textit{NetBreakDB}, in modo da facilitare il tracciamento requisiti-componenti e requisiti-classi.
	
	\subsubsection{Analisi statica dei documenti}
	L’analisi dei documenti mediante la tecnica \textit{Walkthrough} ha reso possibile individuare alcuni errori frequenti, dai quali è stata stilata una lista di controllo alla quale verrà applicata l’\textit{Inspection} per le future attività di verifica.
	
	\subsubsection{Esiti verifiche}
	Di seguito, sono riportati gli esiti delle verifiche sottoposte a tutti i documenti, per il calcolo dell’indice Gulpease.


	\begin{table}[H]
		\begin{longtable}{>{\centering\arraybackslash}p{5cm}|>{\centering\arraybackslash}p{5cm} | >{\centering\arraybackslash}p{5cm}}
			\hline
			\rowcolor{Gray}
			\textbf{Documento} & \textbf{Indice Gulpease} & \textbf{Esito} \\
			\hline
			\textit{\ST} & 67  & Superato\\
			\hline
			\textit{\NdP} & 57  & Superato\\
			\hline
			\textit{\PdP} & 55 & Superato \\
			\hline
			\textit{\PdQ} & 54  & Superato\\
			\hline
			\textit{\AdR} & 70  & Superato \\
			\hline
			\textit{\SdF} & 54  & Superato\\
			\hline
			\textit{\G}& 49 & Superato\\
			\hline
			\textit{Verbale Interno - 27/01/2017}		& 	55	&	Superato	\\
			\hline
			\textit{Verbale Interno - 04/02/2017}		& 	64	&	Superato	\\
			\hline
			\textit{Verbale Interno - 13/02/2017}		& 	58	&	Superato	\\
			\hline
			\textit{Verbale Esterno - 16/02/2017}		& 	60	&	Superato	\\
			\hline
			\textit{Verbale Interno - 17/02/2017}		& 	57	&	Superato	\\
			\hline
			\textit{Verbale Interno - 21/02/2017}		& 	63	&	Superato	\\
			\hline
			\textit{Verbale Interno - 02/03/2017}		& 	61	&	Superato	\\
			\hline
			\textit{Verbale Interno - 03/03/2017}		& 	61	&	Superato	\\
			\hline
		\end{longtable}
		\caption{Resoconto verifiche documenti - Revisione di Progettazione}
	\end{table}