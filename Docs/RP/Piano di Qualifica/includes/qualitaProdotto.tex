\newpage
\section{Qualità di prodotto}
	
Al fine di garantire una buona qualità di prodotto, il \textit{team\ped{G}} ha individuato dallo standard \textit{ISO/IEC 9126\ped{G}} le qualità che ritiene più importanti durante tutto il ciclo di vita del prodotto \progetto. Per ognuna delle qualità individuate, sono stati definiti obiettivi e metriche coerenti con i livelli di qualità dichiarati.

\subsection{Definizione degli obiettivi di qualità}
Elenco obiettivi di qualità individuati:
\begin{itemize}
	\item Funzionalità (6.1);
	\item Affidabilità (6.2);
	\item Usabilità (6.3);
	\item Efficienza (6.4);
	\item Manutenibilità (6.5);
	\item Portabilità (6.6).
\end{itemize}
	
	\subsubsection{Funzionalità}
	Rappresenta la capacità del prodotto nel soddisfare tutti i requisiti descritti nel documento \textsc{AnalisiDeiRequisiti 1\_0\_0.pdf}.  
		
		\begin{itemize}
		\item \textbf{Misura:} quantità di requisiti mappati;
		\item \textbf{Metrica:} la sufficienza è definita dal numero di requisiti obbligatori. Ogni requisito secondario andrà ad aumentare la qualità del prodotto, ma la priorità dovrà essere data ai requisiti obbligatori;
		\item \textbf{Strumenti:} il sistema dovrà superare tutti i test previsti. Gli strumenti utilizzati sono descritti nel documento \textsc{NormeDiProgetto 1\_0\_0.pdf}.
			
		\end{itemize}
	
	\subsection{Affidabilità}
		Il sistema dovrà dimostrarsi il più possibile robusto e garantire, in modo particolare, la disponibilità del servizio di \textit{API Gateway\ped{G}} in percentuale più alta possibile. 
		
		\begin{itemize}
			\item \textbf{Misura:} l'unità di misura utilizzata sarà il numero di chiamate a \textit{microservizi\ped{G}} andate a buon fine diviso il numero di chiamate totali;
			\item \textbf{Metrica: }il limite di sufficienza riguarderà il tasso di successo delle chiamate e il tempo che il sistema impiegherà per portarle a buon fine;
			\item \textbf{Strumenti: }il sistema dovrà superare tutti i test previsti.
			
		\end{itemize}
	
	\subsection{Usabilità}
		L’interfaccia web deve essere di facile utilizzo per l’utente, tenendo però in considerazione il target di utenza prevista. Si presume che il sistema verrà utilizzato da utenti con un livello di conoscenza informatica abbastanza evoluto. 
		
		\begin{itemize}
			\item \textbf{Misura: }l’unità di misura usata sarà una valutazione soggettiva dell’usabilità. Questo
			è dovuto all’inesistenza di una metrica oggettiva adatta allo scopo;
			\item \textbf{Metrica: }non esiste una metrica adeguata per determinare la sufficienza;
			\item \textbf{Strumenti: }si veda il documento \textsc{NormeDiProgetto 1\_0\_0.pdf}.
			
		\end{itemize}
	
	\subsection{Efficienza}
		Il sistema deve fornire tutte le funzionalità nel più breve tempo possibile, riducendo al minimo l’utilizzo di risorse.
		
		\begin{itemize}
			\item \textbf{Misura: }il tempo di latenza di chiamata ad un \textit{microservizio\ped{G}}, e il tempo di latenza nel caricamento delle pagine web;
			\item \textbf{Metrica: }i valori di latenza massimi verranno definiti in base al \textit{microservizio\ped{G}};
			\item \textbf{Strumenti: }si veda il documento \textsc{NormeDiProgetto 1\_0\_0.pdf}.
			
		\end{itemize}
	
	\subsection{Manutenibilità}
		Il sistema deve essere comprensibile ed estensibile in modo facile e verificabile.
		
		\begin{itemize}
			\item \textbf{Misura: }l’unità di misura utilizzata saranno le metriche per il codice descritte nella sezione 3.6.3;
			\item \textbf{Metrica: }il prodotto dovrà avere la sufficienza su tutte le metriche descritte nella sezione 3.6.3;
			\item \textbf{Strumenti: }si veda il documento \textsc{NormeDiProgetto 1\_0\_0.pdf}.
			
		\end{itemize}
	
	\subsection{Portabilità}
		Il \textit{front-end\ped{G}} dovrà essere utilizzabile dalle versioni più recenti dei più comuni browsers in commercio. 
		Il \textit{back-end\ped{G}}, basandosi sul linguaggio \textit{Jolie\ped{G}}, dipenderà dalle risorse necessarie all’utilizzo di questo linguaggio.
		
		\begin{itemize}
			\item \textbf{Misura: }l’unità di misura utilizzata saranno le metriche per il codice descritte nella sezione 3.6.3;
			\item \textbf{Metrica: }il prodotto dovrà avere la sufficienza su tutte le metriche descritte nella sezione 3.6.3;
			\item \textbf{Strumenti: }si veda il documento \textsc{NormeDiProgetto 1\_0\_0.pdf}.
			
		\end{itemize}
	
	\subsection{Altre qualità}
		Saranno inoltre importanti per il prodotto le seguenti qualità:
		
		\begin{itemize}
			\item \textbf{Incapsulamento: }applicare le tecniche di incapsulamento per aumentare la manutenibilità
			e la possibilità di riutilizzo del codice. Sarà quindi favorito l’uso di interfacce ove
			possibile;
			\item \textbf{Coesione: }riguarda le funzionalità che collaborano al fine di raggiungere uno stesso obiettivo. Esse devono risiedere nello stesso componente, ed hanno lo scopo di ridurre l’indice di dipendenza, favorire la semplicità e la manutenibilità.
			
		\end{itemize}
		
	\subsection{Organizzazione (sezione in dubbio)}
		\subsubsection{Criteri di qualità}
		Lo standard \textit{ISO/IEC 9126\ped{G}} descrive gli obiettivi e le relative metriche necessarie a valutarli.
		I criteri valutativi sono così suddivisi:
		\begin{itemize}
			\item \textbf{Qualità esterna:} le metriche esterne, specificate nella norma \textit{ISO/IEC 9126-2\ped{G}}, valutano i comportamenti del prodotto sulla base di prove, dell'operatività e dell'osservazione durante la sua esecuzione, in funzione degli obiettivi stabiliti;
			\item \textbf{Qualità interna:} è specificata nella norma \textit{ISO/IEC 9126-3\ped{G}} e si applica al software non eseguibile
			durante la progettazione e la codifica dello stesso, le misure effettuate consentono di prevedere il livello di qualità esterna ed interna, in quanto gli attributi interni influenzano quelli esterni e di uso;
			\item \textbf{Qualità d'uso:} rappresenta la qualità dal punto di vista dell'utente finale, viene raggiunto quando sono raggiunte la qualità esterna e quella interna, le metriche di valutazione sono fornite nella norma \textit{ISO/IEC 9126-4\ped{G}}.
		\end{itemize}
	
		\subsubsection{Pianificazione strategica e temporale}
		Al fine di ridurre la propagazione di errori, l’attività di verifica di codice e documentazione dovrà essere sistematica. 
		Per ogni attività sarà seguita una procedura che consiste in uno studio preliminare, volto ad evitare errori di natura tecnica o logica, portando ad un alleggerimento dell'attività di verifica finale.
		Inoltre, si ha come obiettivo quello di rispettare le scadenze riportate nel documento \textsc{PianoDiProgetto 1\_0\_0.pdf}.
			
		\subsubsection{Responsabilità}
		Il \textit{\RdP}\ ha il compito di:
		\begin{itemize}
			\item Accertarsi che attività e ruoli vengano rispettate, come definite nel \hbox{\textsc{PianoDiProgetto 1\_0\_0.pdf}};
			\item Accertarsi che le attività di verifica vengano eseguite sistematicamente, come descritto nelle \textsc{NormeDiProgetto 1\_0\_0.pdf};
			\item Approvare e sancire la distribuzione di un documento.
		\end{itemize}
		
		I \textit{\Vers}\ hanno il compito di:
		\begin{itemize}
			\item Effettuare l‘attività di verifica in modo sistematico utilizzando metodi e strumenti descritti nel \textsc{PianoDiQualifica 1\_0\_0.pdf};
			\item Documentare e segnalare ogni errore riscontrato.
		\end{itemize}	
		
	\subsection{Misure e metriche}
	Per conseguire dei risultati concreti, il processo di verifica deve fornire dei dati quantificabili per poter valutare se gli obiettivi sono stati raggiunti o meno. Queste è possibile tramite l’utilizzo di metriche e misure. Considerata la poca esperienza del gruppo, questi valori potrebbero essere inizialmente non molto accurati, ma utilizzando il modello incrementale visto nel \hbox{\textsc{PianoDiProgetto 1\_0\_0.pdf}} si potrà migliorarne la precisione.
	Per ogni metrica sono indicati due range:
	\begin{itemize}
		\item \textbf{Range-accettazione: }rappresenta i valori minimi da raggiungere per il conseguimento degli obiettivi di qualità;
		\item \textbf{Range-ottimale: }rappresenta i valori entro cui dovrebbe collocarsi la misurazione. Non sono vincolanti ma nel caso in cui non si raggiungessero questi valori, sarà necessaria una verifica più accurata e una ulteriore discussione, nella riunione successiva, delle cause di questo scostamento.
	\end{itemize}
	
	\subsubsection{Indice Gulpease}
	L'indice Gulpease è un indice di leggibilità di un testo tarato sulla lingua italiana. Rispetto ad altri ha il vantaggio di utilizzare la lunghezza delle parole in lettere anziché in sillabe, semplificandone il calcolo automatico.
	L'indice di Gulpease considera due variabili linguistiche: la lunghezza della parola e la lunghezza della frase rispetto al numero delle lettere.
	La formula per il suo calcolo è la seguente:
	
	\[ 80+\frac{300*(numero delle frasi)-10*(numero delle lettere)}{numero delle parole} \]
	
	
	I risultati sono compresi tra 0 e 100, dove il valore "100" indica la leggibilità più alta e "0" la leggibilità più bassa. In generale risulta che testi con un indice:
	\begin{itemize}
		\item inferiore a 80 sono difficili da leggere per chi ha la licenza elementare;
		\item inferiore a 60 sono difficili da leggere per chi ha la licenza media;
		\item inferiore a 40 sono difficili da leggere per chi ha un diploma superiore.
	\end{itemize}
	
	I parametri presi in considerazione saranno:
	
	\begin{itemize}
		\item \textbf{Range-accettazione: }[35/100];
		\item \textbf{Range-ottimale: }[45/100].
	\end{itemize}
	
	\subsubsection{Metriche per i processi}
	Le metriche scelte prendono in considerazione tempi e costi, in modo da poter controllare efficacemente i processi e riuscire ad attenersi a quanto deciso nel documento \textsc{PianoDiProgetto 1\_0\_0.pdf}. 
	Per queste metriche non vengono forniti Range-accettazione e Range-ottimale poiché bisognerà consultare il \textsc{PianoDiProgetto 1\_0\_0.pdf} per fare le dovute considerazioni.
	
	\begin{itemize}
		\item \textbf{PPC (Partial Planned Cost): }indica il costo pianificato per lo svolgimento di un sottoinsieme di attività. Si misura in euro e in ore;
		\item \textbf{PV (Planned Value): }indica il valore che si prevede ottenere dal completamento delle attività pianificate. Per questo progetto tale valore corrisponde alla spesa richiesta per il completamento delle attività. Si misura in euro e in ore;
		\item \textbf{EV (Earned Value): }indica il valore ottenuto tramite le attività completate alla data corrente. Per questo progetto tale valore corrisponde alla spesa richiesta per il completamento delle attività. Si misura in euro e in ore;
		\item \textbf{AC (Actual Cost): }indica il costo effettivamente sostenuto alla data corrente. Si misura in euro e in ore. Aiuta a calcolare altre metriche;
		\item \textbf{BAC (Budget at Completion): }costo previsto per portare a termine il progetto. Si misura in euro e in ore. Mantiene traccia della spesa totale preventivata all'inizio del progetto;
		\item \textbf{ETC (Estimate to Complete): }indica i costi pianificati per portare a termine le attività di progetto rimanenti alla data corrente. Corrisponde al PV riesaminato allo stato corrente del progetto ma senza tenere conto delle attività completate. Si misura in euro e in ore;
		\item \textbf{EAC (Estimated at Completion): }revisione del costo stimato per la realizzazione del progetto, ossia il BAC rivisto allo stato corrente del progetto. Si misura in euro e in ore e si ottiene dalla formula: EAC = AC + ETC;
		\item \textbf{SV (Schedule Variance): }è un indicatore di efficacia, mostra se si è o meno in linea con la pianificazione temporale rispetto alle attività nella baseline. Una schedule variance positiva indica che il gruppo è in anticipo rispetto al Piano di Progetto, altrimenti è in ritardo. Si ottiene dalla formula: SV = EV - PV;
		\item \textbf{BV (Budget Variance): }indica se la spesa sostenuta alla data corrente è superiore o inferiore a quella preventivata. Una budget variance positiva indica che si è speso meno di quanto inizialmente previsto,  altrimenti viceversa. Si ottiene dalla formula: BV = PV - AC.
		
	\end{itemize}
	
	
	\subsubsection{Metriche per il codice}
	Le metriche per il software ora descritte non sono definitive, ma saranno affinate successivamente.
	
	\begin{itemize}
		\item \textbf{Attributi per classe: }un grande numero di attributi interni ad una classe mostra probabilmente la necessità di suddividere la classe in più classi relazionate tra loro.
		
		\begin{itemize}
			\item \textbf{Range-accettazione: }[0/18];
			\item \textbf{Range-ottimale: }[2/9].
		\end{itemize}
		
		\item \textbf{Numero livelli annidamento: }mostra il livello di annidamento dei metodi. Un numero alto implica una bassa astrazione del codice ed un' elevata complessità.
		
		\begin{itemize}
			\item \textbf{Range-accettazione: }[1/8];
			\item \textbf{Range-ottimale: }[1/4].
		\end{itemize}
		
		\item \textbf{Numero parametri per metodo: }un valore elevato indica che probabilmente il metodo ha un sovraccarico di funzionalità.
		
		\begin{itemize}
			\item \textbf{Range-accettazione: }[0/8];
			\item \textbf{Range-ottimale: }[0/5].
		\end{itemize}
		
		\item \textbf{Accoppiamento afferente: }indica il numero di classi esterne ad un package che dipendono da esso. Un grande valore indica una forte dipendenza del software per il package in questione, un valore basso invece indica una bassa utilità del package per il resto del software.
		
		\begin{itemize}
			\item \textbf{Range-accettazione: }non ancora definito;
			\item \textbf{Range-ottimale: }non ancora definito.
		\end{itemize}
		
		\item \textbf{Accoppiamento efferente: }il numero di classi di un package che dipendono da package esterni. Un valore basso indica che il package ha numerose funzionalità indipendenti dal resto del software.
		
		\begin{itemize}
			\item \textbf{Range-accettazione: }non ancora definito;
			\item \textbf{Range-ottimale: }non ancora definito.
		\end{itemize}
		
		\item \textbf{Source Line Of Code (SLOC): }il numero di istruzioni presenti nel codice. Questa metrica fornisce una stima della complessità del programma. È utile anche per dare una stima di quanto il codice incrementerà nel tempo, semplificando così la pianificazione.
		
		\begin{itemize}
			\item \textbf{Range-accettazione: }[0/18];
			\item \textbf{Range-ottimale: }[2/9].
		\end{itemize}
		
		\item \textbf{Complessità Ciclomatica: }una metrica sviluppata da Thomas J. McCabe che consente di stimare la complessità di un programma misurando il numero di cammini linearmente indipendenti attraverso il grafo di controllo di flusso.
		
		\begin{itemize}
			\item \textbf{Range-accettazione: }non ancora definito;
			\item \textbf{Range-ottimale: }non ancora definito.
		\end{itemize}
		
		
		
	\end{itemize}
	
	
	\subsection{Analisi}
	
	\subsubsection{Analisi statica}
	L'analisi statica non necessita dell'esecuzione del codice oggetto ed è applicabile sin da subito su codice e documenti prodotti. Essa ha lo scopo di trovare anomalie e può essere eseguita nei due modi seguenti.
	
	\paragraph{{Formal Walkthrough\ped{G}}}
	Questa tecnica consistente nella ricerca a largo spettro di qualsiasi tipo di errore, nel modo più generico possibile. 
	Questa tecnica è utilizzata nelle prime fasi di verifica. Durante ogni fase di verifica verrà stilata una lista degli errori più frequenti, in modo da facilitare l’individuazione delle anomalie nelle fasi successive. 
	Nel momento in cui si avrà a disposizione una lista sufficientemente dettagliata, si potrà passare al metodo Inspection.
	
	
	\paragraph{{Formal Inspection\ped{G}}}
	Questo metodo si basa sulla lista prodotta precedentemente con il metodo \textit{Formal Walkthrough\ped{G}}. In questo modo si andrà a cercare in modo mirato gli errori già individuati in passato, prestando comunque attenzione a nuovi possibili errori, che andranno poi ad arricchire la lista.
	
	
	\subsubsection{Analisi dinamica}
	Da definire
	
	
	
	
	