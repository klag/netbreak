\newpage
\section{Introduzione}

\subsection{Scopo del documento}
Questo documento descrive le scelte e le strategie attutate per permettere di raggiungere determinati obbiettivi di qualità misurabili. Per questo sarà necessario un continuo processo di verifica per individuare e correggere errori ed eventuali sprechi di risorse.

\subsection{Scopo del prodotto}
Lo scopo del prodotto è la realizzazione di un \textit{API Market\ped{G}} per l'acquisto e la vendita di \textit{microservizi\ped{G}}. Il sistema offrirà la possibilità di registrare nuove \textit{API\ped{G}} per la vendita, permetterà la consultazione e la ricerca di \textit{API\ped{G}} ai potenziali acquirenti, gestendo i permessi di accesso ed utilizzo tramite creazione e controllo di relative \textit{API key\ped{G}}. Il sistema, oltre alla web app stessa, sarà corredato di un \textit{API Gateway\ped{G}} per la gestione delle richieste e il controllo delle chiavi, e fornirà funzionalità avanzate di statistiche per il gestore della piattaforma e per i fornitori dei \textit{microservizi\ped{G}}.

\subsection{Riferimenti normativi}
\begin{itemize}
\item \textsc{NormeDiProgetto 1\_0\_0.pdf};
\item \textbf{Capitolato d’appalto C1:} APIM: An API Market Platform\\ \url{http://www.math.unipd.it/~tullio/IS-1/2016/Progetto/C1.pdf};
\end{itemize}

\subsection{Riferimenti informativi}
\begin{itemize}
	\item \textsc{PianoDiProgetto 1\_0\_0.pdf};
	\item \textbf{Qualità di prodotto:} \url{http://www.math.unipd.it/~tullio/IS-1/2016/Dispense/L10.pdf};
	\item \textbf{Qualità di processo} \url{http://www.math.unipd.it/~tullio/IS-1/2016/Dispense/L11.pdf};
	\item \textbf{CapacityMaturityModel:} \url{https://en.wikipedia.org/wiki/Capability_Maturity_Model};
	\item \textbf{CapacityMaturityModel Integration:} \url{https://en.wikipedia.org/wiki/Capability_Maturity_Model_Integration};
	\item \textbf{Standard ISO 9001:} \url{https://en.wikipedia.org/wiki/ISO_9000#Contents_of_ISO_9001};
	\item \textbf{Standard ISO/IEC 9126:2001:} \url{https://en.wikipedia.org/wiki/ISO/IEC_9126};
	\item \textbf{Standard ISO/IEC 15504:} \url{https://en.wikipedia.org/wiki/ISO/IEC_15504};
	\item \textbf{Indice Gulpease:} \url{https://it.wikipedia.org/wiki/Indice_Gulpease};
	
\end{itemize}


\subsection{Glossario}
Per semplificare la consultazione e disambiguare alcune terminologie tecniche, le voci indicate con la lettera \textit{G} a pedice sono descritte approfonditamente nel documento \textsc{Glossario 1\_0\_0.pdf}.