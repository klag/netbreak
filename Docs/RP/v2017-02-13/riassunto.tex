\section{Introduzione}

	\begin{itemize}
		\item Tipologia: riunione decisionale;
		\item Redatto da: \DS;
		\item Data: 13 febbraio 2017;
		\item Luogo: LuF2 - Plesso Paolotti;
		\item Ora inizio: 13.45;
		\item Ora fine: 18.20;
		\item Presenti: \AN, \DAN, \DS, \AS, \NS, \MC;
		\item Assenti: nessuno.		
	\end{itemize}

\section{Riassunto}
Questa riunione aveva lo scopo di decidere una policy di vendita sensata per le API, in modo da poter poi concorrentemente aggiornare gli Use Case di \textsc{AnalisiDeiRequisiti 2\_0\_0.pdf} in base alla decisione presa e poter poi progettare la base di dati con i campi necessari a monitorare l'utilizzo dei microservizi. Abbiamo optato per tre diverse policy di vendita mutuamente esclusive, sulla base di tre tipologie di consumo delle API.


\begin{itemize}
	\item \textbf{Policy per numero di chiamate}: Si paga solo in base al numero di chiamate effettuate.
Questa policy \`{e} stata pensata per API con traffico e tempo di utilizzo costanti. La vendita avverr\`{a} prima dell'uso a blocchi di $1*10^x$ o $2*10^x$ o $5*10^x$, con $x \in [2,3,4]$, resi disponibili a seconda della tipologia di utente e l'utilizzo potr\`{a} essere monitorato in tempo reale, con la possibilit\`{a} da parte di un utente di ricevere un'email raggiunto un opportuno $\epsilon$ di chiamate rimanenti.
	
	\begin{itemize}
		\item \textbf{CM}: Costo mantenimento API Market per chiamata. Serve per evitare che API Market vada in perdita;
		\item \textbf{CS}: Guadagno impostato dallo sviluppatore;
		\item \textbf{\%APIMarket}: Percentuale trattenuta da API Market sul guadagno impostato;
		\item \textbf{NC}: Numero di chiamate che il cliente vuole acquistare.
	\end{itemize}
	formula: $(CM + GS + \%APIMarket) * NC
$

	\item \textbf{Policy per traffico}: Si paga per ogni chiamata e per il traffico generato.
Questa policy \`{e} stata pensata per API che richiedono il passaggio di dati pesanti (esempio: un rielaboratore di immagini). L'acquisto non avviene prima dell'uso, invece l'utilizzo viene monitorato per poi scalare l'importo dovuto dal conto dell'utente. C'\`{e} la possibilit\`{a} di ricevere un'email ogni volta che si supera una threshold pre-stabilita di importo dovuto.
	
	\begin{itemize}
		\item \textbf{CM}: Costo mantenimento API Market per chiamata. Serve per evitare che API Market vada in perdita;
		\item \textbf{GpB}: Guadagno per Byte impostato dallo sviluppatore;
		\item \textbf{\%APIMarket}: Percentuale trattenuta da API Market sul guadagno impostato;
		\item \textbf{NB}: Numero Byte di traffico (ottenuti da Byte dei parametri in entrata + di ritorno).
	\end{itemize}
	formula: $CM + (GpB + \%APIMarket) * NB
$
	
	\item \textbf{Policy per tempo di utilizzo}: Si paga per ogni chiamata e per il tempo di utilizzo.
Questa policy \`{e} pensata per API che eseguono calcoli complessi e quindi richiedono tanto sforzo computazionale, ma non generano tanto traffico. L'acquisto non avviene prima dell'uso, invece l'utilizzo viene monitorato per poi scalare l'importo dovuto dal conto dell'utente. C'\`{e} la possibilit\`{a} di ricevere un'email ogni volta che si supera una threshold pre-stabilita di importo dovuto.
	
	\begin{itemize}
		\item \textbf{CM}: Costo mantenimento API Market per chiamata. Serve ad evitare che API Market vada in perdita;
		\item \textbf{GpMS}: Guadagno per Millisecondo impostato dallo sviluppatore;
		\item \textbf{\%APIMarket}: Percentuale trattenuta da API Market sul guadagno impostato;
		\item \textbf{NMS}: Numero millisecondi trascorsi tra l'invocazione della chiamata ed il ritorno della chiamata.
	\end{itemize}
	formula: $CM + (GpMS + \%APIMarket) * NMS
$
\end{itemize}

