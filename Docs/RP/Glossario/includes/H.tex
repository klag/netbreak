\newglossaryentry{Hammer.js}
{
	name=Hammer.js,
	description={Libreria JavaScript che aiuta lo sviluppatore ad aggiungere all'applicazione il supporto touchscreen per le pagine.}
}

\newglossaryentry{HTML}
{
	name=HTML,
	description={Acronimo per HyperText Markup Language (traduzione letterale: linguaggio a marcatori per ipertesti), è il linguaggio di markup solitamente usato per la formattazione e impaginazione di documenti ipertestuali disponibili nel World Wide Web sotto forma di pagine web. È un linguaggio di pubblico dominio, la cui sintassi è stabilita dal World Wide Web Consortium (W3C).}
}

\newglossaryentry{HTML5}
{
	name=HTML5,
	description={Linguaggio di markup per la strutturazione di pagine web, pubblicato come W3C Recommendation da ottobre 2014. Introduce notevoli migliorie alle versioni precedenti, mantenendo la retrocompatibilità.}
}

\newglossaryentry{HTTP}
{
	name=HTTP,
	description={Acronimo per HyperText Transfer Protocol (protocollo di trasferimento di un ipertesto), protocollo a livello applicativo usato come principale sistema per la trasmissione di informazioni sul web, ovvero in un'architettura tipica client-server. Le specifiche del protocollo sono gestite dal W3C.}
}
