\hypertarget{A}{}

\newglossaryentry{AlexaSDK}
{
	name=AlexaSDK,
	description={Sistema di comando vocale sviluppato da Amazon}
}
\newglossaryentry{Android}
{
	name=Android,
	description={Sistema operativo open source per dispositivi mobili, sviluppato da Google Inc. e basato su kernel Linux. Inoltre, mette a disposizione una piattaforma per lo sviluppo di applicazioni per dispositivi mobili}
}
\newglossaryentry{Angular 2}
{
	name=Angular 2,
	description={Framework gratuito per il front-end web, open source ed evoluzione di AngularJS. \MakeUppercase{è} scritto in linguaggio TypeScript}
}
\newglossaryentry{API}
{
	name=API,
	description={Acronimo per Application Programming Interface, indica un insieme di funzioni software ad alto livello disponibili al programmatore. Solitamente sono composte di poche istruzioni volte alla realizzazione di una specifica azione. Un esempio di API sono le librerie software messe a disposizione da un certo linguaggio di programmazione}
}
\newglossaryentry{API Gateway}
{
	name=API Gateway,
	description={Strumento che filtra e reindirizza le richieste utente per le varie API, fornendo il servizio, anche se questo non è presente sul server del marketplace}
}
\newglossaryentry{API Key}
{
	name=API Key,
	description={Codice che identifica univocamente una specifica API e funge da token segreto che ne regola l'accesso e l'utilizzo}
}
\newglossaryentry{API Market}
{
	name=API Market,
	description={Piattaforma che permette il commercio di API}
}
\newglossaryentry{Asana}
{
	name=Asana,
	description={Applicazione web, disponibile anche per dispositivi mobili, progettata per aiutare i team di progetto a monitorare il proprio lavoro e migliorare la collaborazione. Ogni progetto è composto di più attività, chiamate task, le quali richiedono lo svolgimento di determinati compiti. Gli utenti possono aggiungere note, commenti, allegati e tag, il tutto notificando, via email automatica o notifica push, ogni altro membro del team che lavora al progetto}
}
\newglossaryentry{Astah}
{
	name=Astah,
	description={Strumento di modellazione UML per la creazione di vari tipi di diagrammi. La versione Community è gratuita, a differenza di quella Professional}
}
\newglossaryentry{AWS}
{
	name=AWS,
	description={acronimo per Amazon Web Services, rappresenta una collezione di servizi di cloud computing che compongono la piattaforma on demand (su richiesta) offerta dall'azienda Amazon. Questi servizi sono operativi in 12 regioni geografiche in cui Amazon stessa ha suddiviso il globo. Tra questi servizi i più conosciuti sono Amazon Elastic Compute Cloud (EC2) e Amazon Simple Storage Service (S3)}
}
\newglossaryentry{AWS Lambda}
{
	name=AWS Lambda,
	description={Servizio di elaborazione serverless, che esegue il codice utente in risposta a determinati eventi e gestisce automaticamente le risorse di elaborazione, tra cui la manutenzione di server e del sistema operativo, il provisioning e il ridimensionamento automatico della capacità, lo sviluppo di patch per codice e protezione e il monitoraggio e la creazione di log. Può essere usato per estendere altri servizi AWS con logica personalizzata oppure creare servizi di back-end in grado di operare con la scalabilità, le prestazioni e la sicurezza di AWS}
}

