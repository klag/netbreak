%Stile e dimensione del documento
\documentclass[a4paper,11pt]{article}

%Pacchetti da importare
\usepackage{ifthen}
\usepackage[italian]{babel}
\usepackage[utf8]{inputenc}
\usepackage[T1]{fontenc}
\usepackage{float}
\usepackage{chapterbib}
\usepackage{graphicx}
\usepackage[a4paper,top=2.5cm,bottom=2.5cm,left=2.5cm,right=2.5cm]{geometry}
\usepackage[colorlinks=true, urlcolor=black, citecolor=black, linkcolor=black]{hyperref}
\usepackage{booktabs}
\usepackage{fancyhdr}
\usepackage{totpages}
\usepackage{tabularx, array}
\usepackage{dcolumn}
\usepackage{epstopdf}
\usepackage{booktabs}
\usepackage{fancyhdr}
\usepackage{longtable}
\usepackage{calc}
\usepackage{datatool}
\usepackage[bottom]{footmisc}
\usepackage{listings}
\usepackage{textcomp}
\usepackage{titlesec}
\usepackage{rotating}
\usepackage{multirow}
\usepackage{placeins}
\usepackage{color}
\usepackage[table,usenames,dvipsnames]{xcolor}
\usepackage{hyperref}
\usepackage{makecell}
\usepackage{breakurl}
\usepackage{hyperref}
\usepackage{multirow}
\usepackage{xcolor,colortbl}
\usepackage{afterpage}
\usepackage{mathtools}
\usepackage{verbatim} 
\usepackage[toc,page]{appendix}

%glossary code%
\usepackage[nonumberlist,xindy]{glossaries}

\newglossarystyle{myaltlistgroup}{%
	\setglossarystyle{altlistgroup}%
	\renewcommand*{\glsgroupheading}[1]{%
		
		\newpage
		\item\makebox[\linewidth]{\Large\textbf{\glsgetgrouptitle{##1}}}%
		\vspace*{-\baselineskip}%
		\item\makebox[\linewidth]{\hspace*{3cm}\hrulefill\hspace*{3cm}}%
	}%
}



%Stile fancy per il documento (Header e footer)
\pagestyle{fancy}
%Rimuovo l'indentazione
\setlength{\parindent}{0pt}

%Imposto l'intestazione
\lhead{\Large{\progetto} \\ \footnotesize{\documento}}
%Linea sotto l'intestazione
\renewcommand{\headrulewidth}{0.4pt} 

%Footer
\lfoot{\textit{\gruppoLink}\\ \footnotesize{\email}}
%Footer con numero romano per le prime pagine
\rfoot{\thepage}
\cfoot{}
%Linea sopra il footer
\renewcommand{\footrulewidth}{0.4pt}   

%Imposta il livello degli elenchi 
\setcounter{secnumdepth}{7}
\setcounter{tocdepth}{7}

%Paragrafi impostati come una sezione
\titleformat{\paragraph}{\normalfont\normalsize\bfseries}{\theparagraph}{1em}{}
\titlespacing*{\paragraph}{0pt}{3.25ex plus 1ex minus .2ex}{1.5ex plus .2ex}

\titleformat{\subparagraph}{\normalfont\normalsize\bfseries}{\thesubparagraph}{1em}{}
\titlespacing*{\subparagraph}{0pt}{3.25ex plus 1ex minus .2ex}{1.5ex plus .2ex}

\makeatletter
\newcounter{subsubparagraph}[subparagraph]
\renewcommand\thesubsubparagraph{
  \thesubparagraph.\@arabic\c@subsubparagraph}
\newcommand\subsubparagraph{
  \@startsection{subsubparagraph}
    {6}
    {\parindent}
    {3.25ex \@plus 1ex \@minus .2ex}
    {0.75em}
    {\normalfont\normalsize\bfseries}}
\newcommand\l@subsubparagraph{\@dottedtocline{6}{10em}{5.5em}} 
\newcommand{\subsubparagraphmark}[1]{}
\makeatother

\makeatletter
\newcounter{subsubsubparagraph}[subsubparagraph]
\renewcommand\thesubsubsubparagraph{
  \thesubsubparagraph.\@arabic\c@subsubsubparagraph}
\newcommand\subsubsubparagraph{
  \@startsection{subsubsubparagraph}
    {7}
    {\parindent}
    {3.25ex \@plus 1ex \@minus .2ex}
    {0.75em}
    {\normalfont\normalsize\bfseries}}
\newcommand\l@subsubsubparagraph{\@dottedtocline{7}{10em}{6.5em}}
\newcommand{\subsubsubparagraphmark}[1]{}
\makeatother

\renewcommand\appendixtocname{Appendice}
\renewcommand\appendixpagename{Appendice}